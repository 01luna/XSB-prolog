\documentclass[11pt]{report}

\usepackage{epsf,epsfig,subfigure,latexsym,makeidx,latexsym,xspace,amssymb,alltt}

\pagestyle{headings}

\setlength{\topmargin}{-0.25in}
\setlength{\headheight}{10pt}
\setlength{\headsep}{30pt}
\setlength{\oddsidemargin}{0.0in}
\setlength{\evensidemargin}{0.0in}
\setlength{\textheight}{8.5in}
\setlength{\textwidth}{6.5in}
\setlength{\footskip}{50pt}

% JF: not allowed in Latex2e
%\setlength{\footheight}{24pt}

\setlength{\parskip}{2mm}               % space between paragraphs

\def\cut{\mbox{\tt '!'/0}}
\def\not{\mbox{${\tt '\backslash+'/1}$}}

\newtheorem{example}{Example}[section]

\newenvironment{Prog}{\begin{tt}\begin{tabular}[c]{l}}{\end{tabular}\end{tt}}

\newcommand{\comment}[1]{}
\newcommand{\ourprolog}{XSB}
\newcommand{\smallourprolog}{xsb}
\newcommand{\version}{Version 2.0}
\newcommand{\LRD}{LRD-stratified}

\newcommand{\demo}[1]{\hspace*{1.5cm}{\tt #1}}
\newcommand{\desc}[1]{\item[{\tt #1}]\hspace*{1mm}\newline}
\newcommand{\desce}[1]{\item[{\tt #1}]}
\newcommand{\ourrepeatitem}[1]{\item[{\mbox{\tt #1}}]\ \\ \vspace*{-.35in}}
\newcommand{\ouritem}[1]{\item[{\mbox{\tt #1}}]\ \\}
\newcommand{\ournewitem}[2]{\item[{\mbox{\tt #1}}]\hspace*{\fill}{\mbox{\sf #2}}\ \\}

\newcommand{\stuff}[1]{
        \begin{minipage}{4in}
        {\tt \samepage
        \begin{tabbing}
        \hspace{8mm} \= \hspace{6mm} \= \hspace{10mm} \= \hspace{55mm} \= \kill
        #1 \hfill
        \end{tabbing}
        }
        \end{minipage}
}

\newcommand{\longline}{\noindent\rule{\textwidth}{.01in}}


\begin{document}

%%--------------------------- cut above before incorporating into document



\newenvironment{qrules}{\begin{quote}\sf\begin{tabular}[t]{l}}%
{\end{tabular}\end{quote}}


\newcommand{\obj}{\textit{obj}}
\newcommand{\db}[1]{\ensuremath{\mathcal{#1}}}

\newcommand{\xany}{\textsf{any}}

\newcommand{\xplus}{\ensuremath{^+}}
\newcommand{\xstar}{\ensuremath{^*}}
\newcommand{\xinv}{\ensuremath{^{-1}}}
\newcommand{\xopt}{\ensuremath{^{?}}}

\newcommand{\xto}[1]{\ensuremath{^{#1}}}
\newcommand{\xcond}[1]{\ensuremath{\textsf{if}(#1)}}
\newcommand{\xif}[1]{\ensuremath{\textsf{if}(#1)}}
\newcommand{\xmu}[1]{\ensuremath{\tcmu(#1)}}
\newcommand{\xmuif}[2]{\ensuremath{\tcmu(#1,#2)}}


\newcommand{\xconc}{\ensuremath{{\cdot}}}
\newcommand{\xor}{\ensuremath{|}}

\newcommand{\nnot}{\mbox{$\neg$}}                           % negation
\newcommand{\query}{\mbox{$\, ?\! - \, $}}                  % query
\newcommand{\impl}                                          % implication
  {\mbox{\Large $\; {\bf \leftarrow} \;$}}  
\newcommand{\isa}{\,{\bf{:}}\,}
\newcommand{\subcl}{\,{\bf{::}}\,}
\newcommand{\eq}{\ensuremath{\doteq}}                           % equation

% f-logic arrows

\newcommand{\fd}{\ensuremath{{\rightarrow}}}                   % scalar
\newcommand{\bfd}{\ensuremath{{\bullet\!\!\!\fd}}}            % " + inheritable
\newcommand{\mvd}{\ensuremath{{\rightarrow\!\!\!\!\rightarrow}}}  % multivalued
\newcommand{\bmvd}{\ensuremath{{\bullet\!\!\!\mvd}}}              % " + inheritable
\newcommand{\Fd}{\ensuremath{{\Rightarrow}}}                      % scalar signature
\newcommand{\Mvd}{\ensuremath{{\Rightarrow\!\!\!\!\Rightarrow}}}  % multiv signature



% curved f-logic arrows

\newcommand{\anyd}{\ensuremath{\leadsto}}                       % non-inheritable
\newcommand{\bleadsto}{\ensuremath{\bullet\!\!\!\leadsto}}     % inheritable
\newcommand{\banyd}{\bleadsto}                              % "
\newcommand{\Leadsto}{\ensuremath{\approx}\!\!{>}}            % signature
\newcommand{\Anyd}{\Leadsto}                                % "

\newcommand{\FdConstr}{\ensuremath{\stackrel{constr}{\Fd}}}
\newcommand{\MvdConstr}{\ensuremath{\stackrel{constr}{\Mvd}}}

\newlength{\flogicindent}


\newlength{\flength}
\newlength{\counterlength}


\newcommand{\la}{\ensuremath{\,\leftarrow\,}}

\newcommand{\anon}{\_}

\newcommand{\note}[1]{\textit{[[#1]]}}
\newcommand{\nterm}[1]{\ensuremath{\langle}\textit{#1}\ensuremath{\rangle}}


\newcommand{\NI}{\noindent}

\newcommand{\bs}{\ensuremath{\backslash}}
\newcommand{\FLIP}{{\mbox{\sc Flip}}\xspace}
\newcommand{\FLORA}{{\mbox{\sc Flora}}\xspace}
\newcommand{\FLORID}{{\mbox{\sc Florid}}\xspace}
\newcommand{\fl}{{F-logic}\xspace}


\newcommand{\consts}{\ensuremath{\mathcal{C}}}
\newcommand{\funcs}{\ensuremath{\mathcal{F}}}
\newcommand{\preds}{\ensuremath{\mathcal{P}}}
\newcommand{\vars}{\ensuremath{\mathcal{V}}}

\newcommand{\HU}{\ensuremath{U}}
\newcommand{\HB}{\ensuremath{\mathcal{HB}}}
\newcommand{\ext}{\ensuremath{^{\star}}}

\newcommand{\bksl}{\symbol{92}}
\newcommand{\dq}{\symbol{34}}


\chapter{\FLORA's Secrets}

\begin{center}
{\Large {\bf By Michael Kifer, Bertram Lud\"ascher and Guizhen Yang}}
\end{center}

\section{Introduction}

\FLORA is a sophisticated F-logic to XSB compiler. It translates a program
written in the F-logic language \cite{KLW95} (which must be in a file with
extension {\tt .flr}, {\it e.g.}, {\tt file.flr}) and outputs a file with
extension {\tt .P} ({\it e.g.}, {\tt file.P}), which is a regular XSB
program. This program is then passed to XSB for compilation (yielding {\tt
  file.O}) and execution.

The origins of \FLORA trace back to the \FLIP compiler developed by Bertram
Ludaescher, and the basic architectures of the two compilers are the same.
However, \FLORA has many features not found in \FLIP, it has a much more
optimized compiler and its tokenizer and parser are very different from
\FLIP's.

The programming language supported by \FLORA is mostly the dialect of F-logic
developed by the \FLORID group.\footnote{
  %%
  See {\tt http://www.informatik.uni-freiburg.de/$\sim$dbis/florid/} for more
  details.
  %%
  }
%%
In particular, \FLORA fully supports the versatile syntax for \FLORID path
expressions. However, \FLORA has extensions of itw own, and some features
differ significantly.

\FLORA is part of the official distribution of XSB beginning with version
2.0. It is organized as an XSB package and lives in the directory
%%
\begin{quote}
 \verb|<xsb-installation-directory>/packages/flora/|  
\end{quote}
%%
\FLORA is fully integrated into the XSB system, including its module
system. In particular, \FLORA modules can invoke predicates defined in
other XSB modules, and regular XSB modules can query the objects defined in
\FLORA. At present, XSB is the only platform where \FLORA can run, because
\FLORA heavily relies on tabling and the well-founded semantics for negation
that are available in XSB.

\FLORA comes with a number of demo programs that live in
%%
\begin{quote}
 \verb|<xsb-installation-directory>/packages/flora/demos/|  
\end{quote}
%%
The demos can be run by issuing the command
\verb|rundemo('demo-filename').| at the \FLORA prompt, {\it e.g.},
\verb|rundemo('flogic_basics').|
There is no need to change to the demo directory.

As mentioned earlier, an XSB programmer can invoke \FLORA objects from
other XSB programs. However, the easiest way to get the feel of the system
is to start \FLORA shell and begin to enter queries interactively.  To
this end, you must first invoke XSB and then load the {\tt flora}
package:
%%
\begin{quote}
  \tt
foo>~~xsb  \\
\tt
[ XSB loading messages omitted ]\\
\tt
| ?- [flora].\\
\tt
[flora loaded]\\
\tt
[flrshell loaded]\\
\tt
| ?-
\end{quote}
%%
At this point, it is possible to use a limited number of \FLORA
commands, but to run queries you must enter \FLORA command loop:
%%
\begin{quote}
  \tt
| ?- flora\_shell.  \\
 \tt
[ FLORA chat deleted ] \\
 \tt
FLORA> ?-
\end{quote}
%%

At this point, \FLORA takes over and \fl syntax becomes the
norm. To get back to the XSB command loop, type {\tt Control-D} or 
%%
\begin{quote}
  \tt
| ?- end.  
\end{quote}
%%

\section{\FLORA Shell Commands} \label{sec-shell-commands}

The following \FLORA shell commands are supported:
\texttt{
\begin{tabbing}
  dynconsult('FILE',[...]) \= : \=  dynconsult('FILE') with options [...] \kill
  help                      \> : \> show this info \\
  compile('FILE')           \> : \> compile FILE.P; create FILE.O \\
  flcompile('FILE')         \> : \> compile FILE.flr; create FILE.P and FILE.O \\
  flcompile('FILE',[...])\footnotemark  \> : \> flcompile('FILE') with options [...] \\
  flconsult('FILE')         \> : \> compile FILE.flr; consult FILE.P \\
  flconsult('FILE',[...])   \> : \> flconsult('FILE') with options [...] \\
  load('FILE[.EXT]')\footnotemark       \> : \> consult FILE.flr, FILE.P or FILE.O \\
  \symbol{91}'FILE[.EXT]',...\symbol{93}        \> : \> consult a list of .flr, .P, or .O files \\
  dyncompile('FILE')        \> : \> compile FILE.flr to dynamic code \\
  dyncompile('FILE',[...])  \> : \> dyncompile('FILE') with options [...] \\
  dynconsult('FILE')        \> : \> dyncompile FILE.flr; dynamically load FILE.P \\
  dynconsult('FILE',[...])  \> : \> dynconsult('FILE') with options [...] \\
  dynload('FILE[.EXT]')\footnotemark    \> : \> dynamically load FILE.flr or FILE.P \\
  \symbol{123}'FILE[.EXT]',...\symbol{125}        \> : \> dynload a list of .flr or .P files \\
  rundemo('FILE')           \> : \> flconsult a demo from FLORA demos directory \\
  rundemo('FILE',[...])     \> : \> rundemo('FILE') with options [...] \\
  abolish\_all\_tables\footnotemark        \> : \> flush all tabled data \\
  all                       \> : \> show all solutions (default) \\
  one                       \> : \> show solutions one by one \\
  maxerr(all/N)             \> : \> set/show the max number of errors FLORA reports \\
  end                       \> : \> say CIAO to FLORA \\
  halt                      \> : \> quit FLORA and XSB
\end{tabbing}
}
\addtocounter{footnote}{-3}\footnotetext{Currently supported is equality checking option: eqlevel(N), N=0,1.}
\addtocounter{footnote}{1}\footnotetext{File extension is optional, but must be .flr, .P or .O if supplied.}
\addtocounter{footnote}{1}\footnotetext{File extension is optional, but must be .flr or .P if supplied.}
\addtocounter{footnote}{1}\footnotetext{Tables need to be flushed if the database has been
changed since last evaluation.}

All commands with a FILE argument passed to them use the XSB {\tt library\_directory}
predicate to search for the module, except that the command {\tt rundemo(FILE)}
first looks for {\tt FILE} in the \FLORA demo directory. In general, all XSB commands
can be executed from \FLORA shell, if the corresponding XSB library has already been
loaded.

After a syntax error, parsing error, or compiling error, \FLORA shell will dicard tokens
read from the current input stream until an end of file character or a rule delimeter (.) is
encountered. If \FLORA shell seems to hang forever after the prompt:
\begin{verbatim}
[FLORA: discarding tokens]
\end{verbatim}
hitting the Enter key once, then entering a .\ character and Enter again will normally reset
the current input buffer and have \FLORA shell print out the prompt:
\begin{verbatim}
FLORA> ?-
\end{verbatim}
and wait for a new query or command.

 
\section{\fl and \FLORA by Example}

In the future, this section will contain a number of small
introductory examples illustrating the use of F-logic and \FLORA. Meanwhile, the
reader is referred to the excellent tutorial written by the members of the
\FLORID project.\footnote{
  %%
  See {\tt http://www.informatik.uni-freiburg.de/$\sim$dbis/florid/} for more
  details.
  %%
  }
%%
Since \FLORA and \FLORID share much of their syntax, most examples in that
tutorial are also valid \FLORA programs.



\section{Inside \FLORA}


\FLORA consists of the following modules:
\begin{itemize}
\item \texttt{flrshell.P}: top-level module that provides the \FLORA shell
  commands, i.e., for compiling and consulting \FLORA programs
  (\texttt{flcompile/1}, \texttt{flconsult/1}), setting the ouput mode
  (\texttt{all/0} or \texttt{one/0} solution(s) at a time) and -- last
  but not the least -- for directly issuing queries against the loaded
  database/program (see Section~\ref{sec-shell-commands} for a full description
  of shell commands).
\item \texttt{flrtokens.P}: \FLORA tokenizer.
\item \texttt{flrparser.P}: DCG parser for \fl.
\item \texttt{flrcompiler.P}: \FLORA compiler that translates \fl to XSB.
\item \texttt{flrutils.P}: miscellaneous utility predicates.
\end{itemize}



\subsection{How \FLORA Works}



\paragraph{Overview.}

As an \fl to XSB compiler, \FLORA first parses its argument file and then
compiles it to XSB syntax. For instance the command
\begin{verbatim}
        FLORA> ?- flconsult(myprog).
\end{verbatim}
compiles the program \verb|'myprog.flr'| (using \texttt{flrparser.P}) into
the XSB file \verb|'myprog.P'| (using \texttt{flrcompiler.P}).  Look at
this file to see what has become of your F-logic program! The compilation
consists mainly of a flattening procedure sketched below.  Next,
\verb|'myprog.P'| is compiled by XSB, yielding \verb|'myprog.O'| which is
then loaded and executed.  If \verb|'myprog.flr'| contains queries, they are
immediately executed by XSB (provided there are no errors).

The main purpose of the \FLORA shell, however, is to allow the evaluation
of ad-hoc F-logic queries. For example, after having requested the
execution of the \texttt{'default.flr'} file from the demo directory (using
the command \texttt{FLORA>~?-~rundemo(default).}), you may ask
\begin{verbatim}
    FLORA> ?-  X..kids[                 % Whose kids
                 self -> K;             % ... (list them by name)
                 hobbies ->>            % ... have hobbies
                 {H:dangerous_hobby}    % ... that are dangerous?
    ]. 
\end{verbatim}
\FLORA will parse, flatten, and evaluate this query in the same way as
the queries in a source file.


\paragraph{Flattening F-logic.}

Consider, e.g., the following complex F-logic molecule, representing
facts about the object \texttt{mary} (the syntax of \fl is given in
Section \ref{sec-basic-flogic}):

\begin{quote}
{\small\begin{verbatim}
mary:employee[age->29;kids->>{tim,leo};salary@(1998)->a_lot].
\end{verbatim}}
\end{quote}

As described in \cite{KLW95}, any complex F-logic molecule can be
decomposed into a conjunction of simpler F-logic atoms. These latter atoms
can be directly represented using Prolog syntax.  For the different kinds
of F-logic atoms we use different Prolog predicates. For instance, the
result of translating the above F-molecule might be:

\begin{quote}
{\small \begin{verbatim}
isa_(mary,employee).                    % mary:employee.
fd_(mary,@(age),29).                    % mary[age->29].
mvd_(mary,@(kids),tim).                 % mary[kids->>{tim}].
mvd_(mary,@(kids),leo).                 % mary[kids->>{leo}].
fd_(mary,@(salary,1998),a_lot).         % mary[salary@(1998)->a_lot].
\end{verbatim}}
\end{quote}



\paragraph{Closure Axioms.}

The flattening process alone is not enough to convert an \fl program
into Prolog, because of the additional semantics hidden behind the notions of
the subclass relationship, inheritance, and scalar methods. This semantics
is captured through the facts and rules called \emph{closure axioms}, which
must be explicitly added to the flattened user program.  Closure axioms are
static and reside in the subdirectory \texttt{closure/}. In this way, the
closure axioms rules get appended to every user program. These rules capture:

\begin{itemize}
\item The closure of ``\subcl'', i.e., the subclass hierarchy.  A
  runtime check warns about cycles in the subclass hierarchy.
\item The closure of ``\isa'', i.e., if $X\isa C, C\subcl D$ then
  $X\isa D$. 
\item Monotone and non-monotone inheritance.
\item Closure properties for functional methods.
\end{itemize}


\subsection{\FLORA vs. \FLORID}

The syntax of \FLORA and some of its design decisions are borrowed from
Florid, an \fl interpreter developed at Freiburg University in Germany.
For more information on Florid please visit the project home page at:
\verb|http://www.informatik.uni-freiburg.de/~dbis/florid/|. The following
is a list of differences between these two systems.

\begin{itemize}
\item \FLORID
  \begin{itemize}
  \item (Semi-)naive bottom-up evaluation.
  \item ``Hard-wired'' closure axioms.
  \item Nonmonotonic inheritance (trigger semantics).
  \item C++ based system.
  \end{itemize}
\item \FLORA
  \begin{itemize}
  \item Translation of \fl into XSB rules.
  \item Top-down evaluation of the generated rules. If tabled, the compiled 
    programs can be much more efficient than the corresponding Florid
    programs.
  \item Closure axioms implemented as Prolog rules for easy experimentation.
  \item Non-monotonic inheritance implemented using closure axioms and the
    well-founded semantics.
  \item Module systems.
  \item Full access to the underlying XSB system.
  \end{itemize}
\end{itemize}



\section{Syntax of \FLORA and Path Expressions }

The following is adopted from \cite{ludaescher-himmeroeder-IS-98}.


\subsection{Basic F-logic Syntax}\label{sec-basic-flogic}


\begin{itemize}
\item \emph{Symbols}: The \fl\ alphabet of \emph{object constructors}
  consists of the sets \funcs (function symbols), \preds (predicate symbols
  including $\eq$), and \vars (variables).  Variables are denoted by
  capitalized symbols or an underscore followed by zeor or more letters
  and/or digits (e.g., $X,\textit{Name}, \_, \_v5$)\footnote{$\_$ is the same
  unanimous variable as in Prolog}, whereas all other symbols,
  especially constants (which are 0-ary object constructors) are denoted in
  lowercase (e.g., $a,\textit{john}$).  An expression is called
  \emph{ground} if it involves no variables.  In addition to the usual
  first-order connectives and symbols, there are a number of special
  symbols: ], [, \}, \{, \fd, \mvd, \Fd, \Mvd,
  \isa, \subcl.\footnote{
    %%
    We do not deal with inheritance in this manual, so we
    omit the symbols for \emph{inheritable} methods
    \cite{KLW95}.
    %%
    }
  %%
\item \emph{Id-Terms/Oids}:%
\footnote{
    Numbers (including integers and floats) may also be used as id-terms. But
    such use is not recommended.}%
\medskip
  
  \hfill (1)
  \begin{minipage}[t]{.80\textwidth}
    First-order terms over \funcs\ and \vars\ are called
    \emph{id-terms}, and are used to name objects, methods, and
    classes.  Ground id-terms correspond to \emph{logical object
      identifiers} (\emph{oid}s), also called object \emph{names}.
  \end{minipage}
  \hfill ~
\item \emph{Atoms}: Let $O,M,R_{i},X_{i},C,D,T$ be id-terms.  In
  addition to the usual first-order atoms, like $p(X_1,\dots,X_n)$, there
  are the following basic types of atoms: \medskip

  \begin{math}
    \hfill (2)~O[M\fd R_0] \hfill (3)~O[M\mvd \{R_1,\dots,R_n\}]
    \hfill (4)~C[M\Fd T] \hfill (5)~C[M\Mvd T]. \hfill
  \end{math} \medskip
  
  (2) and (3) are \emph{data atoms}, specifying that a \emph{method} $M$
  applied to an object $O$ yields the result object $R_i$. In (2), $M$ is a
  \emph{single-valued} (or \emph{scalar}) method, i.e., there is
  at most one $R_0$ such that $O[M\fd R_0]$ holds. In contrast, in
  (3), $M$ is \emph{multi-valued}, so there may be several result
  objects $R_i$. For $n=1$ the curly braces may be omitted.\\
  \\
  (4) and (5) denote \emph{signature atoms}, specifying that the
  (single-valued and multi-valued, respectively) method $M$ applied to
  objects of \emph{class} $C$ yields results of type $T$.
  
  The organization of objects in classes is specified by
  \emph{isa-atoms}: \medskip

  \begin{math}
    \hfill (6)~O\isa C \hfill (7)~C\subcl D. \hfill
  \end{math} \medskip

  (6) defines that $O$ is an \emph{instance} of class $C$, while (7)
  specifies that $C$ is a \emph{subclass} of $D$. 
\item \emph{Parameters}: Methods may be \emph{parameterized}, so
  \begin{math}
    M@(P_1,\dots,P_k)
  \end{math} is allowed in (2)~--~(5), where $P_1,\dots,P_k$ are
  id-terms, e.g., \textsf{john[salary@(1998)\fd 50000]}.
  
\item \emph{Programs}: \fl\ \emph{literals}, \emph{rules}, and
  \emph{programs} are defined as usual, based on \fl\ atoms.
\end{itemize}

\NI As a concise notation for several atoms specifying properties of the
same object, \emph{F-molecules} can be used. For instance, instead of
$\textsf{john:person}\land\textsf{john[age\fd
  31]}\land\textsf{john[children\mvd\{bob,mary\}]}$, we can simply write
\textsf{john\isa person[age\fd 31; children\mvd\{bob,mary\}]}.


\begin{example}
  {\bf (Publications Database)}
  Figure~\ref{fig-flogic-model} depicts an \fl representation of a fragment
  of an object-oriented publications database.
\end{example}


\begin{figure}[htbp]
\begin{tabular}{c}
  \begin{tabular}{l}
    {\bf Schema:}\\
    conf\_p\subcl paper. \\
    journal\_p\subcl paper.\\
    paper[authors\Mvd  person; title\Fd string].\\
    journal\_p[in\_vol\Fd volume]. \\
    conf\_p[at\_conf\Fd conf\_proc].\\
    journal\_vol[of \Fd journal; volume\Fd integer; 
               number\Fd integer; year\Fd integer].\\  
    journal[name\Fd string; publisher\Fd string;
            editors@(integer)\Mvd person]. \\
    conf\_proc[of\_conf\Fd conf\_series; year\Fd integer;
               editors@(integer)\Mvd person]. \\
    conf\_series[name\Fd string]. \\
    publisher[name\Fd string].\\
    person[name\Fd string; affil@(integer)\Fd institution]. \\
    institution[name\Fd string; address\Fd string].\smallskip\\

    {\bf Objects:}\\
    $o_{j1}$\isa journal\_p[%
      title\fd ``Records, Relations, Sets, Entities, and Things'';
      authors\mvd$\{o_{mes}\}$; in\_vol\fd $o_{i11}$]. \\
    $o_{di}$\isa conf\_p[
      title\fd ``DIAM II and Levels of Abstraction'';
      authors\mvd$\{o_{mes},o_{eba}\}$; at\_conf\fd $o_{v76}$]. \\
    $o_{i11}$\isa journal\_vol[of\fd $o_{is}$; number\fd 1; volume\fd 1; year\fd1975]. \\
    $o_{is}$\isa journal[name\fd``Information Systems''; editors@(...)\mvd $\{o_{mj}\}$]. \\
    $o_{v76}$\isa conf\_proc[of\fd vldb; year\fd 1976; editors\mvd $\{o_{pcl},o_{ejn}\}$].\\
    $o_{vldb}$\isa conf\_series[name\fd``Very Large Databases'']. \\
    $o_{mes}$\isa person[name\fd``Michael E. Senko'']. \\
    $o_{mj}$\isa person[name\fd``Matthias Jarke''; affil@($\dots$)\fd $o_{rwt}$]. \\
    $o_{rwt}$\isa institution[name\fd``RWTH\_Aachen''].
\end{tabular}
\end{tabular}
\caption{A Publications Object Base and its Schema Represented 
  Using \fl}\label{fig-flogic-model}
\end{figure}





\subsection{Path Expressions}

In addition to the basic \fl syntax, the \FLORA  system also supports
\emph{path expressions} to simplify object navigation along
single-valued and multi-valued method applications, and to avoid
explicit join conditions \cite{frohn-lausen-uphoff-VLDB-94}.  The
basic idea is to allow the following \emph{path expressions} wherever
id-terms are allowed:
%%

  \medskip

\begin{math}
  \hfill (8)~O.M \hfill (9)~O..M \hfill
\end{math} \medskip

\NI The path expression in (8) is \emph{single-valued}; it refers to
the unique object $R_0$ for which $O[M\fd R_0]$ holds; (9) is a
\emph{multi-valued} path expression; it refers to each $R_i$ for which
$O[M\mvd\{R_i\}]$ holds. $O$ and $M$ may be id-terms or path
expressions; moreover, $M$ may be parameterized, i.e., of the form
$M@(P_1,\dots,P_k)$.
  
In order to obtain a unique syntax and to specify different orders of
method applications, parentheses are used: By default, path
expressions associate to the left, so $a.b.c$ is equivalent to
$(a.b).c$ and specifies the unique object $o$ such that $a[b\fd x]
\land x[c\fd o]$ holds (note that $x=a.b$). In contrast, $a.(b.c)$ is
the object $o'$ such that $b[c\fd x'] \land a[x'\fd o']$ holds (here,
$x'=b.c$); generally, \ensuremath{o' \neq o}. Note that in $(a.b).c$, $b$ is a
method name, whereas in $a.(b.c)$ it is used as an object name.
Observe that function symbols can also be applied to path
expressions, since path expressions (like id-terms) are used to
reference objects.
  
As path expressions and \fl atoms can be arbitrarily nested, this leads
to a concise and very flexible specification language for object
properties, as illustrated in the following example.

\begin{example}[Path Expressions]\label{Ex:PathExpr}
  Consider again the schema given in Figure~\ref{fig-flogic-model}.
  Given the name $n$ of a person, the following path expression
  references all editors of conferences in which $n$ had a
  paper:\footnote{Each occurrence of ``\_'' denotes a distinct
    don't-care variable (existentially quantified at the
    innermost level).}
\begin{qrules}
  \anon\isa conf\_p[authors\mvd\{\anon [name\fd $n$]\}].at\_conf..editors
\end{qrules}
Therefore, the answer to the \emph{query}
\begin{qrules}
  ?- P\isa conf\_p[authors\mvd\{\anon [name\fd
  $n$]\}].at\_conf[editors\mvd\{E\}].
\end{qrules}
is the set of all pairs (\textsf{P},\textsf{E}) such that \textsf{P}
is (the logical oid of) a paper written by $n$, and \textsf{E} is the
corresponding proceedings editor.  If one is also interested in the
affiliations of the above editors when the papers were published, we only
need to slightly modify our query:
\begin{qrules}
  ?- P\isa conf\_p[authors\mvd\{\anon [name\fd
  $n$]\}].at\_conf[year\fd Y]..editors[affil@(Y)\fd A].
\end{qrules}
\end{example}
Thus, \FLORA's path expressions support navigation and specification
of object properties along two dimensions: the ``depth'' dimension
corresponds to navigation along method applications (``.''  and
``..''), while the bracketed specification lists specify
properties of the intermediate objects (the ``breadth'' dimension). Note
that constraints \emph{within} the expressions can be stated using
variables.

To access intermediate objects that arise implicitly in the middle
of a path expression, one can define the method \textsf{self} a
$X[\textsf{self}\fd X]$ and then simply
write $\dots[\textsf{self}\fd O]\dots$ anywhere in a complex
path expression. This would bind the id of the current object to the
variable $O$.\footnote{
  %%
  A similar feature is used in other
  languages, e.g., XSQL \cite{xsql-92}.
  %%
  }
%%

\begin{example}[Path Expressions with \textsf{self}]\label{ex-path-self}
  Recall the second query in Example~\ref{Ex:PathExpr}. If the user is
  also interested in the respective conferences, the query can be
  reformulated as
\begin{qrules}
   ?- P\isa conf\_p[authors\mvd\{\anon [name\fd
   $n$]\}].at\_conf[self\fd C; year\fd Y]..editors[affil@(Y)\fd A]. 
\end{qrules}
\end{example}

\subsection{References: Truth Value vs.\ Object Value}\label{sec-references}

Id-terms, F-logic atoms, and path expressions can all be used to
reference objects. This is obvious for id-terms (1) and path
expressions (8~--~9). Similarly, F-logic atoms (2~--~7) have not only a
truth value, but also reference objects, i.e., yield an object value.
For example, $o\isa c[m\fd r]$ is a reference to $o$ and additionally,
it specifies $o$'s membership in class $c$ and the value of the attribute $m$.

Consequently, all F-logic expressions of the form (1~--~9) are called
\emph{references}. F-logic references have a dual reading: Given an
\fl\ database \db I (see below), a reference has
\begin{itemize}
\item an \emph{object value}, which yields the name(s) of the objects
  reachable in \db I by the corresponding reference, and 
\item a \emph{truth value} like any other literal or molecule of the
  language; in particular, a reference $r$ evaluates to \emph{false} if
  there exists no object that is referenced by $r$ in \db I.
\end{itemize}
Thus, a path expression can be viewed as a logical formula
(\emph{the deductive perspective}), or as a name for a number of objects
(\emph{the object-oriented perspective}).

Consider the following path expression and the equivalent (with respect to
the truth value) flattening:

\begin{displaymath}
a..b[c\mvd\{d.e\}] \quad\ \Leftrightarrow \quad\  a[b\mvd\{X_{ab}\}]
\land d[e\fd X_{de}] \land X_{ab}[c\mvd\{X_{de}\}]. \hspace{4em} (*)
\end{displaymath}


Such flattening is used to determine the truth value of
arbitrarily complex path expressions in the \emph{body} of a rule.
The object values \obj\ of a path expression are
the names of the referenced objects: e.g., for $(*)$ we have
\begin{displaymath}
\obj(a..b) = \{x_{ab} \mid \db I \models a[b\mvd\{x_{ab}\}]\}
\qquad\textrm{ and }\qquad \obj(d.e) = \{x_{de} \mid \db I \models d[e\fd 
x_{de}]\} ~,
\end{displaymath}
%
where $\db I \models \varphi$ means that $\varphi$ holds in \db I.
Observe that $\obj(d.e)$ contains at most one element because the
\emph{single-valued} method $e$ is applied to the unique oid $d$.  In
general, for an \fl\ database \db I, the object values of ground
expressions are given by the following mapping \obj\ from ground
references to sets of ground references:
%
\begin{displaymath}
  \begin{array}{cll@{\hspace{4em}}c}
    \obj(t) & := & \{t' \mid t'=t \textrm{ and } \db I\models t' \}, 
     \textrm{ for a ground id-term $t$}  & (1) \\   
                                %
    \obj(o[\dots]) & :=& \{o'\in\obj(o) \mid \db I \models o'[\dots]
    \}& (2),...,(5)\\  
                                %
    \obj(o\isa c) & := & \{o'\in\obj(o) \mid \db I \models o'\isa c\} &
    (6) \\ 
                                %
    \obj(c\subcl d) & := & \{c'\in\obj(c) \mid \db I \models c'\subcl
    d\} &  (7)\\ 
                                %
    \obj(o.m) & :=  & \{r'\in\obj(r) \mid \db I \models o[m\fd
    r]\}  &  (8)\\ 
                                %
    \obj(o..m) & := &  \{ r'\in\obj(r) \mid \db I \models
    o[m{\mvd}\{r\}] \} & (9)   
  \end{array}
\end{displaymath}
Observe that if $t$ does not hold (i.e., occur) in \db{I}, then $\obj(t)$ is
$\emptyset$.  Conversely, a ground reference $r$ is called \emph{active} if
$\obj(r)\neq\emptyset$. A reference, $r$, can be classified as either
single-valued or multi-valued:
\begin{itemize}
\item $r$ is called \emph{multi-valued} if
 \begin{itemize}
  \item it has the form $o..m$, or 
  \item it has one of the forms $\underline{o}[\dots]$,
    $\underline{o}\isa c$, $\underline{c}\subcl d$, or
    $\underline{o}.\underline{m}$, and any of the underlined
    subexpressions is multi-valued;
 \end{itemize}
\item in all other cases, $r$ is \emph{single-valued}.
\end{itemize}

\section{Various Extensions}

\subsection{FLORA Syntax}

\paragraph{Atoms}

Like XSB atoms, \FLORA atoms begin with a lowercase letter followed by zero or more letters
($A \ldots Z, a \ldots z$), digits ($0 \ldots 9$), or underscores (\_), e.g.,
$\textit{student}, \textit{apple\_pie}$. Atoms may also be \emph{any} sequence
of characters enclosed by a pair of single quotes, e.g.,
$\textit{'JOHN SMITH'},\textit{'default.flr'}$.

\begin{table}[htb]
\begin{center}
\texttt{ \small
\begin{tabular}{|c|r@{\hspace{1.5cm}}|@{\hspace{5mm}}l@{\hspace{5mm}}|}
\hline
Escaped String &
  \multicolumn{1}{c|@{\hspace{5mm}}}{ASCII (decimal)} &
  \multicolumn{1}{c|}{Symbol} \\ \hline
{\bksl}{\bksl} &  92 & {\bksl} \\ \hline
{\bksl}n &  10 &		 NewLine \\ \hline
{\bksl}N &  10 &		 NewLine \\ \hline
{\bksl}t &   9 &		 Tab \\ \hline
{\bksl}T &   9 &		 Tab \\ \hline
{\bksl}r &  13 &		 Return \\ \hline
{\bksl}R &  13 &		 Return \\ \hline
{\bksl}v &  11 &		 Vertical Tab \\ \hline
{\bksl}V &  11 &		 Vertical Tab \\ \hline
{\bksl}b &   8 &		 Backspace \\ \hline
{\bksl}B &   8 &		 Backspace \\ \hline
{\bksl}f &  12 &		 Form Feed \\ \hline
{\bksl}F &  12 &		 Form Feed \\ \hline
{\bksl}e &  27 &		 Escape \\ \hline
{\bksl}E &  27 &		 Escape \\ \hline
{\bksl}d & 127 &		 Delete \\ \hline
{\bksl}D & 127 &		 Delete \\ \hline
{\bksl}s &  32 &		 Whitespace \\ \hline
{\bksl}S &  32 &		 Whitespace \\
\hline
\end{tabular}
}
\end{center}
\caption{Escaped Character Strings and Their Corresponding Symbols}
\label{tab:tab-esc-str}
\end{table}

\FLORA also recognizes escaped characters inside a pair of single quotes ($\texttt{'}$).
An escaped character normally begins with a backslash ($\texttt{\bksl}$).
Table~\ref{tab:tab-esc-str}
lists the special escaped character strings and their corresponding special symbols. An
escaped character may also be any ASCII character. Such a character is entered by
a backslash together with a lowercase $\texttt{x}$ (or uppercase $\texttt{X}$) followed
by one or two hexadecimal symbols representing its ASCII value. For example,
$\texttt{{\bksl}xd}$
is the ASCII character Carriage Return, whereas $\texttt{{\bksl}x3A}$ represents the
semicolon. In case that a backslash is unable to escape any following character, it is
recognized as itself. 

One exception is that inside a quoted atom, a single quote character is escaped by another
signle quote, e.g., $\texttt{'isn''t'}$.

\paragraph{Strings (Character Lists)}

Like XSB strings, \FLORA strings are enclosed by a pair of double quotes ($\texttt{{\dq}}$).
These strings are represented internally as lists of ASCII characters, e.g.,
\mbox{\texttt{[102,111,111]}} corresponds to the string \texttt{{\dq}foo{\dq}}.

\FLORA strings recognize the same escaped characters as described above for \FLORA atoms.
But inside a string, a single quote character does not need to be escaped. A double quote
character, however, needs to be escaped by another double quote, e.g.,
$\texttt{{\dq}{\dq}{\dq}foo{\dq}{\dq}{\dq}}$.

\paragraph{Numbers}

Normal \FLORA integers are decimals represented by a sequence of digits, e.g., $892, 12$.
\FLORA also recognizes integers in other bases (2 through 36). The base is specified by
a decimal integer followed by a single quote ($\texttt{'}$). The digit string immediately
follows the single quote. The letters $A \ldots Z$ or $a \ldots z$ are used to represent
digits greater than 9. Table~\ref{tab:tab-int-rep} lists a few example integers.
\begin{table}[htb]
\begin{center}
\texttt{ \small
\begin{tabular}{|r@{'}l|r@{\hspace{1.5cm}}|c|}
\hline
  \multicolumn{2}{|c|}{Integer} &
  \multicolumn{1}{c|}{Base (decimal)} &
  \multicolumn{1}{c|}{Value (decimal)} \\ \hline
\multicolumn{1}{|r}{} & \multicolumn{1}{@{}l|}{1023} &  10 & 1023 \\ \hline
2 & 1111111111 & 2 & 1023 \\ \hline
8 & 1777 & 8 & 1023 \\ \hline
16 & 3FF &  16 & 1023 \\ \hline
32 & vv & 32 & 1023 \\
\hline
\end{tabular}
}
\end{center}
\caption{Integers Representions}
\label{tab:tab-int-rep}
\end{table}

Underscore (\texttt{\_}) can be put inside any sequence of digits as delimeters. It is used
to partition some long numbers, e.g., $\texttt{2'11\_1111\_1111}$ is recognized the same as
$\texttt{2'1111111111}$. But it cannot be the first symbol of an integer, since variables may
start with an underscore. For example, $1\_2\_3$ represents the number $123$ whereas $\_12\_3$
represents a variable named $\_12\_3$.

Floating numbers normally look like $24.38$. The decimal point must be preceded by an integral
part, even if it is $0$, i.e., $0.3$ must be entered as $0.3$ but not $.3$. Each float may also
have an optional exponent. It begins with a lowercase $e$ (or uppercase $E$) followed by an
optional minus sign ($-$) or plus sign ($+$) and an integer. This exponent is recognized as
in base 10. For example, \mbox{\tt 2.43E2=243} whereas \mbox{\tt 2.43e-2=0.0243}. 

\paragraph{Comments}

\FLORA supports three kinds of comments: (1) all characters following the {\tt \%} symbol are
interpreted as a line of comment; (2) all characters following the {\tt //} symbol are also
interpreted as a line of comment; (3) all characters enclosed by a pairs of {\tt /*} and
{\tt */} are interpreted as comments. Only (3) may span across multiple lines.

Comments are considered as white spaces. Therefore, tokens are also deliminated by comments.

\subsection{Path Expression in Rule Head}

Only single-valued path expressions are allowed in rule head. Set-valued path expressions
are not allowed since the semantics is not always clear.

The following shows an example with path expression in rule head (it means that the mother
of person X has grandson Y, if X has son Y):
\begin{qrules}
X.mother[grandson{\mvd}Y] :- X{\isa}person[son{\mvd}Y].
\end{qrules}
The complication arises if later on we specify that:
\begin{qrules}
john[mother{\fd}mary]. \\
john[son{\mvd}david].
\end{qrules}
and ask the following query:
\begin{qrules}
?- mary[grandson{\mvd}S].
\end{qrules}

In the above case, we should be able to relate ${\tt mary}$ to ${\tt john.mother}$ to
complete the deductive process. To deal with single-valued path expressions in rule head,
\FLORA uses a \emph{skolemization} technique and also requires \emph{equality checking}. But a
user does not need to worry about this. All this will be done automatically by the \FLORA
compiler. If path expression in rule head is detected, \FLORA compiler will perform
skolemization upon this rule head and finally append to the compiled .P XSB program a set of
rules for equality checking.

A user, however, has to be aware that equality checking could cost time. To improve system
performance, the first thing to do may be to eliminate path expressions in rule heads. In some
cases, our experiments have shown that without equality checking it runs 10 times faster.

\subsection{Aggregation}

\FLORA adopts basically the same syntax for aggregation from \FLORID. An aggregate
looks like this:
\begin{qrules}
agg\{X[Gs]; body\}
\end{qrules}
${\it agg}$ represents the aggregate operator. $X$ is called the aggregate variable; $Gs$ is
a list of comma seperated variables called grouping variable; ${\it body}$ is list of literals
specifying the conditions. The grouping variables are optional, i.e., the entire part of
$[Gs]$ may be omitted.

All the variables appearing in ${\it body}$ but not in $X$ and $Gs$ are considered as
existentially quantified. Furthermore, the syntax of an aggregate must satisfy the following
conditions: (1) Both $X$ and $Gs$ must appear in ${\it body}$; (2) $Gs$ should not contain $X$.

The following aggregate operators are supported: (1) {\it min}; (2) {\it max}; (3) {\it count};
(4) {\it sum}; (5) {\it avg}. {\it min} and {\it max} can be performed on any list of terms,
and the mininum and maximum in the order as specified by XSB's {\tt @=<} operator (for numbers
this order will be preserved) will be returned, respectively. {\it sum} and {\it avg}, however,
can only take numbers. If the aggregate variable is instantiated to some non-number,
{\it sum} and {\it avg} will both discard it and generate a runtime warning message.

In general, aggregates can appear anywhere a number is allowed. Therefore, aggregates can
be nested. The following examples illustrate the use of aggregates (some of them are borrowed
from the \FLORID manual):
\begin{qrules}
?- Z = min\{S; john[salary@(Year){\fd}S]\}. \\
?- Z = count\{Year; john.salary@(Year) $<$ max\{S; john[salary@(Y){\fd}S], Y$<$Year\}\}. \\
?- avg\{S[Who]; Who{\isa}employee[salary@(Year){\fd}S]\} $>$ 20000.
\end{qrules}

If an aggregate contains grouping varialbes, and these grouping variables are not bound
by any preceding subgoal, then this aggregate may backtrack upon the grouping variables. In
other words, they are existentially quantified. Consider the last query in the above example.
The aggregate will backtrack upon the variable {\it who}. If both {\it john's} and {\it mary's}
average salary is greater than 20000, this query will backtrack and print out both answers.

\subsection{Arithmetic Expressions}

Unlike XSB, in \FLORA arithmetic expressions are always evaluated (in XSB, $+$ may be used
as a binary functor). Both single-valued and multi-valued path expressions are allowed in
arithmetic expressions. But the semantics is that all objects (variables) are considered as
existentially quantified. For example, the following query
\begin{qrules}
?- john..bonus $+$ mary..bonus $>$ $1000$.
\end{qrules}
is actually equivalent to
\begin{qrules}
?- john[bonus{\mvd}{\tt V1}], mary[bonus{\mvd}{\tt V2}], ${\tt V1}+{\tt V2} > 1000$.
\end{qrules}
The only difference is that the values of {\tt V1} and {\tt V2} may be printed out for the
latter query, but not for the former one.

Order matters in \FLORA. All variables appearing in an arithmetic expression must be
instantiated at the time of evaluation. Otherwise, a runtime error will be generated.

\FLORA allows arithmetic expressions to appear in path expressions, since arithmetic
expressions are always evaluated (they are consideredd as representing a number), and \FLORA
also recognizes numbers as object names ({\it oid}).

Take $o_1.m_1+o_2.m_2.method$ as an example. Since \FLORA allows path expressions inside
arithmetic expressions, and {\it vice versa}, it is not very clear whether the previous
example means the arithmetic expression $(o_1.m_1)+(o_2.m_2.method)$, or the path expression
$(o_1.m_1+o_2.m_2).method$.

One more confusing example is $2.3.4$. Does it mean $(2).(3).(4)$, or (2.3).4, or 2.(3.4)?
In \FLORA, $2.3.4$ alone means $(2.3).4$, since all tokens like integers , floats and
operators, $\ldots$, etc.\ are first processed by \FLORA tokenizer and then passed to \FLORA
parser for parsing. In general, the interpretation of .\ as a decimal point takes precedence
over the interpretation of it as part of a single-valued path expression.

Another ambiguious situation is the symbol $-$ ($+$). It may be used as a minus sign (plus
sign), e.g., $-3$ ($+3$), or as a binary arithmetic subtraction (addition) operator, e.g.,
$4-7$ ($4+7$). Actually, minus sign (plus sign) is defined as a unary
operator in \FLORA and always takes precedence over binary operators.

Table~\ref{tab:tab-op-pre} lists various operators in non-increasing precedence order and their
associativity and arity.
\begin{table}[htb]
\begin{center}
\texttt{ \small
\begin{tabular}{|c|c|c|c|c|}
\hline
Precedence & Operator & Use & Associativity & Arity \\ \hline
1 & () & parentheses & not applied & not applied\\ \hline
2 & . & decimal point & not applied & not applied \\ \hline
3 & $-$ & minus sign & right & unary \\ \cline{2-5}
  & $+$ & plus sign & right & unary \\ \hline
4 & . & path expression & left & binary \\ \hline
5 & $*$ & multiplication & left & binary \\ \cline{2-5}
  & $/$ & division & left & binary \\ \hline
6 & $-$ & subtraction & left & binary \\ \cline{2-5}
  & $+$ & addition & left & binary \\ \hline
  & =< & less than or equal to & not applied & binary \\ \cline{2-5}
  & >= & greater than or equal to & not applied & binary \\ \cline{2-5}
7 & =:= & equal to & not applied & binary \\ \cline{2-5}
  & ={\bksl}= & unequal to & not applied & binary \\ \cline{2-5}
  & := & assignment & not applied & binary \\ \cline{2-5}
  &is & \multicolumn{3}{c|}{same as :=} \\
\hline
\end{tabular}
}
\end{center}
\caption{Operators in Non-Increasing Precedence Order and Their Associativity and Arity}
\label{tab:tab-op-pre}
\end{table}

Wherever ambiguity may arise, parentheses can be used to avoid misleading expressions. The
following lists more examples of legal expressions accepted by \FLORA:
\begin{qrules}
($o_1$.$m_1$+$o_2$.$m_2$).{\tt method} \\
2.(3.4) \\
$3+--2$ (equivalent to $3+(-(-2)$) \\
$5*-6$ (equivalent to $5*(-6)$) \\
$5.(-6)$
\end{qrules}

Note that both minus sign ($-$) and plus sign ($+$) are defined as unary operators. Therefore,
$-6$ is not a token, but an arithmetic expression instead. To further avoid misleading
expressions, \FLORA requires that all arithmetic expressions be enclosed by parentheses if
they appear as {\it oid} in any path expression. According to this rule, although $5.$$-6$
seems legal by Table~\ref{tab:tab-op-pre}, it has to be entered as $5.(-6)$.

\section{Compiled Code vs. Dynamic Code}

A \FLORA program usually consists of facts and rules. All these facts and rules become the
runtime database of \FLORA when they are loaded. Conceptually, the runtime system of \FLORA
is partitioned into two areas: static area and dynamic area. Compiled code can be compiled by
\emph{flcompile}, loaded into the static area by \emph{flconsult}, \emph{load}, or {\tt [~]},
while dynamic code can be compiled by \emph{dyncompile}, loaded into the dynamic area by
\emph{dynconsult}, \emph{dynload}, or \texttt{\symbol{123}~\symbol{125}}.
Section~\ref{sec-shell-commands} lists the syntax and meaning of these commands (predicates).
These predicates can also be called from within a \FLORA program. But except {\tt [~]} and
\texttt{\symbol{123}~\symbol{125}}, all of them must first be imported from \emph{flrutils}
(see Section~\ref{sec-module} for details).

Although compiled code resides in static area while dynamic code resides in dynamic area,
they are considered as a whole and executed all together. A small example would help
illustrate this. Suppose there are two programs, {\it static.flr} and {\it dynamic.flr},
as shown in Figure~\ref{fig:fig-static-dynamic-code}.
\begin{figure}[htb]
\begin{center}
\begin{tabular}{l}
{\bf static.flr:}\\ \\
department[faculty{\Mvd}professor; coursesOffered{\Mvd}string]. \\
professor[teaches@(string,number){\Mvd}string]. \\
\\
X{\isa}department[coursesOffered{\mvd}C] :-
	X..faculty[teaches@(S,Y){\mvd}C]. \\ \\

cse{\isa}department[faculty{\mvd}smith]. \\
smith{\isa}professor. \\
smith[teaches@(fall,1998){\mvd}cse220]. \\
smith[teaches@(spring,1999){\mvd}cse310]. \\
smith[teaches@(spring,1999){\mvd}cse530]. \\
\\
{\bf dynamic.flr:}\\ \\
math{\isa}department[faculty{\mvd}john]. \\
john{\isa}professor. \\
john[teaches@(spring,1999){\mvd}math230]. \\
john[teaches@(spring,1999){\mvd}math101].
\end{tabular}
\end{center}
\caption{Static Code vs. Dynamic Code} \label{fig:fig-static-dynamic-code}
\end{figure}

Start XSB and \FLORA shell from the current directory where both \emph{static.flr} and
\emph{dynamic.flr} reside, then from \FLORA shell enter:
\begin{verbatim}
FLORA> ?- flconsult(static).
[ FLORA chat deleted ]

Yes.

FLORA> ?- dynconsult(dynamic).
[ FLORA chat deleted ]

Yes.

FLORA> ?- D:department[coursesOffered->>C].

D = cse
C = cse220

D = cse
C = cse310

D = cse
C = cse530

D = math
C = math101

D = math
C = math230

[ FLORA chat deleted ]

Yes.

FLORA> ?-
\end{verbatim}

Furthermore, \FLORA provides users with several predicates to modify the runtime database.
These predicates can be executed either from the static area or the dynamic area. But
\emph{only} facts in the dynamic area can be asserted/retracted (in the furture, \FLORA
may support dynamically asserting/retracting rules to/from the dynamic area). The following
lists the syntax and meaning of the database modification predicates supported by \FLORA:

\paragraph{$\mathtt{assert(P_1,\ldots,P_n)}$} asserts a list of facts into the dynamic area.
$\mathtt{P_i~(i=1{\ldots}n)}$ can be any \fl molecule or user defined predicate, e.g.,
$\mathtt{assert(david{\isa}professsor[teaches@(fall,1999){\mvd}cse505])}$.

\paragraph{$\mathtt{retract(P_1,\ldots,P_n | C_1,\ldots,C_n)}$} retracts the \emph{ground}
facts corresponding to $\mathtt{P_1,\ldots,P_n}$ if the conjunction of
$\mathtt{P_1,\ldots,P_n,C_1,\ldots,C_n}$ succeeds. $\mathtt{C_1,\ldots,C_n}$ can be
considered as the conditions qualifying the facts to be retracted.

Special built-in predicates like arithmetic comparison operators can not be retracted.
If $\mathtt{P_i}$ happens to be one of those special predicates, \FLORA compiler will
interprete it as an additional condition $\mathtt{C_i}$ and generate a warning message.

For example,
\begin{displaymath}
\mathtt{retract(john[teaches@(S,Y){\mvd}C],Y\leq1999)}
\end{displaymath}
is equivalent to
\begin{displaymath}
\mathtt{retract(john[teaches@(S,Y){\mvd}C]|Y\leq1999)};
\end{displaymath}
$\mathtt{retract(john[teaches@(S,Y){\mvd}C]|smith[teaches@(S,Y){\mvd}C])}$
 retracts those of {\it john}'s teaching records if he and {\it smith} taught the same course
for the same semester of the same year, whereas
$\mathtt{retract(john[teaches@(S,Y){\mvd}C],smith[teaches@(S,Y){\mvd}C])}$
retracts the teaching records of both.

\paragraph{$\mathtt{retractall(P_1,\ldots,P_n | C_1,\ldots,C_n)}$} retracts \emph{all ground}
facts corresponding to $\mathtt{P_1,\ldots,P_n}$ if the conjunction of
$\mathtt{P_1,\ldots,P_n,C_1,\ldots,C_n}$ succeeds. The difference between $\mathtt{retract}$
and $\mathtt{retractall}$ is that: $\mathtt{retract}$ retracts facts one by one and fails
if it is unable to retract any facts, whereas $\mathtt{retractall}$ always succeeds no
matter what facts reside in the database. Actually, $\mathtt{retractall}$ is implemented
using $\mathtt{retract}$, as follows (with arguments omitted):
\begin{qrules}
retractall :- retract, fail. \\
retractall.
\end{qrules}

\paragraph{$\mathtt{erase(P_1,\ldots,P_n | C_1,\ldots,C_n)}$} retracts \emph{all ground} facts
as $\mathtt{retract}$ does. Moreover, it will trace the object reference links and retract
all ground facts referenced along those paths. To erase $O_1{\isa}O_2$ or $O_1{\subcl}O_2$,
the object reference links of \emph{only} $O_1$ are traced. For all other \fl facts such as
$O_1[method{\fd}O_2],O_1[method{\mvd}O_2]$, the object reference links of \emph{only} $O_2$
are traced. 

To see the effects of $\mathtt{erase}$, continue
the example of Figure~\ref{fig:fig-static-dynamic-code}:
\begin{verbatim}
FLORA> ?- erase(cse[faculty->>smith]).

No.

FLORA> ?- erase(math[faculty->>john]).

Yes.

FLORA> ?- P:professor[teaches@(Semester,Year)->>Course].

P = smith
Semester = fall
Year = 1998
Course = cse220

P = smith
Semester = spring
Year = 1999
Course = cse310

P = smith
Semester = spring
Year = 1999
Course = cse530

[ FLORA chat deleted ]

Yes.

FLORA> ?- P:professor.

P = smith

[ FLORA chat deleted ]

Yes.

FLORA> ?- 
\end{verbatim}

\paragraph{$\mathtt{eraseall(P_1,\ldots,P_n | C_1,\ldots,C_n)}$} erases \emph{all ground}
facts corresponding to $\mathtt{P_1,\ldots,P_n}$ if the conjunction of
$\mathtt{P_1,\ldots,P_n,C_1,\ldots,C_n}$ succeeds. Like $\mathtt{retractall}$,
$\mathtt{eraseall}$ always succeeds and is implemented using $\mathtt{erase}$, as follows
(with arguments omitted):
\begin{qrules}
eraseall :- erase, fail. \\
eraseall.
\end{qrules}

\section{\FLORA Modules and Interaction with XSB}\label{sec-module}

Besides static area and dynamic area, \FLORA also has its own module system, which is basically
borrowed from XSB. The import/export compiler directives are created for the module system.
\FLORA modules communicate with each other by importing/exporting either \emph{ground} \fl
signatures or normal Prolog predicates, while \FLORA module and XSB module communicate with
each other \emph{only} by normal Prolog predicates.

\begin{figure}[htb]
\begin{center}
\begin{tabular}{l}
{\bf module1.flr:}\\ \\
:- import employee[salary@(number){\Fd}number] from module2. \\
john{\isa}employee. \\
john[salary@(1994){\fd}70]. \\
john[salary@(1995){\fd}80]. \\
john[salary@(1996){\fd}70]. \\
john[salary@(1997){\fd}50]. \\
john[salary@(1998){\fd}80]. \\
\\
{\bf module2.flr:}\\
\\
:- export employee[salary@(number){\Fd}number]. \\
employee[salary@(number){\Fd}number]. \\
mary{\isa}employee. \\
mary[salary@(1994){\fd}60]. \\
mary[salary@(1995){\fd}60]. \\
mary[salary@(1996){\fd}70]. \\
mary[salary@(1997){\fd}80]. \\
mary[salary@(1998){\fd}90].
\end{tabular}
\end{center}
\caption{Example \FLORA Modules} \label{fig:fig-module}
\end{figure}

Take as an example the two \FLORA modules \emph{module1.flr} and \emph{module2.flr} in
Figure~\ref{fig:fig-module}. Start XSB and \FLORA shell in the current directory where
both \emph{module1.flr} and \emph{module2.flr} reside, from \FLORA shell loop, enter:
\begin{verbatim}
FLORA> ?- flcompile(module2).
[ FLORA chat deleted ]

Yes.

FLORA> ?- [module1].
[ FLORA chat deleted ]

Yes.

FLORA> ?- X=count{Year; john.salary@(Year) < mary.salary@(Year)}.

X = 2
Year = Year (unbound)

[ FLORA chat deleted ]

Yes.

FLORA> ?-
\end{verbatim}

The import/export directives can take a list of predicate/arity pairs (as XSB does) and/or
\emph{ground} \fl signatures (no variables are allowed in the signatures to be
imported/exported). For example, :- import tc/2, student[grade@(string){\Fd}number], p/1.

For a \FLORA module that exports, a copy of its \emph{compiled} code must exist, since this is
the copy to be executed by the module system. And a \FLORA module can not both import and
export the same \fl signature at the same time, due to the way XSB's module system works and
the way \FLORA translates import/export directives.

Import directives can appear in both static code and dynamic code. But all export directives
as well as queries will be ignored when a \FLORA module is compiled as dynamic code and/or
dynamically loaded into the dynamic area.

Since \FLORA supports import/export directives much the same way as XSB does, \FLORA modules
have full access to the underlying XSB's functionalities. Although it still remains to be seen
whether it is a good programming practice, users can freely mix \FLORA code and XSB code
together, as long as the needed XSB predicates have been imported from the corresponding
modules and used \emph{correctly}.
\begin{figure}[htb]
\begin{center}
\begin{tabular}{l}
{\bf mix.flr:}\\ \\
:- import findall/3 from setof. \\
\\
edge(a,b). \\
edge(b,c). \\
edge(c,b). \\
\\
string[reachableTo{\Mvd}string]. \\
\\
X{\isa}activeNode[reachableTo{\mvd}Y] :- edge(X,Y). \\
X{\isa}activeNode[reachableTo{\mvd}Y] :- edge(X,Z), Z[reachableTo{\mvd}Y]. \\
\\
tc(X,Y) :- X[reachableTo{\mvd}Y]. \\
\\
show(X) :- \\
\hspace{1cm} X{\isa}activeNode, \\
\hspace{1cm} write(X), \\
\hspace{1cm} write({\tt'}[reachableTo{\mvd}{\tt \{'}), \\
\hspace{1cm} findall(Y,tc(X,Y),L), \\
\hspace{1cm} writelist(L), \\
\hspace{1cm} writeln({\tt '\}}]{\tt '}). \\
\\
writelist([X]) :- write(X). \\
writelist([$X_1,X_2|$Xs]) :- write($X_1$), write(','), writelist([$X_2|$Xs]).
\end{tabular}
\end{center}
\caption{Mix \FLORA code with XSB code} \label{fig:fig-mix}
\end{figure}

As the example in Figure~\ref{fig:fig-mix} illustrates:
\begin{verbatim}
FLORA> ?- [mix].
[ FLORA chat deleted ]

Yes.

FLORA> ?- show(a).
a[reachableTo->>{b,c}]

Yes.

FLORA> ?- show(b).
b[reachableTo->>{b,c}]

Yes.

FLORA> ?-
\end{verbatim}

However, \FLORA is still not 100\% compatible with XSB, because of the double roles that
\mbox{\fl} molecules play, i.e., both as \emph{oid} and as truth value. If an \fl molecule
appears as some predicate's argument, \FLORA \emph{always} assumes that its \emph{oid} is
intended. Therefore, if not used carefully, some XSB library predicates would give very
weird results due to the way \FLORA flattens an \fl molecule as an argument. For
example, for the \FLORA program shown in Figure~\ref{fig:fig-mix},
$\mathtt{findall(Y,X[reachableTo{\mvd}Y],L)}$ won't work, since it would be flattened
into \linebreak $\mathtt{X[reachableTo{\mvd}Y],findall(Y,X,L)}$.

\section{Powered by Tabling}

All \fl atoms are flatened to predicates that are tabled by default. For all other user
defined predicates, they have to be tabled explicitly as necessary. \FLORA programs accepts
the same tabling directives as XSB does (Section~\ref{sec-comp-directives} lists all the
compiler directives).

Because of the current implementation of XSB, cuts (!) can not cut across tabled predicates.

\section{\FLORA Compiler} \label{sec-comp-directives}

Like XSB programs, \FLORA programs can take compiler directives. All compiler directives must
begin with {\tt :-} (while all queries must begin with {\tt ?-}). The following lists all the
compiler directives supported by \FLORA:

\paragraph{Tabling Directive} Tabling directive may be either \emph{{\tt :-} auto\_table.}\ 
which lets XSB automatically decide which predicates should be tabled, or
\emph{{\tt :-} table p\_a\_list.}\ , where \emph{p\_a\_list} is a list of \mbox{predicate/arity}
pairs specifying those predicates to be tabled. All \fl atoms are flattened to predicates that
are tabled by default.

\paragraph{Import Directive} Import directive looks like
\emph{{\tt :-} import sig\_pa\_list.}\ , where \emph{sig\_pa\_list} is a list of \emph{ground} \fl
signatures \mbox{and/or} \mbox{predicate/arity} pairs.

\paragraph{Export Directive} Export directive looks like
\emph{{\tt :-} export sig\_pa\_list.}\ , where \emph{sig\_pa\_list} is a list of \emph{ground} \fl
signatures \mbox{and/or} \mbox{predicate/arity} pairs. All export directives are ignored if a
\FLORA module is compiled as dynamic code and/or dynamically loaded.

\paragraph{Equality Checking Directive} Equality checking directive looks like
\emph{{\tt :-} eqlevel(N).}\ , where the level number \emph{N} specifies the level of equality
checking to be enforced. Currently, only two levels of equality checking are supported: 0
(no equality checking) and 1 (full equality checking).

Equality checking directive may appear in multiple places in a \FLORA program. But the
greatest level number overrides and specifies the final intended level of equality checking.

If there is no equality checking directive, the default level number is 0, unless \FLORA has
detected path expression in rule head and performed skolemization. In the latter case, \FLORA
compiles the program as if the program had a \emph{{\tt :-} eqlevel(1).} compiler directive,
since the corretness of skolemization requires full equality checking.

The level of equality checking may also be specified as part of compiler options that\linebreak
are passed to commands/predicates such as \emph{flcompile} and \emph{flconsult} (see
Section~\ref{sec-shell-commands}), e.g., \linebreak \mbox{flcompile(benchmark,[eqlevel(1)])}.
If multiple equality checking options are specified, only the maximum is selected. Furthermore,
if the program to be compiled also specifies a level of equality checking, then the maximum of
these two level numbers is passed to the compiler.

%------------------- cut below
\end{document}
