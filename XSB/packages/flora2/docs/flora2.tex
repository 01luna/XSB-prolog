\documentclass[11pt]{article}

\usepackage{epsf,epsfig,subfigure,latexsym,makeidx,latexsym,xspace,amssymb,alltt}

\pagestyle{headings}
\bibliographystyle{plain}

%%\setlength{\topmargin}{-0.25in}
\setlength{\headheight}{10pt}
\setlength{\headsep}{30pt}
\setlength{\oddsidemargin}{0.0in}
\setlength{\evensidemargin}{0.0in}
\setlength{\textheight}{8.5in}
\setlength{\textwidth}{6.5in}
\setlength{\footskip}{50pt}


\setlength{\parskip}{2mm}               % space between paragraphs

\def\cut{\mbox{\tt '!'/0}}

\newtheorem{example}{Example}[section]

\newenvironment{Prog}{\begin{tt}\begin{tabular}[c]{l}}{\end{tabular}\end{tt}}

\newcommand{\comment}[1]{}

\newcommand{\demo}[1]{\hspace*{1.5cm}{\tt #1}}
\newcommand{\desc}[1]{\item[{\tt #1}]\hspace*{1mm}\newline}
\newcommand{\desce}[1]{\item[{\tt #1}]}
\newcommand{\ourrepeatitem}[1]{\item[{\mbox{\tt #1}}]\ \\ \vspace*{-.35in}}
\newcommand{\ouritem}[1]{\item[{\mbox{\tt #1}}]\ \\}
\newcommand{\ournewitem}[2]{\item[{\mbox{\tt #1}}]\hspace*{\fill}{\mbox{\sf #2}}\ \\}

\newcommand{\stuff}[1]{
        \begin{minipage}{4in}
        {\tt \samepage
        \begin{tabbing}
        \hspace{8mm} \= \hspace{6mm} \= \hspace{10mm} \= \hspace{55mm} \= \kill
        #1 \hfill
        \end{tabbing}
        }
        \end{minipage}
}

\newcommand{\longline}{\noindent\rule{\textwidth}{.01in}}


\newenvironment{qrules}{\begin{quote}\sf\begin{tabular}[t]{l}}%
{\end{tabular}\end{quote}}


\newcommand{\obj}{\textit{obj}\xspace}
\newcommand{\db}[1]{\ensuremath{\mathcal{#1}}}

\newcommand{\xany}{\textsf{any}}

\newcommand{\xplus}{\ensuremath{^+}}
\newcommand{\xstar}{\ensuremath{^*}}
\newcommand{\xinv}{\ensuremath{^{-1}}}
\newcommand{\xopt}{\ensuremath{^{?}}}

\newcommand{\xto}[1]{\ensuremath{^{#1}}}
\newcommand{\xcond}[1]{\ensuremath{\textsf{if}(#1)}}
\newcommand{\xif}[1]{\ensuremath{\textsf{if}(#1)}}
\newcommand{\xmu}[1]{\ensuremath{\tcmu(#1)}}
\newcommand{\xmuif}[2]{\ensuremath{\tcmu(#1,#2)}}


\newcommand{\xconc}{\ensuremath{{\cdot}}}
\newcommand{\xor}{\ensuremath{|}}

\newcommand{\nnot}{\mbox{$\neg$}}                           % negation
\newcommand{\query}{\mbox{$\, ?\! - \, $}}                  % query
\newcommand{\impl}                                          % implication
  {\mbox{\Large $\; {\bf \leftarrow} \;$}}  
\newcommand{\isa}{\,{\bf{:}}\,}
\newcommand{\subcl}{\,{\bf{::}}\,}
\newcommand{\eq}{\ensuremath{\doteq}}                           % equation

% f-logic arrows

\newcommand{\fd}{\ensuremath{{\rightarrow}}}                   % scalar
\newcommand{\bfd}{\ensuremath{{\bullet\!\!\!\fd}}}            % " + inheritable
\newcommand{\mvd}{\ensuremath{{\rightarrow\!\!\!\!\rightarrow}}}  % multivalued
\newcommand{\bmvd}{\ensuremath{{\bullet\!\!\!\mvd}}}              % " + inheritable
\newcommand{\Fd}{\ensuremath{{\Rightarrow}}}                      % scalar signature
\newcommand{\Mvd}{\ensuremath{{\Rightarrow\!\!\!\!\Rightarrow}}}  % multiv signature



% curved f-logic arrows

\newcommand{\anyd}{\ensuremath{\leadsto}}                       % noninheritable
\newcommand{\bleadsto}{\ensuremath{\bullet\!\!\!\leadsto}}     % inheritable
\newcommand{\banyd}{\bleadsto}                              % "
\newcommand{\Leadsto}{\ensuremath{\approx}\!\!{>}}            % signature
\newcommand{\Anyd}{\Leadsto}                                % "

\newcommand{\FdConstr}{\ensuremath{\stackrel{constr}{\Fd}}}
\newcommand{\MvdConstr}{\ensuremath{\stackrel{constr}{\Mvd}}}

\newlength{\flogicindent}


\newlength{\flength}
\newlength{\counterlength}


\newcommand{\la}{\ensuremath{\,\leftarrow\,}}

\newcommand{\anon}{\_}

\newcommand{\note}[1]{\textit{[[#1]]}}
\newcommand{\nterm}[1]{\ensuremath{\langle}\textit{#1}\ensuremath{\rangle}}



\newcommand{\bs}{\ensuremath{\backslash}}
\newcommand{\FLIP}{{\mbox{\sc Flip}}\xspace}
\newcommand{\FLORA}{{\mbox{${\cal F}${\sc lora}\rm\emph{-2}}}\xspace}
\newcommand{\FLORAone}{{\mbox{${\cal F}${\sc lora}}}\xspace}
\newcommand{\FLORID}{{\mbox{\sc Florid}}\xspace}
\newcommand{\fl}{\mbox{F-logic}\xspace}


\newcommand{\consts}{\ensuremath{\mathcal{C}}}
\newcommand{\funcs}{\ensuremath{\mathcal{F}}}
\newcommand{\preds}{\ensuremath{\mathcal{P}}}
\newcommand{\vars}{\ensuremath{\mathcal{V}}}

\newcommand{\HU}{\ensuremath{U}}
\newcommand{\HB}{\ensuremath{\mathcal{HB}}}
\newcommand{\ext}{\ensuremath{^{\star}}}

\newcommand{\bksl}{\symbol{92}}
\newcommand{\dq}{\symbol{34}}


\title{\FLORA: Users Manual}

\author{
  Guizhen Yang
  \and
  Michael Kifer
  }
  
\makeindex
\begin{document}

\maketitle
\thispagestyle{empty}

\newpage
\pagenumbering{roman}
\setcounter{page}{1}

\tableofcontents

\newpage

\pagenumbering{arabic}
\setcounter{page}{1}


\section{Introduction}

\FLORA is a sophisticated \fl to XSB compiler and application
development platform. It translates a program written in the \fl
language \cite{KLW95} (which must be in a file with extension {\tt .flr},
{\it e.g.}, {\tt file.flr}) and outputs a regular XSB program (with
extension {\tt .P}). This program is
then passed to XSB for compilation (yielding {\tt file.O}) and execution.

\index{FLIP}
\index{FLORID}
%%
\FLORA was implemented by Guizhen Yang, but its origins trace back to the
\FLIP compiler developed by Bertram Lud\"aescher.  The programming language
supported by \FLORA is a dialect of \fl with some extensions borrowed
from \FLORID, a C++-based \fl system developed at Freiburg
University.\footnote{
  %%
  See {\tt http://www.informatik.uni-freiburg.de/$\sim$dbis/florid/} for more
  details.
  %%
  }
%%
In particular, \FLORA fully supports the versatile syntax of \FLORID path
expressions. However, \FLORA has numerous extensions of its own.  Of
particular importance is the support for HiLog \cite{hilog-jlp} and
Transaction Logic \cite{trans-dbpl93,trans-iclp93,trans-tcs94}, both of
which are smoothly integrated with \fl. Extensions aside, the syntax of
\FLORA also differs in some important ways both from \FLORID, the original
version of \fl as described in \cite{KLW95}, as well as from \FLORAone,
Release 1. These syntactic changes were needed in order to bring the syntax
of \FLORA closer to that of Prolog and make it possible to include typical
Prolog programs into \FLORA programs without choking the compiler.
It was also the result of the added support for HiLog, which made the use
of the ``@'' sign in method invocations superfluous.

\FLORA is part of the official distribution of XSB beginning with version
2.3. It is organized as an XSB package and lives in the directory
%%
\begin{quote}
 \verb|<xsb-installation-directory>/packages/flora2/|  
\end{quote}
%%
\FLORA is fully integrated into the XSB system, including its module
system. In particular, \FLORA modules can invoke predicates defined in
other XSB modules, and regular XSB modules can query the objects defined in
\FLORA modules. At present, XSB is the only platform where \FLORA can run,
because it heavily relies on tabling and the well-founded semantics for
negation that at the moment are available only in XSB.

As mentioned earlier, an XSB programmer can invoke \FLORA methods from
other XSB programs. However, the easiest way to get a feel of the system
is to start \FLORA shell and begin to enter queries interactively.  To
this end, you must first invoke XSB and then load the {\tt flora2}
package:
%%
\begin{quote}
  \tt
foo>~~xsb  \\
\tt
... XSB loading messages omitted ...\\
\tt
| ?- [flora2].\\
\tt
[flora2 loaded]\\
\tt
| ?-
\end{quote}
%%
At this point, it is possible to use a limited number of \FLORA
commands, but to run queries you must enter the \FLORA command loop:
%%
\begin{quote}
  \tt
| ?- flora\_shell.  \\
 \tt
... FLORA messages omitted ... \\
 \tt
flora2 ?-
\end{quote}
%%

At this point, \FLORA takes over and \fl syntax becomes the
norm. To get back to the XSB command loop, type {\tt Control-D} or 
%%
\begin{quote}
  \tt
| ?- end.  
\end{quote}
%%

If you are using \FLORA shell frequently, it pays to define an alias, say,
%%
\begin{quote}
 {\tt
   alias flora2='xsb -e "[flora2], flora\_shell."'
   }
\end{quote}
%%
\FLORA can then be invoked directly from the shell prompt by typing
\begin{quote}
  \tt
foo>~~flora2
\end{quote}
%%
It is even possible to tell it to execute commands on start-up.
For instance, 
%%
\begin{quote}
 \tt
 foo>~~flora2 -e "help."
\end{quote}
%%
will cause the system to execute the help command right after the start.
Then the usual \FLORA shell prompt is displayed.

\noindent
\FLORA comes with a number of demo programs that live in
%%
\begin{quote}
 \verb|<xsb-installation-directory>/packages/flora2/demos/|  
\end{quote}
%%
The demos can be run by issuing the command
``\verb|rundemo(demo-filename).|''
at the \FLORA prompt, {\it e.g.},
%%
\begin{quote}
 \verb|flora2 ?- rundemo(flogic_basics).|
\end{quote}
%%
There is no need to change to the demo directory, as {\tt rundemo} knows
where to find the demos.


\section{\FLORA Shell Commands} \label{sec-shell-commands}

The following \FLORA shell commands are supported:
%\begin{verbatim}
\begin{itemize}
\item {\tt help}:
    show this help info
\item {\tt compile(FILE[.P])}:
    compile FILE.P; create FILE.O
\item {\tt flcompile(FILE[.flr])}:
    compile FILE.flr for the default module `main'
\item {\tt flcompile(FILE[.flr],Module)}:
    compile FILE.flr for the module `Module'
\item {\tt flcompinc(FILE[.fli])}:
    compile \FLORA template include file
\item {\tt flconsult(FILE[.flr])}:
    load FILE.flr into the default module `main'
\item {\tt flload(FILE[.flr],Module)}:
    load FILE.flr into the module `Module'
\item {\tt flload(FILE[.\{P$|$O$|$flr\}])}:
    load FILE.flr (into the default module `main'), or FILE.P or FILE.O
\item {\tt [FILE[.\{P$|$O$|$flr\}][$>>$ Module],...]}:
    load a list of .flr (into the module `Module'), .P, or .O files
\item {\tt rundemo(FILE)}:
    consult a demo from \FLORA demos directory
\item {\tt flloadtrailer(\{standard$|$equality\},Module)}:
    load the standard/equality trailer into the module `Module'
\item {\tt flloadtrailer(\{standard$|$equality\})}:
    load the standard/equality trailer into the default module `main'
\item {\tt abolish\_all\_tables}:
    flush all tabled data
\item {\tt prolog Functor/Arity}:
    define Functor/Arity as Prolog in shell mode
\item {\tt florapred Functor/Arity}:
    define Functor/Arity as non-HiLog in shell mode
\item {\tt sign Functor(\{oid$|$form\}[,\{oid$|$form\}]*)}:
    define predicate signature in shell mode
\item {\tt op(Precedence,Associativity,Operator)}:
    define an operator in shell mode
\item {\tt reset(prolog$|$florapred$|$sign$|$op)}:
    clear all dynamic prolog/florapred/sign/op definitions in shell mode
\item {\tt all}:
    show all solutions (default)
\item {\tt one}:
    show solutions one by one
\item {\tt maxerr(all$|$N)}:
    set/show the maximum number of errors \FLORA reports
\item {\tt fltrace/flrnotrace}:
    turn on/off \FLORA trace
\item {\tt chatter/nochatter}:
    turn on/off \FLORA chatter
\item {\tt end}:
    say Ciao to \FLORA
\item {\tt halt}:
    quit both \FLORA and XSB
\end{itemize}
%\end{verbatim}

All commands with a FILE argument passed to them use the XSB
{\tt library\_directory} predicate to search for the file, except that the
command {\tt rundemo(FILE)} first looks for {\tt FILE} in the \FLORA demo
directory. In general, all XSB commands can be executed from \FLORA shell,
if the corresponding XSB library has already been loaded.

After a syntax error, parsing error, or compiling error, \FLORA shell will
discard tokens read from the current input stream until the end of file or a
rule delimiter (``.'') is encountered. If \FLORA shell seems to hang forever
after the prompt:
\begin{quote}
\begin{verbatim}
++FLORA Warning: discarding tokens (rule delimeter `.' or EOF expected)
\end{verbatim}
\end{quote}
hitting the Enter key once, then entering a ``.'' character and Enter again
will normally reset the current input buffer and cause \FLORA issue a
command prompt:
\begin{quote}
\begin{verbatim}
flora2 ?-
\end{verbatim}
\end{quote}

 
\section{\fl and \FLORA by Example}


In the future, this section will contain a number of small
introductory examples illustrating the use of \fl and \FLORA. Meanwhile, the
reader is referred to the excellent tutorial written by the members of the
\FLORID project.\footnote{
  %%
  See {\tt http://www.informatik.uni-freiburg.de/$\sim$dbis/florid/} for more
  details.
  %%
  }
%%
Since \FLORA and \FLORID share much of the same syntax, most examples in that
tutorial can be made into valid \FLORA programs by changing the separator
``;'' used in F-molecules into ``,'' and by eliminating the ``@''
sign in method invocations.



\section{Inside \FLORA}


\FLORA consists of the following modules:
\begin{itemize}
\item \texttt{flrshell.P}: the top level module that provides the \FLORA shell
  commands for compiling and consulting \FLORA programs, for setting the output mode,
  and -- last but not the least -- for directly issuing queries against the loaded
  database/program (see Section~\ref{sec-shell-commands} for a full
  description of shell commands).
\item \texttt{flrlexer.P}: the \FLORA lexer.
\item \texttt{flrcomposer.P}: the \FLORA composer that parses tokens according to operator grammar.
\item \texttt{flrparser.P}: the \FLORA parser.
\item \texttt{flrcompiler.P}: the \FLORA compiler that translates \fl to XSB.
\item \texttt{flrcoder.P}: the \FLORA coder that generates XSB code.
\item \texttt{flrutils.P}: miscellaneous utility predicates.
\end{itemize}
%%
Additional system libraries are located in the {\tt syslib/} subdirectory.
The sbudirectory {\tt trailer/} contains two trailers that can be
loaded by the \FLORA commands {\tt flloadtrailer/1} and \linebreak
{\tt flloadtrailer/2}. There
is also a number of files in the {\tt closure/} subdirectory that serve as
headers and trailers that are automatically included into the {\tt .P} files by the
\FLORA compiler (explained later).


\subsection{How It Works}


\paragraph{Overview.}

As an \fl-to-XSB compiler, \FLORA first parses its source file,
compiles it into XSB syntax and then outputs XSB code. For instance the command
\begin{quote}
\verb|flora ?- flconsult(myprog).|
\end{quote}
would compile the \FLORA program ``{\tt myprog.flr}'' and generate the
following files: ``{\tt myprog.flh}'', ``{\tt myprog.P}'',
``{\tt myprog\_main.P}'', and ``{\tt myprog.fdb}'' (if ``{\tt myprog.flr}''
contains \fl facts).  By default, {\tt flconsult(myprog)} loads the
program into the default module named ``main''. ``{\tt myprog.flh}''
contains this module name definition.  If ``{\tt myprog.flr}''
contains \fl facts, all these facts will be compiled separately into
the XSB file ``{\tt myprog.fdb}'' that is to be dynamically loaded at
runtime. When the compilation process continues, the file ``{\tt
myprog\_main.P}'' will be generated and passed to the XSB compiler,
yielding byte code ``{\tt myprog\_main.O}'', which is then loaded and
executed.  If ``{\tt myprog.flr}'' contains queries, they are
immediately executed by XSB (provided there are no errors).

In the module system of \FLORA, a program can be loaded into any
arbitrarily named module. Even the same program can loaded into
different modules at the same time. For each module, a different byte
code will be generated (module name is appended to the source file
name to generate the name of the file containing the byte code). This
is achieved by encoding the name of the module into all predicates
(not function symbols) appearing in the program. Take a look at
``{\tt myprog\_main.P}'' to see what has become of your \FLORA program!

The main purpose of the \FLORA shell, however, is to allow the evaluation
of ad-hoc \fl queries. For example, after having consulted the
the \texttt{'default.flr'} file from the demo directory by launching
the command \texttt{flora2~?-~rundemo(default).}, you may ask
\begin{verbatim}
    flora2 ?-  X..kids[                 % Whose kids
                 self -> K,             % ... (list them by name)
                 hobbies ->>            % ... have hobbies
                   H:dangerous_hobby    % ... that are dangerous?
    ]. 
\end{verbatim}
\FLORA will parse, flatten, and evaluate this query in the same way as
those queries in a source program.


\paragraph{\FLORA compilation.}
The basic idea behind the implementation of \fl by translating it into
predicate calculus is described in \cite{KLW95}. It consists of two parts:
translation of F-molecules into various kinds of Prolog predicates, and
addition of appropriate ``closure rules'' that implement the
object-oriented semantics of the logic.

Consider, for instance, the following complex F-molecule, representing
facts about the object \texttt{mary} (the syntax of \fl is given in
Section \ref{sec-basic-flogic}):

\begin{quote}
{\small\begin{verbatim}
mary:employee[age->29, kids->>{tim,leo}, salary(1998)->a_lot].
\end{verbatim}}
\end{quote}

As described in \cite{KLW95}, any complex F-molecule can be
decomposed into a conjunction of simpler \fl atomic formulas. These
latter atoms can be directly represented using Prolog syntax.  For
different kinds of \fl atoms we use different Prolog predicates. For
instance, the result of translating the above F-molecule might be:

\begin{quote}
{\small
\begin{verbatim}
isa(mary,employee).           % mary:employee.
fd(mary,age,29).              % mary[age->29].
mvd(mary,kids,tim).           % mary[kids->>{tim}].
mvd(mary,kids,leo).           % mary[kids->>{leo}].
fd(mary,salary(1998),a_lot).  % mary[salary(1998)->a_lot].
\end{verbatim}
  }
\end{quote}
%%

\index{closure axioms}
The closure axioms are intended to provide the following semantic features:
%%
\begin{itemize}
\item Compute transitive closure of ``\subcl'' (the subclass relationship).  A
  runtime check warns about cycles in the subclass hierarchy.
\item Compute closure of ``\isa'' with respect to ``\subcl'', i.e., if $X\isa C,
  C\subcl D$ then $X\isa D$.
\item Perform monotonic and non-monotonic inheritance.
\item Make sure that scalar methods are, indeed, scalar.
\end{itemize}
%%
Templates for these axioms reside in the subdirectory \texttt{closure/}.

The above is a much simplified picture of the inner-workings of \FLORA. The
actual translation into Prolog and the form of the closure rules is very
complex.  Some of this complexity exists to ensure good performance.  Other
complications come from the need to provide a module system of \FLORA and
integrate it into the XSB system.  The module system serves two purposes.
First, it promotes modular design for \FLORA programs, making it possible
to split the code into separate files and import objects defined in other
modules. Second, it allows \FLORA programs to communicate with XSB by using
the predicates defined in XSB programs and letting XSB programs use \FLORA
objects.  Some of these implementation issues are described in
\cite{guiz-flora-00}.


\subsection{Equality Maintenance}


Equality of object identities (oids) stems in the first place from the
closure axioms stupilated on scalar methods in \fl. For instance, consider
the following \fl facts:
\begin{quote}
mary[spouse{\fd}john]. \\
mary[spouse{\fd}joe]. \\
john[son{\mvd}frank].
\end{quote}
Because ``{\tt spouse}'' is a scalar method, i.e., single-valued method,
we know that the oids ``{\tt john}'' and ``{\tt joe}'' represent
the same object. Therefore, whatever is true about ``{\tt john}'' should
be also true about ``{\tt joe}'', and vice versa. Thus we can derive
that {\tt joe[son{\mvd}frank]}.

What is illustrated above is just a very simple scenario of equality
maintenance. Once oids are identified, it needs to propogate to all
\fl structures, including the subclass hierarchy, the ISA hierarchy, etc.
However, users do not have to worry about the various aspects of equality
maintenance. The equality maintenance rules and other closure axioms
are already implemented in the trailers and will be automatically included
into the final file by the \FLORA compiler.

By default, \FLORA does not use the trailer with equality maintenance,
since equality maintenance consumes more resources. To inform the
\FLORA compiler of the interest in equality maintenance, just include
the following line in the source program:
\begin{quote}
\verb|o:=:o.|
\end{quote}
In fact, the operator ``{\tt :=:}'' can be used to explicitly define
equality in a program. If the \FLORA compiler sees any equality
definition in a program, it will automatically use the trailer with
equality maintenance. Since {\tt o:=:o} only trivially identifies
the oid {\tt o} with itself, its net effect is just to force the
\FLORA compiler to use the trailer with equality maintenance. Note
that the reflexive and transitive aspects of equality maintenance are
already automatically taken care of in the \FLORA trailer, users do
not need to implement them in the program.


\section{Basic \fl Syntax}


\subsection{\fl Vocabulary}\label{sec-basic-flogic}


\begin{itemize}
\item \emph{Symbols}: The \fl alphabet of \emph{object constructors}
  \index{object constructor}
  consists of the sets \funcs (function symbols), \preds (predicate symbols
  including $=$), and \vars (variables).  Variables begin with a
  capitalized letter or an underscore, followed by zero or more letters
  and/or digits and/or underscores (e.g., $\tt X, Name, \_, \_v\_5$).
  All other symbols, including the constants (which are 0-ary object
  constructors), are symbols that start with a lowercase letter (e.g., {\tt a,
  john}). Constants can also be any string of symbols enclosed in single
  quotes (e.g., {\tt 'AB@*c'}). 
  In addition to the usual first-order connectives and symbols, there is a
  number of special symbols:
  ], [, \}, \{, ``,'', ``;'', \#, \_\#, \fd, \mvd, \Fd,
  \Mvd, \isa, \subcl. Later we will explain other symbols introduced by
  the inheritance mechanism.
  
\item \emph{Variables}: The variable ``\_'' is called \emph{anonymous}
  variable. It is used whenever a \emph{unique} new variable is needed.
  In particular, two different occurrences of ``\_'' in the same clause are
  treated as different variables. The variables that start with an
  underscore, e.g., {\tt \_foo}, are \emph{not} anonymous and two
  different occurrences of such a variable in the same clause refer to the
  same variable. Nevertheless, such variables have special status as far as
  error checking is concerned. The practice of logic programming shows that
  a singleton occurrence of a variable in a clause is often a mistake
  due to misspelling. Therefore, \FLORA issues a warning when
  it finds that some variable is mentioned only once in a clause. If such
  an occurrence is truly intended, it must be replaced by an anonymous
  variable or a variable that begins with the underscore to avoid the
  warning message from \FLORA.

  %%
  \index{id-term}
  \index{oid}
  \index{object identifier}
\item \emph{Id-Terms/Oids}:
    First-order terms over \funcs\ and \vars\ are called \emph{id-terms},
    and are used to name objects, methods, and classes.  Ground id-terms
    (i.e., terms with no variables) correspond to \emph{logical
      object identifiers} (\emph{oid}s), also called object \emph{names}.
    Numbers (including integers and floats) can also be used as id-terms,
    but such use might be confusing and is not recommended.
  \index{atomic formula!in \fl}
\item \emph{Atomic formulas}: Let $O,M,R_{i},X_{i},C,D,T$ be id-terms.  In
  addition to the usual first-order atomic formulas, like
  $p(X_1,\dots,X_n)$, there are the following basic types of formulas:
  \medskip

  \begin{math} \sf
    \hfill (1)~~O[M\fd R_0] \hfill (2)~~O[M\mvd \{R_1,\dots,R_n\}]
    \hfill (3)~~C[M\Fd T] \hfill (4)~~C[M\Mvd T]. \hfill
  \end{math} \medskip
  
  \index{data atom}
  \index{atom!data}
  \index{method}
  \index{method!single-valued}
  \index{method!scalar}
  \index{method!multi-valued}
  %
  (1) and (2) are \emph{data atoms}, which specify that a \emph{method} $M$
  applied to an object $O$ yields the result object $R_i$. In (1), $M$ is a
  \emph{single-valued} (or \emph{scalar}) method, i.e., there is
  at most one $R_0$ such that $O[M\fd R_0]$ holds. In contrast, in
  (2), $M$ is \emph{multi-valued}, so there can be several result objects
  $R_i$. For $n=1$ the curly braces can be omitted.

  \index{atom!signature}
  \index{signature!in \fl}

  \medskip

  (3) and (4) denote \emph{signature atoms}. They specify that method $M$,
  applied to objects of \emph{class} $C$, yields results of type $T$.
  In (3), $M$ is declared as single-valued, while in (4) as set-valued.
  \medskip

  \index{atom!isa}
  %%
  Objects are grouped into classes using \emph{ISA-atoms}:
  \medskip

  \begin{math}
    \sf \hfill (5)~~O\isa C \hfill (6)~~C\subcl D. \hfill
  \end{math} \medskip

  \index{class}
  \index{subclass}
  \index{class!subclass}
  \index{class!instance}
  (5) defines that $O$ is an \emph{instance} of class $C$, while (6)
  specifies that $C$ is a \emph{subclass} of $D$. 
\item \emph{Parameters}: Methods can have arguments, i.e., 
  \begin{math}
    M(P_1,\dots,P_k)
  \end{math}
  is allowed in (1)~--~(4), where $P_1,\dots,P_k$ are
  id-terms, e.g., {\sf john[salary(1998)\fd50000]}.
  
\item \emph{Programs}: \fl\ \emph{literals}, \emph{rules}, and
  \emph{programs} are defined as usual, based on \fl atoms.
\end{itemize}

\noindent
\index{F-molecule}
\emph{F-molecules} provide a shortcut for specifying properties of the same
object. For instance, instead of
{\sf john{\isa}person},~{\sf john[age{\fd}31]},~{\sf john[children\mvd\{bob,mary\}]},
we can simply write \linebreak
\mbox{\sf john{\isa}person[age{\fd}31, children\mvd\{bob,mary\}]}.


\begin{example}
  {\bf (Publications Database)} \rm Figure~\ref{fig-flogic-model} depicts
  an \fl representation of a fragment of an object-oriented publications
  database.
\end{example}


\begin{figure}[htb]
\begin{tabular}{c}
  \begin{tabular}{l}
    {\bf Schema:}\\
    conf\_p\subcl paper. \\
    journal\_p\subcl paper.\\
    paper[authors\Mvd  person, title\Fd string].\\
    journal\_p[in\_vol\Fd volume]. \\
    conf\_p[at\_conf\Fd conf\_proc].\\
    journal\_vol[of \Fd journal, volume\Fd integer, 
               number\Fd integer, year\Fd integer].\\  
    journal[name\Fd string, publisher\Fd string,
            editors\Mvd person]. \\
    conf\_proc[of\_conf\Fd conf\_series, year\Fd integer,
               editors\Mvd person]. \\
    conf\_series[name\Fd string]. \\
    publisher[name\Fd string].\\
    person[name\Fd string, affil(integer)\Fd institution]. \\
    institution[name\Fd string, address\Fd string].\smallskip\\

    {\bf Objects:}\\
    $o_{j1}$\isa journal\_p[%
      title\fd 'Records, Relations, Sets, Entities, and Things',
      authors\mvd$\{o_{mes}\}$, in\_vol\fd $o_{i11}$]. \\
    $o_{di}$\isa conf\_p[
      title\fd 'DIAM II and Levels of Abstraction',
      authors\mvd$\{o_{mes},o_{eba}\}$, at\_conf\fd $o_{v76}$]. \\
    $o_{i11}$\isa journal\_vol[of\fd $o_{is}$, number\fd 1, volume\fd 1, year\fd1975]. \\
    $o_{is}$\isa journal[name\fd'Information Systems', editors\mvd $\{o_{mj}\}$]. \\
    $o_{v76}$\isa conf\_proc[of\fd vldb, year\fd 1976, editors\mvd $\{o_{pcl},o_{ejn}\}$].\\
    $o_{vldb}$\isa conf\_series[name\fd'Very Large Databases']. \\
    $o_{mes}$\isa person[name\fd'Michael E. Senko']. \\
    $o_{mj}$\isa person[name\fd'Matthias Jarke', affil(1976)\fd $o_{rwt}$]. \\
    $o_{rwt}$\isa institution[name\fd'RWTH\_Aachen'].
\end{tabular}
\end{tabular}
\caption{A Publications Object Base and its Schema Represented 
  Using \fl}\label{fig-flogic-model}
\end{figure}



\subsection{Path Expressions in the Rule Body}


\index{path expression}
\index{path expression!in rule body}
In addition to the basic \fl syntax, the \FLORA  system also supports
\emph{path expressions} to simplify object navigation along
single-valued and multi-valued method applications, and to avoid
explicit join conditions \cite{frohn-lausen-uphoff-VLDB-94}.  The
basic idea is to allow the following \emph{path expressions} wherever
id-terms are allowed:
%%

  \medskip

\begin{math} \sf
  \hfill (7)~~O.M \hfill (8)~~O..M \hfill
\end{math} \medskip

\noindent
The path expression in (7) is \emph{single-valued}; it refers to the unique
object $R_0$ for which $O[M\fd R_0]$ holds; (8) is a \emph{multi-valued}
path expression; it refers to each $R_i$ for which $O[M\mvd\{R_i\}]$ holds.
The symbols $O$ and $M$ stand for an id-term or path a expression.
Moreover, $M$ can be a method that takes arguments, i.e.,
$O..M(P_1,\dots,P_k)$ is a valid path expression.
  
In order to obtain a unique syntax and to specify different orders of
method applications, parentheses can be used. By default, path expressions
associate to the left, so $a.b.c$ is equivalent to $(a.b).c$ and specifies
the unique object $o$ such that $a[b\fd x] \land x[c\fd o]$ holds (note
that $x=a.b$). In contrast, $a.(b.c)$ is the object $o1$ such that $b[c\fd
x1] \land a[x1\fd o1]$ holds (here, $x1=b.c$). In general, these can be
different objects. Note that in $(a.b).c$, $b$ is a method name, whereas in
$a.(b.c)$ it is used as an object name.  Observe that function symbols can
also be applied to path expressions, since path expressions (like id-terms)
are used to refer objects. Thus, $f(a.b)$ is valid.

As path expressions and F-molecules can be arbitrarily nested, this leads
to a concise and very flexible specification language for object
properties, as illustrated in the following example.

\begin{example}[Path Expressions]\label{Ex:PathExpr}
  \rm
  Consider again the schema given in Figure~\ref{fig-flogic-model}.
  Given the name $n$ of a person, the following path expression
  references all editors of conferences in which $n$ had a
  paper:\footnote{Each occurrence of ``\_'' denotes a distinct
    don't-care variable (existentially quantified at the
    innermost level).}
\begin{qrules}
  \anon\isa conf\_p[authors\mvd\{\anon [name\fd $n$]\}].at\_conf..editors
\end{qrules}
Therefore, the answer to the \emph{query}
\begin{qrules}
  ?- P\isa conf\_p[authors\mvd\{\anon [name\fd
  $n$]\}].at\_conf[editors\mvd\{E\}].
\end{qrules}
is the set of all pairs (\textsf{P},\textsf{E}) such that \textsf{P}
is (the logical oid of) a paper written by $n$, and \textsf{E} is the
corresponding proceedings editor.  If one is also interested in the
affiliations of the above editors when the papers were published, we only
need to slightly modify our query:
\begin{qrules}
  ?- P\isa conf\_p[authors\mvd\{\anon [name\fd
  $n$]\}].at\_conf[year\fd Y]..editors[affil(Y)\fd A].
\end{qrules}
\end{example}
Thus, \FLORA path expressions support navigation 
along the method application dimension using the operators
``.''  and
``..''. In addition, intermediate objects through which such navigation
takes place can be selected by specifying the properties of such objects
inside square brackets.

\index{method!self}
To access intermediate objects that arise implicitly in the middle
of a path expression, one can define the method \textsf{self} as
{\sf X[self{\fd}X]} and then simply
write $\dots${\sf [self{\fd}O]}$\dots$ anywhere in a complex
path expression. This would bind the id of the current object to the
variable {\sf O}.\footnote{
  %%
  A similar feature is used in other
  languages, e.g., XSQL \cite{xsql-92}.
  %%
  }
%%

\begin{example}[Path Expressions with \textsf{self}]\label{ex-path-self}
  \rm
  Recall the second query in Example~\ref{Ex:PathExpr}. If the user is
  also interested in the respective conferences, the query can be
  reformulated as follows:
\begin{qrules}
  X[self\fd X].\\
   ?- P\isa conf\_p[authors\mvd\{\anon [name\fd
   $n$]\}].at\_conf[self\fd C, year\fd Y]..editors[affil(Y)\fd A]. 
\end{qrules}
\end{example}


\subsection{Path Expressions in the Rule Head}


Only single-valued path expressions are allowed in a rule head. Set-valued
path expressions are not allowed because the semantics is not always clear
in such cases.

\index{path expression!in rule head}
The following is an example of a path expression in rule head. For any
person X, this rule defines the grand children for X's mother.
\begin{qrules}
X.mother[grandson{\mvd}Y] :- X{\isa}person[son{\mvd}Y].
\end{qrules}
%%
Complications arise if we specify the following facts later on:
%%
\begin{qrules}
john[mother{\fd}mary]. \\
john[son{\mvd}david].
\end{qrules}
and ask the following query:
\begin{qrules}
?- mary[grandson{\mvd}S].
\end{qrules}

Here, we should be able to identify ${\tt mary}$ and ${\tt john.mother}$,
since the attribute {\tt mother} is scalar. To deal with single-valued path
expressions in rule heads, \FLORA \emph{skolemizes} ${\tt john.mother}$ and
derives the requisite equalities.  All this is done by the
\FLORA compiler transparently to the user: if a path expression in rule head
is detected, \FLORA replaces this expression with a Skolem function and
appends the trailer with equality maintenance to the final {\tt .P} file
to ensure that proper equalities are maintained.

The user must be aware though, that the use of Skolem functions implies
the need for \emph{equality maintenance}, which may be costly.  Performance can
thus be improved if path expressions in the rule heads are avoided.  Our
experiments show that without equality checking \FLORA can run 2 -- 10 times
faster in some cases.


\subsection{Truth Value vs.\ Object Value}\label{sec-references}


Id-terms, \fl atoms, and path expressions can all be used 
as objects. This is obvious for id-terms and path
expressions (7~--~8). Similarly, \fl atoms (1~--~6) have not only a
truth value, but they each have an object value as well.
For example, {\sf o{\isa}c[m{\fd}r]} has object value {\sf o} and a truth value.
The latter depends on whether {\sf o} belongs to class {\sf c} and on the value of
the attribute {\sf m}.

Consequently, all F-molecules of the form (1~--~8) 
have dual reading. Given an
\fl\ database \db I (see below), an expression has:
\begin{itemize}
  \index{molecule!object value}
\item An \emph{object value}, which yields the name(s) of the object(s)
  reachable in \db I by the corresponding expression, and 
  \index{molecule!truth value}
\item A \emph{truth value}, like any other literal or molecule of the
  language. In particular, a molecule $r$ evaluates to \emph{false} if
  \db I has no object corresponding to $r$.
\end{itemize}
%%
Thus, a path expression can be viewed as a logical formula (\emph{the
  deductive perspective}), or as an expression that represents one or more
objects (\emph{the object-oriented perspective}).

Consider the following path expression and an equivalent, decomposed
expression:

\begin{displaymath}
\sf
a..b[c\mvd\{d.e\}] \quad\ \Leftrightarrow \quad\  a[b\mvd\{X_{ab}\}]
\land d[e\fd X_{de}] \land X_{ab}[c\mvd\{X_{de}\}]. \hspace{4em} (*)
\end{displaymath}

\noindent
Such decomposition is used to determine the truth value of arbitrarily complex
path expressions in the \emph{body} of a rule.  Let $\sf \obj(path)$ denote
the ids of all objects represented by the path expression. Then, for $(*)$,
we have:
\begin{displaymath} \sf
\obj(a..b) = \{x_{ab} \mid \db I \models a[b\mvd\{x_{ab}\}]\}
\qquad\textrm{ and }\qquad \obj(d.e) = \{x_{de} \mid \db I \models d[e\fd 
x_{de}]\} ~,
\end{displaymath}
%
where $\db I \models \varphi$ means that $\varphi$ holds in \db I.
Observe that $\obj(d.e)$ contains at most one element because the
\emph{single-valued} method $e$ is applied to a single oid $d$. Thus, two
formulas might be equivalent logically, but their values as objects might
be different!

In general, for an \fl\ database \db I, the object values of ground
path expressions are given by the following mapping \obj from ground molecules
to sets of ground oids:
%
\begin{displaymath}
  \begin{array}{cll@{\hspace{4em}}c}
    \obj(t) & := & \{t \mid  \db I\models t[] \}, 
     \textrm{ for a ground id-term $t$}  \\   
                                %
    \obj(o[\dots]) & := & \{o1 \mid o1\in\obj(o), \db I \models o1[\dots]
    \} \\  
                                %
    \obj(o\isa c) & := & \{o1 \mid o1\in\obj(o), \db I \models o1\isa c\}
     \\ 
                                %
    \obj(c\subcl d) & := & \{c1 \mid c1\in\obj(c), \db I \models c1\subcl
    d\} \\ 
                                %
    \obj(o.m) & :=  & \{r1 \mid r1\in\obj(r), \db I \models o[m\fd
    r]\} \\ 
                                %
    \obj(o..m) & := &  \{r1 \mid  r1\in\obj(r), \db I \models
    o[m{\mvd}\{r\}] \}
  \end{array}
\end{displaymath}

Observe that if $\tt t[]$ does not occur in \db{I}, then $\obj(t)$ is
$\emptyset$.  Conversely, a ground molecule $r$ is called \emph{active} if
$\obj(r)$ is not empty. A molecule, $r$, can be 
single-valued or multi-valued:
%%
\begin{itemize}
\item $r$ is called \emph{multi-valued} if
 \begin{itemize}
  \item it has the form $o..m$, or 
  \item it has one of the forms $\underline{o}[\dots]$,
    $\underline{o}\isa c$, $\underline{c}\subcl d$, or
    $\underline{o}.\underline{m}$, and any of the underlined
    subexpressions is multi-valued;
 \end{itemize}
\item in all other cases, $r$ is \emph{single-valued}.
\end{itemize}

In \FLORA, \fl path expressions can appear wherever id-terms are
allowed. Consequently, the question arises whether a path expression
is intended to indicate a truth value or an object value. For
instance, we may want to call the following XSB {\sf findall/3}
predicate to retrieve the oids of those managers whose hobbies include
tennis:
\begin{qrules}
findall(P,P{\isa}manager[hobbies{\mvd}\{tennis\}],L)
\end{qrules}
In the above example, {\sf P{\isa}manager[hobbies{\mvd}\{tennis\}]}
should evaluate to true in order for the bindings of {\sf P} to be
collected into the list {\sf L}. Therefore, the formula corresponding
to the F-molecule needs to be compiled into the predicate
{\sf findall/3} for evaluation at runtime.

\index{predicate signature}
\index{compiler directive!{\tt sign}}
%
By default, when a path expression appears as an argument to a
predicate or function symbol, the \FLORA compiler will flatten it and
pass its object value to the predicate or function. If the truth value
of a path expression is intended, the compiler directive {\sf sign}
can be used to define the signature of a predicate, i.e., to signal
how the compiler should flatten the arguments. For instance, the
following compiler directive is needed to make {\sf findall/3}
function correctly in \FLORA:
\begin{qrules}
:- sign findall(oid,form,oid).
\end{qrules}
which indicates that the \FLORA compiler should compile
any second argument of {\sf findall/3} into a formula, while the
first and third arguments into oids.



\subsection{Inheritance}


\index{inheritance!structural}
\index{inheritance!behavioral}
\fl identifies two types of inheritance: \emph{structural} and
\emph{behavioral}.  Structural inheritance applies to signatures only. For
instance, if {\sf student::person} and a program defines the signature
{\sf person[name{\Fd}string]} then the query {\sf ?- student[name{\Fd}X]}
succeeds with {\sf X=string}.

\index{inheritance!non-monotonic}
Behavioral inheritance is much more complicated. The problem  is that it is
\emph{non-monotonic}. That is, the addition of new facts might obviate previously
established inferences.

\index{attribute!inheritable}
\index{attribute!non-inheritable}
\fl (and \FLORA) distinguishes between attributes and methods that can
inherit values from superclasses and those that do not. The syntax that we
have seen so far applies to \emph{non-inheritable} attributes only.
\emph{Inheritable attributes} are declared using the {\tt *=>} and {\tt *=>>}
style arrows and defined using the {\tt *->} and {\tt *->>} style arrows. For
instance, the following is a \FLORA program for the classical {\sf Royal
Elephant} example:
%%
\begin{quote}
\begin{verbatim}
elephant[color*=>color].
royal_elephant::elephant.
clyde:elephant.
elephant[color*->gray].
\end{verbatim}
\end{quote}
%%
The question is what is the color of {\sf Clyde}?
{\sf Clyde}'s color has not been defined in the above program. However, since
{\sf Clyde} is an elephant and the default color for elephants is gray, {\sf Clyde}
must be gray. Thus, we can derive:
%%
\begin{quote}
\begin{verbatim}
clyde[color->gray].  
\end{verbatim}
\end{quote}
%%
Observe that when inheritable methods are inherited from a class by its
members, the attribute becomes non-inheritable. On the other hand, when
such a method is inherited by a subclass from its superclass, then the
method is still inheritable, so it can be further inherited by the members
of that subclass or by its subclasses. For instance, if we have
%%
\begin{quote}
\begin{verbatim}
circus_elephant::elephant.
\end{verbatim}
\end{quote}
%%
then we can derive 
%%
\begin{quote}
\begin{verbatim}
circus_elephant[color*->gray].  
\end{verbatim}
\end{quote}
%%

Non-monotonicity of behavioral inheritance becomes apparent when certain new
information gets added to the knowledge base. For instance, suppose we
learn that
%%
\begin{quote}
\begin{verbatim}
royal_elephant[color*->white].  
\end{verbatim}
\end{quote}
%%
Although we have previously established that {\sf Clyde} is gray, this
new information renders our earlier conclusion invalid. Indeed, Since
{\sf Clyde} is a royal elephant, he must be white, while being an
elephant he must be gray.  The conventional wisdom in object-oriented
languages, however, is that inheritance from more specific classes
must take precedence. Thus, we must withdraw our earlier conclusion
that {\sf Clyde} is gray and infer that he is white:
%%
\begin{quote}
\begin{verbatim}
clyde[color->white].    
\end{verbatim}
\end{quote}
%%

Behavioral inheritance in \fl is discussed at length in \cite{KLW95}.
The above problem of non-monotonicity is just a tip of the iceberg. Much
more difficult problems arise when inheritance interacts with the regular
deduction. To illustrate, consider the following program:
%%
\begin{quote}
\begin{verbatim}
b[m*->>c].
a:b.
a[m->>d] :- a[m->>c].
\end{verbatim}
\end{quote}
%%
In the beginning, it seems that \verb|a[m->>c]| should be derived by
inheritance, and so we can derive \verb|a[m->>d]|. Now, however, we can
reason in two different ways:
%%
\begin{enumerate}
\item \verb|a[m->>c]| was derived based on the belief that attribute {\tt
    m} is not defined for the object {\tt a}. However, once inherited,
  necessarily we must have \verb|a[m->>{c,d}]|. So, the value of
  attribute {\tt m} is not really that produced by inheritance. In other
  words, inheritance of \verb|a[m->>c]| negates the very premise on which
  the original inheritance was based, so we must undo the operation and the
  ensuing rule application.
\item We did derive \verb|a[m->>d]| as a result of inheritance, but
  that's OK --- we should not really be looking back and undo previously
  made inheritance inferences. Thus, the result must be \verb|a[m->>{c,d}]|.
\end{enumerate}
%%
A semantics that favors the second interpretation was proposed in
\cite{KLW95}. This approach is based on a fixpoint computation of
non-monotonic behavioral inheritance.  However, this semantics is very hard
to implement efficiently, especially using a top-down deductive engine
provided by XSB. Thus, \FLORA uses a different, more cautious semantics for
inheritance, which favors the first interpretation above.  The idea can be
summarized using the following rules, which define how class instances
inherit from the classes they belong to:
%%
\begin{quote}
\begin{verbatim}
// inheritance rules for scalar attributes
:- table defined/2, overwritten/3, conflict/3.

Obj[A->V] <- Obj:Class, Class[A*->V], tnot defined(Obj,A),
             tnot overwritten(Obj,Class,A), tnot conflict(Obj,Class,A).

defined(Obj,A) <- Obj[A->V].

overwritten(Obj,Class,A) <- Obj:Class1, Class1::Class,
                            Class1[A*->W], Class1 \= Class.

conflict(Obj,Class,A) <- Obj:Super, Super[A*->V],
                         tnot Super::Class, tnot Class::Super.

// inheritance rules for multi-valued attributes
:- table definedSet/2, overwrittenSet/3, conflictSet/3.

Obj[A->>V] <- Obj:Class, Class[A*->>V], tnot definedSet(Obj,A),
              tnot overwrittenSet(Obj,Class,A),
              tnot conflictSet(Obj,Class,A).

definedSet(Obj,A) <- Obj[A->>V].

overwrittenSet(Obj,Class,A) <- Obj:Class1, Class1::Class,
                               Class1[A*->>W], Class1 \= Class

conflictSet(Obj,Class,A) <- Obj:Super, Super[A*->V],
                            tnot Super::Class, tnot Class::Super.
\end{verbatim}
\end{quote}
%%
\index{well-founded semantics}
Negation here is implemented using the {\em well-founded semantics}
for negation \cite{gelder-alternating-89,gelder-ross-schlipf-91} (as
indicated by the {\tt tnot} operator).  Similar rules are needed to
describe how classes inherit from superclasses.

In the examples that we have seen so far, path expressions use only
non-inheritable attributes. Clearly, there is no reason to disallow
inheritable attributes in such expressions. To distinguish inheritable
attributes from non-inheritable ones in path expressions, \FLORA uses
``{\tt !}'' and ``{\tt !!}''. For instance,
%%
\begin{quote}
\begin{verbatim}
clyde!color           means: some X, such that clyde[color*->X]}.
obj!!attr             means: some Y, such that obj[attr*->>Y].
\end{verbatim}
\end{quote}
%%


\subsection{Type Checking}


\index{type checking}
%%
Although \FLORA allows specification of object types through signatures,
type correctness is not being checked automatically. So, what are the
signatures good for then? One answer is that future versions of \FLORA
might support some forms of type checking. However, because \fl can
naturally support powerful meta-programming, even the current level
of support for signatures is useful. For instance, users can write
simple queries to check the types of methods that might look suspicious.
Here is one way to construct such a type-checking query:
%%
\begin{verbatim}
scalar_type_incorrect(O,M,R) :- O[X->R], O:C, C[X=>D], tnot R:D.
?- scalar_type_incorrect(obj,meth,Result).
\end{verbatim}
%%
Here, we define what it means to violate type checking using the usual
\fl semantics. The corresponding predicate can then be queried. A
``{\sf no}'' answer means that the corresponding attribute \emph{does not}
violate the typing rules.

In this way, one can easily consruct special purpose type checkers.  This
feature is particularly important when dealing with \emph{semistructured}
data. (Semistructured data has object-like structure but normally does not
need to conform to any type; or if it does, the type would normally cover
only certain portions of the object structure.)


\section{Syntax of \FLORA}


\subsection{Symbols, Strings, and Comments}


\index{symbol}
%%
\paragraph{Symbols.}
\FLORA symbols (that are used for the names of constants, predicates, and
object constructors) begin with a lowercase letter followed by zero or more
letters ($\tt A \ldots Z, a \ldots z$), digits ($\tt 0 \ldots 9$), or underscores
(\_), e.g., \texttt{student}, \texttt{apple\_pie}. Symbols can also be
\emph{any} sequence of characters enclosed in a pair of single quotes,
e.g., \texttt{'JOHN SMITH'},\texttt{'default.flr'}.  Internally, \FLORA
symbols are represented as \emph{XSB symbols},\footnote{
  %%
  Symbols are called ``atoms'' in XSB, which contravenes the use of this
  term for atomic formulas in classical logic and \fl.
  We avoid the use of the term ``atom'' in reference to symbols.
  %%
  }
%%
which are used there as names of predicates and function symbols.

\begin{table}[htb]
\center
\texttt{ \small
\begin{tabular}{|c|r@{\hspace{1.5cm}}|@{\hspace{5mm}}l@{\hspace{5mm}}|}
\hline
Escaped String &
  \multicolumn{1}{c|@{\hspace{5mm}}}{ASCII (decimal)} &
  \multicolumn{1}{c|}{Symbol} \\ \hline
{\bksl}{\bksl} &  92 & {\bksl} \\ \hline
{\bksl}n &  10 &		 NewLine \\ \hline
{\bksl}N &  10 &		 NewLine \\ \hline
{\bksl}t &   9 &		 Tab \\ \hline
{\bksl}T &   9 &		 Tab \\ \hline
{\bksl}r &  13 &		 Return \\ \hline
{\bksl}R &  13 &		 Return \\ \hline
{\bksl}v &  11 &		 Vertical Tab \\ \hline
{\bksl}V &  11 &		 Vertical Tab \\ \hline
{\bksl}b &   8 &		 Backspace \\ \hline
{\bksl}B &   8 &		 Backspace \\ \hline
{\bksl}f &  12 &		 Form Feed \\ \hline
{\bksl}F &  12 &		 Form Feed \\ \hline
{\bksl}e &  27 &		 Escape \\ \hline
{\bksl}E &  27 &		 Escape \\ \hline
{\bksl}d & 127 &		 Delete \\ \hline
{\bksl}D & 127 &		 Delete \\ \hline
{\bksl}s &  32 &		 Whitespace \\ \hline
{\bksl}S &  32 &		 Whitespace \\
\hline
\end{tabular}
}
\caption{Escaped Character Strings and Their Corresponding Symbols}
\label{tab:tab-esc-str}
\end{table}

\index{escaped character}
\FLORA also recognizes escaped characters inside single quotes
(\texttt{'}).  An escaped character normally begins with a backslash
(\texttt{\bksl}).  Table~\ref{tab:tab-esc-str} lists the special escaped
character strings and their corresponding special symbols. An escaped
character may also be any ASCII character. Such a character is preceded
with a backslash together with a lowercase \texttt{x} (or an uppercase
\texttt{X}) followed by one or two hexadecimal symbols representing its
ASCII value. For example, \texttt{{\bksl}xd} is the ASCII character
Carriage Return, whereas \texttt{{\bksl}x3A} represents the semicolon. In
other cases, a backslash is recognized as itself.

One exception is that inside a quoted symbol, a single quote character is
escaped by another single quote, e.g., \texttt{'isn''t'}.

\paragraph{Strings (character lists).}

\index{string}
\index{character list}
%
Like XSB strings, \FLORA strings are enclosed in a pair of double quotes
(\texttt{\dq}).  These strings are represented internally as lists of
ASCII characters. For instance, \mbox{\texttt{[102,111,111]}} is the same
as \texttt{{\dq}foo{\dq}}.

Escape characters are recognized inside \FLORA strings similarly to
\FLORA symbols.  However, inside a string, a single quote character does
not need to be escaped. A double quote character, however, needs to be
escaped by another double quote, e.g.,
\texttt{{\dq}{\dq}{\dq}foo{\dq}{\dq}{\dq}}.

\paragraph{Numbers.}

\index{number}
%%
\index{integer}
%%
\index{floating number}
%%
Normal \FLORA integers are decimals represented by a sequence of digits,
e.g., \texttt{892, 12}.  \FLORA also recognizes integers in other bases (2 through
36). The base is specified by a decimal integer followed by a single quote
(\texttt{'}). The digit string immediately follows the single quote. The
letters $\tt A \ldots Z$ or $\tt a \ldots z$ are used to represent digits greater
than 9.  Table~\ref{tab:tab-int-rep} lists a few example integers.
%%
\begin{table}[htb]
\center
\texttt{ \small
\begin{tabular}{|r@{'}l|r@{\hspace{1.5cm}}|c|}
\hline
  \multicolumn{2}{|c|}{Integer} &
  \multicolumn{1}{c|}{Base (decimal)} &
  \multicolumn{1}{c|}{Value (decimal)} \\ \hline
\multicolumn{1}{|r}{} & \multicolumn{1}{@{}l|}{1023} &  10 & 1023 \\ \hline
2 & 1111111111 & 2 & 1023 \\ \hline
8 & 1777 & 8 & 1023 \\ \hline
16 & 3FF &  16 & 1023 \\ \hline
32 & vv & 32 & 1023 \\
\hline
\end{tabular}
}
\caption{Representation of Integers}
\label{tab:tab-int-rep}
\end{table}

Underscore (\texttt{\_}) can be put inside any sequence of digits as
delimiters. It is used to partition some long numbers. For instance,
$\texttt{2'11\_1111\_1111}$ is the same as $\texttt{2'1111111111}$.
However, ``\texttt{\_}'' cannot be the first symbol of an integer, since
variables can start with an underscore. For example, $1\_2\_3$ represents
the number $123$ whereas $\_12\_3$ represents a variable named $\_12\_3$.

Floating numbers normally look like {\tt 24.38}. The decimal point
must be preceded by an integral part, even if it is 0, e.g., {\tt 0.3}
must be entered as {\tt 0.3}, but not as {\tt .3}. Each floating
number may also have an optional exponent. It begins with a lowercase
{\tt e} or an uppercase {\tt E} followed by an optional minus sign
({\tt -}) or plus sign ({\tt +}) and an integer. This exponent is
recognized as in base 10. For example,
\mbox{\tt 2.43E2 is 243} whereas
\mbox{\tt 2.43e-2 is 0.0243}.

\paragraph{Comments.}

\index{comment}
%
\FLORA supports three kinds of comments: (1) all characters following
{\tt \%} until the end of the line; (2) all characters following
{\tt //} until the end of the line; (3) all characters inside a pair of
{\tt /*} and {\tt */}. Note that only (3) can span multiple lines.

Comments are recognized like whitespaces by the compiler.  Therefore,
tokens can also be delimited by comments.


\subsection{Operators}


\index{operators}
%
All programming constructs in \FLORA, such as ``{\tt ,}'', ``{\tt;}'',
``{\tt +}'', ``{\tt .}'', ``{\tt ->}'', and ``{\tt ::}'', are now
carefully organized as operators.  This brings the programming
practice in \FLORA much closer to that in Prolog. All the operator
definitions reside in the file ``{\tt flroperator.P}''. These
definitions consititute a grammar on which the \FLORA compiler relies
to parse a source program.

As in Prolog, there are three kinds of operators in \FLORA: infix,
prefix, and postfix. An infix operator appears between its two
arguments, while a prefix operator before its single argument and a
postfix operator after its single argument.

\index{operators!precedence level}
\index{operators!type}
%
Each operator has a \emph{precedence level}, which is a positive
integer. Each operator also has a \emph{type}. The possible types for
infix operatos are: {\tt xfx}, {\tt xfy}, {\tt yfx}, while the
possible types for prefix operators are: {\tt fx}, {\tt fy}, and the
possible types for postfix operators are: {\tt xf}, {\tt yf}. The type
of an operator is coded in a schematic expression in which {\tt f}
stands for the operator, {\tt x} and {\tt y} stand for the arguments.
Therefore, from the code of a type we can know whether an operator
is binary or unary.

The precedence levels and types together decide how an expression is
parsed when the structure of the term denoted is not made explicit
through the use of parentheses. The general rule is that the operator
with the highest precedence level is the principal functor, and the
subexpressions so determined will be its arguments. The code of a type
also carries extra information: {\tt x} means that the precedence
level of the functor of this argument should be \emph{smaller} than
that of the principle functor, while {\tt y} means that the precedence
level of the functor of this argument should be \emph{no greater} than
that of the principle functor.

For example, in \FLORA, {\tt -} has precedence level {\tt 800} and
type {\tt yfx}, {\tt *} has precedence level {\tt 700} and type
{\tt yfx}, {\tt ->} has precedence level 1100 and type {\tt xfx}.
Therefore, {\tt 8-2-3*4} is the same as {\tt -(-(8,2),*(3,4))} in
prefix notation, and {\tt a -> b -> c} will generate a parsing error.

\index{compiler directive!{\tt op}}
%
An atom named {\sf Name} can be declared as an operator of type {\sf Type}
and precedence level {\sf Precedence} by the following compiler directive:
\begin{qrules}
{\tt :- op(Precedence,Type,Name).}
\end{qrules}

As a notational convenience, the argument {\sf Name} can also be a list of
names of operators of the same type and precedence level, for instance,
\begin{qrules}
{\tt :- op(800,yfx,[+,-]).}
\end{qrules}

It is possible to have more than one operator of the same name. But the
\FLORA built-in operators are not allowed to be redefined.


\subsection{Logic Expressions}


\index{logic expressions}
%
In a \FLORA program, any combination of conjunction, disjunction, and
negation of literals can appear wherever a logial formula is allowed,
e.g., in a rule body.

Conjunction can be made through the infix operator ``{\tt ,}'', which
has precedence level {\tt 1300} and type {\tt xfy}, while disjunction
can be made through the infix operator ``{\tt ;}'', which has
precedence level {\tt 1400} and type {\tt xfy}, and negation can be
made through the prefix operators ``{\verb|~|}'', ``{\verb|\+|}'',
``{\tt not}'', and ``{\tt tnot}'', which all have precedence level {\tt
1200} and type {\tt fy}. Refer to Section~\ref{sec:negation} for more
information on the difference between negation operators.

For example, in \FLORA the following expression: \verb|a, b; c, ~d|,
is equivalent to the the logical formula:
$\tt (a \wedge b) \vee (c \wedge (\neg d))$.

\index{molecule!logic expressions}
%
As a syntactic sugar, any combination of conjunction, disjunciton, and
negation can also appear inside the specification of an object. For
instance, the following F-molecule:
\begin{qrules}
o.m[att1{\fd}val1, att2{\mvd}val2; meth{\fd}res]
\end{qrules}
is equivalent to the following formula:
\begin{qrules}
o.m[att1{\fd}val1], o.m[att2{\mvd}val2]; o.m[meth{\fd}res]
\end{qrules}

In fact, when used properly, this feature can help specifiy some
formula that would otherwise be very inconvenient to write. For example,
the following F-molecule:
\begin{qrules}
o.m[tnot att1{\fd}val1; tnot att2{\mvd}val2]
\end{qrules}
would otherwise have to be written as the following equivalent formula:
\begin{qrules}
o[m{\fd}X], (tnot X[att1{\fd}val1]; tnot X[att2{\mvd}val2])
\end{qrules}


\subsection{Arithmetic Expressions}


\index{arithmetic expression}
In \FLORA arithmetic expressions are \emph{not} always evaluated. As
in XSB, the arithmetic operators such as {\tt +}, {\tt -}, {\tt /},
and {\tt *}, are defined as normal binary functors. However, to
evaluate arithmetic expressions, \FLORA provides two operators:
{\tt is} and {\tt :=}, which are just different names for the same
operator. For example, {\tt X := 3+4} will bind {\tt X} to the value
{\tt 7}.

Order matters in \FLORA. All variables appearing in an arithmetic
expression must be instantiated at the time of evaluation. Otherwise,
a runtime error will occur.

Both single-valued and multi-valued path expressions are allowed in
arithmetic expressions, and all objects (variables) are considered to
be existentially quantified. For example, the following query
\begin{qrules}
?- john..bonus $+$ mary..bonus $>$ 1000.
\end{qrules}
is actually equivalent to
\begin{qrules}
?- john[bonus{\mvd}{\tt V1}], mary[bonus{\mvd}{\tt V2}], ${\tt V1}+{\tt V2} > 1000$.
\end{qrules}
The only difference is that the values of {\tt V1} and {\tt V2} will be
printed out for the latter query, but not for the former one.

\FLORA recognizes numbers as oids, and allows arithmetic expressions
to appear in path expressions, which are just like any complex id-terms
in path expressions.

Confustion may arise while combining arithmetic and path expressions together.
To illustrate, consider the following example:
%%
\begin{qrules}
1.m+2.n.k
\end{qrules}
%%
It is not immediately obvious whether the previous
example stands for the arithmetic expression {\tt (1.m)+(2.n.k)}, or for the
path expressions {\tt (1.m+2.n).k}, {\tt (1.m + 2).n.k}, or {\tt 1.(m+2).n.k}.
The correct answer is the first arithmetic expression, because in \FLORA
``{\tt .}'' is also defined as an operator whose precedence level is smaller
than that of ``{\tt +}''.

One more confusing example is {\tt 2.3.4}. Does it mean {\tt (2).(3).(4)}, or
{\tt (2.3).4}, or {\tt 2.(3.4)}?  In \FLORA, {\tt 2.3.4} actually means
{\tt (2.3).4}, because all tokens, like integers, floating numbers,
operators, etc., are first processed by the \FLORA lexer and then passed on to
the \FLORA parser. In general, the interpretation of ``{\tt .}'' as a decimal
point is done by the lexer and therefore precedes its interpretation as a object
reference operator, which is done by the parser.

Another ambiguous situation arises when the symbols {\tt -} and {\tt +}
are used.  Indeed, they can be used as minus and plus signs,
respectively, e.g., {\tt -3} and {\tt +3}, or as binary arithmetic
operators, e.g., {\tt 4-7} and {\tt 4+7}. Actually, the minus and plus
signs are defined in \FLORA as unary operators whose precedence levels
are smaller than those of binary operators.

Table~\ref{tab:tab-op-pre} lists various operators in decreasing precedence
order, their associativity, and arity.
%%
\begin{table}[htb]
\center
\texttt{ \small
\begin{tabular}{|c|c|c|c|c|}
\hline
%%
Precedence  & Operator & Use & Associativity & Arity \\ \hline
not applied & () & parentheses & not applied & not applied\\ \hline
not applied & . & decimal point & not applied & not applied \\ \hline
            & .   & single-valued object reference & left & binary \\ \cline{2-5}
300         & ..   & multi-valued object reference & left & binary \\ \cline{2-5}
            & :    & ISA specification & left & binary \\ \cline{2-5}
            & ::   & subclass specification & left & binary \\ \hline
600         & - & minus sign & right & unary \\ \cline{2-5}
            & + & plus sign & right & unary \\ \hline
700         & * & multiplication & left & binary \\ \cline{2-5}
            & / & division & left & binary \\ \hline
800         & - & subtraction & left & binary \\ \cline{2-5}
            & + & addition & left & binary \\ \hline
            & =< & less than or equal to & not applied & binary \\ \cline{2-5}
            & >= & greater than or equal to & not applied & binary \\ \cline{2-5}
1000        & =:= & equal to & not applied & binary \\ \cline{2-5}
            & ={\bksl}= & unequal to & not applied & binary \\ \cline{2-5}
            & := & assignment & not applied & binary \\ \cline{2-5}
            &is & \multicolumn{3}{c|}{same as :=} \\
\hline
\end{tabular}
}
\caption{Operators in Non-Increasing Precedence Order and Their Associativity and Arity}
\label{tab:tab-op-pre}
\end{table}

Wherever ambiguity may arise, parentheses can be used to avoid misleading
expressions. Here are more examples of valid expressions in \FLORA:
\begin{quote}
\begin{verbatim}
o1.m_1+o2.m2.method
2.(3.4)
3 + - - 2              equivalent to 3+(-(-2))
5 * - 6                equivalent to 5*(-6)
5.(-6)                 method -6 applied to object 5
\end{verbatim}
\end{quote}

Note that in the above examples, the parentheses in {\tt 5.(-6)} are needed,
because otherwise ``{\tt .-}'' would be recognized as one single token by
the \FLORA lexer. Similarly, the whitepsaces in \mbox{``{\tt + - -}''} and
\mbox{``{\tt * -}''} are also needed to deliminate the operators.


\subsection{Aggregates}


The syntax for aggregates in \FLORA is similar to that in \FLORID. An
aggregate in \FLORA has the following form:
%%
\begin{qrules}
agg\{X[Gs] $|$ body\}
\end{qrules}
%%
\index{aggregation!aggregate operator}
\index{aggregation!grouping}
%
where {\sf agg} represents the aggregate operator, {\sf X} is called the
aggregation variable, {\sf Gs} is a list of comma-separated grouping
variables, and {\sf body} is a logical formula that specifies the
query conditions. The grouping variables, {\sf Gs}, are optional. {\sf body}
can be any combinaiton of conjunction, disjunction, and negation of literals.

All the variables appearing in {\sf body} but not in {\sf X} or {\sf Gs} are
considered to be existentially quantified. Furthermore, the syntax of an
aggregate must satisfy the following conditions:
\begin{enumerate}
\item All the variable names in both {\sf X} and {\sf Gs} must
appear in {\sf body};
\item {\sf Gs} should not contain {\sf X}.
\end{enumerate}

Aggregates in \FLORA are evaluated as follows: First, the query
conditions specified by {\sf body} are evaluated to obtain all the bindings
for the tuple of the form {\sf \texttt{<}X, Gs\texttt{>}}; Then, these
tuples are grouped according to each distinct binding for
{\sf \texttt{<}Gs\texttt{>}}; Finally, for each group the aggregate operator
is applied the list of bindings for the aggregate variable.


\index{aggregates!min}
\index{aggregates!max}
\index{aggregates!count}
\index{aggregates!sum}
\index{aggregates!avg}
\index{aggregates!collectset}
\index{aggregates!collectbag}
%
The following aggregate operators are supported in \FLORA: {\sf min}, {\sf max},
{\sf count}, {\sf sum}, {\sf avg}, {\sf collectset} and {\sf collectbag}.

The operators {\sf min} and {\sf max} can apply to any list of
terms. The order among terms is defined by the XSB operator {\tt @=<}.  In
contrast, the operators {\sf sum} and {\sf avg} can take numbers only. If
the aggregate variable is instantiated to something other than a
number, {\sf sum} and {\sf avg} will discard it and generate a runtime
warning message.

For each group, the operator {\sf collectbag} collects all the bindings of
the aggregation variable into a list. The operator {\sf collectset} works
similarly to {\sf collectbag}, except that all the duplicates are removed
from the result list.

In general, aggregates can appear wherever a number or a list is
allowed. Therefore, aggregates can be nested. The following examples
illustrate the use of aggregates (some borrowed from the \FLORID manual):
%%
\begin{quote}
\begin{verbatim}
?- Z = min{S|john[salary(Year)->S]}.
?- Z = count{Year|john.salary(Year) < max{S|john[salary(Y)->S], Y < Year}}.
?- avg{S[Who]|Who:employee[salary(Year)->S]} > 20000. 
\end{verbatim}
\end{quote}
%%
If an aggregate contains grouping variables that are \emph{not} bound
by a preceding subgoal, then this aggregate would backtrack over such
grouping variables (In other words, they are considered to be
existentially quantified). For instance, in the last query above, the
aggregate will backtrack over the variable {\sf Who}. Thus, if
{\sf john}'s and {\sf mary}'s average salary is greater than {\sf 20000},
this query will backtrack and return both {\sf john} and {\sf mary}.

The following example is a query that for each employee asks for a list of
years when this employee had salary less than 60. This illustrates the use
of the {\sf collectset} aggregate.
%%
\begin{quote}
\begin{verbatim}
?- Z = collectset{Year[Who]|Who[salary(Year)->X], X < 60}.
Z = [1990,1991]
Who = mary

Z = [1990,1991,1997]
Who = john
\end{verbatim}
\end{quote}
%%

\index{aggregates!multi-valued methods}
\index{multi-valued methods!aggregation}
%
To retrieve all the return values of multi-valued methods, \FLORA provides
two new operators, {\tt ->->} and {\tt *->->}, for non-inheritable and
inheritable multi-value methods, respectively. 

In fact, {\tt ->->} and {\tt *->->} are just syntactic sugars and can be
defined using the {\tt collectset} aggregate, as follows:
\begin{quote}
\begin{verbatim}
O[M->->L] :- L=collectset{V|O[M->>V]}

O[M*->->L] :- L=collectset{V|O[M*->>V]}
\end{verbatim}
\end{quote}

The only restriction is that {\tt ->->} and {\tt *->->} cannot appear in
the head of a rule.


\subsection{Boolean Methods}


\index{method!boolean}
%
As a syntactic sugar, \FLORA introduces boolean methods, which can be
considered as scalar methods that return some fixed value, e.g.,
{\tt true}. For example, the following facts:
\begin{quote}
\verb|john[is_tall -> true].| \\
\verb|john[loves(tennis) -> true].|
\end{quote}
can be rewritten using boolean methods as follows:
\begin{quote}
\verb|john[is_tall].| \\
\verb|john[loves(tennis)].|
\end{quote}

Conceptually, boolean methods represent statements about objects whose
truth value is of primary concern. However, boolean methods are not
completely interchangeable with scalar methods in that boolean methods
do not return any value (not even the value {\tt true}). Therefore,
boolean methods can only appear inside the specification part for an
object, not as part of a path navigation like
\mbox{\tt john.is\_vegetarian}.

Like inheritable scalar methods, there are also inheritable boolean
methods. To make a boolean method inheritable, a ``\verb|*|'' sign
needs to be added before it. For instance, the following facts:
\begin{quote}
\begin{verbatim}
buddhist[*is_vegetarian].
john:buddhist.
\end{verbatim}
\end{quote}
say that all Buddhists are vegetarian and {\sf John} is a
Buddhist. Since \verb|is_vegetarian| is inheritable, it follows that
{\sf John} is also a vegetarian, i.e., \verb|john[is_vegetarian]|.


\subsection {Sets vs. Lists}


\index{multi-valued methods!set notation}
%
As mentioned in Section~\ref{sec-basic-flogic}, \fl uses the set
notation, \{$\ldots$\}, to specify a subset of the return values of a
multi-valued method, e.g.,
\begin{quote}
\verb|mary[children->>{john,frank}].|
\end{quote}
says that the children of {\sf Mary} \emph{include} {\sf John} and
{\sf Frank}.

However, since \fl does not support any operator to compose or
decompose a set, and Prolog does not support the set notation either,
breaking down of a set is done by the \FLORA compiler. This drawback
leads to less flexibility in programing style.

\index{multi-valued methods!list notation}
%
But in Prolog there is a data structure very close to sets --- lists.
Note that lists are indeed complex terms and potentially can also be
used as oids. To avoid this ambiguity, \FLORA introduces a new
operator, {\tt +>>}, to specify that all members of a list constitute
a subset of the return values of a multi-valued method. For example,
the following query tests whether all {\sf Mary}'s children are also
{\sf John}'s children:
\begin{quote}
\begin{verbatim}
?- findall(X,mary[children->>X],L), john[children+>>L].
\end{verbatim}
\end{quote}

Corresponding to inheritable multi-valued methods, \FLORA introduces
another new operator, {\tt *+>>}, for use in the list notation.

The only limitation to the use of {\tt +>>} and {\tt *+>>} is that
they cannot appear in a fact or the head of a rule.


\subsection{HiLog and Related Issues} \label{sec:hilog}


\index{HiLog}
%
\FLORA now supports HiLog. In HiLog, complex terms can appear wherever
a function symbol is allowed, e.g., {\sf group(X)(Y,Z)}, where the
functor is the complex term {\sf group(X)}. In \FLORA, any id-term,
including function symbols, predicate symbols, and their arguments,
can be a HiLog term.

\index{HiLog!translation}
%
In fact, HiLog has a higher order syntax but a first order semantics
\cite{hilog-jlp}. Any HiLog term can be consistently translated into
a Prolog term. For instance, {\sf group(X)(Y,Z)} can be represented
by the Prolog term {\sf flapply(flapply(group,X),Y,Z)}. The translation
scheme is pretty straightforward and described in \cite{hilog-jlp}.

In \FLORA any term is considered as a HiLog term and therefore subject
to translation. That is, even a normal Prolog term will by default be
represented using the HiLog translaiton, e.g., {\sf foo(a)} will be
represented as {\sf flapply(foo,a)}. This is to guarantee that no
extra translation will be needed for unification to work correctly at
runtime. For instance, {\sf foo(a)} will unify with {\sf F(a)} and
bind the variable {\sf F} to {\sf foo}.

\index{compiler directive!{\tt prolog}}
%
However, if the default HiLog translation is not intended, the following
compiler directive can be used to define a traditional Prolog term:
\begin{quote}
\verb|:- prolog functor/arity.|
\end{quote}
where {\sf functor} stands for the functor of a Prolog term and
{\sf arity} stands for its arity.

Note that the above {\sf prolog} compiler directive does not affect
the translation of its arguments. For instance, if nothing more than
the following directive is defined in a \FLORA program:
\begin{quote}
\verb|:- prolog member/2.|
\end{quote}
then a term like {\sf member(foo(a),X)} in this program will be translated
into {\sf member(flapply(foo,a),X)}.

\index{compiler directive!{\tt prologall}}
%
In the case where all Prolog terms should stay as they are, the following
compiler directive can be used:
\begin{quote}
\verb|:- prologall.|
\end{quote}
However, for those non-Prolog terms, they will still be represented using
the HiLog translation.

For a list of built-in Prolog term definitions in \FLORA, refer to the
file {\tt flrprolog.P}.

\index{HiLog!unification}
%
The following problem arises as far as unificatoin is concerned:
Should {\sf foo} unify with {\sf foo()}, which both seem like 0-ary
terms? Moreover, if the answer is yes, then should {\sf foo} unify
with {\sf foo()()}, which is also a valid HiLog term?  In \FLORA, the
answers depend on whether these 0-ary terms appear as predicate calls
in rule heads and bodies, or just as id-terms, i.e., arguments.

In general, the presence of a pair of parentheses in a term would imply
one application of the HiLog translation during
compilation. Therefore, {\sf foo()} would be represented as
{\sf flapply(foo)}, {\sf foo()()} as {\sf flapply(flapply(foo))},
and {\sf foo} as it is, i.e, {\sf foo}.  So {\sf foo} does not unify
with {\sf foo()} or {\sf foo()()} while they all act as id-terms. One
of the advantages of keeping them differently is to make a \FLORA module
communicate easily with normal XSB modules, which do not fully support
HiLog. Refer to Section~\ref{sec:flora-modules} for more information
on \FLORA modules.

However, to make programming easy in the higher order fashion, 0-ary
Prolog terms are translated like 0-ary HiLog terms when they appear
as predicate calls in rule heads and bodies. For instance, the
following two rules are equivalent in \FLORA:
\begin{quote}
\begin{verbatim}
foo :- body.
foo() :- body.
\end{verbatim}
\end{quote}
and can be invoked in a rule body by either {\sf foo}, {\sf foo()},
or more fancily, {\sf X()}, where {\sf X} is a variable to be bound
to {\sf foo} when the call succeeds.

{\sf foo} never unifies with {\sf foo()()} in \FLORA.


\subsection{Meta-programming in \FLORA}


\index{meta-programming}
%
The syntax of \fl and HiLog lends itself naturally to meta-programming. For
instance, it is easy to examine the methods and types defined for the
various classes.  Here are some simle examples:
%%
\begin{quote}
\begin{verbatim}
// all unary scalar methods defined for John
?- john[M(_) -> _].

// all unary scalar methods that apply to John,
// for which a signature was declared
?- john[M(_) => _].

// all method signatures that apply to John,
// which are either declared explicitly or inherited
?- john[M => _].

// all method invocations defined for John
?- john[M -> _].
\end{verbatim}
\end{quote}
%%

However, a number of meta-programming primitives are still needed
since they cannot be directly expressed in \fl. Many such features are
provided by the underlying XSB system and \FLORA simply takes advantage of
them: 
%%
\begin{quote}
\begin{verbatim}
?- functor(X,f,3).
X = f(_h455,_h456,_h457)
Yes.

?- compound(f(X)).
X = _h472
Yes.

?- X =.. [f,a,b].
X = f(a,b)
Yes.
\end{verbatim}
\end{quote}
%%
Note that these primitives are used for Prolog terms only and
described in the XSB manual. Meta-programming support for HiLog terms
and F-molecules will be provided through future enhancement.


\subsection{Anonymous Oid}


\index{anonymous oid}
%
For applications where oids are not important, \FLORA provides the
compiler directive \verb|_#| to automatically generate a new
oid. \verb|_#| can be used wherever an id-term is allowed. Like the
anonymous variable \verb|_|, each occurrence of \verb|_#| represents
an anonymous oid. The difference is that such an oid is not only
``unique'' in each rule, but in the source program as well.

However, ``unique'' is just the approximate saying. The \FLORA
compiler does not really check for uniqueness of those newly generated
oids.  When \FLORA compiles a source program, it first initiates a
sequential counter.  Each occurrence of \verb|_#| causes the counter
to be incremented by one, and a new oid to be generated by
concatenating a fixed prefix, e.g, \verb|_$_$_flora'newoid|, and the
content of the counter.

For example, the following program:
\begin{quote}
\begin{verbatim}
_#[ssn->123, father->_#[name->john, spouse->_#[name->mary]]].
foo[_#(X)->Y] :- foo(X,Y).
\end{verbatim}
\end{quote}
is the same as the following one, where the anonymous oids are explicitly
replaced by the corresponding encrypted strings that would be generated
by the \FLORA compiler:
\begin{quote}
\begin{verbatim}
'_$_$_flora''newoid1'[ssn->123, father->'_$_$_flora''newoid2'].
'_$_$_flora''newoid2'[name->john, spouse->'_$_$_flora''newoid3'].
'_$_$_flora''newoid3'[name->mary].
foo['_$_$_flora''newoid4'(X)->Y] :- foo(X,Y).
\end{verbatim}
\end{quote}


\subsection{Loading Files}


\index{loading files}
%
Like XSB, \FLORA also supports the list notation to load a list of
files. For instance, the following command:
\begin{quote}
\verb|[myprog1, myprog2]|
\end{quote}
will try to load the two programs {\tt myprog1} and {\tt myprog2}.

Both XSB files and \FLORA files can appear in the list. Note
that the names of XSB source files have {\tt .P} as extension,
while XSB byte code files have {\tt .O}, and \FLORA source files
have {\tt .flr}. If a file name is not specified with an extension,
\FLORA will search the follwing files in order until one is found
or none: \FLORA source files, XSB source files, and XSB byte code
files.

For \FLORA programs, an optional module name can be specified using
the syntax bellow:
\begin{quote}
\verb|[..., program >> module, ...]|
\end{quote}
For instance, the following command:
\begin{quote}
\verb|[myprog >> foomod]|
\end{quote}
will try to load the \FLORA program {\tt myprog} into the \FLORA
module named {\tt foomod}.

If a module name is specified using the {\tt >>} syntax, then
\FLORA will only search for \FLORA programs. On the other hand,
if a module name is not specified for a \FLORA program, then this
program will be loaded into the default \FLORA module named
{\tt main}. As an example, the following command:
\begin{quote}
\verb|['myprog.flr', mydb >> foomod]|
\end{quote}
will load the \FLORA file {\tt myprog.flr} into module {\tt main} and
{\tt mydb.flr} into module {\tt foomod}.


\section{\FLORA Modules} \label{sec:flora-modules}

\index{module}
%
\FLORA modules are different from XSB modules in the following two
ways:
\begin{enumerate}
\item In general, module names in XSB are determined by file names.
      For example, if the file {\tt myprog.P} is loaded, then the name
      of the module that this file resides in is named {\tt myprog}.
      On the contrary, module names in \FLORA are more conceptual and
      have nothing to do with file names at all. Any program can be
      loaded into any arbitrarily named module as long as the name of
      the module is provided.

\item XSB modules are static. Once a program is loaded as a module (not
      into {\tt usermod}), no other program (e.g., a new version of
      the same program) can be loaded to replace the previous program.
      It remains in the module until the XSB system exits. \FLORA modules,
      however, can be dynamically loaded during runtime. When a program
      is loaded into a module, its effect is just like that of the XSB
      command {\tt consult} --- new predicates in the program will wipe
      out those old predicates already loaded into the module if they
      share the same functor and arity.
\end{enumerate}

Since a program can be loaded into any module, different byte code of
the same program needs to be generated for each different module.  The
command {\tt flcompile(file,module)} can be called to generate the
byte code for the program to be loaded into the module named
{\tt module}. The name of the byte code file is the module name appended
to the program file name. For instance, {\tt flcompile(myprog,foomod)}
would generate the byte code {\tt myprog\_foomod.O}.


\subsection{Compilation} \label{sec:flora-modules-compilation}


\index{module!compilation}
%
To compile a program file for a module, \FLORA actually encodes the
module name into all predicate names in the program, and into all
those predicate names used for translating \fl literals. Different
modules will cause different names to be encoded, thus avoiding
predicate name clashes and rendering the program loadable to different
modules at runtime.  For example, the following \FLORA program:
\begin{quote}
\begin{verbatim}
foo(a).
john[father->smith].
\end{verbatim}
\end{quote}
will be compiled into the following XSB program, assuming it is to
loaded into the module named {\tt foomod}:
\begin{quote}
\begin{verbatim}
'_$_$_flora''foomod''flapply'(foo,a).
'_$_$_flora''foomod''fd'(john,father,smith).
\end{verbatim}
\end{quote}

\index{compiler directive!{\tt florapred}}
%
In the above examples, the translation of {\tt foo(a)} into
\verb|'_$_$_flora''foomod''flapply'(foo,a)| is due to the HiLog
translation. If this is not wanted, the compiler directive,
{\tt florapred functor/arity}, can be used to turn off the HiLog
translation, as illustrated as follows:
\begin{quote}
\begin{verbatim}
:- florapred foo/1.
foo(a).
john[father->smith].
\end{verbatim}
\end{quote}
The above program will be compiled into the following XSB program,
assuming it is to be loaded into the module named {\tt foomod}:
\begin{quote}
\begin{verbatim}
'_$_$_flora''foomod''foo'(a).
'_$_$_flora''foomod''fd'(john,father,smith).
\end{verbatim}
\end{quote}

Note that the compiler directive \mbox{\tt prolog functor/arity} is used
to define a Prolog term, no matter whether it appears as
a predicate call or as an argument. However, \mbox{\tt florapred functor/arity}
only turns off HiLog translation for predicate calls, not for arguments.
Moreover, the presence of a {\tt prolog} directive will obviate
the {\tt florapred} directive for the same functor and arity pair. 

Shown below is an example to illustrate the different effects of the
{\tt prolog} and {\tt florapred} compiler directives:
\begin{quote}
\begin{verbatim}
:- prolog foo/1.
:- prolog goo/1.
:- florapred foo/1.
:- florapred hoo/1.
foo(X) :- hoo(goo(X)), goo(hoo(X)).
\end{verbatim}
\end{quote}
Assuming the above program is to be loaded into the module named
{\tt foomod}, it will be compiled into the following XSB program:
\begin{quote}
\begin{verbatim}
foo(X) :- '_$_$_flora''foomod''hoo'(goo(X)), goo(flapply(hoo,X)).
\end{verbatim}
\end{quote}

\index{compiler directive!{\tt florapredall}}
%
To turn off the HiLog translation for all predicate calls in a program,
the following compiler directive can be used:
\begin{quote}
\begin{verbatim}
:- florapredall.
\end{verbatim}
\end{quote}


\subsection{Reference to Literals in Other Modules}


\index{module!reference}
%
The syntax to refer to literals defined in other modules is as follows:
\begin{quote}
\verb|literal @ module| \footnote{
%
If the name of a \FLORA module contains the character ``{\tt '}'', the
\FLORA undefined predicate handler may not work properly in some cases.
%
}
\end{quote}
where {\tt literal} can be any predicate or F-molecule. For instance,
\verb|foo(a) @ foomod| refers to {\tt foo(a)} in the module named
{\tt foomod}, be it specified as a fact or defined by a rule.

\index{module!rules}
%
By default, if not followed by the ``\verb|@|'' sign and a module
name, all literals only refer to the module where they reside. To specify
literal references in a \FLORA program, the following rules need to be
followed:
\begin{enumerate}
\item The name of the module can be specified using an atom or a variable.
      If a variable is used, then it must be bound to an atom at runtime
      when this reference is evaluated.

\item Literal reference cannot appear in a rule head or be specified as
      a fact. For example, the following program will generate
      parsing errors in \FLORA:
      \begin{quote}
      \verb|john[father->smith] @ foomod.| \\
      \verb|foo(X) @ foomod :- goo(X).|
      \end{quote}

\item Module name specification is distributive over logical connectives,
      including the conjunction operator, ``\verb|,|'', the disjunction operator,
      ``\verb|;|'', and the negation operators, ``\verb|~|'', ``\verb|\+|,
      ``\verb|not|'', and ``\verb|tnot|''. For example, the formula below:
      \begin{quote}
      \verb|(john[father->smith], tnot smith[spouse->mary]) @ foomod|
      \end{quote}
      is equivalent to the following formula:
      \begin{quote}
      \verb|john[father->smith] @ foomod; tnot (smith[spouse->mary] @ foomod)|
      \end{quote}

\item Module name specifications can be nested. The one closest to a literal
      takes effect. For example, the formula below:
      \begin{quote}
      \verb|(foo(a), goo(b) @ goomod, hoo(c)) @ foomod|
      \end{quote}
      is equivalent to the following formula:
      \begin{quote}
      \verb|foo(X) @ foomod, goo(b) @ goomod, hoo(c) @ foomod|
      \end{quote}

\item Module name specification propogates to any F-molecule appearing
      in the argument position of a predicate for which the module name is
      specified. For example, the formula below:
      \begin{quote}
      \verb|foo(a.b[c->d]) @ foomod|
      \end{quote}
      is equivalent to the following formula:
      \begin{quote}
      \verb|a[b->X] @ foomod, X[c->d] @ foomod, foo(X) @ foomod|
      \end{quote}

\item However, module name specification does not affect the translation of
      any term in the argument position unless a module name is explicitly
      specified for it. For example, the \FLORA queries below:
      \begin{quote}
      \verb|?- foo(goo(a[c->d])) @ foomod.| \\
      \\
      \verb|?- foo(goo(a[c->d]) @ goomod) @ foomod.|
      \end{quote}
      will be compiled into the following XSB queries, respectively:
      \begin{quote}
      \verb|?- '_$_$_flora''foomod''fd'(a,b,c),| \\
      \verb|   '_$_$_flora''foomod''flapply'(foo,flapply(goo,a)).| \\
      \\
      \verb|?- '_$_$_flora''goomod''fd'(a,b,c),| \\
      \verb|   '_$_$_flora''foomod''flapply'(foo,'_$_$_flora''goomod''flapply'(goo,a)).|
      \end{quote}
\end{enumerate}


\subsection{Calling XSB from \FLORA}


\index{module!calling XSB}
\index{module!import}
\index{compiler directive!{\tt import}}
%
Like XSB, \FLORA supports the compiler directive,
\mbox{\tt import functor/arity from module}, to call a predicate
with functor {\tt functor} and arity {\tt arity} that is defined in an
XSB module named {\tt module}. For example, the following line:
\begin{quote}
\verb|:- import member/2 from basics.|
\end{quote}
will tell the \FLORA compiler to compile {\tt member/2} as a predicate
that refers to the XSB module named {\tt basics}.

In fact, the above compiler directive will cause \emph{every}
occurrence of {\tt member/2} in a \FLORA program to be compiled into a
predicate that refers to the XSB module {\tt basics}. Moreover, its
side effect is also to define {\tt member/2} as a Prolog
term. It is equivalent to saying the following two lines of
compiler directives:
\begin{quote}
\begin{verbatim}
:- prolog member/2.
:- import member/2 from basics.
\end{verbatim}
\end{quote}


\index{module!{\tt prologlib}}
%
To avoid the above mentioned side effect, the following syntax, which
is an extension to the module name specification, can be exploited to
make individual XSB calls in a \FLORA program:
\begin{quote}
\begin{verbatim}
xsb_predicate @ prologlib xsb_module
\end{verbatim}
\end{quote}
where {\tt prologlib} is a fixed keyword, and \verb|xsb_predicate| is a
predicate defined in the XSB module named \verb|xsb_module|. Note that
the translation of other predicates that have the same functor and
arity as \verb|xsb_predicate| will not be affected by this call. For
example, the \FLORA query below:
\begin{quote}
\verb|?- member(a,[a]) @ prologlib basics, member(X,[a]).|
\end{quote}
will be compiled into the following XSB query:
\begin{quote}
\begin{verbatim}
?- import member/2 from basics, member(a,[a]),
   '_$_$_flora''main''flapply(member,X,[a]).
\end{verbatim}
\end{quote}

Some XSB predicates, including those standard predicates like
{\tt write/1}, can be called directly without being imported from a
module. To call such XSB predicates in a \FLORA program, simply omit
the XSB module name and replace it with a pair of parentheses {\tt ()},
e.g,
\begin{quote}
\verb|write(X) @ prologlib ()|
\end{quote}


\subsection{Calling \FLORA from XSB}

\index{module!calling \FLORA}
%
The major problem with calling \FLORA from XSB is that XSB does not
understand the \fl syntax. However, \FLORA supports the XSB syntax and
since every \FLORA program is finally compiled into an XSB program, it
can loaded and used like an XSB program.

First, the \FLORA program needs to declare the predicates to be called
as Prolog terms. If the calling XSB module needs to access \fl data,
then new predicates should be introduced to contain the relevant
F-molecules.  For example, the following program segment shows how to
define a new Prolog predicate to carry course selection information
about students:
\begin{quote}
\begin{verbatim}
:- prolog studinfo/4.

studinfo(SSN,Year,Semester,CourseCode) :-
        _Person:student[ssn->SSN,Year(Semester)->>CourseCode].
\end{verbatim}
\end{quote}

Once the predicates of interest have been properly defined in a \FLORA
program, the calling XSB program needs to import them using the
compiler directive \mbox{\tt import functor/arity from module}, where
{\tt module} must be either {\tt usermod} or {\tt global}. This means
that the calling XSB module must explicitly load the \FLORA program 
at runtime by issuing the command {\tt consult/1}.

\index{module!file name}
%
The only thing that needs extra care is the name of the file that
should be passed to \mbox{\tt consult/1}. As pointed out earlier at
the beginning of Section~\ref{sec:flora-modules}, module names are
appended to \FLORA file names to generate the names for the byte code
files. Therefore, the name of final byte code file, but not the
original \FLORA file, should be used in {\tt consult/1}.

For instance, suppose the name of the \FLORA program is
{\tt studentdb.flr}, and the byte code is generated using the \FLORA
command {\tt flcompile(studentdb,studmod)}, then the XSB program
calling predicates in it should use the command
\verb|consult(studentdb_studmod)| to load the file.


\subsection{Including Files into \FLORA Programs}


There are programs that encode a lot of common functionalities but
need to access predicates in the calling program. These programs are
like an integral part of the calling program and therefore cannot be
used like a \FLORA module or an XSB module. Ideally these programs
need to be embeded into the calling program to function correctly.

\index{module!include}
%
The only way to use these programs is to include their sources into
the calling program. To include a file into a program, the following
preprocessing command can be used:
\begin{quote}
\verb|#include "file_name"|
\end{quote}
where \verb|file_name| is the name of the program file to be included.

In fact, before parsing a program, \FLORA \emph{always} calls the
{\sf gpp} preprocessor to process its input program. Therefore, any
preprocessing command that is recognized by {\sf gpp} can appear in
a \FLORA program (refer to the XSB manual for more information on
{\sf gpp}). The above mentioned \verb|#include "file_name"| is one
of the preprocessing commands supported by {\sf gpp}.


\subsection{This Module Name}


\index{module!{\tt flthismod}}
%
Although a \FLORA program can be loaded into any arbitrary module, the
program itself can call the predicate {\tt flthismod/1} to retrieve
the name of the module into which it is loaded at runtime.

By default, {\tt flthismod/1} is declared for each compiled \FLORA program
as follows:
\begin{quote}
\verb|:- florapred flthismod/1.|
\end{quote}
It accepts one argument and will bind it to the name of the \FLORA
module.


\section{\FLORA and Tabling}\label{sec-tabling-flora}


\subsection{Tabling in a Nut Shell}


\index{tabling}
%
Tabling is a technique that enhances top-down evaluation with a mechanism
that remembers the calls previously made during query evaluation.
This technique is known to be essentially equivalent to the Magic Sets
method for bottom-up evaluation. However, tabling combined with top-down
evaluation has the advantage of being able to utilize highly optimized
compilation techniques developed for Prolog. The result is a very efficient
deductive engine.

XSB lets the user specify which predicates must be tabled.  The \FLORA
compiler automatically tables the predicates used to flatten F-molecules.
However, the user is responsible for telling the system which other
predicates must be tabled. (Normally, these are predicates defined by the
user.)  \FLORA accepts the same tabling directives as XSB does
(Section~\ref{sec-comp-directives} lists all the compiler directives).

\index{table}
\index{compiler directive!{\tt table}}
%
Due to the current implementation of HiLog and module system in \FLORA,
a HiLog predicate cannot be tabled directly (without hacking into the
\FLORA implementation). To table a predicate, this predicate must be
declared either as a Prolog term using the \mbox{\tt prolog functor/arity}
compiler directive, or as a non-HiLog term using the
\mbox{\tt florapred functor/arity} compiler directive (refer to
Section~\ref{sec:hilog} and Section~\ref{sec:flora-modules-compilation}
for more information on these compiler directives). For instance,
the following \FLORA program:
\begin{quote}
\begin{verbatim}
:- prolog tc/2.
:- table tc/2.

tc(X,Y) :- edge(X,Y).
tc(X,Y) :- edge(X,Y), tc(Y,Z).
\end{verbatim}
\end{quote}
turns off the HiLog translation for the predicate {\tt tc/2} and
renders it as a Prolog term to be tabled directly by XSB.

It is important to keep in mind that XSB does not do reordering of
objects and predicates during joins. Instead, all joins are performed
left-to-right.  Thus, program clauses must be written in such a way as
to ensure that smaller predicates and classes appear early on in the
join. Also, even though XSB tables the results obtained from previous
queries, the current tabling engine has several limitations. In
particular, when a new query comes in, XSB tries to determine if this
query is ``similar'' to one that already has been answered (or is in
the process of being evaluated).  Unfortunately, the current notion of
similarity used by XSB is fairly weak, and many unnecessary
recomputations might result. This problem will be corrected in a
future release.

\index{abolish\_all\_tables}
%%
It is also important to be aware that when XSB (and \FLORA) evaluate a program,
all tabled predicates are partially materialized and all the computed
tuples are stored in XSB tables. Thus, if you change the set of facts, the
existing tables must be discarded in order to allow XSB to recompute the
results. This is accomplished by issuing the predicate
{\tt abolish\_all\_tables/0} described in the XSB manual.


\subsection{Procedural Methods}


\index{method!procedural}
%
Unfortunately, tabling sometimes has undesirable effects in
``real-world'' programming, especially on those predicates with
non-logical ``side effects'' (e.g., writing or reading a file).  If a
tabled predicate has such side effects, then the first time the predicate
is called the side effect will be performed, but the second time the call
simply returns with success or failure (depending on the outcome of the
first call).  Thus, if the predicate is intended to perform the side
effect each time it is called, it will not operate correctly.

In the object-oriented programming style, people tend to define
methods with side effects and attach them to objects. These methods
are \emph{procedural} in nature and should be performed each time they
are called. Because by default \FLORA tables everything that comes
from F-molecules, these procedural methods are potentially subject to
the same problem as described above.

To solve this program, \FLORA introduces a new syntax to identify
procedural methods --- simply put a ``\verb|#|'' sign in front of a
procedural method. For instance, the following rule defines an
output method that for every object writes out its oid in canonical
form:
\begin{quote}
\verb|O[# output] :- write_canonical(O).|
\end{quote}

Like boolean methods in \FLORA, procedural methods do not return any
value but can take any number of arguments. The only difference is that
procedural methods are \emph{not} tabled, while boolean methods are.


\subsection{Cuts}


\index{cuts in \FLORA}
\index{cutting across tables}
%%
No discussion of a logic programming language is complete without a few
words about the infamous Prolog cut (!). Although Prolog cut has been
(mostly rightfully) excommunicated by as far as Database Query Languages
are concerned, it is sometimes indispensable when doing ``real work'', like
pretty-printing \FLORA programs or implementing a pattern matching
algorithm.  To facilitate this kind of tasks, \FLORA lets the programmer
use cuts.  However, the current implementation of XSB has a limitation that
Prolog cuts cannot ``cut across tabled predicates.''  Without trying to
pretend to be experts, we refer the reader to the XSB manual for details on
this obscure problem. The XSB team is considering correcting this problem
in a future release.

For now, enjoy your cut.  If you get an error message telling something
about cutting across the tables --- you know that you may have cut too much.
The basic rule that can keep you out of trouble is: do not put a cut
in the body of a rule \emph{after} any F-molecule. However, it is (usually)
OK to to put a cut before any F-molecule. It is even OK to have a cut in
the body of a rule that \emph{defines} an F-molecule (again, provided that
the body has no F-molecule to the left of that cut).


\section{Updates}


\index{update}
%
\FLORA provides the users with several predicates to update the runtime
database. The first note about these predicates is that they operate
on \emph{base} facts only, but not on derived facts. That is, an
insertion predicate only inserts facts into the database, while a
deletion predicate never remove derived facts from the database.

These update predicates can be categorized into two different groups,
based on their effects upon backtracking: non-backtrackable and
backtrackable.


\subsection{Non-backtrackable Updates} \label{sec:non-backtrackable-updates}


\index{update!non-backtrackable}
%
The effects of non-backtrackable updates persist once the operations are
executed, even when a later failure causes the system to backtrack.

\FLORA supports the following non-backtrackable update predicates:
{\tt insert}, {\tt insertall}, {\tt delete}, {\tt deleteall},
{\tt erase}, {\tt eraseall}.

\index{non-backtrackable update!insert}
\index{non-backtrackable update!insertall}
%
\paragraph{Insertion.} The syntax of an insertion predicate is as follows:
\begin{quote}
\begin{verbatim}
insop{literals [| formula]}
\end{verbatim}
\end{quote}
where {\tt insop} stands for either {\tt insert} or {\tt insertall},
{\tt literals} stands for any number of comma separated literals,
including predicates and \mbox{F-molecules}, and {\tt [| formula]}
stands for an optional logic formula following the vertical bar
(``{\tt |}'') to specify additional quantifying conditions for the
literals that will be inserted. For instance,
\begin{quote}
\begin{verbatim}
insert{p(a),mary[spouse->smith,children->>frank]}
insertall{P[spouse->S] | S[spouse->P]}
\end{verbatim}
\end{quote}

While {\tt insert} inserts only one instance of the literals that
satisfy the formula, {\tt insertall} inserts \emph{all} instances
of the literals that satisfy the formula.

Note that each literal appearing inside the insertion predicate is
treated as a separate fact. Thus these literals should follow the
syntactic rules for facts and literals in the rule head (e.g,
multi-valued path expressions are not allowed), except that module
names can be specified for them, which means it is allowed to update
the facts for other modules. Moreover, module name specification here
is distributive over multiple literals enclosed by a pair of
parentheses. For example, the insertion predicate below:
\begin{quote}
\begin{verbatim}
insert{(mary[children->>frank], john[father->smith]) @ foomod}
\end{verbatim}
\end{quote}
is equivalent to the following one:
\begin{quote}
\verb|insert{mary[children->>frank] @ foomod, john[father->smith] @ foomod}|
\end{quote}


\index{non-backtrackable update!delete}
\index{non-backtrackable update!deleteall}
\index{non-backtrackable update!erase}
\index{non-backtrackable update!eraseall}
%
\paragraph{Deletion.} The syntax of a deletion predicate is as follows:
\begin{quote}
\begin{verbatim}
delop{literals [| formula]}
\end{verbatim}
\end{quote}
where {\tt delop} stands for any of the following four deletion
operations: {\tt delete}, {\tt deleteall}, {\tt erase}, and
{\tt eraseall}, {\tt literals} stands for any number of comma
separated literals, including predicates and \mbox{F-molecules},
and {\tt [| formula]} stands for an optional logic formula following
the vertical bar (``{\tt |}'') to specify additional quantifying
conditions for the literals that will be deleted.

For instance, the following predicate:
\begin{quote}
\begin{verbatim}
deleteall{john[Year(Semester)->>Course] | Year < 2000}
\end{verbatim}
\end{quote}
will delete {\sf John}'s course selection history before the year
2000.

Note that the comma separated literals are actually in conjunctive
form specifying the conditions that the literals should hold jointly
to be deleted. That is, if variables in these literals have the same
name, then they should share the same binding. For example, suppose
the database contains the following facts:
\begin{quote}
\begin{verbatim}
p(a). p(b). q(a). q(c).
\end{verbatim}
\end{quote}
then the predicate below:
\begin{quote}
\begin{verbatim}
delete{p(X),q(X)}
\end{verbatim}
\end{quote}
will succeed with the variable {\tt X} bound to {\tt a}.  However, if
the database contains only the facts {\tt p(b)} and {\tt q(c)}, then
the above predicate will fail.

\FLORA supports four different deletion operations, {\tt delete},
{\tt deleteall}, {\tt erase}, and {\tt eraseall}:
\begin{enumerate}
\item If there are serveral bindings or matches for
      the literals to be deleted, then {\tt delete} will choose only
      one of them and delete it. The choice is nondeterministic.
      For instance, suppose the database contains the following facts:
      \begin{quote}
      \verb|p(a). p(b). q(a). q(b).|
      \end{quote}
      then
      \begin{quote}
      \verb|delete{p(X),q(X)}|
      \end{quote}
      will succeed with {\tt X} bound to either {\tt a} or {\tt b},
      depending on the ordering of facts in the database at runtime.

\item On the other hand, {\tt deleteall} will try to delete ``all''
      bindings or matches. However, the meaning of ``all'' here is
      limited to some ``minimum'' semantics. The problem arises when
      a fact appears in several bindings or matches for the literals.
      For instance, suppose the database contains the following facts:
      \begin{quote}
      \verb|p(a). q(b). q(c).|
      \end{quote}
      then both \verb|<p(a),q(b)>| and \verb|<p(a),q(c)>| are good
      bindings for \verb|<p(a),q(X)>|. But the question is: What will
      \verb|deleteall{p(a),q(X)}| do? Should it delete both or just one?

      \smallskip

      In fact, \FLORA will randomly choose one of them and delete it.
      The idea is to remove the \emph{minimum} amount of facts to
      \emph{invalidate} the literals taken together as a conjunctive
      query. For the above example, after the two facts {\tt p(a)} and
      {\tt q(b)} (or {\tt q(c)}) are deleted, there will be no binding for
      \verb|<p(a),q(X)>|. Therefore, further deleting {\tt q(c)}
      (or {\tt q(b)}) seems like an overkill.

      \smallskip

      Unlike the {\tt delete} predicate, {\tt deleteall} \emph{always}
      succeeds. However, when it succeeds, {\tt deleteall} will leave
      all variables unbound.

\item {\tt erase} works like {\tt delete}, but with extra functionalities
      built for \fl facts. While deleting an \fl fact, {\tt erase}
      will traverse the object tree starting from the object of
      interest specified in this fact, and delete \emph{all} \fl facts
      related those objects encountered on its path.

\item {\tt eraseall} is the full version of {\tt erase}, just like {\tt deleteall}
      is the full version of {\tt delete}. {\tt eraseall} will try to erase
      ``all'' bindings or matches. It always succeeds and will leave all
      variables unbound when it returns.
\end{enumerate}


\subsection{Backtrackable Updates}


\index{update!backtrackable}
%
The effects of backtrackable updates are undone upon backtracking,
i.e., a previous inserted item will be removed from the database, and
a previous deleted item will be put back into the database when the
system backtracks.

\index{backtrackable update!btinsert}
\index{backtrackable update!btinsertall}
\index{backtrackable update!btdelete}
\index{backtrackable update!btdeleteall}
\index{backtrackable update!bterase}
\index{backtrackable update!bteraseall}
%
In \FLORA, the syntax of backtrackable update predicates is similar
to that of non-backtrackable update predicates, except that the
predicate names are different. The syntax of a backtrackable insertion
predicate is:
\begin{quote}
\begin{verbatim}
btinsop{literals [| formula]}
\end{verbatim}
\end{quote}
while the syntax of a backtrackable deletion predicate is:
\begin{quote}
\begin{verbatim}
btdelop{literals [| formula]}
\end{verbatim}
\end{quote}
where {\tt btinsop} stands for either {\tt btinsert} or {\tt btinsertall},
{\tt btdelop} stands for any of the following four backtrackable
deletion operations: {\tt btdelete}, {\tt btdeleteall}, {\tt bterase}, and
{\tt bteraseall}. The meanings of {\tt literals} and {\tt [| formula]} are
explained in Section~\ref{sec:non-backtrackable-updates}.

{\tt btinsert}, {\tt btinsertall}, {\tt btdelete}, {\tt btdeleteall},
{\tt bterase}, and {\tt bteraseall} work similarly to {\tt insert},
{\tt delete}, {\tt deleteall}, {\tt erase}, and {\tt eraseall},
respectively, except that the former is backtrackable while the latter
is not. Refer to Section~\ref{sec:non-backtrackable-updates} for
details of these operations.

To illustrate the difference between backtrackable and
non-backtrackable updates, consider the following execution trace
immediately after the \FLORA system starts:
\begin{quote}
\begin{verbatim}
flora2 ?- insert{p(a)}, fail.

No.

flora2 ?- p(a).

Yes.

flora2 ?- btinsert{q(a)}, fail.

No.

flora2 ?- q(a).

No.
\end{verbatim}
\end{quote}
In the above example, when the first \verb|fail| executes, the system
backtracks to {\tt insert\{p(a)\}} and does nothing. Thus the insertion
of {\tt p(a)} persists and the following query \verb|p(a)| returns
with {\tt Yes}. However, when the second \verb|fail| executes, the
system backtracks to {\tt btinsert\{q(a)\}} and removes {\tt q(a)} that
was previously inserted into the database. Thus the next query
\verb|q(a)| returns with {\tt No}.


\subsection{Updates and Tabling}


To implement object properties, \FLORA relies on a feature of XSB called
\emph{tabling} (see Section~\ref{sec-tabling-flora} for more details).
Unfortunately, tabling and database updates do not mix well.
The problem is that the results from previous queries are stored in
XSB tables, and database updates do not modify XSB tables.

Thus, a user might get the following counterintuitive result:
\begin{quote}
\begin{verbatim}
flora2 ?- insert{o[m->v]}.

Yes.

flora2 ?- o[m->v].

Yes.

flora2 ?- delete{o[m->v]}, o[m->v].

Yes.
\end{verbatim}
\end{quote}
The reason for the last positive answer here is that XSB tables remember
that the query {\tt o[m->v]} is true. So, when the same query is asked
after {\tt delete}, a ``wrong'' result is returned from the tables. Similarly,
tabling might interact poorly with {\tt insert}:
\begin{quote}
\begin{verbatim}
flora2 ?- o[m->v].

No.

flora2 ?- insert{o[m->v]}, o[m->v].

No.
\end{verbatim}
\end{quote}
The reason for the last result here is, again, that XSB tables
remember that {\tt o[m->v]} is false. However, this is no longer
correct after the insertion operation.

\index{abolish\_all\_tables}
%%
In a future release, \FLORA will provide a workaround for these
problems (and it is even possible that a future release of XSB will
start doing the right thing in these situations). For now, the only
remedy is to use a call to {\tt abolish\_all\_tables}, which will
clear all tables.  However, at present, the only safe way to do this
is by executing {\tt abolish\_all\_tables} as a separate query.

Because of the way a tabled predicate is executed, it is a bad
programming practice to define a tabled predicate that depends on
database updates and/or procedural methods. For now, the \FLORA
compiler does not automatically check this kind of dependency. Such
a feature will be supported in a future release.


\section{Negation} \label{sec:negation}


\FLORA supports two kinds of negation: negation based on SLD resolution,
and negation based on well-founded semantics, which relies on the
underlying XSB tabling system.

For negation based on SLD resolution, it is specified using any of the
following operators: \verb|~|, \verb|\+|, and {\tt not}. Usually, these
operators are applied to procedural methods and database updates.

For negation based on well-founded semantics, it is specified using the
operator {\tt tnot}. Usually, {\tt tnot} is applied to predicates that
are tabled. Note that except procedural methods, all \fl methods are
tabled, thus {\tt tnot} can be applied to them.

All tabled predicates should use {\tt tnot} for negation. The
semantics is not clear when negation based on SLD resolution is
applied to a tabled predicate.

In \FLORA, {\tt tnot} can also be specified for a non-tabled
predicate, or a logic formula. The effect is such that the system will
first generate a new tabled predicate to hold the results from the
non-tabled predicate or the logic formula, then apply {\tt tnot} to
this new predicate.

In fact, {\tt tnot} in \FLORA is implemented using the XSB operator
\verb|sk_not|. Refer to the XSB manual for more information on
negation, well-founded semantics, and the XSB operators {\tt tnot} and
\verb|sk_not|.


\section{\FLORA Compiler Directives} \label{sec-comp-directives}

\index{compiler directive}
%
\index{compiler directive!{\tt auto\_table}}
\index{compiler directive!{\tt dynamic}}
\index{compiler directive!{\tt index}}
%
Like XSB compiler, \FLORA compiler can take compiler directives. All
such directives must begin with {\tt :-} (while all queries must begin with
{\tt ?-}). The following is a list of all the compiler directives supported
by \FLORA:
\begin{itemize}
\item \verb!auto_table!
\item \verb!dynamic functor/arity, ..., functor/arity!
\item \verb!export functor/arity, ..., functor/arity!
\item \verb!import functor/arity, ..., functor/arity from module!
\item \verb!index(functor/arity,trie)!
\item \verb!florapred functor/arity, ..., functor/arity!
\item \verb!florapredall!
\item \verb!op(precedence,type,operator)!
\item \verb!op(precedence,type,[operator, ..., operator])!
\item \verb!prolog functor/arity, ..., functor/arity!
\item \verb!prologall!
\item \verb!sign functor(oid|form, ..., oid|form)!, where \verb!oid|form! means
      either \verb|oid| or \verb!form!
\item \verb!table functor/arity, ..., functor/arity!
\end{itemize}


\section{\FLORA Debugger}
\index{debugging}


*********** Must be fixed when the debugger is adapted **********


\FLORA debugger is essentially a presentation layer on top of the XSB
debugger, so familiarity with the latter is highly recommended (XSB Manual,
Part I). Here we sketch only a few basics.

The debugger has two facilities: tracing and spying. Tracing allows the
user to watch the program being executed step by step, and spying allows
one to tell \FLORA that it must pose when execution reaches certain 
predicates or object methods. The user can trace the execution from then
on. At present, only the tracing facility has been implemented.

\index{tracing}
To start tracing, you must issue the command {\tt flora\_trace} at the
\FLORA prompt. It is also possible to put the subgoal {\tt flora\_trace} in
the middle of the program. In tat case, tracing will start after this
subgoal gets executed. This is useful when you know where exactly you want
to start tracing the program. To stop tracing, type {\tt flora\_notrace}.

During tracing, the user is normally prompted at the four ports of subgoal
execution: {\tt Call} (when a subgoal is first called), {\tt Exit} (when
the call exits), {\tt Redo} (when the subgoal is tried with a different
binding on backtracking), and {\tt Fail} (when a subgoal fails).
At each of the prompts, the user can issue a number of commands. The most
common ones are listed below. See the XSB manual for more.
%%
\begin{itemize}
  \item {\tt carriage return (creep)}:  to go to the next step  
  \item {\tt s (skip)}: execute this subgoal non-interactively; prompt
    again when the call exits (or fails)
  \item {\tt S (verbose skip)}: like {\tt s}, but also show the trace
    generated by this execution
  \item {\tt l (leap)}: stop tracing and execute the remainder of the
    program
\end{itemize}
%%
The behavior of the debugger is controled by the predicate {\tt
  debug\_ctl}. For instance, executing {\tt debug\_ctl(profile, on)} at the
\FLORA prompt tells XSB to measure the CPU time it takes to execute each
call. This is useful for tuning your program for performance. Other useful
controls are: {\tt debug\_ctl(prompt, off)}, which causes the trace to be
generated without user intervention; and {\tt debug\_ctl(redirect,
  foobar)}, which redirects debugger output to the file named {\tt foobar}.
The latter feature is usually useful only in conjunction with the
  aforesaid prompt-off mode. See the XSB manual for additional information
  on debugger control.


\section{Emacs Support}

Editing and debugging \FLORA programs can be greatly simplified with the
help of \emph{flora-mode}, a special Emacs editing mode designed
specifically for \FLORA programs. Flora-mode provides support for syntactic
highlighting, automatic indentation, and the ability to run \FLORA programs
right out of the Emacs buffer.


\subsection{Instalation}


To install \emph{flora-mode}, you must perform the following steps. Put the
file
%%
\begin{quote}
  {\tt XSB/packages/flora2/emacs/flora.el} 
\end{quote}
%%
found in your XSB distribution on the load path of Emacs or XEmacs
(whichever you are using). The best way to work with Emacs is to make a
separate directory for Emacs libraries (if you do not have one), and put
{\tt flora.el} there. Such a directory can be added to emacs search path by
putting the following command in the file \verb|~/.emacs| (or
\verb|~/.xemacs|, if you are running one of the newer versions of XEmacs):
%%
\begin{quote}
  \tt
   (setq load-path (cons "your-directory" load-path)) 
\end{quote}
%%
It is also a good idea to compile emacs libraries. To compile flora.el,
use this:
%%
\begin{quote}
  \tt
   emacs -batch -f batch-byte-compile flora.el 
\end{quote}
%%
If you are using XEmacs, use {\tt xemacs} instead of {\tt emacs} above ---
the two emacsen often use incompatible byte code.

Finally, you must tell X/Emacs how to recognize \FLORA program files, so
Emacs will be able to invoke the Flora major mode automatically when you
are editing such files:
%%
\begin{verbatim}
(setq auto-mode-alist (cons '("\\.flr$" . flora-mode) auto-mode-alist))
(autoload 'flora-mode "flora" "Major mode for editing Flora programs." t)
\end{verbatim}
%%$

To enable syntactic highlighting of Emacs buffers (not just for \FLORA
programs), you can do the following:
%%
\begin{itemize}
  \item  {\sf In Emacs:} select {\tt Help.Options.Global Font Lock} on
    the menubar.  To enable highlingting permanently, put 
    %%
    \begin{quote}
      \tt
      (global-font-lock-mode t)
    \end{quote}
    %%
    in \verb|~/.emacs|.
  \item {\sf In XEmacs:} select {\tt Options.Syntax
        Highlighting.Automatic} in the menubar. To enable this permanently, put
      %%
      \begin{quote}
        \tt
        (add-hook 'find-file-hooks 'turn-on-font-lock)
      \end{quote}
      %%
      in \verb|~/.emacs| or \verb|~/.xemacs| (whichever is used by your
      XEmacs).
\end{itemize}
%%


\subsection{Functionality}


\paragraph{Menubar menu.}
Once flora editing mode is installed, it provides a number of functions.
First, whenever you edit a \FLORA program, you will see the ``Flora'' menu
in the menubar. This menu provides commands for controlling the Flora
process (i.e., XSB with the \FLORA shell). You can start and stop
this process, type queries to it, and you can tell it to consult regions of
the buffer you are editing, the entire buffer, or some other file.

Because Emacs provides automatic file completion and allows you to edit
what you typed, performing these functions right out of the buffer takes
much less effort than typing the corresponding commands on XSB command
line.

\paragraph{Keyboard functions.}
In addition to the menu, \emph{flora-mode} lets you execute most of the
menu commands using the keyboard. Once you get the hang of it, keyboard
commands are much faster to invoke:
%%
\begin{verbatim}
Consult file:                 Ctl-c Ctl-f
Consult file dynamically:     Ctl-u Ctl-c Ctl-f
Consult buffer:               Ctl-c Ctl-b
Consult buffer dynamically:   Ctl-u Ctl-c Ctl-b
Consult region:               Ctl-c Ctl-r
Consult region dynamically:   Ctl-u Ctl-c Ctl-r
\end{verbatim}
%%
When you invoke any of the above commands, a \FLORA process is started,
unless it is already running. However, if you want to invoke this process
explicitly, type
%%
\begin{verbatim}
ESC x run-flora  
\end{verbatim}
%%
You can control the \FLORA process using the following commands:
%%
\begin{verbatim}
Interrupt Flora Process:      Ctl-c Ctl-c
Quit Flora Process:           Ctl-c Ctl-d
Restart Flora Process:        Ctl-c Ctl-s
\end{verbatim}
%%
Interrupting \FLORA is equivalent to typing {\tt Ctl-c} at the \FLORA
prompt, quitting the process stops XSB, and restarting the process shuts
down the old XSB process and starts a new one with \FLORA shell running.

\paragraph{Indentation.}
Flora editing mode understands some aspects of the \FLORA syntax, which
enables it to provide correct indentation of program lines (in many cases).
In the future, flora mode will know more about the syntax, which will let
it provide even better support for indentation.

The most common use of \FLORA indentation facility is by typing the {\tt
  TAB}-key. If \emph{flora-mode} manages to understand where the cursor is,
it will indent the line accordingly. Another way is to put the following in
your emacs startup file (\verb|~/.emacs| or \verb|~/.xemacs|):
%%
\begin{verbatim}
    (setq flora-electric t)  
\end{verbatim}
%%
In this case, whenever you type the return key, the next line will be
indented automatically.





\section{Future Enhancements}

\FLORA is work in progress. We are still experimenting with features and
nothing is cast in stone. So, although we do not intend to make the life of 
\FLORA users harder than it already is, we cannot give a guarantee of
backward compatibility. The following enhancements and features are among
those currently planned:

%%
\begin{description}
  \item[Aggregates:] \FLORA will provide built-in functions that will
    directly apply to the outcome of the aggregates {\tt collectbag} and
    {\tt collectset}. This will make it possible to do grouping once and
    then compute multiple aggregates over the groups.
  \item[Negation:] Currently \FLORA does not support negation of mixed
    F-molecules, i.e., molecules that contain tables and non-tabled
    things..
\end{description}
%%


\bibliography{../../../docs/userman/manual}

\printindex

\end{document}
