\section{XSB Caching for XJ}
	   
When XSB is used in conjunction with (XJ and) Interprolog, caching may
be useful for efficiency and consistency.  Information in the CDF is
maintained in XSB, so that Java calls XSB goals in order to create or
refresh displays.  When displays are refreshed, XSB goals may be
repeatedly reexecuted.  Caching helps such goals to be re-executed
efficiently, and also helps ensure consistency in the order tuples are
returned from goals.  The caching system also maintains information
that may be used by a graphical interface such as the number of
answers to a goal, along with a numbering of these answers.  Thus the
user can ask for a range of answers for a goal, a feature that is
useful, for instance in the lazy lists that form the basis of various
XJ components.  Finally, the caching system uses {\tt sorting templates}
\index{sorting templates}
to indicate whether answers are to be indexed in ascending or
descending sorted order, or simply in the order that they are derived.

%--------------------------
\subsection{Maintaining Correct Caches in the Presence of Updates}
\label{sec:cache-maint}

While caches are useful, it must be ensured that a cache for a goal
$G$ is invalidated when an update to the knowledge base may add or
delete answers to $G$.  For example, suppose we have a rule defining
{\tt p(X,Y) :- r(X,Z),s(Z,Y)}. and we are caching a call to {\tt p/2}.
If {\tt s/2} is modified in such a way that the cache for a {\tt p/2}
goal may be changed, we want to invalidate the cache for that goal.
One way to do this is for a user to invalidate caches herself, using
the predicate {\tt cache\_invalidate(Goal)} which invalidates all
caches that unify with {\tt Goal}.  Alternately, the user can
invalidate caches bases on a predicate {\tt
cache\_inv(MutableGoal,CachedGoal)} that for each form of a mutable
goal, indicates the forms of the cached goals that should be
invalidated.  So for the example above, if {\tt s/2} is mutable, we
will have a fact: {\tt cache\_inv(s(\_,Y),p(\_,Y))}. which indicates
that if, say, {\tt s(b,a)} is asserted, then all goals of the form
{\tt p(\_,a)} in the cache must be invalidated.

The predicate {\tt generate\_cache\_invalidators(File)} generates the
appropriate \pred{cache\_inv/2} facts.  The file it is given must
contain the rules that define the cached predicates. It must in
addition define 3 more predicates: {\tt cached/1} which provides the
skeletons of all the cached predicates; {\tt mutable/1} which provides
the skeletons of all predicates that can be modified; and {\tt
immutable/1} which contains the skeletons of all predicates that are
used in the definitions of the cached predicates but not defined in
this file.  It need not include standard XSB predicates.  Predicates
in {\tt immutable/1} are assumed to be universally true.  {\tt
generate\_cache\_invalidators(File)} dynamically loads the indicated
File and generates a file called \file{cache\_inv\_gen.P} that
contains the definition of {\tt cache\_inv/2}.  The algorithm used is
quite simple.  It can do a little optimization for simple selections,
but in more complex cases will just invalidate all cached goals for a
predicate.

In {\tt File}, the user may also place \pred{cache\_inval/2} facts.
These are user-defined invalidation facts of the same form as will be
generated for {\tt cache\_inv/2} in the external file.  These facts
are simply checked to be sure the first argument is a mutable
predicate term and the second is a cached predicate term.  The
algorithm used by {\tt generate\_cache\_invalidators/1} generates {\tt
cache\_inval} facts for all cached predicates that do not have facts
defined for them in {\tt cache\_inval/2}.  (That is, if the user gives
explicit invalidate facts, then none are automatically generated.)

\begin{description}
\ourpredmoditem{cache\_set\_auto\_commit/1}{prologCache}
{\tt cache\_set\_auto\_commit(?OnOrOff)} sets the auto\_commit state
for caching notification on or off.  If it is called with a variable,
it returns the current auto\_commit state.  If auto\_commit is {\tt
on}, then every update that invalidates a cache causes notifications
from XSB to Jave to be invoked (sent).  If auto\_commit is {\tt off},
then notifications are accumulated as pending but not sent.  They
later can be sent with an explicit call to
\pred{cache\_invoke\_pending\_notifications/0}.  If there are pending
notifications when {\tt cache\_set\_auto\_commit/1} is called, it
fails.

\ourpredmoditem{cache\_count/4}{prologCache}
{\tt cache\_count(+Goal,+Template,+NotifierGoal,-Count)} takes a goal
{\tt Goal}, a {\tt Template} indicating how answers to the goal are to
be sorted, and a notifier goal, and returns the number of (not
necessarily distinct) answers for {\tt Goal}.  If {\tt Goal} has been
cached, this information is found from the cache; if not, the
predicate creates a cache for {\tt Goal} and returns the count.  In
either case, the cache is sorted according to {\tt Template}.  If {\tt
Template} is a constant, the cache is unsorted; if the main functor
symbol of {\tt Template} is {\tt asc}, the tuples are sorted in
ascending order; otherwise they are sorted in descending order.  {\tt
NotifierGoal} is a goal that will be called when the cache for {\tt
Goal} becomes invalid due to underlying updates.

\ourpredmoditem{cache\_range/5}{prologCache}
{\tt cache\_range(\#Goal,+SortTemp,+Notifier,+From,+To)} takes a goal
{\tt Goal}, a template for sorting, a notifier goal, and a pair of
integers {\tt From} and {\tt To}.  Upon success, this predicate
unifies {\tt Goal} with the element of the cache indexed by {\tt
From}; upon backtracking {\tt Goal} is unified with successive
elements of the cache up to the element indexed by {\tt To}.  For
example, a call to {\tt cache\_range(p(X,Y),asc(\_),cacheWarn,25,50)}
eventually returns different bindings for {\tt X} and {\tt Y}, from
the call to {\tt p(X,Y)} (assuming it includes at least 50 tuples.)
If {\tt Goal} is not cached, {\tt cache\_range/3} creates a cache for
{\tt Goal} and then returns bindings in the requested range; otherwise
the tuples are retrieved from the cache and returned.  {\tt Template}
and {\tt Notifier} are as described in {\tt cache\_count/4}.

% TLS: see skeleton in XSB manual.
\ourpredmoditem{cache\_index/4}{prologCache}
{\tt cache\_index(+Goal,+Sort,\#Skel,-Index)} is called with a cached
goal, {\tt Goal},a sorting template (as elsewhere), and a (normally
ground) skeleton, {\tt Skel}.  {\tt cache\_index} will return the
successive indices of the {\tt Skel}, binding variables if {\tt Skel}
is not ground.

% TLS: some questions about notification.

\ourpredmoditem{nocache\_but\_notify/2}{prologCache}
{\tt nocache\_but\_notify(+Goal,+Notifier)} saves {\tt Notifier} to call when
this goal is invalidated.  It then calls {\tt Goal}.  This is for use
instead of {\tt cache\_range/5}, when notification on invalidation is
required, but caching is not desired.

\ourpredmoditem{just\_notify/2}{prologCache}
{\tt just\_notify(+Goal,+Notifier)} saves Notifier to call when this
goal is invalidated.  It does NOT call the {\tt Goal}.


\ourpredmoditem{cache\_remove\_notifier/4}{prologCache}
{\tt cache\_remove\_notifier(+Goal,+Template,+Notifier,+IfCache)} is
called with a {\tt Goal}, which must be a cached (or nocached) goal, a
sorting template (as elsewhere), a notifier goal, and an indication of
whether {\tt Goal} is cached ({\tt cache}) or not cached ({\tt
nocache}). {\tt Notifier} is assumed to be assoiciated with {\tt
Goal}, and {\tt cache\_remove\_notifier/4} removes that notifier.  It
could delete the cache if there are no other notifiers associated with
it, but at this time, this is not done.

\ourpredmoditem{cache\_invalidate/1}{prologCache}
 {\tt cache\_invalidate(+Goal)} removes from the cache all goals (and
 their answers) that unify with Goal. If auto\_commit is set on, then
 it invokes the notifiers of the invalidated caches.  If auto\_commit
 is set off, then the notifiers are saved as pending.

\ourpredmoditem{generate\_cache\_invalidators/1}{prologCache}
{\tt generate\_cache\_invalidators(+File)} takes as input a file, and
generates cache invalidators as described in
Section~\ref{sec:cache-maint}.

\ourpredmoditem{cache\_invalidate\_for/1}{prologCache}
{\tt cache\_invalidate\_for(UpdGoal)} assumes {\tt cache\_inv/2} is
defined, and uses it to invalidate all caches that could be changed by
an assert or retract of (any instance of) {\tt UpdGoal}.

\ourpredmoditem{cache\_print\_goals/0}{prologCache}
{\tt cache\_print\_goals} prints out all goals that are currently in the
cache and the count of tuples in each.

\mycomment{
\ourpredmoditem{cache\_print/0}{prologCache}
{\tt cache\_print} prints out all goals that are currently in the cache
and all their tuples.  (This is mostly a debugging predicate, since
normally one doesn't want to print so much data.
}

\end{description}