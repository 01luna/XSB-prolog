\section{Leader Determination (D.S.Warren 1 July 2019)} \label{sec:leader}

This section describes the algorithm now used (as of 7/1/2019) to
determine leaders.  XSB maintains a ``completion stack'' which
contains a node for each active tabled call.  The completion stack
corresponds to a subset of the choice point stack.  To determine when
a call is completely evaluated, the system keeps track of ``back
dependency'' information, i.e., when a tabled call depends on (i.e.,
calls) another already existing tabled call that is deeper in the
stack than itself.  This means that the current tabled call cannot be
completely evaluated until (at least) that deeper tabled call is
completely evaluated.  A leader is a tabled call that does not depend
on any tabled call that is deeper than it, and for which no shallower
tabled call depends on a tabled call deeper than it.  This means that,
when all answers have been returned to all calls at this level or
shallower, all tables at the leader and shallower on the stack are
completely evaluated and can be marked as completed.  (It gets more
complicated when delay and simplification are included, but the
concept of a leader is the same and still required.)

The previous algorithm for determining leaders was simple and
straightforward, but led to quadratic behavior.  We describe it here.
That algorithm is implemented in XSB by the adjust\_level macro,
defined in emu/tab\_structs.h.  That algorithm numbers each completion
frame, and in each completion frame keeps a variable, called
level\_num, containing the number of the completion frame that is its
current leader, i.e., the deepest frame it, or a shallower frame,
depends on.  The value is initialized to its own frame number.  The
value is maintained (by invocation of adjust\_level) each time a call
is made to a table whose frame is deeper in the stack.  At this time
the level\_num of every completion frame between the current call and
deeper table is updated to be the level\_num of the completion frame
of the deeper table.  A leader is then a table whose completion frame
has its own number as level\_num.  I.e., it and no descendant depends
on any earlier (deeper) table.  This algorithm is simple and clearly
correct.  However, it may require updating the level\_num field of
linearly many completion frames at each call to a deeper table.  It is
straightforward to construct a sequence of propositional programs for
which this algorithm results in quadratic complexity.

Consider a propositional program with N proposition symbols, $p_i$,
with rules of the form:
\begin{enumerate}
\item $p_i~:-~p_{i+1}$. for $1 \leq i < N$, and
\item $p_i~:-~p_{i-N/2}$. for $N/2 < i < N$.
\end{enumerate}
(We note, as an aside, that all propositions are false in the least
model of this program.)  A call to $p_1$ will call $p_2$, etc. up to
$p_{N/2}$. From there on, there is a call to the next (new)
proposition, and one to an earlier proposition with an existing table.
Each of these calls will require setting the level\_num fields of all
calls between them.  This will require O($N^2$) operations in total.

So we seek an algorithm for leader determination that takes
significantly less time.  In each completion frame we keep a field
that contains a pointer to another completion frame:
\begin{enumerate}
\item If the completion frame is not (currently) a leader, then this
  points to a chain of completion frames that ends at the leader.
\item If the completion frame is (currently) a leader, then this
  pointer is tagged (to indicate current leader) and it points to the
  next shallower completion frame on the stack that is a (current)
  leader.
\end{enumerate}
So at each point we can determine a current leader by checking that
its own pointer is tagged as a leader, and we can find a frame's
leader by running its chain to its leader.  Each new completion frame
has its pointer initialized to a leader (i.e. tagged) and added to the
end of the (deep-to-shallow) leader chain.  When the adjust\_level
macro is called (which now calls adjust\_level\_ptrs, defined in
emu/tables.c), the leader of the deeper called table is found, and the
leader chain is traversed to the leader of the shallower calling
table; each intermediate leader pointer on this chain is changed to be
an untagged pointer to the deeper called table leader (the beginning
of the leader chain), and that deeper leader is made the end-of-leader
chain.  So this has updated the structure so that all the frames
between the shallow calling table and the deep called table will now
follow their chains to their new leader.

This algorithm, as it stands, may also exhibit quadratic behavior; a
chain to a leader may be O(N) long and need to be traversed O(N)
times.  However, if each traversal of such a chain to a leader
collapses the chain by pointing every node along the chain to the
final leader, we can avoid the multiple traversals and the quadratic
time.  We can see that this algorithm is a version of the union-find
algorithm, with path compression but without ranking.  The union
operation is K-way when doing adjust\_level\_ptrs for a deep pointer
and there are K leader nodes on the chain from the deeper node to the
top of the stack.  These K sets are unioned to become children of the
deepest node, regardless of their sizes.  Since we ignore the sizes of
the sets being unioned, this is the ``without ranking'' union
operation.  It is known that this algorithm may add a log factor to
the complexity, so this algorithm is at worst O(N*log N).  But it
seems to be only for programs with quite complex dependency structures
that this algorithm is actually worse than linear.

We could, relatively easily, change this algorithm to add ranking, by
adding another field to the completion frame, which would point to its
leader, and then when doing the union, find the largest set (easy to
find by the difference of this and the next node's addresses) and
point all the others to this largest one, which we can call the
``representative'' of the set.  And then we set the representative's
new leader pointer to point to the actual leader.  Finding the leader
of a completion frame would run its chain to the representative, and
then follow its new leader pointer to the actual leader.  But I don't
think the gain (in the worst case, changing the log factor to an
``inverse Ackermann's'' factor) for most programs would be worth the
overhead (minimal though it might be).  (Of course, I could be wrong
on this.)

