\section{Unification and Comparison of Terms} \label{Comparison}
\index{terms!comparison of}
\index{terms!unification of}

The predicates described in this section allow unification and
comparison of terms~\footnote{Arithmetic comparison predicates that
may evaluate terms before comparing them are described in
Section~\ref{sec:arith-eval}.}.

\index{occurs check}
\index{terms!cyclic}
\index{Prolog flags!{\tt unify\_with\_occurs\_check}}
Like most Prologs, default unification in XSB does not perform a
so-called {\em occurs check} --- it does not handle situations where a
variable {\tt X} may be bound to a structure containing {\tt X} as a
proper subterm.  For instance, in the goal

{\tt X = f(X)}    {\em \% incorrect!}

{\tt X} is bound to {\tt f(X)} creating a term that is either cyclic
or infinite, depending on one's point of view.  Prologs in general
perform unification without occurs check since without occurs check
unification is linear in the size of the largest term to be unified,
while unification with occurs check may be exponential in the size of
the largest term to be unified.  Most Prolog programmers will rarely,
need to concern themselves with cyclic terms or unification with
occurs check.  However, unification with occurs check can be important
for certain applications, in particular when Prolog is used to
implement theorem provers or sophisticated constraint handlers.  As a
result XSB provides an ISO-style implementation of the predicate {\tt
  unify\_with\_occurs\_check/2} described below, as well as a Prolog
flag {\tt unify\_with\_occurs\_check} that changes the behavior of
unification in XSB's engine.

As opposed to unification predicates, term comparison predicates
described below take into account a standard total ordering of terms,
which has as follows:
%
\[ 
variables \ {\tt @<} \ floating \ point \ numbers \ {\tt @<} \ integers \ {\tt @<} \ atoms \ {\tt @<} \ compound \ terms 
\] 
%
Within each one of the categories, the ordering is as follows:
\begin{itemize}
\item ordering of variables is based on their address within the
  SLG-WAM --- the order is {\em not\/} related to the names of
  variables.  Thus note that two variables are identical only if they
  share the same address -- only if they have been unified or are the
  same variable to begin with.  As a corollary, note that two
  anonymous variables will not have the same address and so will not
  be considered identical terms.  As with most WAM-based Prologs, the
  order of variables may change as variables become bound to one
  another.  If the order is expected to be invariant across variable
  bindings, other mechanisms, such as attributed variables, should be
  used.
\item floating point numbers and integers are put in numeric order,
  from $-\infty$ to $+\infty$.  Note that a floating point number is
  always less than an integer, regardless of their numerical values.
  If comparison is needed, a conversion should be performed
  (e.g. through {\tt float/1}).
\item	atoms are put in alphabetical (i.e.\ ASCII) order;
\item	compound terms are ordered first by arity, then by the name of their
	principal functor and then by their arguments (in a left-to-right 
	order).
\item	lists are compared as ordinary compound terms having arity 2 and 
	functor {\tt '.'}.
\end{itemize}

For example, here is a list of terms sorted in increasing standard order:
\begin{center}
	{\tt [ X, 3.14, -9, fie, foe, fum(X), [X], X = Y, fie(0,2), fie(1,1) ]}
\end{center}
The basic predicates for unification and comparison of arbitrary terms are:
\begin{description}

\standarditem{X = Y}{=/2}
    Unifies {\tt X} and {\tt Y} without occur check.

\standarditem{unify\_with\_occurs\_check(One,Two)}{unify\_with\_occurs\_check/2}
%
Unifies {\tt One} and {\tt Two} using an occur check, and failing if
{\tt One} is a proper subterm of {\tt Two} or if {\tt Two} is a proper
subterm of {\tt One}.  

{\bf Example:}
    {\footnotesize
     \begin{verbatim}
                | ?- unify_with_occurs_check(f(1,X),f(1,a(X))).

                no
                | ?- unify_with_occurs_check(f(1,X),f(1,a(Y))).

                X = a(_h165)
                Y = _h165

                yes
                | ?- unify_with_occurs_check(f(1,a(X)),f(1,a(X))).

                X = _h165

                yes
  \end{verbatim}}

\standarditem{T1 == T2}{==/2}
%
    Tests if the terms currently instantiating {\tt T1} and {\tt T2}
    are literally identical (in particular, variables in equivalent positions
    in the two terms must be identical).
    For example, the goal:

    \demo{$|$ ?- X == Y.}

    \noindent
    fails (answers no) because {\tt X} and {\tt Y} are distinct variables.
    However, the question

    \demo{$|$ ?- X = Y, X == Y.}

    \noindent
    succeeds because the first goal unifies the two variables.

\ournewitem{X $\backslash$\,= Y}{ISO}\index{$\backslash$\texttt{=/2}}
\index[pred]{$\backslash$\texttt{=/2}}
    Succeeds if~{\tt X} and~{\tt Y} are not unifiable,
    fails if~{\tt X} and~{\tt Y} are unifiable.
    It is thus equivalent to {\tt $\backslash$+}\/{\tt (X = Y)}.

\ournewitem{T1 {$\backslash$==} T2}{ISO} \index{$\backslash$\texttt{==/2}} 
\index[pred]{$\backslash$\texttt{==/2}}
    Succeeds if the terms currently instantiating {\tt T1} and {\tt T2}
    are not literally identical.

\standarditem{Term1 ?= Term2}{?=/2}
%
Succeeds if the equality of {\tt Term1} and {\tt Term2} can be
compared safely, i.e. whether the result of {\tt Term1 = Term2} can
change due to further instantiation of either term. It is defined as
by {\tt ?=(A,B) :- (A==B ; A \= B), !.} 

\ourmoditem{unifiable(X, Y, -Unifier)}{unifiable/3}{constraintLib}
%
If {\tt X} and {\tt Y} can unify, succeeds unifying {\tt Unifier} with
a list of terms of the form {\tt Var = Value} representing a most
general unifier of {\tt X} and {\tt Y}.\  {\tt unifiable/3} can handle
cyclic terms. Attributed variables are handles as normal
variables. Associated hooks are not executed~\footnote{In \version ,
  {\tt unifiable/3} is written as a Prolog predicate and so is slower
  than many of the predicates in this section.}.

\standarditem{T1 @$<$ T2}{"@$<$/2}
    Succeeds if term {\tt T1} is before term {\tt T2} in the standard order.

\standarditem{T1 @$>$ T2}{"@$>$/2}
    Succeeds if term {\tt T1} is after term {\tt T2} in the standard order.

\standarditem{T1 @$=<$ T2}{"@$=</2$}
    Succeeds if term {\tt T1} is not after term {\tt T2} in the standard order.

\standarditem{T1 @$>=$ T2}{"@$>=/2$}
    Succeeds if term {\tt T1} is not before term {\tt T2} in the standard order.

\standarditem{T1 @$=$ T2}{"@$=/2$}
    Succeeds if {\tt T1} and {\tt T2} are identical variables, or if
    the main structure symbols of {\tt T1} and {\tt T2} are identical.

\standarditem{compare(?Op, +T1, +T2)}{compare/3}
    Succeeds if the result of comparing terms {\tt T1} and {\tt T2} 
    is {\tt Op}, where the possible values for {\tt Op} are:
    \begin{description}
    \item[`='] if {\tt T1} is identical to {\tt T2},
    \item[`$<$'] if {\tt T1} is before {\tt T2} in the standard order,
    \item[`$>$'] if {\tt T1} is after {\tt T2} in the standard order.
    \end{description}
    Thus {\tt compare(=, T1, T2)} is equivalent to {\tt T1==T2}.
    Predicate {\tt compare/3} has no associated error conditions.

\index{terms!cyclic}
\standarditem{ground(+X)}{ground/1}
%
Succeeds if {\tt X} is currently instantiated to a term that is
completely bound (has no uninstantiated variables in it); otherwise it
fails.  While {\tt ground/1} has no associated error conditions, it is
not safe for cyclic terms: if cyclic terms may be an issue use {\tt
  ground\_or\_cyclic/1}.

\standarditem{ground\_or\_cyclic(+X)}{ground\_or\_cyclic/1} 
%
Succeeds if {\tt X} is currently instantiated to a term that is
completely bound (has no uninstantiated variables in it) {\em or} is a
cyclic term; otherwise it fails.  The predicate has no associated
error conditions.

{\tt ground\_or\_cyclic/1} is written to be as efficient as possible,
and requres a single traversal of a term, regardless of whether the
term is ground, nonground or cyclic.  However, due to the nature of
checking for cyclicity, {\tt ground\_or\_cyclic/1} is slower than the
unsafe {\tt ground/1}.

\ourmoditem{subsumes(?Term1, +Term2)}{subsumes/2}{subsumes}
%\predindex{subsumes/2~(L)}
    Term subsumption is a sort of one-way unification.  Term {\tt Term1}
    and {\tt Term2} unify if they have a common instance, and unification
    in Prolog instantiates both terms to that (most general) common instance.
    {\tt Term1} subsumes {\tt Term2} if {\tt Term2} is already an instance of
    {\tt Term1}.  For our purposes, {\tt Term2} is an instance of {\tt Term1}
    if there is a substitution that leaves {\tt Term2} unchanged and makes
    {\tt Term1} identical to {\tt Term2}.  Predicate {\tt subsumes/2} does
    not work as described if {\tt Term1} and {\tt Term2} share common
    variables.

\ourrepeatmoditem{subsumes\_chk(+Term1, +Term2)}{subsumes\_chk/2}{subsumes}%
\isoitem{subsumes\_term(+Term1, +Term2)}{subsumes\_term/2}
%
    The {\tt subsumes\_chk/2} predicate is true when {\tt Term1} subsumes 
    {\tt Term2}; that is, when {\tt Term2} is already an instance of
    {\tt Term1}.  This predicate simply checks for subsumption and 
    does not bind any variables either in {\tt Term1} or in {\tt Term2}.
    {\tt Term1} and {\tt Term2} should not share any variables.

    Examples:
    {\footnotesize
     \begin{verbatim}
            | ?- subsumes_chk(a(X,f,Y,X),a(U,V,b,S)).

            no
            | ?- subsumes_chk(a(X,Y,X),a(b,b,b)).

            X = _595884
            Y = _595624
     \end{verbatim}}

\ourmoditem{variant(?Term1, ?Term2)}{variant/2}{subsumes}
%\predindex{variant/2~(L)}
    This predicate is true when {\tt Term1} and {\tt Term2} are 
    alphabetic variants.  That is, you could imagine that {\tt variant/2}
    as being defined like:
    \begin{center}
    \begin{minipage}{3.5in}
    \begin{verbatim}
        variant(Term1, Term2) :-
             subsumes_chk(Term1, Term2),
             subsumes_chk(Term2, Term1).
    \end{verbatim}
    \end{minipage}
    \end{center}
    but the actual implementation of {\tt variant/2} is considerably more
    efficient.  However, in general, it does not work for terms that share
    variables; an assumption that holds for most (reasonable) uses of
    {\tt variant/2}.

\end{description}

\subsection{Sorting of Terms}

Sorting routines compare and order terms without instantiating them.
Users should be careful when comparing the value of uninstantiated
variables.  The actual order of uninstantiated variables may change in
the course of program evaluation due to variable aliasing, garbage
collection, or other reasons.

\begin{description}
\standarditem{sort(+L1, ?L2)}{sort/2}
    The elements of the list {\tt L1} are sorted into the standard order,
    and any identical (i.e.\ `==') elements are merged, yielding the 
    list~{\tt L2}.  The time to perform the sorting is $O(n log n)$ where 
    $n$ is the length of list {\tt L1}.  

    Examples:
    {\footnotesize
     \begin{verbatim}
                | ?- sort([3.14,X,a(X),a,2,a,X,a], L).

                L = [X,3.14,2,a,a(X)];

                no
     \end{verbatim}}
    Exceptions:
    \begin{description}
    \item[{\tt instantiation\_error}]
	Argument 1 of {\tt sort/2} is a variable or is not a proper list.
    \end{description}

\standarditem{keysort(+L1, ?L2)}{keysort/2}
    The list {\tt L1} must consist of elements of the form \verb'Key-Value'.
    These elements are sorted into order according to the value of {\tt Key},
    yielding the list~{\tt L2}.  The elements of list {\tt L1} are scanned
    from left to right.  Unlike {\tt sort/2}, in {\tt keysort/2} no
    merging of multiple occurring elements takes place.  The time to perform
    the sorting is $O(n log n)$ where $n$ is the length of list {\tt L1}.  
    Note that the elements of {\tt L1} are sorted only according to the
    value of {\tt Key}, not according to the value of {\tt Value}.  The 
    sorting of elements in {\tt L1} is not guaranteed to be stable.

    Examples:
    {\footnotesize
     \begin{verbatim}
                | ?- keysort([3-a,1-b,2-c,1-a,3-a], L).

                L = [1-b,1-a,2-c,3-a,3-a];

                no \end{verbatim}}
     Exceptions: 
\begin{description} 
\item[{\tt instantiation\_error}]
     {\tt L1} {\tt keysort/2} is a variable or is not a proper
     list.  
\item[{\tt domain\_error(key\_value\_pair,Element)}] {\tt
     L1} contains an element {\tt Element} that is not of the
     form \verb'Key-Value'.  
\end{description}

\ourmoditem{parsort(+L1, +SortSpec, +ElimDupl, ?L2)}{parsort/4}{machine}
%
    {\tt parsort/4} is a very general sorting routine.  The list {\tt
    L1} may consist of elements of any form.  {\tt SortSpec} is the
    atom {\tt asc}, the atom {\tt desc}, or a list of terms of the
    form {\tt asc(I)} or {\tt desc(I)} where {\tt I} is an integer
    indicating a sort argument position.  The elements of list {\tt
    L1} are sorted into order according to the sort specification.
    {\tt asc} indicates ascending order based on the entire term; {\tt
    desc} indicates descending order.  For a sort specification that
    is a list, the individual elements indicate subfields of the
    source terms on which to sort.  For example, a specification of
    {\tt [asc(1)]} sorts the list in ascending order on the first
    subfields of the terms in the list.  {\tt [desc(1),asc(2)]} sorts
    into descending order on the first subfield and within equal first
    subfields into ascending order on the second subfield.  The order
    is determined by the standard predicate {\tt compare}.  If {\tt
    ElimDupl} is nonzero, merging of multiple occurring elements takes
    place (i.e., duplicate (whole) terms are eliminated in the
    output).  If {\tt ElimDupl} is zero, then no merging takes place.
    A {\tt SortSpec} of {\tt []} is equivalent to ``asc''.  The time
    to perform the sorting is $O(n log n)$ where $n$ is the length of
    list {\tt L1}.  The sorting of elements in {\tt L1} is not
    guaranteed to be stable. {\tt parsort/4} must be imported from
    module {\tt machine}.

    Examples:
    {\footnotesize
     \begin{verbatim}
            | ?- parsort([f(3,1),f(3,2),f(2,1),f(2,2),f(1,3),f(1,4),f(3,1)],
                 [asc(1),desc(2)],1,L). 

            L = [f(1,4),f(1,3),f(2,2),f(2,1),f(3,2),f(3,1)];

            no \end{verbatim}}

{\bf Error Cases:}
\begin{description} 
\item[{\tt instantiation\_error}]
     {\tt L1} is a variable or not a proper list.  
% TLS: not sure how to make sense of this...
%\item[{\tt type\_error}] The elements of {\tt L1} are not terms with at
%     least as many arguments as required by {\tt SortSpec}, or {\tt
%     SortSpec} is not of an allowed form.  
\end{description}
\end{description}

