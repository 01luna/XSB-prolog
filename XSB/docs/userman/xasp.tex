\chapter{XASP: Answer Set Programming with XSB and Smodels}
%===================================
\label{xasp}

\begin{center}
{\Large {\bf By Luis Castro, Terrance Swift, David
    S. Warren}~\footnote{ Thanks to Barry Evans for helping
    resuscitate the XASP installation procedure.}}
\end{center}

The term {\em Answer Set Programming (ASP)} describes a paradigm in
which logic programs are interpreted using the (extended) stable model
semantics.  While the stable model semantics is quite elegant, it has
radical differences from traditional program semantics based on
Prolog.  First, stable model semantics applies only to ground
programs; second stable model semantics is not goal-oriented --
determining whether a stable model is true in a program involves
examing each clause in a program, regardless of whether the goal would
depends on the clause in a traditional evaluation.

Despite (or perhaps because of) these differences, ASP has proven to
be a useful paradigm for solving a variety of combinatorial programs.
Indeed, determining a stable model for a logic program can be seen as
an extension of the NP-complete problem of propositional
satisfiability, so that satisfiability problems that can be naturally
represented as logic programs can be solved using ASP.  

The current generation of ASP systems are very efficient for
determining whether a program has a stable model (analogous to whether
the program, taken as a set of propositional axioms, is satisfiable).
However, ASP systems have somewhat primitive file-based interfaces.
XSB is a natural complement to ASP systems.  Its basis in Prolog
provides a procedural couterpart for ASP, as described in Chapter 5 of
Volume 1 of this manual; and XSB's computation of the Well-founded
semantics has a well-defined relationship to stable model semantics.
Furthermore, deductive-database-like capabilities of XSB allow it to
be an efficient and flexible grounder for many ASP problems.

The XASP package provides various mechanisms that allow tight linkage
of XSB programs to the SModels \cite{smodels:engine} stable model
generator.  The main interface is based on a store of clauses that can
be incrementally asserted or deleted by an XSB program.  Clauses in
this store can make use of all of the cardinality and weight
constraint syntax supported by SModels, in addition to default
negation.  When the user decides that the clauses in a store are a
complete representation of a program whose stable model should be
generated, the clauses are copied into SModels buffers.  Using the
Smodels API, the generator is invoked, and information about any
stable models generated are returned.  This use of XASP is roughly
analagous to building up a constraint store in CLP, and periodically
evaluating that store, but integration with the store is less
transparent in XASP than in CLP.  In XASP, clauses must be explicitly
added to a store and evaluated; furthermore clauses are not removed
from the store upon backtracking, unlike constraints in CLP.

The XNMR interpreter provides a second, somewhat more implicit use of
XASP.  In the XNMR interface a query $Q$ is evaluated as is any other
query in XSB.  However, conditional answers produced for $Q$ and for
its subgoals, upon user request, can be considered as clauses and sent
to SModels for evaluation.  In backtracking through answers for $Q$,
the user backtracks not only through answer substitutions for
variables of $Q$, but also through the stable models produced for the
various bindings. 

\section{Installing the Interface}

Installing the Smodels interface of XASP sometimes can be tricky, for
two reasons.  First, XSB must dynamically load the Smodels library,
and dynamic loading introduces platform dependencies.  Second since
Smodels is written in C++ and XSB is written in C, the load must
ensure that names are properly resolved and that C++ libraries are
loaded, steps that may addressed differently by different
compilers~\footnote{XSB's compiler can automatically call foreign
  compilers to compile modules written in C, but in \version{} of XSB
  C++ modules must be compiled with external commands, such as the
  {\tt make} command shown below.}.

In order to use the Smodels interface, several steps must be
performed.  {\em Note: this interface has not yet been ported to Windows
Cygwin.}

\begin{enumerate}
\item {\em Creating a library for Smodels.} Smodels itself must be
  compiled as a library and installed in the proper directory.  This
  can be done in the {\tt \$SMODELS} directory (NOT the XASP
  directory), using the Smodels makefile, possibly after modifing it.
  On some platforms the command {\tt make libinstall} may work
  properly, making and installing the Smodels library (by default
  installed into {\tt /usr/local/lib}).  On other platforms, {\tt make
    lib} may work properly, but the installation may need to be done
  by hand.

  If the compilation step ran successfully, there should be a file
  {\tt libsmodels.so} (or {\tt libsomodels.dylib} on Mac OS X or {\tt
    libsmodels.dll} on Windows...) in the appropriate directory.

  The Smodels makefile included in version 2.31 can be problematic on
  Mac OS X.  The directory {\tt \$XSBDIR/packages/xasp/makefiles} has
  makefiles that work for Mac OS X 10.3 and 10.4.

\item {\em Making the Smodels include files visible to XASP.}  This
  step may not be necessary on all systems, but it is simple to do by
  executing the command
%
\begin{verbatim}
sh makelinks.sh <path-to-smodels>
\end{verbatim}
%
which copies them to the XASP directory.

\item {\em Compiling the XASP files} Compilation of the XASP files
  depends on the {\tt smoMakefile} that is generated by XSB's
  configuration procedure.  Thus, reconfigure XSB with the option
\begin{verbatim}
  --with-smodels=<path-to-smodels> 
\end{verbatim}
  along with any other desired options (Note: the Smodels interface
  has not been tested with the multi-threaded engine, and Smodels
  itself is not thread-safe).  The main reason for this configuration
  is to make the {\tt smoMakefile}, so XSB does not need to be
  recompiled -- although recompilation wont hurt.

  Once the {\tt smoMakefile} is obtained, it should be moved into the
  {\tt xasp} directory from \\ {\tt \$XSBDIR/config/<configuration+tag>}.
  Then type the following two commands:
%
\begin{verbatim}
  make -f smoMakefile module

  make -f smoMakefile all
\end{verbatim}
%
\item {\em Checking the Installation} 
%
To see if the installation is working properly, cd to the subdirectory
{\tt tests} and type: 

{\tt sh testsuite.sh <\$XSBDIR>}

If the test suite succeeded it will print out a message along the lines of 

\begin{small}
{\tt PASSED testsuite for /Users/terranceswift/XSBNEW/XSB/config/powerpc-apple-darwi\
n7.5.1/bin/xsb}
\end{small}

\end{enumerate}


\section{The Smodels Interface}

The Smodels interface contains two levels: the \emph{cooked} level and
the \emph{raw} level.  The cooked level interns rules in an XSB
\emph{clause store}, and translates general weight constraint rules
\cite{SiNS02} into a \emph{normal form} that the Smodels engine can
evaluate.  When the programmer has determined that enough clauses have
been added to the store to form a semantically complete sub-program,
the program is \emph{committed}.  This means that information in the
clauses is copied to Smodels and interned using Smodels data
structures so that stable models of the clauses can be computed and
examined.  By convention, the cooked interface ensures that the atom
{\tt true} is present in all stable models, and the atom {\tt false}
is false in all stable models.  The raw level models closely the
Smodels API, and demands, among other things, that each atom in a
stable sub-program has been translated into a unique integer.  The raw
level also does not provide translation of arbitrary weight constraint
rules into the normal form requried by the Smodels engine.  As a
result, the raw level is significantly more difficult to directly use
than the cooked level.  While we make public the APIs for both the raw
and cooked level, we provide support only for users of the cooked
interface.

As mentioned above Smodels extends normal programs to allow weight
constraints, which can be useful for combinatorial problems.  However,
the syntax used by Smodels for weight constraints does not follow ISO
Prolog syntax so that the XSB syntax for weight constraints differs in
some respects from that of Smodels.  Our syntax is defined as follows,
where \emph{A} is a Prolog atom, \emph{N} a non-negative integer, and
\emph{I} an arbitrary integer.

\begin{itemize}

\item {\em GeneralLiteral ::= WeightConstraint $|$ Literal }

\item {\em WeightConstraint ::= weightConst(Bound,WeightList,Bound) }

\item {\em WeightList ::= List of WeightLiterals }

\item {\em WeightLiteral ::= Literal $|$ weight(Literal,N) }

\item {\em Literal ::= A $|$ not(A) }

\item {\em Bound ::== I $|$ {\tt undef} }

\end{itemize}

Thus an example of a weight constraint might be: 
\begin{itemize}
\item {\tt weightConst(1,[weight(a,1),weight(not(b),1)],2)}
\end{itemize}
We note that if a user does not wish to put an upper or lower bound on
a weight constraint, she may simply set the bound to {\tt undef} or to
an integer less than {\tt 0}.  
 
The intuitive semantics of a weight constraint
{\tt weightConst(Lower,WeightList,Upper)}, in which {\tt List} is is
list of \emph{WeightLiterals} that it is true in a model \emph{M} whenever
the sum of the weights of the literals in the constraint that are true
in \emph{M} is between the lower {\tt Lower} and {\tt Upper}.  Any literal
in a \emph{WeightList} that does not have a weight explicitly attached
to it is taken to have a weight of \emph{1}.

In a typical session, a user will initialize the Smodels interface,
add rules to the clause store until it contains a semantically
meaningful sub-problem.  He can then specify a compute statement if
needed, commit the rules, and compute and examine stable models via
backtracking.  If desired, the user can then re-initialize the
interface, and add rules to or retract rules from the clause store
until another semantically meaningful sub-program is defined; and then
commit, compute and examine another stable model \footnote{Currently,
only normal rules can be retracted.}.

The process of adding information to a store and periodically
evaluating it is vaguely reminiscent of the Constraint Logic
Programming (CLP) paradigm, but there are important differences.  In
CLP, constraints are part of the object language of a Prolog program:
constraints are added to or projected out of a constraint store upon
forward execution, removed upon backwards execution, and iteratively
checked.  When using this interface, on the other hand, an XSB program
essentially acts as a compiler for the clause store, which is treated
as a target language.  Clauses must be explicitly added or removed
from the store, and stable model computation cannot occur
incrementally -- it must wait until all clauses have been added to the
store.  We note in passing that the {\tt xnmr} module provides an
elegant but specialized alternative.  {\tt xnmr} integrates stable
models into the object language of XSB, by computing ""relevant""
stable models from the the residual answers produced by query
evaluation.  It does not however, support the weighted constriant
rules, compute statements and so on that this module supports.

Neither the raw nor the cooked interface currently supports explicit
negation.

Examples of use of the various interfaces can be found in the
subdirectory {\tt intf\_examples}

\begin{description}
\ouritem{smcInit}
\index{\texttt{smcInit/0}} 
%
Initializes the XSB clause store and the Smodels API.  This predicate
must be executed before building up a clause store for the first time.
The corresponding raw predicate, {\tt smrInit(Num)}, initializes the
Smodels API assuming that it will require at most {\tt Num} atoms.

\ouritem{smcReInit}
\index{\texttt{smcReInit}} 
%
Reinitializes the Smodels API, but does \emph{not} affect the XSB
clause store.  This predicate is provided so that a user can reuse
rules in a clause store in the context of more than one sub-program.

\ouritem{smcAddRule(+Head,+Body)}
\index{\texttt{smcAddRule/2}} 
%
Interns a ground rule into the XSB clause store.  {\tt Head} must be a
\emph{GeneralLiteral} as defined at the beginning of this section, and
     {\tt Body} must be a list of \emph{GeneralLiterals}.  Upon
     interning, the rule is translated into a normal form, if
     necessary, and atoms are translated to unique integers.  The
     corresponding raw predicates, {\tt smrAddBasicRule/3}, {\tt
       smrAddChoiceRule/3}, {\tt smrAddConstraintRule/4}, and {\tt
       smrAddWeightRule/3} can be used to add raw predicates
     immediately into the SModels API.

\ouritem{smcRetractRule(+Head,+Body)}
\index{\texttt{smcRetractRule/2}} 
%
Retracts a ground (basic) rule from the XSB clause store.  Currently,
this predicate cannot retract rules with weight constraints: {\tt
  Head} must be a \emph{Literal} as defined at the beginning of this
section, and {\tt Body} must be a list of \emph{GeneralLiterals}.

\ouritem{smcSetCompute(+List)}
\index{\texttt{smcCompute/1}} 
%
Requires that {\tt List} be a list of literals -- i.e. atoms or the
default negation of atoms).  This predicate ensures that each literal
in {\tt List} is present in the stable models returned by Smodels.  By
convention the cooked interface ensures that {\tt true} is present and
{\tt false} absent in all stable models.  After translating a literal
it calls the raw interface predicates {\tt smrSetPosCompute/1} and
{\tt smrSetNegCompute/1}

\ouritem{smcCommitProgram}
\index{\texttt{smcCommitProgram/0}} 
%
This predicate translates all of the clauses from the XSB clause store
into the data structures of the Smodels API.  It then signals to the
API that all clauses have been added, and initializes the Smodels
computation.  The corresponding raw predicate, {\tt smrCommitProgram},
performs only the last two of these features.

\ouritem{smComputeModel}
\index{\texttt{smcComputeModel/0}} 
%
This predicate calls Smodels to compute a stable model, and succeeds
if a stable model can be computed.  Upon backtracking, the predicate
will continue to succeed until all stable models for a given program
cache have been computed.  {\tt smComputeModel/0} is used by both the
raw and the cooked levels.

\ouritem{smcExamineModel(+List,-Atoms)}
\index{\texttt{smcExamineModel/2}} 
%
{\tt smcExamineModel/(+List,-Atoms)} filters the literals in {\tt
  List} to determine which are true in the most recently computed
stable model.  These true literals are returned in the list {\tt
  Atoms}.  {\tt smrExamineModel(+N,-Atoms)} provides the corresponding
raw interface in which integers from {\tt 0} to {\tt N}, true in the
most recently computed stable model, are input and output.

\ouritem{smEnd}
\index{\texttt{smcEnd/0}} 
%
Reclaims all resources consumed by Smodels and the various APIs.  This
predicate is used by both the cooked and the raw interfaces.

\ouritem{print\_cache}
\index{\texttt{print\_cache/0}} 
%
This predicate can be used to examine the XSB clause store, and may
be useful for debugging.

\end{description}

\section{The xnmr\_int Interface}.

This module provides the interface from the {\tt xnmr} module to
Smodels.  It does not use the {\tt sm\_int} interface, but rather
directly calls the Smodels C interface, and can be thought of as a
special-purpose alternative to {\tt sm\_int}.

\begin{description}
\ouritem{init\_smodels(+Query)}
\index{\texttt{init\_smodels/1}}
%
Initializes smodels with the residual program produced by evaluationg
{\tt Query}.  {\tt Query} must be a call to a tabled predicate that is
currently completely evaluated (and should have a delay list)

\ouritem{atom\_handle(?Atom,?AtomHandle)} 
\index{\texttt{atom\_handle/2}}
% 
The {\em handle} of an atom is set by {\tt init\_smodels/1} to be an
integer uniquely identifying each atoms in the residual program (and
thus each atom in the Herbrand base of the program for which the
stable models are to be derived.  The initial query given to
{\tt init\_smodels} has the atom-handle of 1.

\ouritem{in\_all\_stable\_models(+AtomHandle,+Neg)}
\index{\texttt{in\_all\_stable\_models/2}}
%
{\tt in\_all\_stable\_models/2} returns true if {\tt Neg} is 0 and the
atom numbered {\tt AtomHandle} returns true in all stable models (of
the residual program set by the previous call to {\tt
  init\_smodels/1}).  If {\tt Neg} is nonzero, then it is true if the
atom is in NO stable model.

\ouritem{pstable\_model(+Query,-Model,+Flag)}
\index{\texttt{pstable\_model/3}}
%
returns nondeterministically a list of atoms true in the partial
stable model total on the atoms relevant to instances of {\tt Query},
if {\tt Flag} is 0.  If {\tt Flag} is 1, it only returns models in
which the instance of {\tt Query} is true.

\ouritem{a\_stable\_model}
\index{\texttt{a\_stable\_model/0}}
%
This predicate invokes Smodels to find a (new) stable model (of the
program set by the previous invocation of {\tt init\_smodels/1}.)  It
will compute all stable models through backtracking.  If there are no
(more) stable models, it fails.  Atoms true in a stable model can be
examined by {\tt in\_current\_stable\_model/1}.

\ouritem{in\_current\_stable\_model(?AtomHandle)} 
\index{\texttt{in\_current\_stable\_model/1}}
%
This predicate is true of handles of atoms true in the current stable
model (set by an invocation of {\tt a\_stable\_model/0}.)

\ouritem{current\_stable\_model(-AtomList)}
\index{\texttt{current\_stable\_model/1}}
%
returns the list of atoms true in the current stable model.

\ouritem{print\_current\_stable\_model}
\index{\texttt{print\_current\_stable\_model/0}}
%
prints the current stable model to the stream to which answers are
sent (i.e {\tt stdfbk})

\end{description}
