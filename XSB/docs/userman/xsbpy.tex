
\newcommand{\xsbpyversion}{Version 1.0}

\begin{center}
\chapter[XSB and Python]{{\tt xsbpy}: The XSB-Python 3 Interface} \label{chap:xsbpy}
\end{center}

\vspace*{-.30in} 
\begin{center}
{\Large {\bf  \xsbpyversion}}
\end{center}

\begin{center}
  {\Large {\bf By Theresa Swift, Muthukumar Suresh, Carl Andersen}}
\end{center}

\noindent
%{\large {\bf {\em This chapter documents beta-level software.}}}

%Both XSB and Python are written in C, which makes possible a robust
%and efficient interface.  
%

The new {\tt xsbpy} package provides an efficient and easy way for XSB
to call Python 3 functions and methods.  {\tt xsbpy} leverages the
fact that XSB and most Pythons are written in C, so that both systems
can be readily loaded into the same process. The core interface
routines are also written entirely in C, so the interface is very
efficient and -- it is hoped -- very robust within its known
limitations.\footnote{The {\tt xsbpy} package was partly funded by BBN
  Technologies.}

This chapter first describes how to configure {\tt xsbpy}, followed by
introductory examples.  Next is a more precise description of its
functions, its current limitations followed by applications and
further examples.

\section{Configuration and Loading}

Configuration processes of {\tt xsbpy} have been written for Linux, Mac
and Windows.  On these platforms, {\tt xsbpy} has been tested using
versions 3.5 -- 3.10 of Python and its libraries.\footnote{The {\tt
    xsbpy} configuration script was written by Michael Kifer.  It has
  been tested on Windows 10; on Ubuntu, Fedora and CentOs Linuxes; and
  on on Intel-based Mac.}  The {\tt xsbpy} configuration script is
called by the main XSB configuration script, and can be found in {\tt
  XSB/packages/xsbpy}.  {\tt xsbpy} configuration can be run
separately from the main configuration script if desired.

\subsection{Installing {\tt xsbpy} to work with Anaconda}

For most users, installing {\tt xsbpy} under Anaconda is
recommended.  Anaconda comes with all of the Python development
extensions needed for {\tt xsbpy}.  To make use of them, simply
activate your conda environment, (re)configure and (re)make XSB as
described in the manual.

As long as the Anaconda environment is active when you configure,
that's all you need to do!

\subsection{Configuring {\tt xsbpy} for {\tt venv} or without a virtual environmnet} \label{sec:xsbpy-linux}

If you use {\tt venv} as your environment manager or if you are
configuring a system installation of {\tt xsbpy} for several users,
configuration may be slightly more complicated than using Anaconda.
In principle, when XSB is configured on Linux or the Mac, {\tt xsbpy}
will also be configured so that it can be used like any other XSB
module.  However, for this configuration to work several pieces of
software much be in place.

\begin{itemize}
\item {\tt git} and {\tt gcc} (or some other compiler) must be present
  and usable by the user performing the configuration.  Most of these
  tools need to be installed on Linux or macOS to configure and
  compile XSB, so they will not usually be an issue.

%\item As discussed below, a version of C-Python 3.x (the usual
\item A version of C-Python (the usual implementation of Python) 3.x
  with development extensions must be present and usable.\footnote{In
    particular {\tt libpython} and {\tt Python.h} are needed from the
    development version.}  Unfortunately, Python development
  extensions are not contained in a {\tt venv} environment (unlike
  with Anaconda) so they will need to be installed.\footnote{{\tt
      venv} is not designed to support applications like {\tt xsbpy}
    that embed Python in another process.  When a virtual environment
    is created using {\tt venv}, the virtual environment contains a
    Python executable, but not {\tt libpython} or {\tt Python.h} which
    must be linked or included from system directories.}

  \item {\tt xsbpy} configuration uses a Python package called {\tt
    find\_libpython}, and will try to do a {\tt pip} installation if
    it is not present.  {\tt pip} will install this package under the
    {\tt venv} or user's {\tt site-packages} directory; otherwise a
    system {\tt site-packages} directory will be used.\footnote{Some
      Linuxes like Ubuntu do not include {\tt pip} in their default
      distribution(!), in which case the Linux package for {\tt pip} must
      be installed.}
\end{itemize}


%\subsubsection{Installing {\tt xsbpy} to work with a system Python}

\subsubsection{Troubleshooting}

%If you have {\tt sudo} privileges and want to make {\tt xsbpy} work
%with a system version of Python there are a few more steps needed than
%when working with Anaconda.  First, your system needs a development
%version of Python~\footnote{If several versions of Python 3.x are
%  installed, the most recent will be used by the configure script.}
%The Python development extensions are needed because they contain
%Python as a shared object ({\tt .so}) file along with Python header
%({\tt .h}) files.  While some version of Python-3 is included in
%almost any version of Linux, additional Linux package installations
%either may be needed for these development extensions or to install a
%more recent version of Python.
%  
%  As an example, suppose version 3.8 of Python is desired.  If it is
%  not present on the platform, it may be installed with its
%  development extensions as:

%  {\tt sudo apt-get install python3.8-dev}
%
%\noindent
%  in Ubuntu; or
%
%  {\tt sudo dnf install python3.10-dev}
%
%\noindent
%  in Fedora (newer versions) or CentOs.
%
%%Otherwise, if Python3.8 has been installed but the development version
%%is not present the development extensions ({\tt libpython3.8.so} and
%{\tt Python.h}) can be installed as:
%
%{\tt apt-get install libpython3.8-dev}.

If there are difficulties in configuring {\tt xsbpy}, it is likely
that one of the above requirements is not met.  However, if these
requirements are met, but the configuration does not work properly, it
is best to check the file {\tt packages/xsbpy/xsb2py\_connect\_defs.h}
to check whether its definitions are correct ({\tt
  xsb2py\_connect\_defs.h} is created when {\tt xsbpy} is
configured).\footnote{If you can determine that the required software
  is present and that there is a problem with the configuration please
  report the problem at {\tt https://sourceforge.net/p/xsb/bugs}.}

The definitions in {\tt xsb2py\_connect\_defs.h} are used in {\tt
  get\_compiler\_options/2} in {\tt init\_xsbpy.P} to properly compile
{\tt xsbpy} and link in {\tt libpython3x.so}.  These final steps are
done automatically when {\tt xsbpy} is consulted into XSB via a normal
consult or {\tt ensure\_loaded/[1,2]}.  As part of this process, in
{\tt init\_xsbpy.P} the main C code for {\tt xsbpy} is compiled using
gcc compiler options that are appropriate for the version of {\tt
  libpython} used.\footnote{See {\tt get\_compiler\_options/2} in {\tt
    init\_xsbpy.P}.}  When the {\tt xsbpy} module is consulted into
XSB, the {\tt libpython} shared object file is loaded dynamically
along with the {\tt xsbpy} shared object file.

%If installing {\tt xsbpy} on Ubuntu, the first step is to ensure that
%the proper Python libraries and header files are present.  If using
%Python version 3.7 this command is:
%
%{\tt sudo apt-get install libpython3.7-dev}
%
%For {\tt xsbpy} to work properly in Linux:
%\begin{itemize}
%\item The {\tt xsbpy} code must be compiled and dynamically loaded into XSB.
%\item A version of {\tt libpython.so} also needs to be dynamically
%  loaded into XSB, and configured with the proper C library paths.
%\item The proper paths must be set for the Python Standard Library.
%\item Paths to {\tt xsbpy} and to {\tt xsbpy/apps} must be set up to
 % access both XSB and Python modules.
%\end{itemize}

%\begin{verbatim}
%-I <path to XSB's emu directory>  
%-I/usr/include/python3.7m  -I/usr/include/python3.7  
%-lpython3.7m -L/usr/lib/python3.7 
%-Wl -rpath=/usr/lib/python3.7 -lpython3.7m -L/usr/lib/python3.7 
%-Wl -rpath=/usr/lib/python3.7
%\end{verbatim}

\noindent
%In the above compilation options, {\tt xsbpy} initialization
%configures the path to the XSB {\tt emu} directory but all other paths
%are currently hard-coded.

\subsection{Installing a development version of Python under  macOS (if Anaconda is not used)}

Most of the discussion in Section~\ref{sec:xsbpy-linux} pertains
directly to macOS.  As with Linux, installation of {\tt xsbpy} under
Anaconda will be easiest for most users, and it is worthwhile noting
that whether or not Anaconda is used both XSB and {\tt xsbpy} can be
compiled using either {\tt clang} or {\tt gcc} on both macOS or Linux.
%When XSB is configured on the Mac, the configuration script also tries
%to configure {\tt xsbpy} so that it can be loaded and used like any
%other module.  However for this process to work correctly additional
%software must be present.
%
%\subsubsection{A Development Version of Python}
%
However, if XSB is not configured using an Anaconda version of Python
a development version of Python 3.x needed, but is not included by
default in {\tt macOS}, so that code needed by {\tt xsbpy} like {\tt
  Python.h} or {\tt libpython<xxx>.dylib} needs to be installed.  This
software is not always contained in {\tt xtools} either.

Fortunately, there are several installation tools available for the
Mac.  One of the most popular is {\tt brew}.  Once {\tt brew} itself
is installed it will be used to make two further installations.
First, type:

{\tt brew install autoconf automake libtool}

\noindent
which is a prerequisite for a development version of Python to be
installed.  Next, type 

{\tt brew install python@3.8}

\noindent
to install the Python itself (other version numbers of Python can be used)
and save the output of this installation.  Brew installs this Python in a
directory that for many users is not on the execution path determined by
the \texttt{PATH} environment variable.  Therefore the directory
\emph{Python-installation-folder}\texttt{/bin/} will need to be added to
the user's execution path (to the \texttt{PATH} variable), as instructed in
the brew installation output that appears after installing Python.
This can be done, for instance, by executing
%% 
\begin{verbatim}
echo 'export PATH="/usr/local/opt/python@3.8/bin:$PATH"' >> ~/.zshrc
echo 'export LDFLAGS="-L/usr/local/opt/python@3.8/lib"' >> ~/.zshrc
\end{verbatim}
%% 
if the development version of
Python is installed in \texttt{/usr/local/opt/python@3.8}. This
addition is important, because XSB's configuration checks the user's
execution paths as one way to find software.

%\subsubsection{A C Compiler and Loader}
%
%Second -- naturally -- there must be a C compiler installed on your
%Mac.  C compilers can be installed on the Mac through Xcode or Apple's
%Command Line Tools.  (A user needn't be familiar with how a compiler
%works -- the compiler just needs to be installed).  Two compilers are
%available on macOS, Clang and GCC; XSB and {\tt xsbpy} have been
%tested using both compilers.
%
%\subsubsection{The Configuration Itself}
%
%Once a C compiler has been installed; and a development version of
%Python has been installed and added to the user's execution paths,
%simply run the {\tt configure} and {\tt makexsb} scripts in {\tt
%  \$XSB\_ROOT/XSB/build} and {\tt xsbpy} will be built automatically.
%
%As you can see, the entire process is simplicity itself!
%
\subsection{Configuring {\tt xsbpy} under Windows}

To run the {\tt xsbpy} interface under Windows 10, you must first
download and install a development version of Python as described
above.  You must also have the Microsoft {\tt MSVC} compiler
installed.  The Visual Studio Community Edition is free (under some
licensing terms) and can be downloaded from

{\tt https://visualstudio.microsoft.com/downloads/}.

In the examples below, we assume Python version 3.9 has been installed
and YOURUSERNAME should be replaced with your Windows user name.
You must adjust the example paths to correctly
reflect where Python is installed on your machine.

With MSVC and Python (development version) installed, you must then
define a new environment variable named {\tt PYTHON\_LIBRARY} with
value

{\tt C:/Users/YOURUSERNAME/AppData/Local/Programs/Python/Python39/libs/python3.lib}.

To install and use Python packages, you will need to have the {\tt
  PYTHONPATH} environment variable defined.  A definition that worked
for me was (your mileage will vary):
\begin{verbatim}
PYTHONPATH=C:/Users/YOURUSERNAME/AppData/Local/Packages/
       PythonSoftwareFoundation.Python.3.9_qbz5n2kfra8p0/
       LocalCache/local-packages/Python39/site-packages
\end{verbatim}

The environment variable {\tt PATH} must include the python directories:\\
{\tt C:/Users/YOURUSERNAME/AppData/Local/Programs/Python/Python39/Scripts} and\\
{\tt C:/Users/YOURUSERNAME/AppData/Local/Programs/Python/Python39}.

To see if the interface is installed correctly and is functional, it
is suggested (just below) that you run: {\tt bash} test.sh in
directory {\tt <XSB>/packages/xsbpy/test/}.  While this can be made to
work when cygwin is installed, under windows, it is easier to run {\tt
  xsb} in that directory and do:
\begin{verbatim}
| ?- [testSuite], testSuite.
\end{verbatim}
This will compile and load the Python interface and run a few tests,
printing out results.  If it succeeds without errors and without
printing failed test notifications, the interface is available for
use.

At this initial writing, the interface works when loaded by a unix
shell running with cygwin.  However, when running in a native windows
cmd shell, the system throws an error: {\tt Bad Image} claiming that
the python library is not designed to run on Windows or contains an
error.  Dispatching the pop-up error box allows the interface to run
correctly.

\subsection{Testing whether the {\tt xsbpy} configuration was successful}

No matter what platform {\tt xsbpy} is installed on, to test out
whether {\tt xsbpy} has been loaded and is working properly, simply
change the working directory to {\tt packages/xsbpy/test} and execute the
command

\begin{verbatim}
  bash test.sh
\end{verbatim}


\section{Introductory Examples}

We introduce some of the core functionality of {\tt xsbpy} via a
series of simple examples.  As background, when {\tt xsbpy} is loaded,
Python is also loaded and initialized within the XSB process, the core
Prolog modules of {\tt xsbpy} are loaded into XSB, and paths to {\tt
  xsbpy} and its subdirectories are added both to Prolog and to Python
(the latter paths are added by modifying Python's {\tt
  sys.path}). Later, XSB calls Python, Python will search for modules
and packages in the same manner as if it were stand-alone.

\begin{example} \rm {\bf Calling a Python Function (I)} \label{ex:xsbpy-json}
  %

  {{\em The translation of JSON through xsbpy in this example,
      although it is functional, is mainly presented for pedagogic
      purposes.  In general, we recommend using XSB's native JSON
      interface described in Chapter \ref{chap:json} of this manual.
  }}

  
\noindent  
Consider the following call:

 \begin{verbatim}
pyfunc(xp_json,
       loads('{"name": "Bob", "languages": ["English","French","GERMAN"]}'),
       Ret)
\end{verbatim}
 
\noindent 
which loads the Python {\tt xp\_json} module in {\tt
  packages/xsbpy/starters} if needed (and by extension the Python
system {\tt json} module), then calls the Python function

{\tt xp\_json.loads()}

\noindent
with the JSON string
\verb|'{"name": "Bob", "languages": ["English","French","GERMAN"]}'|
as the argument.
This call converts the argument to a Python
dictionary.  In this case, the dictionary would have the Python form:
\begin{verbatim}
{
  "name":"Bob",
  "languages":["English", "French","GERMAN"]
}
\end{verbatim}
Next,
{\tt xsbpy} translates this dictionary to a Prolog term, that can
be pretty printed as:
\begin{verbatim}
pyDict([
        ''(name,'Bob'),
        ''(languages,['English','French','GERMAN'])
        ]).
\end{verbatim}
%%
Note that \verb|''| here is an empty string (i.e., a Prolog 0-length
atom).  XSB uses \verb|''| as a functor for terms that are converted
from or to Python tuples: in this case \verb|''/2| is used.  We call a
term that maps to a Python dictionary either {\em a Python dictionary
  in term form} or just {\em a Prolog dictionary} although the latter
slightly abuses terminology.
\end{example}

Note that although the above example used {\tt xp\_json.loads()}, the
Python system package call {\tt json.loads()} could also have been
used.  Such a switch would not require writing {\em any} special
new Prolog or Python code.  This is in part because Python's basic data
structures -- dictionaries, lists, tuples, sets and so on -- are
mapped to Prolog terms (cf. Section~\ref{sec:bi-translation}).  As a
result, calling Python is often a simple matter of setting up input
terms for a Python function, and processing the terms that Python
returns to Prolog.\footnote{The module {\tt xp\_json} basically
  imports and re-exports {\tt loads()}.}

\begin{example} \rm {\bf Calling a Python Function (II): Glue Code} \label{xsbpy-examp:glue}

\noindent
  A slightly more complex call to Python is:
 \begin{verbatim}
pyfunc(xp_json,prolog_load('test.json'),Ret)
\end{verbatim}

\noindent
which loads a JSON string from the file {\tt test.json} into a Prolog
term.  However, the Python function {\tt json.load()} call requires a
Python file pointer as its input, and Python file pointers do not
correspond to XSB I/O streams.  As we shall see in
Example~\ref{xsbpy-examp:method}, a reference to a Python file pointer
could be passed back to XSB, but in most cases it is probably easiest
to write some simple Python glue code such as:

\begin{verbatim}   
def prolog_load(File):
    with open(File) as fileptr:
        return(json.load(fileptr))
\end{verbatim}
\noindent
As in the previous example, the above Prolog goal produces a Prolog
dictionary corresponding to the JSON file.
\end{example}

\begin{example} \rm {\bf Calling a Python Function (III): Keyword Arguments}
  
\noindent
Python functions often make heavy use of keyword arguments.  These can
be easily handled by {\tt pyfunc/4}:
\begin{verbatim}
pyfunc(xp_json,prolog_dump(Dict,'new.json'),[indent=2],Ret)
\end{verbatim}

\noindent
in which the third argument is a list of {\tt =/2} terms.  This list
is turned into a Prolog dictionary and then translated into Python.
Because {\tt json.dump()} needs a file pointer in the same way {\tt
  json.load()}, glue code will also be needed, but the glue code passes
keyword arguments in the usual manner of Python:
\begin{verbatim}
def prolog_dump(Dict,File,**Features):
    with open(File,"w") as fileptr:
        ret = json.dump(Dict,fileptr,**Features)
        return(ret)
\end{verbatim}
\end{example}

The previous examples have sketched an approach that can efficiently
call virtually any Python function or method,\footnote{{\tt xsbpy}
  does not currently support Python's binary types.}  although it
might require a small amount of glue code.  However, Python methods can
also be called directly.

\begin{example} \rm {\bf Calling a Python Method} \label{xsbpy-examp:method}

\noindent
Consider the following simple Python class:

\begin{verbatim}
class Person:
  def __init__(self, name, age, ice_cream=None):
    self.name = name
    self.age = age
    if favorite_ice_cream is None:
      favorite_ice_cream = 'chocolate'
    self.favorite_ice_cream = favorite_ice_cream

  def hello(self,mytype):
    return("Hello my name is " + self.name + " and I'm a " + mytype)
\end{verbatim}

\noindent
The call

\begin{verbatim}
    pyfunc('Person','Person'(john,35),Obj),
\end{verbatim}
\noindent
creates a new instance of the {\tt Person} class, and returns a
reference to this instance which has a form such as {\tt
  pyObj(p0x7fb1947b0210)}.  XSB can later use this reference to call a
method:
\begin{verbatim}
    pydot('Person',pyObj(p0x7fb1947b0210),hello(programmer),Ret2).
\end{verbatim}

\noindent
which returns the Prolog atom:

{\tt 'Hello my name is john and I'm a programmer'}

\noindent
Although Python methods, like Python functions, can include keyword
arguments, {\tt xsbpy} does not support keyword arguments in {\tt
  pydot/4} because Version 3.9.4 of the Python C API does not permit
this.
\end{example}

\begin{example} \rm {\bf Examining a Python Object} \label{xsbpy-examp:exam-object}

\noindent
Example \ref{xsbpy-examp:method} showed how to create a Python object,
pass it back to Prolog and apply a method to it.  Suppose we create
another {\tt Person} instance:

\begin{verbatim}
    pyfunc('Person','Person'(bob,34),Obj),
\end{verbatim}
\noindent
and later want to find out all attributes of {\tt bob} both explicitly
assigned, and default.  This is easily done by {\tt
  xp\_utils:obj\_dict/2}.  Assuming that {\tt pyObj(p0x7f386e1e9650)}
is the object reference for {\tt bob} in Prolog, the call
\begin{verbatim}
obj_dict(pyObj(p0x7f386e1e9650) ,ObjD ) .
\end{verbatim}
returns
\begin{verbatim}
    ObjD = pyDict([(name,bob),(age,34),(favorite_ice_cream,chocolate)])
\end{verbatim}

There are times when using the dictionary associated with a class is
not appropriate.  For instance, not all Python classes have {\tt
  \_\_dict\_\_} methods defined for them, or only a single attribute
of an object might be required.  In these cases, {\tt pydot/4} can be
used:

\begin{verbatim}
    pydot('Person',pyObj(p0x7f386e1e9650),favorite_ice_cream,I)
\end{verbatim}
\noindent
returns {\tt I = chocolate}.

\noindent

Summarizing from Example~\ref{xsbpy-examp:method} and the above
paragraph, {\tt pydot/4} can be used in two ways.  If the third
argument, {\tt arg3}, in a call to {\tt pydot/4} is a Prolog
structure, {\tt arg3} is interpreted as a method.  In this case, a
Python method is applied to the object, and its return is unified with
the last argument of {\tt pydot/4}. If {\tt arg} is a Prolog atom,
{\tt arg3} is interpreted as attribute of the object.  In this case,
the attribute is accessed and unified with {\tt arg3}.  Note that the
functionality of {\tt pydot/4} is overloaded in analogy to the
functionality of the {\tt '.'} connector in Python.
\end{example}

A great deal of Python functionality is directly available via {\tt
  pyfunc/[3,4]} and {\tt pydot/4}.  In our experience so far, many
Python libraries can be called directly and will ``just work''
immediately.  Cases where glue code is needed include the following.

\begin{itemize}
\item In a case like Example \ref{xsbpy-examp:glue} where a Python
  method or function like {\tt json.load()} requires a Python resource
  as input, a small amount of code might be useful to, say, open a
  file and perform an operation.  However as an alternative, the file
  might be opened, the file pointer passed back to XSB, and the
  function called directly from XSB using the file pointer.

  \item As mentioned, {\tt pydot/4} does not support keyword
    arguments, due to restrictions in the Python C API.

  \item Suppose a class with several attributes is defined as a
    subclass of, say a string type.  Currently {\tt xsbpy} will simply
    pass back such objects as strings, rather than as object
    references.  An example of this in fact occurs in the sample
    interface {\tt packages/xsbpy/starters/xp\_rdflib} (see
    Section~\ref{secLxp-rdflib}).  In the {\tt rdflib} package {\tt
      rdflib.Literal} objects are in fact subclasses of a string type.
    These {\tt rdflib.Literal} objects have additional attributes
    representing language tags and data types; and these attributes
    are critical for RDF I/O from Prolog.  An example of how to handle
    this is seen in {\tt xp\_rdflib.py}, where slightly more elaborate
    glue code is needed to marshal an object's attributes as elements
    of a tuple, and passed back along with the object.\footnote{This
      behavior may change in future versions.}
\end{itemize}

With those disclaimers in mind, all glue code that we have needed to
write so far has been simple and straightforward.

\section{Bi-translation between Prolog Terms and Python Data Structures} \label{sec:bi-translation}

{\tt xsbpy} takes advantage of a C-level bi-translation of a large
portion of Prolog terms and Python data structures: i.e., Python
lists, tuples, dictionaries, sets and other data structures are
translated to their Prolog term forms, and Prolog terms of restricted
syntax are translated to lists, tuples, dictionaries, sets and so on.
Bi-translation is recursive in that any of these data structures can
be nested in any other data structures (subject to limitations on
mutables in Python).
     
Due to syntactic similarities between Prolog terms and Python data
structures, the Prolog term forms are easy to translate and use -- and
sometimes appear syntactically identical.

\index{xp\_struct}
\index{px\_term}
\index{term form (of a structure}
%
As terminology, when a Python data structure $D$, say a dictionary, is
translated into a Prolog term $T$, $T$ is sometimes called the {\em
  term form} of $D$.  The type representing any Python structure that
can be translated to Prolog is called {\em xp\_struct} while the type
representing a Prolog term that can be translated into a Python data
structure is called a {\em px\_term}

\subsection{The Bi-translation Specification}

Bi-translation between Prolog and Python can be described from the
viewpoint of Python types as follows:

\begin{itemize}
       \item {\em Numeric Types}: Python integers and floats are
         bi-translated to Prolog integers and floats.  Python complex
         numbers are not (yet) translated, and integers are only
         supported for integers between XSB's minimum and maximum
         integer~\footnote{These integers can be obtained by querying
           {\tt current\_prolog\_flag/2}.}
         \begin{itemize}
           \item {\em Boolean Types} are
             translated to integer values: {\tt True} as {\tt 1}
             and {\tt False} as {\tt 0}.
         \end{itemize}
       \item {\em String Types}: Python string types are bi-translated
         to Prolog atoms.  This translation assumes UTF-8 encoding on
         both sides.

         Note that a Python string can be enclosed in either double
         quotes (\verb|''|) or single quotes (\verb|'|).  In
         translating from Python to Prolog, the outer enclosure is
         ignored, so Python {\tt "'Hello'"} is translated to the
         Prolog {\tt '\textbackslash{}'Hello\textbackslash{}'{}'},
         while the Python {\tt '"Goodby"'} is translated to the Prolog
         {\tt '"Goodby"'}.
       \item {\em Sequence Types}:
         \begin{itemize}
           \item Python lists are bi-translated as Prolog lists and
             the two forms are syntactically identical.
           \item A Python tuple of arity {\tt N} is bi-translated with
             a compound Prolog term \verb|''/N| (i.e., the functor is
             the empty string, denoted by two apostrophes).
             \item Python ranges are not (yet) translated (i.e., they
               are returned as terms with functor {\tt pyObj/1}).
         \end{itemize}
       \item {\em Mapping Types}: A Python dictionary is translated
         into the term form:

         {\tt pyDict(DictList)}

         where {\tt DictList} is a list of tuples in term form: 

         {\tt ''(Key,Value)}

         {\tt Key} and {\tt Value} are the translations of any Python
         data structures that are both allowable as a dictionary key
         or value, and supported by {\tt xsbpy}.  For instance, {\tt
           Value} can be (the term form of) a list, a set, a tuple or
         another dictionary.

       \item {\em Set Types}: A Python set {\em S} is translated to
         the term form

         {\tt pySet(SetList)}

         where {\em SetList} is the list containing exactly the
         translated elements of $S$.  Due to Python's implementation
         of sets, there is no guarantee that the order of elements
         will be the same in $S$ and $SetList$.
       \item {\em None Types.} The Python keyword {\tt None} is
         translated to the Prolog atom {\tt 'None'}. 
       \item {\em Binary Types:} are not yet supported.  There are no
         current plans to support this type.
     \item Any Python object {\tt Obj} that is a non-primitive type,
       or of a type that is not translated to a specific Prolog term
       as indicated above is translated to the Prolog term {\tt
         pyObj(Obj)}.  This {\tt pyObj(Obj)} term can be passed back to
       Python and used for a method call or other purpose.
\end{itemize}

Additionally, a user with a minimal knowledge of C can change parts of
the syntax used in Prolog term forms.  The outer functors {\tt
  pyDict}, {\tt pySet} and {\tt pyObj} and the constant {\tt None} can
all be redefined my modifying the file {\tt xsbpy\_defs.h} in the {\tt
  xsbpy} directory.

\section{Usage}

\begin{description}

\ourrepeatmoditem{pyfunc(+Module,+Function,+Kwargs,+Prolog\_Opts,?Return)}{pyfunc/5}{xsbpy}
\ourrepeatmoditem{pyfunc(+Module,+Function,+Kwargs,?Return)}{pyfunc/4}{xsbpy}
\indourmoditem{pyfunc(+Module,+Function,?Return)}{pyfunc/3}{xsbpy}
%
 Ensures that the Python module {\tt Module} is loaded, and calls {\tt
   Module.Function} unifying the return of {\tt Function} with {\tt tt
   Return}.  Lists of keyword arguments ({\tt Kwargs}) and Prolog
 options ({\tt Prolog\_Opts}) may or may not be included.  For example
 the goal

\begin{verbatim}
pyfunc(xp_rdflib,rdflib_write_file(Triples,'new_sample.ttl'),
       [format=turtle],Ret).
\end{verbatim}

calls the function {\tt xp\_rdflib.rdflib\_write\_file} to write
{\tt Triples}, a list of triples in Prolog format, to the file {\tt
  new\_sample.ttl} using the {\tt turtle} format.  This format is
specified as a keyword argument to {\tt rdflib\_write\_file()} in the
third argument of {\tt pyfunc/4}.

In general, {\tt Module} must be the name of a Python module or path
represented as a Prolog atom, and {\tt Function} is the invocation of
a Python function in {\tt Module}, where {\tt Function} is a compound
Prolog structure.  Optional keyword arguments are passed in the third
argument as lists of {\tt Key = Value} terms; if no such arguments are
needed, {\tt Kwargs} can be an empty list -- or {\tt pyfunc/3} may be
used.  Finally the return value from {\tt Function} is unified with
{\tt Return}.

The only Prolog option currently allowed is {\tt sizecheck(true)},
which traverses the Python data structure to determine its size before
returning the data structure to XSB.  (See
Section~\ref{sec:xsbpy-memory} for details of this option.)

Python modules are searched for in the paths maintained in Python's
{\tt sys.path} list.  As indicated below, these Python paths can be
queried from XSB via {\tt py\_lib\_dir/1} and modified via {\tt
  add\_py\_lib\_dir/1}.
     
{\bf Error Cases}
\bi
\item {\tt Module} cannot be found in the current Python search paths:
\bi
\item {\tt misc\_error}
\ei
\item {\tt Function} does not correspond to a Python function in {\tt Module}
\bi
\item {\tt misc\_error}
  \ei \ei
%
  In addition, errors thrown by Python are
  caught by XSB and thrown as {\tt misc\_error} errors.

\ourrepeatmoditem{pydot(+Module,+ObjRef,+MethAttr,+Prolog\_Opts,?Ret)}{pydot/5}{xsbpy}
\indourmoditem{pydot(+Module,+ObjRef,+MethAttr,?Ret)}{pydot/4}{xsbpy}
%
Applies a method to {\tt ObjRef}, or obtains an attribute from it.  As
with {\tt pyfunc/[3,4]}, {\tt Module} is a Python module or
path. However, {\tt ObjRef} is a Python object reference in term form
(i.e., a term of the form {\tt pyObj(Ref)} where {\tt Ref} is a Prolog
atom depicting a reference to a Python object). {\tt pydot/4} acts in
one of two ways:
\begin{itemize}
\item If {\tt MethAttr} is a Prolog compound term corresponding to a
  Python method for {\tt ObjRef}, the method is called and its return
  unified with {\tt Ret}.

  Unfortunately, limitations in the Python C API version 3.9.4 lead to
  two limitations in calling Python methods from XSB through C.
  First, keyword arguments cannot be used when calling Python method
  as they can for Python functions.  Second, {\tt MethAttr} must have
  3 or fewer arguments.\footnote{The reason for this is that to
    execute Python n-ary methods from the C-API a variadic function
    call must be made, and variadic function {\em calls} cannot be
    contsructed dynamically in C (or at any rate, I don't know how to
    do this).}
%
\item If {\tt MethAttr} is a Prolog atom corresponding to the name of
  an attribute of {\tt ObjRef}, the attribute is accessed and unified
  with {\tt Ret}.
\end{itemize}

Both the Prolog options ({\tt Prolog\_Opts}) and the handling of Python
paths is as with {\tt pyfunc/[3-5]}.

{\bf Error Cases}
\bi
\item {\tt Module} cannot be found in the current Python search paths:
\bi
\item {\tt misc\_error}
\ei
\item {\tt ObjRef} is not a Python object reference in Prolog term form:
\bi
\item {\tt misc\_error}
\ei
\item {\tt MethAttr} is neither a Prolog compound term nor a Prolog atom:
\bi
\item {\tt misc\_error}
\item {\tt MethAttr} has arity greater than 3.
\bi
\item {\tt misc\_error}
  \ei
  \ei \ei In addition, errors thrown by Python are
  caught by XSB and re-thrown as {\tt misc\_error} errors.

\ourrepeatmoditem{pp\_py(+Stream,+Term)}{pp\_py/2}{xsbpy}
\indourmoditem{pp\_py(Term)}{pp\_py/1}{xsbpy}
%
Pretty prints the Prolog translation of a Python data structure in
Python-like syntax.  For instance, the term

\begin{verbatim}
pydict([''(name,'Bob'),''(languages,['English','French','GERMAN'])]).
\end{verbatim}

\noindent
is printed as 
\begin{verbatim}
{
  name:'Bob',
  languages:[
   'English','
   'French',
   'GERMAN'
  ]
} 
\end{verbatim}

Such pretty printing can be useful for developing applications such as
with {\tt xp\_elastic}, the {\tt xsbpy} Elasticsearch interface.

\indourmoditem{add\_py\_lib\_dir(+Path)}{add\_py\_lib\_dir/1}{xsbpy}
%
This convenience predicate allows the user to add a path to the Python
library directories in a manner similar to {\tt add\_lib\_dir/1},
which adds Prolog library directories.

\indourmoditem{py\_lib\_dirs(?Path)}{py\_lib\_dirs/1}{xsbpy}
%
This convenience predicate returns the current Python library
directories as a Prolog list.

\indourmoditem{values(+Dict,+Path,?Val)}{values/3}{xp\_utils}
%
  Convenience predicate to obtain a value from a (possibly nested)
  Prolog dictionary.  The goal

  {\tt values(D,key1,V)}

\noindent
  is equivalent to the
  Python expression {\tt D[key1]} while

  {\tt values(D,[key1,key2,key3],V)}

\noindent
is equivalent to the Python expression

{\tt D[key1][key2][key3]}.

There are no error conditions associated with this predicate.

\ourrepeatmoditem{keys(+Dict,?Keys)}{keys/2}{xp\_utils}
\ourrepeatmoditem{key(+Dict,?Keys)}{key/2}{xp\_utils}
\indourmoditem{items(+Dict,?Items)}{items/2}{xp\_utils}
%
Convenience predicates (for the inveterate Python programmer) to
obtain a list of keys or items from a Prolog dictionary.  There are no
error conditions associated with these predicates.

The predicate {\tt key/2} returns each key of a dictionary on
backtracking, rather than returning all keys as one list, as in {\tt keys/2}.



\indourmoditem{obj\_dict(+ObjRef,-Dict)}{obj\_dict/2}{xp\_utils}
%
Given a reference to a Python object as {\tt ObjRef}, this predicate
returns the dictionary of attributes of {\tt ObjRef} in {\tt Dict}.
If no {\tt \_\_dict\_\_} attribute is associated with {\tt ObjRef} the
predicate fails.

{\tt obj\_dict/2} is a convenience predicate, and could be written
using {\tt pydot/4} as:

\begin{verbatim}
  pydot('__main__',Obj,'__dict__',Dict).
\end{verbatim}

\indourmoditem{obj\_dir(+ObjRef,-Dir)}{obj\_dir/2}{xp\_utils}
%
Given a reference to a Python object as {\tt ObjRef}, this predicate
returns the list of attributes of {\tt ObjRef} in {\tt Dir}.  If no
\_\_dir\_\_ attribute is associated with {\tt ObjRef} the predicate
fails.

{\tt obj\_dir/2} is a convenience predicate, and could be written
using {\tt pydot/4} as:

\begin{verbatim}
  pydot('__main__',Obj,'__dir__'(),Dir).
\end{verbatim}

\end{description}

\section{Performance and Space Management} \label{sec:xsbpy-perf}

The core {\tt xsbpy} routines -- {\tt pyfunc/[3,4]} and {\tt pydot/4}
-- are written almost entirely in C, have shown good performance so
far, and continue to be optimized.  Calling a simple Python function
to increment a number from XSB and then returning the incremented
value to XSB should take about a microsecond on a reasonably fast
machine.  Of course, the overhead for passing large terms from and to
Python will be somewhat higher.  For instance, the time to pass a list
of integers from Python to XSB has been timed at about 20-30
nanoseconds per list element.  Nonetheless, for nearly any practical
application the time to perform useful functionality within Python
will far outweigh any {\tt xsbpy} overhead.

Apart from system resource limitations, there is virtually no upper
limit on the size of Python structures passed back to Prolog: stress
tests have passed lists of initegers of length 100 million from Python
to Prolog without problems. However, it should be noted that {\tt
  xsbpy} must ensure that XSB's heap is properly expanded before
copying a large structure from Python to XSB, a topic to which we now
turn.

\subsection{Memory Management} \label{sec:xsbpy-memory}

\subsubsection{Space Management for XSB's Heap}
Because Python data structures are directly copied onto the XSB heap
stack, the heap serves as a buffer for the return of information from
Python.  XSB currently relies on the Python C API size routine {\tt
  Py\_SIZE} to estimate the size of a structure, since accessing this
routine has a constant-time overhead.\footnote{XSB estimates the size
  multiplying {\tt Py\_SIZE} by a constant factor.}  However, {\tt
  Py\_SIZE} only returns the length of a structure, and not its exact
size.  Accordingly long lists of large structures, heavily nested
dictionaries and other such structures may present a problem.  It
should be noted that when {\tt xsbpy} is initialized, XSB's default
heap margin is reset to to 1 megabyte so that any data structure whose
size is less than 1 megabyte will be copied safely, even if its size
is under-estimated.\footnote{XSB's default heap margin is 64 kbytes.}

Fortunately, this default works for most users.  In using {\tt xsbpy}
the only times large data structures have proven a problem is when
returning large bulk queries from Elasticsearch, or returning large
sets of ontology instances from Wikidata.  In such a case there are
two options.  First, one may reset the heap margin to an even larger
value (see Volume 1 for details).  Alternately, one may use the option
{\tt sizecheck(true)} for those {\tt pyfunc} or {\tt pydot} calls that
are expected to return large data structures.  If this option is
specified, the call performs two traversals of the data structure to
be returned: one to determine its size and ensure heap space, and another
to copy the data structure.

\subsection{Python Space Management}

WHen using Python's C interface, every Python object that is declared
in C must be explicitly released, a requirement that supports Python's
garbage collection.  {\tt xsbpy} memory usage has been checked using
the {\tt guppy} package, which indicates that there are no memory
leaks.

%\subsection{Allowing Python to Reclaim Space}
%
%TBD: Discuss space issues for Python how they are now addressed and
%how they will be addressed in the future.

\section{Interfaces to Python Libraries}

The {\tt packages/xsbpy/starters} directory contains code to interface
to various Python libraries---to help users start projects using {\tt
  xsbpy}.  Some of the files implement useful higher level mappings
that translate say, embedding spaces or SpaCy graphs to Prolog graphs,
or translate RDF graphs to lists of Prolog structures.  Others are
simple collections of examples to show how to query or update
Elasticsearch, to detect the language of input text or to perform
machine translation.  Nearly all of the interfaces have been a
starting point for research or commercial
applications.  \footnote{Testing has been done of the interfaces, but
  the testing has not been exhaustive.  As a result, please
  double-check any results, and report bugs -- or improvements -- to
  {\tt xsb.sorceforge.net}.}

When {\tt xsbpy} is loaded, both the {\tt xsbpy} directory and its
{\tt packages/xsbpy/starters} subdirectory is added to the Prolog and
Python paths.  As a result, modules in these subdirectories can be
loaded into XSB and Python without changing their library paths.

Note that most of these applications require the underlying Python
libraries to have been installed via a {\tt pip} or {\tt conda}
install.

\subsection{Fasttext Queries and Language Detection: {\tt xp\_fasttext}}
%
Facebook's {\tt fastText} provides a collection of functionality that
includes querying pre-trained word vectors in over a hundred
languages~\cite{FBFJM18}, training sets of vectors, aligning vector
embeddings~\cite{MUSE2018}, and identifying languages via {\tt
  lid.176.bin}.  This XSB module uses the Python module {\tt fasttext}
and allows an XSB programmer to immediately start using fastText's
pre-trained word embeddings.  A related module, {\tt xp\_faiss}
provides an interface to Facebook's dense vector management system
Faiss.  The distinction between the two is that Faiss can manage
vectors read in from a file, and provides batch-oriented operations;
the fastText module relies on fastText's binary format and provides
simpler, though useful, query support.

\paragraph{Queries to Word Embeddings}
\begin{description}
  \indourmoditem{load\_model(+BinPath,+Name)}{load\_model/2}{xp\_fasttext}
  Loads a word embedding model in fastText binary form, the path of
  which is {\tt BinPath}.  {\tt Name} is an atom to be used as a
  Prolog referent, and so easily allows use of more than one word
  embedding model at a time.

  \indourmoditem{get\_nearest\_neighbors(+Name,+Word,-Neighbors)}{get\_nearest\_neighbors/3}{xp\_fasttext}
  Returns the 10 nearest neighbors of {\tt Word} in the model {\tt
    Name}.  This feature is useful for determining other words that
  are distributionally similar to {\tt Word}.  {\tt Neighbors} is a
  list of tuples (terms with functor {\tt ''/2}) containing a
  neighboring word and its cosine similarity to {\tt Word}.   Although
  {\tt Word} must be a Prolog atom, it need not be an actual English
  word.  Because fastText uses subword embeddings rather than word
  embeddings \cite{BGJM17}, {\tt Word} need not have been in the
  training set of the model.  This feature can sometimes be useful for
  correcting misspellings and other purposes.

  \indourmoditem{cosine\_similarity(+Name,+WordList,-SimMat)}{cosine\_similarity/3}{xp\_fasttext}
  For a model {\tt Name} and {\tt WordList} a list of atoms of length
  $N$, this predicate returns a (cosine) similarity matrix of
  dimension $N \times N$.
    
  \indourmoditem{get\_word\_vec(+Name,+Word,-Vec)}{get\_word\_vec/3}{xp\_fasttext}
  Returns a the vector for {\tt Word} in the model {\tt Name} as a
  Prolog list of floats.  In general, if a computation on word vectors
  can be done wholly on the Python side, it is much faster to do so,
  rather than manipulating vectors in XSB.  This is because the word
  vectors are actually kept as {\tt numpy} arrays and computations
  performed in C rather than in Python.
\end{description}

\paragraph{Language Identification via {\tt lid.176.bin}}
Assuming that Fasttext's language identification module is in the
current directory, the command:
\begin{verbatim}
 pyfunc(fasttext,load_model('./lid.176.bin'),Obj).
\end{verbatim}
Loads the model and unifies {\tt Obj} with a reference to the loaded
module which might look like {\tt pyObj(p0x7faca3428510)}.  
Next, a call to the example python module {\tt xp\_fasttext}:
\begin{verbatim}
pyfunc(xp_fasttext, fasttext_predict(pyObj(p0x7faca3428510),
       'xsbpy is a really useful addition to XSB! But language detection
        requires a longer string than I usually want to type.'),Lang).  
\end{verbatim}
returns the detected language and confidence value, which in this case

\begin{verbatim}
''('__label__en',0.93856)}.
\end{verbatim}

Note that loading the model can be done by calling the Python {\tt
  fasttext} module directly.  In fact, the only reason that the module
{\tt xp\_fasttext} needs to be used (as opposed to calling the Python
functionality directly) is because the confidence of the language
detection is returned as a {\tt numpy} array, which {\tt xsbpy} does
not currently translate automatically. \footnote{The examples {\tt
    xp\_fasttext} and {\tt googleTrans} were written by Albert Ki;
  {\tt xp\_faiss} was written by Albert Ki and Theresa Swift.}

\subsection{Dense Vector Queries with xp\_faiss}
The dense-vector query engine Faiss \cite{JDH17}, developed by
Facebook offers an efficient way to perform nearest neighbor searches
in vector spaces produced by word, network, tuple, or other
embeddings.  The {\tt xp\_faiss} example provides XSB predicates to
initialize a Faiss index from a text file of vectors, perform queries
to the index, and to make a weighted Prolog graph out of the vector
space.  

As with many machine-learning tools, Faiss expects that each of the
vectors is referenced by an integer.  For instance, a vector for the
string {\em cheugy} would be referenced by an integer, say 37.  The
XSB programmer thus would be responsible for associating the string
{\em cheugy} with 37 in order to use Faiss.  The main predicates
exported by {\tt xp\_faiss.P} include:

\begin{itemize}
\item {\tt faissInit(+XbFile,+Dim)} initializes a Faiss index where
  {\tt XbFile} is a text file containing the vectors to be indexed and
  {\tt Dim} is the dimension of these vectors. ({\tt xb} is Faiss
  terminology for the set of {\em base}, i.e., indexed, vectors.)
  This predicate also creates a {\tt numpy} array with a set of query
  vectors {\tt xq} consisting of the same vectors.  When the query and
  index vectors are set up in this manner, a nearest-neighbor search
  can be performed for any of the indexed vectors.  With this, the
  vector space can be explored, visualized, and so on.

  After execution of this predicate, a fact for the predicate {\tt
    xp\_faiss:xq\_num/1} contains the number of query vectors ({\tt
    xq}), which is the same as the number of indexed vectors ({\tt
    xb}).

\item {\tt get\_k\_nn(+Node,+K,-Neighbors)} finds the {\tt K} nearest
  neighbors of a node.  The predicate takes as input {\tt Node}, the
  integer identifier of a node, and {\tt K} the number of nearest
  neighbors to be returned.  The return structure {\tt Neighbors} is
  the Prolog representation of a 2-ary Python tuple (i.e., {\tt ''/2})
  containing as its first argument a list of {\tt K} distances and as
  it second argument a list of {\tt K} neighbors.


\item {\tt make\_vector\_graph(K)} Given a Faiss index, this predicate
  asserts a weighted graph in Prolog by obtaining the nearest {\tt K}
  neighbors for each indexed vector.  Edges of the graph have the form:

  {\tt vector\_edge\_raw(From,To,Dist)}

  \noindent
  Where {\tt From} and {\tt To} are integer referents for indexed
  vectors, and {\tt Dist} is the Euclidean distance between the vector
  with referent {\tt From} and the vector with referent {\tt To}.
  Each fact of {\tt vector\_edge\_raw/3} is indexed both on its first
  and second argument.

  If both the number of indexed vectors and {\tt K} are large,
  construction of the Prolog vector graph may take a few minutes.
  Construction time is almost wholly comprised of the time to find the
  set of {\tt K} nearest neighbors for each node.

\item {\tt vector\_edge(Node1,Node2,Dist)}.  The vector graph, which
  represents distances is undirected.  However to save space, the {\tt
    vector\_edge\_raw/3} facts are asserted so that if {\tt
    vector\_edge\_raw(Node1,Node2,Dist}) has been asserted, {\tt
    vector\_edge\_raw(Node2,Node1,Dist}) will not be asserted.  {\tt
    vector\_edge/3} calls {\tt vector\_edge\_raw/3} in both
  directions, and should be used for querying the vector graph.
  
\item {\tt write\_vector\_graph(+File,+Header)} writes out the vector
  graph to {\tt File}.  This predicate ensures that {\tt File}
  contains the proper indexing directive for {\tt vector\_edge\_raw/3}
  as well a directive to the compiler describing how to dynamically
  load {\tt File} in an efficient manner.  Because of these
  directives, the file can simply be consulted or ensure\_loaded and
  the user does not need to worry about which compiler options should
  be used.  The graph is loaded into the module {\tt vector\_graph}.

  {\tt Header} is simply a string that is written as a comment to the
  first line of {\tt File} that can serve to contain any necessary
  provenance information.
\end{itemize}  

\subsection{Translating Between RDF and Prolog: xp\_rdflib} \label{secLxp-rdflib}
This module interfaces to the Python {\tt rdflib} library to read RDF
information from files in Turtle, N-triples and N-quads format, and to
write files in Turtle and N-triples format.  In addition, RDF hdt
files can be loaded and queried using rdflib-hdt.  As such {\tt
  xp\_rdflib} augments XSB's RDF package (Chapter~\ref{chapter:RDF})
which handles XML-RDF.

Within a triple, URIs and blank nodes are returned as Prolog atoms,
while literals are returned as terms with functor {\tt ''/3} (the
Prolog representation of a 3-ary tuple) in which the first argument is
the literal's string as a Prolog atom, the second argument is its
datatype, and the third argument its language. If the data type or
language are not included, the argument will be null.  As examples:

\begin{verbatim}
"That Seventies Show"^^<http://www.w3.org/2001/XMLSchema#string> 
\end{verbatim}
is returned as 
\begin{verbatim}
('"That Seventies Show"','<http://www.w3.org/2001/XMLSchema#string>',) 
\end{verbatim}
while 
\begin{verbatim}
"That Seventies Show"@en
\end{verbatim}
is returned as
\begin{verbatim}
('"That Seventies Show"',,en) 
\end{verbatim}

The file {\tt xp\_rdflib.P} contains predicates {\tt test\_nt/0}, {\tt
  test\_ttl/0}, {\tt test\_nq/0} to test reading and writing.  Note
that Python options needed to deserialize an RDFllb graph write are
specific to the RDFlib plug-in for a particular format, and these
plug-ins are not always consistent with one another.  As a result, if
other formats are desired, minor modifications of {\tt xp\_rdflib} may
be necessary.

The use of {\tt xp\_rdflib} differs on the RDF format used.  For the
turtle, ntriples and nquads formats a file is read into XSB as a
(large) list, and an XSB list of terms of the proper form can be
transformed into RDF and written to a file.  For the hdt format the
usage pattern is different: when an hdt file is loaded, it is simply
memory mapped into a process and facts are loaded into a rdflib graph
(and into XSB) purely on demand~\footnote{At least I think that's how
  it works...}.

\subsubsection{Functionality for Turtle, Ntriples and Nquads Formats}

\paragraph{Reading RDF}
\begin{itemize}
\item {\tt read\_rdf(+File,+Format,-TripleList)} reads RDF from a file
  containing an RDF graph formatted as {\tt Format}, where the formats
  {\tt turtle}, {\tt nt} and {\tt nquads} have been tested (the format
  {\tt ttl} is also allowed as a substitute for {\tt turtle}.  These
  formats can be tested on {\tt sample.ttl}, {\tt sample.nt} and {\tt
    sample.nq}, all of which are in the {\tt packages/xsbpy/starters}
  directory.

  Due to the structure of the Python {\tt RDFlib} graph, no guarantee
  is made that the order of facts in {\tt File} will match the order
  of facts in {\tt TripleList}.
\end{itemize}
  
{\bf Error Cases}
\bi
\item {\tt Format} is not {\tt nt}, {\tt turtle}, {\tt ttl}, or {\tt
  nquads} \bi
\item {\tt misc\_error}
\ei
\end{itemize}

\paragraph{Writing RDF}

If {\tt TripleList} is a list of terms, structured as {\tt ''/3} terms
described above, it can be easily be written to {\tt File} as properly
formatted RDF.  The Python function {\tt
  rdflib\_write\_file\_no\_decode()} can be called directly as:
\begin{verbatim}
pyfunc(xp_rdflib,rdflib_write_file_no_decode(+TripleList,+File),[format=+Fmt],-Ret).
\end{verbatim}
where {\tt Fmt} is {\tt turtle} or {\tt nq}.  {\tt
  rdflib\_write\_file\_no\_decode()} is a simple function that creates
an RDFlib graph out of {\tt TripleList}, serializes the graph and
prints it out.  The Python options needed to write to a file are
specific to the RDFlib plug-in for a particular format, so if other
formats are desired, minor modifications of {\tt xp\_rdflib} may be
necessary.

Due to the structure of the Python {\tt RDFlib} graph, no guarantee is
made that the order of facts in {\tt File} will match the order of
facts in {\tt TripleList}.
  
\subsubsection{Functionality for the Hdt Format}

The RDF hdt format is intended to support large, read-only knowledge
bases, such as Wikidata, that may contain billions of triples.  A hdt
file is a compressed binary serialization that can be directly browsed
and queried.  The advantage of querying over compressed data is that
large data stores become manageable that otherwise wouldn't be.  For
instance, a Wikidata snapshot that contains several billion rows along
with indexes takes up about 160 Gbytes on disk and takes about 3
seconds to initialize into (xp\_)rdflib.  Furthermore, the data is
loaded into RAM only as needed for query evaluation.  

{\tt xp\_rdflib} offers two main predicates for use with rdflib hdt:

\begin{description}
  \indourmoditem{hdt\_load(+Store,-Obj)}{hdt\_load/2}{xsbpy}
  %
  Initializes rdflib for the hdt file {\tt Store} and creates a rdflib
  graph in which to store the results of queries.  {\tt Obj} is the
  Python reference to the data store.
  
  \indourmoditem{hdt\_query(?Arg1,?Arg2,?Arg3,-List)}{hdt\_query/4}{xsbpy}
  %

  Allows a user to query a HDT store using Prolog-like syntax and
  returning the results of the query in {\tt List}.  For instance the query
\begin{verbatim}
    hdt_query('http://www.wikidata.org/entity/Q144',Pred,Obj,List)
\end{verbatim}
finds all triples having the above URI as their subject.  In this
case, {\tt List} would be unified with a long list beginning with
\begin{footnotesize}
\begin{verbatim}
[('http://www.wikidata.org/entity/Q144','http://schema.org/name',''(dog,'',en))
 ('http://www.wikidata.org/entity/Q144','http://schema.org/description',''('domestic animal,'',en))
...
\end{verbatim}
\end{footnotesize}
\end{description}

\subsection{xp\_spacy}
SpaCy is widely used tool that exploits neural language models to
analyze text via dependency parses, named entity recognition, and much
else.  Although SpaCy is a Python tool, much of it is written in
C/Cython which makes it highly efficient.  The {\tt xp\_spacy} package
offers a flexible and efficient means to use SpaCy from Prolog (once
SpaCy has been properly installed for Python, along with appropriate
SpaCy language models).

In SpaCy, a user first loads one or more language models for the
language(s) of interest and of a size suitable to the application.
Text is then run through this language model and through other SpaCy
code producing a {\tt Document} object containing a great amount of
detail about the sentences in the text, tokens in the sentence and
their relations to one another.

Reflecting this sequence, the predicate {\tt load\_model/1} is used to
load a SpaCy model into the XSB/Python session:

\begin{verbatim}
load_model(en_core_web_sm) 
\end{verbatim}

\noindent
On the Python side the identifier {\tt en\_core\_web\_sm} is
associated with a {\tt Language} object, and using this
association the same atom can be used to process text throughout the
session.  Multiple models can be loaded and used to process different
text or files in different languages or for different purposes.  For
instance, the command:

\begin{verbatim}
proc_string(en_core_web_sm,'She was the youngest of the two daughters of a
most affectionate, indulgent father; and had, in consequence of her sister's
marriage, been mistress of his house from a very early period.',Doc)
\end{verbatim}

\noindent
processes the above text, and unifying {\tt Doc} with the referent to
the resulting SpaCy {\tt Document} object, which contains the textual
analysis of the string.  The predicate {\tt proc\_file/3} works
similarly for textual files.

At this point, a user of {\tt xp\_spacy} has two options: she can
either query the {\tt Document} object directly or call {\tt
  token\_assert/1} to assert information from the {\tt Document}
object into a Prolog graph that can be conveniently analyzed.  If
querying a {\tt Document} object directly, a small amount of code may
need to be written because SpaCy's API often returns generators to its
data structures rather than its data structures themselves.  Generally
the code that needs to be written is extremely simple and consists of
little more than a list comprehension: the functions {\tt get\_nps()},
{\tt get\_ents()} and {\tt get\_token\_info()} in {\tt xp\_spacy.py}
provide examples of this.

%The predicate {\tt load\_and\_proc/2} loads a SpaCy model and
%processes text, asserting Python references both to the document and
%to the function used to invoke the language model on text.  At this
%point, a user of {\tt xsbpy} has two options: she can either query the
%document directly or call {\tt token\_assert} to assert information
%from the document into a Prolog graph that can be conveniently
%analyzed.  If querying a document directly, a small amount of code may
%need to be written because SpaCy's API often returns generators to its
%data structures rather than its data structures themselves.  Generally
%the code that needs to be written is extremely simple and consists of
%little more than a list comprehension: the functions {\tt get\_nps()},
%{\tt get\_ents()} and {\tt get\_token\_info()} in {\tt xp\_spacy.py}
%provide examples of this.

For most purposes however, it is easier to call the XSB predicate {\tt
  token\_assert(Doc)} that asserts tokens and their dependency parse
edges into XSB as explained below.  As an example of how to navigate
this graph, {\tt show\_all\_trees/0} and its supporting predicates
provide a simple but clear representation of the SpaCy dependency
parse in constituency tree form using the Prolog version of the parse.
Example~\ref{spacy-examp} below shows a similar sequence as it might
be executed in a simple session.


%The code in this package originated as part of a large research
%project and reflects the needs of that project.  Other applications
%may have slightly different needs, and the {\tt xp\_spacy} code can be
%easily refactored to support such needs.

As a final point before presenting the the main predicates, note that
if text from different languages is to be analyzed, the package {\tt
  xp\_fasttext} can be used to determine the language of a text
string, and the text can then be sent to one of several language
models. 

\paragraph{{\tt xp\_spacy} Predicates}

\begin{description}
  \ourrepeatmoditem{load\_model(+Model,+Options)}{load\_model/2}{xp\_spacy}
  \indourmoditem{load\_model(+Model)}{load\_model/1}{xp\_spacy}
%  
Loads the SpaCy model {\tt Model} and associates the Prolog atom {\tt
  Model} with the corresponding SpaCy {\tt Language} object.
Currently, the only form for {\tt Options} is a (possibly empty) list
of terms of the form {\tt pipe(Pipe)} where {\tt Pipe} is the name of
a SpaCy pipe, i.e., a process to add to the NLP pipeline of the SpaCy
Language object {\tt Model}.

{\tt load\_model(Model)} is a convenience predicate for {\tt
  load\_model(Model,[])}.

\ourrepeatmoditem{proc\_string(+Model,+Atom,-Doc,+Options)}{proc\_string/4}{xp\_spacy}
\indourmoditem{proc\_string(+Model,+Atom,-Doc)}{proc\_string/3}{xp\_spacy}
%
Processes the text {\tt Atom} using the model {\tt Model} and unifying
{\tt Doc} with the resulting SpaCy {\tt Document} object.  The only
option currently allowed in {\tt Options} is {\tt
  token\_assert}, which in addition asserts information from {\tt Doc}
into a Prolog graph (after removing information about any previous
dependency graphs).

{\tt proc\_string(+Model,+File,-Doc)} is a convenience predicate for\\
{\tt proc\_string(+Model,+Atom,-Doc,[])}.

If {\tt Model} has not been loaded, {\tt proc\_string/[3,4]} will try
to load it before processing.  If {\tt Model} cannot be found, a
Python {\tt NameError} error is thrown as an XSB miscellaneous error.

\ourrepeatmoditem{proc\_file(+Model,+File,-Doc,+Options)}{proc\_file/4}{xp\_spacy}
\indourmoditem{proc\_file(+Model,+File,-Doc)}{proc\_file/3}{xp\_spacy}
%
Opens {\tt File} and processes its contents using the model {\tt
  Model} and unifying {\tt Doc} with the resulting SpaCy {\tt
  Document} object. {\tt File} is opened in Python, and the stream for
{\tt File} is closed automatically.  The only option currently allowed
in {\tt Options} is {\tt token\_assert}, which in addition asserts
information from {\tt Doc} into a Prolog graph.

{\tt proc\_file(+Model,+File,-Doc)} is a convenience predicate for\\
{\tt proc\_file(+Model,+File,-Doc,[])}.

If {\tt Model} has not been loaded, {\tt proc\_file/[3,4]} will try to
load it before processing.  If {\tt Model} cannot be found, a Python
{\tt NameError} error is thrown as an XSB miscellaneous error.

\indourmoditem{token\_assert(+Doc)}{token\_assert/1}{xp\_spacy}
%    
This predicate accesses the SpaCy Document object {\tt Doc}, then
queries the dependency graph and other information from {\tt Doc}, and
asserts it to Prolog as a graph (after retracting information from any
previous dependency graphs).  The Prolog form of the graph uses two
predicates.  The first:
\begin{verbatim}  
    token_info_raw(Index,Text,Lemma,Pos,Tag,Dep,EntType)
\end{verbatim}
represents the nodes of the graph; For a given SpaCy {\tt token}
object the fields in the corresponding {\tt token\_info\_raw/7} fact are:
are as follows:
\begin{itemize}
  \item {\tt Index} ({\tt token.idx}) is the character offset of {\tt
    token} within the document (i.e., the input file or atom), and
    serves as an index for the token both in SpaCy and in its Prolog
    representation.
  \item {\tt Text} ({\tt token.text}) the verbatim form of {\tt token}
    in the text that was processed.
  \item {\tt Lemma} ({\tt token.lemma\_}) the base form of {\tt
    token}.  If {\tt token} is a verb, {\tt Lemma} is its stem, if
    {\tt token} is a noun, {\tt Lemma} is its singular form.
  \item {\tt Pos} ({\tt token.pos\_ }) is the coarse-grained part of
    speech for {\tt token} according to {\tt
      https://universaldependencies.org/docs/u/pos}
  \item {\tt Tag} ({\tt token.tag\_ }) The fine-grained part of speech
    for {\tt token} that contains some morphological analysis in
    addition to the part-of-speech.
    Cf.

    {\tt https://stackoverflow.com/questions/37611061/spacy-token-tag-full-list}

      for a discussion of its meaning and use.
  \item {\tt Dep} ({\tt token.dep\_ }) The type of relation that {\tt
    token} has with its parent in the dependency graph.
  \item {\tt EntType} ({\tt token.ent\_type\_ }) The SpaCy named entity
    type, e.g., person, organization, etc.
\end{itemize}  

Edges of the Prolog graph have the form:
\begin{verbatim}
token_childOf(ChildIndex,ParentIndex)
\end{verbatim}
where {\tt ChildIndex}, and {\tt ParentIndex} are indexes for {\tt
  token\_info\_raw/7} facts.

Note that SpaCy tokens have many other attributes, of which the above
are some of the more useful.  If other attributes are needed, the {\tt
  xp\_spacy} code can easily be expanded to include them.  However
many aspects of the parse can be easily reconstructed by the Prolog
graph and don't need to be materialized in Prolog.  For instance the
code for {\tt show\_all\_trees/0} in {\tt xp\_spacy.P} contains code
for constructing sentences, subtrees of a given token and so on.

\ourrepeatmoditem{get\_text(Index,Text)}{get\_text/2}{xp\_spacy}
\ourrepeatmoditem{get\_lemma(Index,Lemma)}{get\_lemma/2}{xp\_spacy}
\ourrepeatmoditem{get\_pos(Index,Pos)}{get\_pos/2}{xp\_spacy}
\ourrepeatmoditem{get\_tag(Index,Tag)}{get\_tag/2}{xp\_spacy}
\ourrepeatmoditem{get\_dep(Index,Dep)}{get\_dep/2}{xp\_spacy}
\ourrepeatmoditem{get\_ner\_type(Index,NER)}{get\_ner\_type/2}{xp\_spacy}
\indourmoditem{\tt token\_info(Index,Text,Lemma,Pos,Tag,Dep,Ent\_type)}{token\_info/7}{xp\_spacy}
%
Various convenience predicates for accessing {\tt token\_info\_raw/7}.
{\tt token\_info/7} is a convenience predicate that calls {\tt
  token\_info\_raw/7} and filters out spaces and punctuation.  {\tt
  get\_text/2}, {\tt get\_lemma/2} etc. get the appropriate field from
a {\tt token\_info\_raw/7} fact indexed by {\tt Index}.a
  
\indourmoditem{\tt show\_all\_trees()}{show\_all\_trees/0}{xp\_spacy}
%
  Given a SpaCy graph asserted to Prolog as described above, {\tt
    show\_all\_trees/0} navigates the graph, and for each sentence in
  the graph converts the dependency graph to a tree and prints it out.
  This predicate is useful for reviewing parses, and its code in {\tt
    xp\_spacy.P} can be modified and repurposed for other needed
  functionality.

\indourmoditem{sentence\_roots(-RootList)}{sentence\_root/1}{xp\_spacy}
%
Returns a list of the dependency graph nodes (i.e., {\tt
  token\_info\_raw/7} terms) that are roots of a sentence in the
Prolog dependency graph.  By backtracking through {\tt RootList},
sentence by sentence processing can be done for a document.

\indourmoditem{dependent\_tokens(+Root,-Toklist)}{dependent\_tokens/2}{xp\_spacy}
%
Given the index of token {\tt Root}, returns a sorted list of the
tokens dependent on {\tt Root}.  If {\tt Root} is the root of a
sentence, {\tt Toklist} will be the words in the sentence; if {\tt
  Root} is the root of a noun phrase, {\tt Toklist} will be the words
in the noun phrase, etc.

\end{description}  

\begin{example} \rm \label{spacy-examp}
We provide an example session where {\tt xp\_spacy} is used.  For a
session like this to work SpaCy would need to be installed along with
the SpaCy model {\tt en\_core\_web\_sm}.  The session would start by
consulting the appropriate files and model:
\begin{verbatim}
| ?- [xsbpy,xp_spacy].
:
| ?- load_model(en_core_web_sm).
\end{verbatim}
Next SpaCy is used to process a string (i.e., a Prolog atom):
\begin{verbatim}
| ?- proc_string(en_core_web_sm,'She was the youngest of the two daughters of a most affectionate,indulgent father; and had, in consequence of her sister''''s marriage,been mistress of his house from a very early period.  ',Doc,[token_assert]).

Doc = pyObj(p0x7f36ed5b3580)

yes
\end{verbatim}
The option {\tt token\_assert} automatically loads the SpaCy
dependency graph and other information to Prolog.  Alternately, one
could omit this option and later call {\tt
  token\_assert(pyObj(p0x7f36ed5b3580))}: i.e., call {\tt
  token\_assert/1} with the first argument as the reference to the
SpaCy document object in Python.  Either way, once the dependency
graph has been loaded into XSB the command: 
\begin{verbatim}
show_all_trees().
\end{verbatim}
will print out the list of tokens for this sentence followed by:
    {\footnotesize
\begin{verbatim}
token_info(245,was,be,AUX,VBD,ROOT,)
   token_info(241,She,she,PRON,PRP,nsubj,)
   token_info(253,youngest,young,ADJ,JJS,attr,)
      token_info(249,the,the,DET,DT,det,)
      token_info(262,of,of,ADP,IN,prep,)
         token_info(273,daughters,daughter,NOUN,NNS,pobj,)
            token_info(265,the,the,DET,DT,det,)
            token_info(269,two,two,NUM,CD,nummod,CARDINAL)
            token_info(283,of,of,ADP,IN,prep,)
               token_info(317,father,father,NOUN,NN,pobj,)
                  token_info(286,a,a,DET,DT,det,)
                  token_info(293,affectionate,affectionate,ADJ,JJ,amod,)
                     token_info(288,most,most,ADV,RBS,advmod,)
                  token_info(307,indulgent,indulgent,ADJ,JJ,amod,)
   token_info(325,and,and,CCONJ,CC,cc,)
   token_info(375,been,be,VERB,VBN,conj,)
      token_info(329,had,have,AUX,VBD,aux,)
         token_info(334,in,in,ADP,IN,prep,)
            token_info(337,consequence,consequence,NOUN,NN,pobj,)
               token_info(349,of,of,ADP,IN,prep,)
                  token_info(365,marriage,marriage,NOUN,NN,pobj,)
                     token_info(356,sister,sister,NOUN,NN,poss,)
                        token_info(352,her,her,PRON,PRP$,poss,)
                        token_info(362,'s,'s,PART,POS,case,)
      token_info(380,mistress,mistress,NOUN,NN,attr,)
         token_info(389,of,of,ADP,IN,prep,)
            token_info(396,house,house,NOUN,NN,pobj,)
               token_info(392,his,his,PRON,PRP$,poss,)
      token_info(402,from,from,ADP,IN,prep,)
         token_info(420,period,period,NOUN,NN,pobj,)
            token_info(407,a,a,DET,DT,det,)
            token_info(414,early,early,ADJ,JJ,amod,)
               token_info(409,very,very,ADV,RB,advmod,)
\end{verbatim}
    }

A similar sequence, but with {\tt
  proc\_file(en\_core\_web\_sm,'emma.txt',Doc,[token\_assert])} parses
the sentences in {\tt emma.txt} and loads the results into XSB.  In
this case the command {\tt show\_all\_trees()} displays the dependency
graph for each sentence in tree form.
  \end{example}
 
\subsection{xp\_json}
This module contains an interface to the Python {\tt json} module,
with predicates to read JSON from and write JSON to files and strings.
The {\tt json} module transforms JSON objects into and from Python
dictionaries, which the interface maps to and from their term forms.
This module can be used to help understand how Python dictionaries
relate to XSB terms, or as an alternative to XSB's Json package ({\tt
  json} Chapter~\ref{chap:json}).  For instance, while for most
purposes XSB's {\tt json} package should be used, {\tt xp\_json} can
be useful if the json constructed and read comes from another {\tt
  xsbpy} application such as {\tt xp\_elastic}.  This is because the
format used by {\tt xp\_json} maps directly to a Python dictionary,
while that of the {\tt json} package maps to other (very useful)
formats.

The {\tt xp\_json} functions are written in Python and can be
called directly from Prolog.
\begin{itemize}
\item {\tt pyfunc(xp\_json,prolog\_load(+File),+Features,-Json)}
  opens and reads {\tt File} and returns its Json content in {\tt
    Json} as a Prolog dictionary term.
\item {\tt pyfunc(xp\_json,prolog\_dump(+Dict,+File),+Features,-Ret)}
  converts {\tt Dict} to a Json object, write it to {\tt File} and
  returns the result of the operation in {\tt Ret}.
\item {\tt pyfunc(xp\_json,prolog\_loads(+Atom),+Features,-Json)}
  reads the atom {\tt Atom} and returns its Json content in {\tt Json}
  as a Prolog dictionary term.
\item {\tt  pyfunc(xp\_json,prolog\_dumps(+Dict),+Features,-JsonAtom)}
  converts {\tt Dict} to a Json string, and returns the string as the
  Prolog atom {\tt JsonAtom}.
\end{itemize}  

\subsection{Other Examples and Demos}

\paragraph{xp\_elastic}
This module contains example code for using the Python {\tt
  elasticsearch} package.  A step by step description shows how a
connection is opened, and index is created and a document added and
committed.  The example then shows how the document can be searched in
two ways, and finally deleted.

Much of the information that Elasticsearch reads and writes is in JSON
format, which the Python interface transforms to dictionaries, and
{\tt xsbpy} transforms these dictionaries to and from their term form.
Thus although this example is short, the ideas in it can easily be
extended to a full interface.\footnote{This has already been done by
  one company that uses XSB.}  Often the {\tt elasticsearch} functions
can be called directly, but in certain cases simple Python functions
must be written to handle default positional arguments. \footnote{{\tt
    xsbpy} correctly handles default keyword arguments, but the Python
  C API does not seem to support default positional arguments.}


%\paragraph{mpl}
%This file has a simple example of how to create, display, and save as
%pdf a simple matplotlib document.

\paragraph{Reading XML files as {\tt xsbpy} Dictionaries}
%
Although XSB's {\tt SGML} package allows XML files to be read, the
ability to read XML structures as {\tt xsbpy} dictionaries can be
convenient, especially if an application already must navigate through
{\tt xsbpy} dictionaries for other purposes.  The module {\tt
  xp\_xmldict}, based on the Python package {\tt
  xmltodict} \footnote{{\tt https://pypi.org/project/xmltodict}}
provides a simple implementation of this based on Python's {\tt Expat}
XML parser, and so retains the advantages of {\tt Expat} in terms of
reliability, Unicode support and speed.

\begin{description}
  \indourmoditem{xmldict\_read(+File,-Dict)}{xmldict\_read/2}{xp\_xmldict}
  
  Given an XML file {\tt File}, this predicate opens {\tt File},
  parses its contents, transforms the contents into a dictionary, and
  unifies {\tt Dict} with dictionary.  
\end{description}

\noindent
It is worthwhile noting that the Python {\tt xmltodict} package offers
several keyword arguments and other options for parsing XML files and
strings, that can be easily accessed via user-written {\tt xsbpy}
calls.

\paragraph{Wikidata queries with {\tt xp\_wdi}}

{\sc Preliminary: package is still under development}

This package provides a simple interface to the Wikidata website via
the Python library {\tt wikidataintegrator}.  It is one of two ways in
which {\tt xsbpy} can be used to query Wikidata: the other is to query
a compressed local snapshot of Wikidata via the {\tt hdt}
functionality in {\tt xp\_rdflib}.  Each approach has advantages and
disadvantages.  The use of {\tt hdt} can be much faster: in part
because it requires no webservice calls, but also because the Wikidata
site slows down responses to requests from a session that is using the
site heavily.  On the other hand, to use {\tt hdt} the a Wikidata {\tt
  hdt} file must be locally mounted; and when a process loads the hdt
file, it must allocate a large amount of virtual memory, although this
memory does not usually affect RAM usage.\footnote{In Linux, this
  means that the process has a large virtual memory size, but its
  resident set size is low.}  So for applications that take place on
servers or that use Wikidata extensively the {\tt hdt} approach for
{\tt xp\_rdflib} is best; for other uses {\tt xp\_wdi} may be more
convenient.

\begin{description}
\indourmoditem{wdi\_get\_dict(+Qnode,-Dict))}{wdi\_get\_dict(+Qnode,-Dict)}{xp\_wdi}

\indourmoditem{wdi\_get\_entity(+Qnode,-EDict))}{wdi\_get\_entity/2}{xp\_wdi}

\indourmoditem{wdi\_sparql\_query(+Qnode,+PropertyNode,-Ret)}{wdi\_sparql\_query/3}{xp\_wdi}

\end{description}

\paragraph{xp\_googleTrans}
This example provides demo code to access Google's web-services for
language translation and language detection using {\tt xsbpy}.



\section{Current and Future Work}

\begin{itemize}
\item A callback mechanism is under development.  This mechanism
  allows XSB and Python to recursively call each other.  Our intention
  is to make our callback mechanism consistent with PyXSB, {\tt
    pypi.org/project/py-xsb}, but currently PyXSB and {\tt xsbpy} are
  independent of each other.

\item A possible future version may include a hook in XSB's atom
  garbage collection to list Python objects that may be garbage
  collected, and to send this information back to Python.  
\end{itemize}  

%%% Local Variables: 
%%% mode: latex
%%% TeX-master: "manual2"
%%% End: 
