\section{The Compiler} \label{the_compiler} \index{Compiler}
%===========================================================

The \ourprolog\ compiler translates \ourprolog\ source files into
byte-code object files.  It is written entirely in Prolog.
Both the sources and the byte code\index{byte code!files!compiler}
for the compiler can be found in the \ourprolog\ system directory
{\tt cmplib}\index{cmplib@{\tt cmplib}}.

The following sections describe the various aspects of the compiler 
in more detail.


\subsection{Invoking the Compiler} \label{compiler_invoking}
\index{invoking the Compiler}\index{Compiler!invoking}
%=====================================================

The compiler is invoked directly at the interpreter level (or in a 
program) through the Prolog predicates {\tt compile/[1,2]}.  

The general forms of predicate {\tt compile/2} are:
\begin{center}{\tt	
	compile(+File, +OptionList) \\
	compile(+FileList, +OptionList)
}
\end{center}
and at the time of the call both of its arguments should be ground.

The second form allows the user to supply a proper list of file names as
the parameter for {\tt compile/[1,2]}.  In this case the compiler will
compile all the files in {\tt FileList} with the compiler
options specified in {\tt OptionList}.

\demo{$|$ ?- compile(Files).} 

\noindent
is just a notational shorthand for the query:

\demo{$|$ ?- compile(Files, []).}

The standard predicates {\tt consult/[1,2]} call {\tt compile/1} (if
necessary).  Argument {\tt File} can be any syntactically valid UNIX
or Windows file name (in the form of a Prolog atom), but the user can also
supply a module name.

The list of compiler options {\tt OptionList}, if specified, 
should be a proper Prolog list, i.e.\ a term of the form:
\begin{center}
	{\tt [ $option_1$, $option_2$, $\ldots$, $option_n$ ].}
\end{center}
where $option_i$ is one of the options described in
Section~\ref{compiler_options}.

The source file name corresponding to a given module is obtained by 
concatenating a directory prefix and the extension {\tt .P} (or {\tt .c}) 
to the module name.  The directory prefix must be in the
dynamic loader path (see Section~\ref{LibPath}).
Note that these directories are searched in a predetermined
order (see Section~\ref{LibPath}), so if a module with the same name
appears in more than one of the directories searched, the compiler 
will compile the first one it encounters.  In such a case, the user can 
override the search order by providing an absolute path name.

If {\tt File} contains no extension, an attempt is made to compile the 
file {\tt File.P} (or {\tt File.c}) before trying compiling the file 
with name {\tt File}.  

We recommend use of the extension {\tt .P} for Prolog source file to
avoid ambiguity.  Optionally, users can also provide a header file for
a module (denoted by the module name suffixed by {\tt .H}).  In such a
case, the \ourprolog\ compiler will first read the header file (if it
exists), and then the source file.  Currently the compiler makes no
special treatment of header files.  They are simply included in the
beginning of the corresponding source files, and code can, in
principle, be placed in either.  In future versions of \ourprolog\ the
header files may be used to check interfaces across modules, hence it
is a good programming practice to restrict header files to
declarations alone.
 
The result of the compilation (an SLG-WAM object code file) is stored
in a ($\langle$filename$\rangle$.O), but {\tt compile/[1,2]} does {\em
not\/} load the object file it creates.  (The standard predicates {\tt
consult/[1,2]} and {\tt reconsult/[1,2]} both recompile the source
file, if needed, and load the object file into the system.)  The
object file created is always written into the directory where the
source file resides (the user should therefore have write permission
in that directory).
 
If desired, when compiling a module (file), clauses and directives can be
transformed as they are read.  This is indeed the case for definite clause
grammar rules (see Chapter~\ref{DCGs}), but it can also be done for clauses
of any form by providing a definition for predicate {\tt term\_expansion/2}
(see Section~\ref{DCG_builtins}).

Predicates {\tt compile/[1,2]} can also be used to compile foreign
language modules.  In this case, the names of the source files should
have the extension {\tt .c} and a {\tt .P} file must {\em not\/}
exist.  A header file (with extension {\tt .H}) {\em must} be present
for a foreign language module (see Chapter~\ref{foreign}).


\subsection{Compiler Options}\label{compiler_options}
\index{Compiler!options}\index{options!Compiler}
%=================================================

The following options are currently recognized by the compiler:
\begin{description}
\item[{\tt optimize}]\index{{\tt optimize}}
	When specified, the compiler tries to optimize the object code.
	In \version, this option optimizes predicate calls, among other
	features, so execution may be considerably faster for recursive
	loops.  However,
	due to the nature of the optimizations, the user may not be able to
	trace all calls to predicates in the program.  Also the Prolog code
	should be {\em static}.  In other words, the user is {\em not} allowed
	to alter the entry point of these compiled predicates by asserting new
	clauses.  As expected, the compilation phase will also be slightly
	longer.  For these reasons, the use of the {\tt optimize} option may
	not be suitable for the development phase, but is
	recommended once the code has been debugged.

\item[{\tt cpp\_on}]\index{{\tt cpp_on}}
  Pass the program through the C preprocessor before sending it to the XSB
  compiler. In this case, the source code can include the usual C
  preprocessor directives, such as \#define, \#ifdef, and \#include.

  Because Prolog syntax is often at odds with the C syntax, the
  following must be kept in mind:
  %%
  \begin{itemize}
    \item  {\bf Quotes:} C requires that quotes come in pairs, which is a
      problem if your program has statements like {\tt 0'x}. While this is
      not fatal, the C preprocessor issues anoying warnings. On top of
      this, it might refuse to apply \#define macros to anything between
      this single quote and the next one (say, another {\tt 0'p} hundred
      lines below). It is recommended to pacify the preprocessor by putting
      a comment of the following form at the end of the line: 
      \mbox{ \%  ' }. For instance, 
      %%
\begin{verbatim}
            put(0'p),       %  '
\end{verbatim}
      %%
      Another way to circumvent the problem is to use \#define, {\it e.g.}, 
      %%
\begin{verbatim}
    #define CH_P  80
         .......
         .......
         put(CH_P),
         .......
\end{verbatim}
      %%
    \item {\bf Comments:} The C preprocessor does not understand Prolog
      syntax (yet). In particular, it does not know that ``\%'' starts a
      comment. Therefore, it treats anything it finds in such Prolog
      comments seriously. One particularly annoying problem is related to
      the previous one: if your program has a line like:
      %%
\begin{verbatim}
         %%  Don't do it!!!
\end{verbatim}
      %%
      then you will see a complaint about unterminated quote. In such a
      case, either double the quote ({\it e.g.}, {\tt Don''t do it!!!} or
      use C-style comments.
  \end{itemize}
  %%

%-------------------------------
\index{tabling!compiler options}
%-------------------------------
\item[{\tt auto\_table}]\index{{\tt auto\_table}}
	When specified as a compiler option, the effect is
	as described in Section~\ref{tabling_directives}.  Briefly, a static
	analysis is made to determine which predicates may loop under Prolog's
	SLD evaluation.  These predicates are compiled as tabled predicates,
	and SLG evaluation is used instead.
%qkostis
\item[{\tt suppl\_table}]\index{{\tt suppl\_table}}
	The intention of this option is to direct the
	system to table for efficiency rather than termination.  When 
	specified, the compiler uses tabling to ensure that no predicate
	will depend on more than three tables or EDB facts (as specified
	by the declaration {\tt edb} of Section~\ref{tabling_directives}).
        The action of {\tt suppl\_table} is independent of that of
	{\tt auto\_table}, in that a predicate tabled by one will not
	necessarily be tabled by the other.
	During compilation, {\tt suppl\_table} occurs after {\tt auto\_table},
	and uses table declarations generated by it, if any.
        
%--------------------------------------
\index{specialisation!compiler options}
%--------------------------------------
\item[{\tt spec\_repr}]\index{{\tt spec\_repr}}
	When specified, the compiler performs specialisation of partially
	instantiated calls by replacing their selected clauses with the
	representative of these clauses, i.e. it performs {\em folding\/}
	whenever possible.  We note in general, the code replacement
	operation is not always sound; i.e. there
	are cases when the original and the residual program are not
	computationally equivalent.  The compiler checks for sufficient (but
	not necessary) conditions that guarantee computational equivalence.
	If these conditions are not met, specialisation is not performed
	for the violating calls.
\item[{\tt spec\_off}]\index{{\tt spec\_off}}
	When specified, the compiler does not perform specialisation of
	partially instantiated calls.
\item[{\tt unfold\_off}]\index{{\tt unfold\_off}}
	When specified, singleton sets optimisations are not performed
	during specialisation.  This option is necessary in \version\
	for the specialisation of {\tt table} declarations that select
	only a single chain rule of the predicate.
%qtls ??
\item[{\tt spec\_dump}]\index{{\tt spec\_dump}}
	Generates a {\tt module.spec} file, containing the result of
	specialising partially instantiated calls to predicates defined
	in the {\tt module} under compilation.  The result is in Prolog
	source code form.

%---------------------------------------------
\index{unification factoring!compiler options}
%---------------------------------------------
\item[{\tt ti\_dump}]\index{{\tt ti\_dump}}
	Generates a {\tt module.ti} file containing the result of applying
	unification factoring to predicates defined in the {\tt module}
	under compilation.  The result is in Prolog source code form.
	See page~\pageref{transformational_indexing} for more information
	on unification factoring.
\item[{\tt ti\_long\_names}]\index{{\tt ti\_long\_names}}
	Used in conjunction with {\tt ti\_dump}, generates names for
	predicates created by unification factoring that reflect the
	clause head factoring done by the transformation.

\item[{\tt init\_var\_off}]\index{{\tt init\_var\_off}}
	When specified, the compiler will give a warning (instead of an
	error) upon finding that a potentially uninitialized variable is
	being used.  {\em Potentially uninitialized variables\/} are
	variables that appear in only one branch of an {\sf or} or an
	{\sf if-then-else} goal in the body, and, furthermore, are used
	after that goal.
	In certain clauses, the variable may always be initialized after
	the {\sf or} or the {\sf if-then-else} goal, because the execution 
	cannot continue through the path of the branch that does not initialize
	the variable.  In these cases, the {\tt init\_var\_off} option can be
	useful, though the user is cautioned against careless use of this
	option.

	{\sc Warning:} The object-file generated by the compiler when this
		option is used may not execute correctly (or even cause
		\ourprolog\ to core dump!) if the variable is indeed
		uninitialized when used.

%---------------------------------------------
\index{mode analysis!compiler options}
%---------------------------------------------
\item[{\tt modeinfer}]\index{{\tt modeinfer}}
	This option is used to trigger mode analysis. For each module
	compiled,  the mode analyzer creates a  {\tt {\em module}.D} file
	that contains the mode information.

	{\sc Warning:}
	Occasionally, the analysis itself may take a long time. 
	As far as we have seen,
	the analysis times are longer than the rest of the compilation time
	only when the module contains recursive predicates of arity $\geq 10$.
	If the analysis takes an unusually long time
	(say, more than 4 times as long as the rest of the compilation)
	you may want to abort and restart compilation without {\tt modeinfer}.
	
\item[{\tt mi\_warn}]\index{{\tt mi\_warn}}
	During mode analysis, the {\tt .D} files corresponding to the
	imported modules are read in. The option {\tt mi\_warn} is used
	to generate warning messages if these {\tt .D} files are 
	outdated --- {\em i.e.}, older
	than the last modification time of the source files.

% tls mention that the following are for hackers ...
\item[{\tt sysmod}] Mainly used by developers when compiling system
	modules. If specified, standard predicates (listed in
	Appendix~\ref{standard_predicates}) are automatically
	available for use only if they are primitive predicates (see
	the file {\tt syslib/machine.P} for a current listing of such
	predicates. When compiling in this
	mode, non primitive standard predicates must be explicitly
	imported from the appropriate system module.
\item[{\tt verbo}] Compiles the files (modules) specified in ``verbose'' mode, 
	printing out information about the progress of the compilation of each 
	predicate.
\item[{\tt profile}] This option is usually used when modifying the
	\ourprolog\ compiler.  When specified, the compiler prints out
	information about the time spent in each phase of the
	compilation process.

\item[{\tt mi\_foreign}] This option is used {\em only\/} when mode analysis
	is performed on \ourprolog\ system modules. This option is
	needed when analyzing {\tt standard} and {\tt machine} in
	{\tt syslib}.

\item[{\tt asm\_dump, compile\_off}] Generates a textual representation of 
	the SLG-WAM assembly code and writes it into the file {\tt module.A}
	where {\tt module} is the name of the module (file) being compiled.  
	
	{\sc Warning:} This option was created for compiler debugging and is
		not intended for general use.  There might be cases where
		compiling a module with these options may cause generation
		of an incorrect {\tt .A} and {\tt .O} file.  In such cases,
		the user can see the SLG-WAM instructions that are
		generated for a module by compiling the module as usual and
		then using the {\tt -d module.O} command-line option of the
		\ourprolog\ emulator (see Section~\ref{emulator_options}).
\item[{\tt index\_off}] When specified, the compiler does not generate indices
	for the predicates compiled.  
\end{description}


\subsection{Specialisation}\label{specialisation}
\index{Compiler!specialisation}\index{specialisation!Compiler}
%=============================================================

From Version 1.4.0 on, the \ourprolog\ compiler automatically performs
specialisation of partially instantiated calls.  Specialisation can be
thought as a source-level program transformation of a program to a
residual program in which partially instantiated calls to predicates
in the original program are replaced with calls to specialised versions
of these predicates.  The expectation from this process is that the
calls in the residual program can be executed more efficiently that
their non-specialised counterparts.  This expectation is justified
mainly because of the following two basic properties of the
specialisation algorithm:
\begin{description}
\item[Compile-time Clause Selection] The specialised calls of the
	residual program  directly select (at compile time) a subset
	containing only the clauses that the corresponding calls of the
	original program would otherwise have to examine during their
	execution (at run time).  By doing so, laying down unnecessary
	choice points is at least partly avoided, and so is the need to
	select clauses through some sort of indexing.
\item[Factoring of Common Subterms] Non-variable subterms of partially
	instantiated calls that are common with subterms in the heads
	of the selected clauses are factored out from these terms
	during the specialisation process.  As a result, some head
	unification ({\tt get\_*} or {\tt unify\_*}) and some argument
	register ({\tt put\_*}) WAM instructions of the original
	program become unnecessary.  These instructions are eliminated
	from both the specialised calls as well as from the specialised
	versions of the predicates.
\end{description}
Though these properties are sufficient to get the idea behind
specialisation, the actual specialisation performed by the \ourprolog\
compiler can be better understood by the following example.  The
example shows the specialisation of a predicate that checks if a list
of HiLog terms is ordered:
\begin{center}
\tt
\begin{tabular}{ccc}
\begin{tabular}{l}
ordered([]). \\
ordered([X]). \\
ordered([X,Y|Z]) :- \\
\ \ \ \ X @=< Y, ordered([Y|Z]). 
\end{tabular}
& $\longrightarrow$ &
\begin{tabular}{l}
ordered([]). \\
ordered([X]). \\
ordered([X,Y|Z]) :- \\
\ \ \ \ X @=< Y, \_\$ordered(Y, Z). \\
\\
:- index \_\$ordered/2-2. \\
\_\$ordered(X, []). \\
\_\$ordered(X, [Y|Z]) :- \\
\ \ \ \ X @=< Y, \_\$ordered(Y, Z).
\end{tabular}
\end{tabular}
\end{center}
The transformation (driven by the partially instantiated call
{\tt ordered([Y|Z])}) effectively allows predicate {\tt ordered/2}
to be completely deterministic (when used with a proper list as its
argument), and to not use any unnecessary heap-space for its
execution.  We note that appropriate {\tt :- index} directives are
automatically generated by the \ourprolog\ compiler for all specialised
versions of predicates.

The default specialisation of partially instantiated calls is without
any folding of the clauses that the calls select.  Using the {\tt
spec\_repr} compiler option (see Section~\ref{compiler_options})
specialisation with replacement of the selected clauses with the
representative of these clauses is performed.  Using this compiler
option, predicate {\tt ordered/2} above would be specialised as follows:
\begin{center}
\begin{minipage}{4.1in}
\begin{verbatim}
ordered([]).
ordered([X|Y]) :- _$ordered(X, Y).

:- index _$ordered/2-2.
_$ordered(X, []).
_$ordered(X, [Y|Z]) :- X @=< Y, _$ordered(Y, Z).
\end{verbatim}
\end{minipage}
\end{center}
We note that in the presense of cuts or side-effects, the code
replacement operation is not always sound, i.e.  there are cases when
the original and the residual program are not computationally equivalent
(with respect to the answer substitution semantics).  The compiler
checks for sufficient (but not necessary) conditions that guarantee
computational equivalence, and if these conditions are not met,
specialisation is not performed for the violating calls.

The \ourprolog\ compiler prints out messages whenever it specialises
calls to some predicate.  For example, while compiling a file
containing predicate {\tt ordered/1} above, the compiler would print
out the following message:
\begin{center}
{\tt	\% Specialising partially instantiated calls to ordered/1}
\end{center}
The user may examine the result of the specialisation transformation
by using the {\tt spec\_dump} compiler option
(see Section~\ref{compiler_options}).

Finally, we have to mention that for technical reasons beyond the scope of
this document, specialisation cannot be transparent to the user; predicates
created by the transformation do appear during tracing.


\subsection{Compiler Directives}\label{compiler_directives}
\index{Compiler!directives}\index{directives!Compiler}
%=====================================================

The following compiler directives are recognized in \version\ of XSB
\footnote{Any parallelisation directives ({\tt parallel}) are simply ignored by the
compiler, but do not result in syntax errors to enhance compatibility with
various other earlier versions of PSB-Prolog.}.  

\subsubsection{Mode Declarations}\label{mode_declarations}
\index{modes!directives}\index{directives!modes}
%-----------------------------------------------------

The \ourprolog\ compiler accepts {\tt mode} declarations of the form:

\demo{:- mode $ModeAnnot_1, \ldots, ModeAnnot_n$.}

\noindent
where each $ModeAnnot$ is a {\em mode annotation\/} (a {\em term
indicator\/} whose arguments are elements of the set {\tt \{+,-,\#,?\}}).
From Version 1.4.1 on, {\tt mode} directives are used by the compiler for 
tabling directives, a use which differs from the standard use of modes
in Prolog systems\footnote{The most common uses of {\tt mode} declarations in 
	  Prolog systems are to reduce the size of compiled code,
	  or to speed up a predicate's execution.}.
See Section~\ref{tabling_directives} for detailed examples.

Mode annotations have the following meaning:
\begin{description}
\item[{\tt +}]
	This argument is an input to the predicate.  In every invocation
	of the predicate, the argument position must contain a non-variable
	term.  This term may not necessisarily be ground, but the 
	predicate is guaranteed not to alter this argument).

	\demo{:- mode see(+), assert(+).}
\item[{\tt -}]
	This argument is an output of the predicate.  In every
	invocation of the predicate the argument
	position {\em will always be a variable\/} (as opposed to 
	the {\tt \#} annotation below).
	This variable is unified with the value returned by the predicate.
	We note that Prolog does not enforce the requirement that output
	arguments should be variables; however, output unification is not
	very common in practice.

	\demo{:- mode cputime(-).}
\item[{\tt \#}]
	This argument is either:
	\begin{itemize}
	\item	An output argument of the predicate for which a non-variable
		value may be supplied for this argument position.  If such a
		value is supplied, the result in this position is unified with
		the supplied supplied value.  The predicate fails if this
		unification fails.  If a variable term is supplied, the
		predicate succeeds, and the output variable is unified with
		the return value.

		\demo{:- mode '='(\#,\#).}
	\item	An input/output argument position of a predicate that has
		only side-effects (usually by further instantiating that
		argument).  The {\tt \#} symbol is used to denote the $\pm$
		symbol that cannot be entered from the keyboard.
	\end{itemize}
\item[{\tt ?}]
	This argument does not fall into any of the above categories. 
        Typical cases would be the following:
	\begin{itemize}
	\item	An argument that can be used both as input and as output
		(but usually not with both uses at the same time).

		\demo{:- mode functor(?,?,?).}
	\item	An input argument where the term supplied can be a variable
		(so that the argument cannot be annotated as {\tt +}), or is
		instantiated to a term which itself contains uninstantiated
		variables, but the predicate is guaranteed {\em not\/} to
		bind any of these variables.

		\demo{:- mode var(?), write(?).}
	\end{itemize}
\end{description}
We try to follow these mode annotation conventions throughout this manual.

Finally, we warn the user that {\tt mode} declarations can be error-prone,
and since errors in mode declarations do not show up while running the
predicates interactively, unexpected behaviour may be witnessed in compiled
code, optimised to take modes into account (currently not performed by
\ourprolog).  However, despite this danger, {\tt mode} annotations can be
a good source of documentation, since they express the programmer's
intention of data flow in the program.


\subsubsection{Tabling Directives}\label{tabling_directives}
\index{tabling!directives}\index{directives!tabling}
%-----------------------------------------------------
\index{{\tt auto\_table}}
Memoization is often necessary to ensure that programs terminate, and
can be useful as an optimization strategy as well.  The underlying
engine of \ourprolog\ is based on SLG, a memoization strategy, which,
in our version, maintains a table of calls and their answers for each
predicate declared as {\em tabled}.  Predicates that are not declared
as tabled execute as in Prolog, eliminating the expense of tabling
when it is unnecessary.

The simplest way to use tabling is to include the directive

\demo{:- auto\_table.}

\noindent
anywhere in the source file.  {\tt auto\_table} declares predicates
tabled so that the program will terminate.

To understand precisely how {\tt auto\_table} does this, it is
necessary to mention a few properties of SLG.  For programs which have
no function symbols, or where function symbols always have a limited
depth, SLG resolution ensures that any query will terminate after it
has found all correct answers.  In the rest of this section, we
restrict consideration to such programs.

Obviously, not all predicates will need to be tabled for a program to
terminate.  The {\tt auto\_table} compiler directive tables only those
predicates of a module which appear to static analysis to contain an
infinite loop, or which are called directly through {\tt tnot/1}.  It
is perhaps more illuminating to demonstrate these conditions through
an example rather than explaining them.  For instance, in the program.

%tls commented out minipage because latex was formatting badly,
\begin{center}
%\begin{minipage}{3in}
\begin{verbatim}
:- auto_table. 

p(a) :- s(f(a)). 

s(X) :- p(f(a)).

r(X) :- q(X,W),r(Y).

m(X) :- tnot(f(X)).

:- mode ap1(-,-,+).
ap1([H|T],L,[H|L1]) :- ap1(T,L,L1).

:- mode ap(+,+,-).
ap([],F,F).
ap([H|T],L,[H|L1]) :- ap(T,L,L1).

mem(H,[H|T]).
mem(H,[_|T]) :- mem(H,T).
\end{verbatim}
%\end{minipage}
\end{center}

\noindent
The compiler prints out the messages
\begin{verbatim}
% Compiling predicate s/1 as a tabled predicate
% Compiling predicate r/1 as a tabled predicate
% Compiling predicate m/1 as a tabled predicate
% Compiling predicate mem/2 as a tabled predicate
\end{verbatim}

Terminating conditions were detected for {\tt ap1/3} and {\tt ap/3}, but
not for any of the other predicates.

{\tt auto\_table} gives an approximation of tabled programs which we
hope will be useful for most programs.  The minimal set of tabled
predicates needed to insure termination for a given program is
undecidible.  
\comment{
Practically, refining the set of tabled predicates
deduced by {\tt auto\_table} is still an open research problem.
}
It should be noted that the presence of meta-predicates such as {\tt
call/1} makes any static analysis useless, so that the {\tt
auto\_table} directive should not be used in such cases.

Predicates can be explicitly declared as tabled as well, through the 
{\tt table/[1,2,3]}.   When {\tt table/1} is used, the directive
takes the form

\demo{:- table(F/A).}

\noindent
where {\tt F} is the functor of the predicate to be tabled, and {\tt A} its
arity.  

\index{{\tt suppl\_table}}
Another use of tabling is to filter out redundant solutions for
efficiency rather than termination.  In this case, suppose that the
directive {\tt edb/1} were used to indicate that certain predicates were
likely to have a large number of clauses.  Then the action of the declaration
{\tt :- suppl\_table} in the program:
\begin{verbatim}
:- edb(r1/2).
:- edb(r2/2).
:- edb(r3/2).

:- suppl_table.

join(X,Z):- r1(X,X1),r2(X1,X2),r3(X2,Z).
\end{verbatim}
would be to table {\tt join/2}.  The {\tt suppl\_table} directive is
the XSB analogue to the deductive database optimization, {\em
supplementary magic templates} \cite{BeRa91}.  {\tt suppl\_table/0} is
shorthand for {\tt suppl\_table(2)} which tables all predicates
containing clauses with two or more {\tt edb} facts or tabled
predicates.  By specifying {\tt suppl\_table(3)} for instance, only
predicates containing clauses with three or more {\tt edb} facts or
tabled predicates would be tabled.  This flexibility can prove useful
for certain data-intensive applications.


\subsubsection{Indexing Directives}\label{indexing_directives}
\index{indexing!directives}\index{directives!indexing}
%-------------------------------------------------------------

The \ourprolog\ compiler usually generates an index on the principal 
functor of the first argument of a predicate.  Indexing on the appropriate 
argument of a predicate may significantly speed up its execution time.  
In many cases the first argument of a predicate may not be the most
appropriate argument for indexing and changing the order of arguments
may seem unnatural.  In these cases, the user may generate an index
on any other argument by means of an indexing directive.  This is a
directive of the form:

\demo{:- index Functor/Arity-IndexArg.}

\noindent
indicating that an index should be created for predicate 
{\tt Functor}/{\tt Arity} on its ${\tt IndexArg}^{\rm th}$ argument.
One may also use the form:

\demo{:- index(Functor/Arity, IndexArg, HashTableSize).}

\noindent
which allows further specification of the size of the hash table to use for
indexing this predicate if it is a {\em dynamic} (i.e., asserted) predicate.
For predicates that are dynamically loaded, this directive can be used to
specify indexing on more than one argument, or indexing on a combination
of arguments (see its description on page~\pageref{index_dynamic}).
For a compiled predicate the size of the hash table is computed automatically,
so {\tt HashTableSize} is ignored.

All of the values {\tt Functor}, {\tt Arity}, {\tt IndexArg} (and possibly
{\tt HashTableSize}) should be ground in the directive.  More specifically,
{\tt Functor} should be an atom, {\tt Arity} an integer in the range 0..255,
and {\tt IndexArg} an integer between 0 and {\tt Arity}.  If {\tt IndexArg}
is equal to~0, then no index is created for that predicate. An {\tt index}
directive may be placed anywhere in the file containing the predicate it
refers to.

As an example, if we wished to create an index on the third argument 
of predicate {\tt foo/5}, the compiler directive would be:

\demo{:- index foo/5-3.}


\subsubsection{Unification Factoring}\label{transformational_indexing}
\index{indexing!transformational}
When the clause heads of a predicate have portions of arguments common
to several clauses, indexing on the principal functor of one argument
may not be sufficient.  Indexing may be improved in such cases by the
use of unification factoring.  Unification Factoring is a program
transformation that ``factors out'' common parts of clause heads,
allowing differing parts to be used for indexing, as illustrated by
the following example:
\begin{center}
\tt
\begin{tabular}{ccc}
\begin{tabular}{l}
p(f(a),X) :- q(X). \\
p(f(b),X) :- r(X).
\end{tabular}
& $\longrightarrow$ &
\begin{tabular}{l}
p(f(X),Y) :- \_\$p(X,Y). \\
\_\$p(a,X) :- q(X). \\
\_\$p(b,X) :- r(X).
\end{tabular}
\end{tabular}
\end{center}
The transformation thus effectively allows $p/2$ to be indexed
on atoms $a/0$ and $b/0$.  Unification Factoring is transparent
to the user; predicates created by the transformation are internal
to the system and do not appear during tracing.

The following compiler directives control the use of unification
factoring:\footnote{Unification factoring was once called
transformational indexing, hence the abbreviation {\tt ti} in the
compiler directives}.
\begin{description}
\item[{\tt :- ti(F/A).}] Specifies that predicate $F/A$ should be
	compiled with unification factoring enabled.
\item[{\tt :- ti\_off(F/A).}] Specifies that predicate $F/A$ should be
	compiled with unification factoring disabled.
\item[{\tt :- ti\_all.}] Specifies that all predicates defined in the
	file should be compiled with unification factoring enabled.
\item[{\tt :- ti\_off\_all.}] Specifies that all predicates defined in
	the file should be compiled with unification factoring disabled.
\end{description}
By default, higher-order predicates (more precisely, predicates named
{\it apply\/} with arity greater than 1) are compiled with unification
factoring enabled.  It can be disabled using the {\tt ti\_off}
directive.  For all other predicates, unification factoring must be
enabled explicitly via the {\tt ti} or {\tt ti\_all} directive.  If
both {\tt :- ti(F/A).} ({\tt :- ti\_all.}) and {\tt :- ti\_off(F/A).}
({\tt :- ti\_off\_all.}) are specified, {\tt :- ti\_off(F/A).} ({\tt
:- ti\_off\_all.}) takes precedence.  Note that unification factoring
may have no effect when a predicate is well indexed to begin
with.  For example, unification factoring has no effect on the
following program:
\begin{center}
\tt
\begin{tabular}{l}
p(a,c,X) :- q(X). \\
p(b,c,X) :- r(X).
\end{tabular}
\end{center}
even though the two clauses have $c/0$ in common.  The user may
examine the results of the transformation by using the {\tt ti\_dump}
compiler option (see Section~\ref{compiler_options}).

\subsubsection{Other Directives} \label{other-directives}
%==============================================

XSB has other directives not found in other Prolog systems.

\begin{description}
\desc{:- hilog $atom_1, \ldots, atom_n$.}
	Declares symbols $atom_1$ through $atom_n$ as HiLog symbols.
	The {\tt hilog} declaration should appear {\em before} any use of
	the symbols.  See Chapter~\ref{Syntax} for a purpose of this
 	declaration.
\desc{:- ldoption($Options$).}
        This directive is only recognized in the header file ({\tt .H} file) 
	of a foreign module. See Chapter~\ref{foreign} for its explanation.
\end{description}

\subsection{Inline Predicates}\label{inline_predicates}
\index{Compiler!inlines}\index{inlines!Compiler}
%======================================================

{\em Inline predicates} represent ``primitive'' operations in the
WAM.  Calls to inline predicates are compiled into a sequence of WAM
instructions in-line, i.e. without actually making a call to the
predicate.  Thus, for example, relational predicates (like {\tt >/2},
{\tt >=/2}, etc.) compile to, essentially, a subtraction followed by
a conditional branch.  Inline predicates are expanded specially by
the compiler and thus {\em cannot be redefined by the user without
changing the compiler}.  The user does not need to import these
predicates from anywhere.  There are available no matter what options
are specified during compiling.

Table~\ref{inlinepredicatetable} lists the inline predicates of
\ourprolog\ \version.  Those predicates that start with \verb|_$|
are internal predicates that are also expanded in-line during
compilation.

\begin{table}[htbp]\centering{\tt
\begin{tabular}{lllll}
\verb|'='/2|    &\verb|'<'/2|	&\verb|'=<'/2|  &\verb|'>='/2| &\verb|'>'/2| \\
\verb|'=:='/2|  &\verb|'=\='/2|	&is/2           &\verb|'@<'/2| &\verb|'@=<'/2|\\
\verb|'@>'/2|	&\verb|'@>='/2|	&\verb|'=='/2|	&\verb|'\=='/2|&fail/0  \\
true/0		&var/1		&nonvar/1	&halt/0	       &'!'/0   \\
'\_\$cutto'/1	&'\_\$savecp'/1	&'\_\$builtin'/1
\end{tabular}}
\caption{The Inline Predicates of \ourprolog}\label{inlinepredicatetable}
\end{table}

We warn the user to be very cautious when defining predicates whose functor
starts with \verb|_$|
%%$
since the names of these predicates may interfere with
some of \ourprolog's internal predicates.  The situation may be particularly
severe for predicates like {\tt '\_\$builtin'/1} that are treated specially
by the \ourprolog\ compiler.


%%% Local Variables: 
%%% mode: latex
%%% TeX-master: "manual1"
%%% End: 
