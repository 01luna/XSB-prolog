\chapter{Exception Handling}\label{chap:exception}
\index{exceptions}

We use the term {\em exceptions} to define errors in program execution
that are handled by a non-local change in execution state.  The
preferred mechanism for dealing with exceptions in XSB is to use the
predicates {\tt catch/3}, {\tt throw/1}, and {\tt
default\_user\_error\_handler/1} together.  These predicates are
ISO-compatable, and their use can give a great deal of control to
exception handling.  At a high level, when an exception is encountered
an error term $T$ is {\em thrown}.  Throwing an error term $T$ causes
XSB to examine its choice point stack until it finds a {\em catcher}
that unifies with $T$.  This catcher then calls a {\em handler}.  If
no explicit catcher for $T$ exists, a default handler is invoked,
which usually results in an abort, and returns execution to the
top-level of the interpreter.

More precisely, a handler is set up when {\tt
catch(Goal,Catcher,Handler)} is called.  At this point a continuation
is saved (i.e. a Prolog choice point), and {\tt Goal} is called.  If
no exceptions are encountered, answers for {\tt Goal} are obtained as
usual.  Within the execution of {\tt Goal}, an exception can be
signaled by a call to {\tt throw(Error)}.  This predicate searches
for an ancestor of the current environment called by {\tt catch/3} and
whose catcher (second argument) unifies with {\tt Error}.  If such an
ancestor is found, program execution reverts to the ancestor and all
intervening choice points are removed.  The ancestor's handler (third
argument) is called and the exception is thereby handled.  On the
other hand, if no ancestor was called using {\tt catch/3} the system
checks whether a clause with head {\tt
default\_user\_error\_handler(Term)} has been asserted, such that {\tt
Term} unifies with {\tt Error}.  If so, this handler is executed.  If
not, XSB's default system error handler in invoked an error message is
output and execution returns to the top level of the interpreter.

The following, somewhat fanciful example, helps clarify these
concepts~\footnote{Code for this example can be found in {\tt
\$XSBDIR/examples/exceptions.P}.}.  Consider the predicate {\tt
userdiv/2} (Figure~\ref{fig:userdiv}) which is designed to be called
with the first argument instantiated to a number.  A second number is
then read from a console, and the first number is divided by the
second, and unified with the second argument of {\tt userdiv/2}.  By
using {\tt catch/3} and {\tt throw/1} together the various types of
errors can be caught.

%------------------------------------------------------------------------------------------
\begin{figure}[hbtp]
\longline
\begin{small}
\begin{verbatim}
:- import error_writeln/1 from standard.
:- import type_error/4 from error_handler.

userdiv(X,Ans):- 
        catch(userdiv1(X,Ans),mydiv1(Y),handleUserdiv(Y,X)).

userdiv1(X,Ans):- 
        (number(X) -> true; type_error(number,X,userdiv1/2,1)),
        write('Enter a number: '),read(Y),
        (number(Y) -> true ; throw(mydiv1(error1(Y)))),
        (Y < 0 -> throw(mydiv1(error2(Y))); true),
        (Y =:= 0 -> throw(error(zerodivision,userdiv/1)); true),
        Ans is X/Y.

handleUserdiv(error1(Y),_X):- 
        error_writeln(['a non-numeric denominator was entered in userdiv/1: ',Y]),fail.
handleUserdiv(error2(Y),_X):- 
        error_writeln(['a negative denominator was entered in userdiv/1: ',Y]),fail.
\end{verbatim}
\end{small}
\caption{The {\tt userdiv/1} program} \label{fig:userdiv}
\longline
\end{figure}
%------------------------------------------------------------------------------------------

The behavior of this program on some representative inputs is shown
below.

\begin{small}
\begin{verbatim}
| ?- userdiv(p(1),F).
++Error[XSB/Runtime/P]: [Type (p(1) in place of number)]  in arg 1 of predicate userdiv1/2
Forward Continuation...
... machine:xsb_backtrace/1
... error_handler:type_error/4
... standard:call/1
... x_interp:_$call/1
... x_interp:call_query/1
... standard:call/1
... standard:catch/3
... x_interp:interpreter/0
... loader:ll_code_call/3
... standard:call/1
... standard:catch/3

no
| ?- userdiv(3,F).
Enter a number: foo.
a non-numeric denominator was entered in userdiv/1: foo

no
|| ?- userdiv(3,F).
Enter a number: -1.
a negative denominator was entered in userdiv/1: -1

no
| ?- userdiv(3,Y).
Enter a number: 2.

Y = 1.5000

yes
\end{verbatim}
\end{small}

\noindent
Note, however the following behavior.

\begin{small}
\begin{verbatim}
| ?- userdiv(3,F).
Enter a number: 0.
++Error[XSB/Runtime/P] uncaught exception: error(zerodivision,userdiv / 1)
Aborting...
\end{verbatim}
\end{small}

\noindent
By examining the program above, it can be seen that if {\tt p(1)} is
entered, the predicate {\tt type\_error/3} is called.  {\tt
  type\_error/3} is an XSB mechanism to throw an ISO-style type error
from Prolog.  Such an error is known to the default system error
handler which prints out a message along with a {\em backtrace} that
indicates the calling context in which the error arose (see
Section~\ref{sec:backtrace}).  Alternately, in the second case, when
{\tt -1} is entered, the error term {\tt mydiv1(error2(-1))} is thrown,
which is caught within {\tt userdiv/2} and handled by {\tt
  handleUserdiv/2}.  Finally, when {\tt 0} is entered for the
denominator, an error term of the form {\tt
  error(zerodivision,userdiv/1)} is thrown, and that this term does
not unify with the second argument of the {\tt catch/3} literal in the
body of {\tt userdiv/1}, or with a known ISO error.  The error is
instead caught by XSB's default system error handler which prints an
uncaught exception message and aborts to the top level of the
interpreter.  

XSB's default system error handler recognizes certain error formats
(see Section \ref{sec:iso-errors}), and handles the rest as uncaught
exceptions, and takes actions that are often appropriate.  On the
other hand, there may be times when an application would like special
default handling: perhaps the application calls XSB from C, so that
aborts are not practical.  Alternately, perhaps XSB is being called
from a graphical user interface via Interprolog~\cite{Cale01} or some
other interface, and in addition to a special abort handling, one
would like to display an error window.  In these cases it is
convenient to make use of the dynamic predicate {\tt
  default\_user\_error\_handler/1}.  {\tt
  default\_user\_error\_handler/1} is called immediately before the
default system error handler, and after it is ascertained that no
catcher for an error term is available via a {\tt catch/3} ancestor.

Accordingly, suppose the following clause is asserted into {\tt
usermod}:
%
\begin{small}
\begin{verbatim}
?- assert((default_user_error_handler(error(zerodivision,Pred)):- 
        error_writeln(['Aborting: division by 0 in: ',Pred]))).
\end{verbatim}
\end{small}
%
The behavior will now be
\begin{small}
\begin{verbatim}
| ?- userdiv(4,F).
Enter a number: 0.
Aborting: division by 0 in: userdiv / 1
\end{verbatim}
\end{small}
The actions of {\tt catch/3} and {\tt throw/1} resemble that of the
Prolog cut in that they remove choice points that lie between a call
to {\tt throw/1} and the matching {\tt catch/3} that serves as its
ancestor.  However, if this process encounters a choice point for an
incomplete table, execution is aborted to the top user level.

\section{Representations of ISO Errors} \label{sec:iso-errors}
\index{ISO!errors}

All exceptions that occur during the execution of an XSB program can
be caught.  However, by structuring error terms in a consistent
manner, different classes of errors can be handled much more easily by
user-defined handlers.  This philosophy partly underlies the ISO
Standard for defining classes of Prolog errors \cite{ISO-Prolog}.
While the ISO standard defines various types of errors and how they
should arise during execution of ISO Prolog predicates, it does not
define the actual error terms a system should use.  Accordingly, we
define the formats for various ISO errors~\footnote{We note that XSB's
  system predicates are in the process of being updated to handle
  these errors.}.  Below, in Section~\ref{sec:errorpredicates} we
provide convenience predicates for throwing various ISO errors and
performing various error checks.

In the following predicates, {\tt Msg} is either a list of HiLog terms
or a comma-list of HiLog terms.  Each of the {\tt error/2} terms below
can also be represented as {\tt error/3} terms, where the third
argument is instantiated to the representation of a backtrace.

\begin{description}
\item [error(domain\_error(Valid\_type,Culprit),Msg)] is the format of
  an ISO type error, where {\tt Valid\_type} is the domain expected
  and {\tt Culprit} is the term observed.  Unlike types, domains can
  be user-defined.
%
\item[error(evaluation\_error(Flag),Msg)] is the format of an ISO
  evaluation error (e.g. overflow or underflow), and {\tt Flag} is the
  type of evaluation error encountered.
%
\item [error(existence\_error(Type,Culprit),Msg)] is the format of an
  ISO type error, where {\tt Type} is the type of a resource and {\tt
    Culprit} is the term observed.
%
\item[error(instantiation\_error,Msg))] is the format of an ISO
  instantiation error.
%
\item [error(permission\_error(Op,Obj\_type,Culprit).Msg)] is the format of
  an ISO permission error, for an operation {\tt Op} applied to an
  object of type {\tt Obj\_type}, where {\tt Culprit} was observed.
%
\item[error(representation\_error(Flag).Msg)] is the format of an ISO
  representation error (e.g. the maximum arity of a predicate has been
  exceeded), and {\tt Flag} is the type of representation error
  encountered.
%
\item[error(resource\_error(Flag).Msg)] is the format of an ISO
  resource error (e.g. too many files are opened), and {\tt Flag} is
  the type of resource error encountered.
%
\item[error(syntax\_error,Msg)] is the format of an ISO syntax error.
%
\item[error(system\_error(Flag),Msg)] is the format of an ISO system error,
  and {\tt Flag} is the type of system error encountered.
%
\item[error(type\_error(Valid\_type,Culprit),Msg)] is the format of an
  ISO type error, where {\tt Valid\_type} is the type expected and
  {\tt Culprit} is the term observed.  This should be used for checks
  of Prolog types only (i.e. integers, floats, atoms, etc.)
%
\end{description}

In addition, XSB's engine also makes use of some other types of
errors.
%
\begin{description}
\item[error(table\_error,Msg)] is the format of an error arising when
  using XSB's tabling mechanism.
%
\item[error(misc\_error,Msg)] is the format of an error that is not
  otherwise classified.
%
\item[error(thread\_cancel,Id)] is the format of an error ball for a
  thread that has been cancelled by XSB thread {\tt Id} (See
  Chapter~\ref{chap:threads} for details on thread cancellation.)
%
\end{description}

In \version{} of XSB, errors for ISO predicates are not always ISO
compliant.  First, when XSB determines it is out of available memory,
recovering from such an error may be difficult at best.  Accordingly
the computation is aborted in the sequential engine, or XSB exits in
the multi-threaded engine.  Second, errors in XSB code sometimes arise
as miscellaneous errors rather than as a designated ISO-error type.
We are, however, in the process of reclassifying errors to their ISO
types.

When XSB generates a memory exception, it prints out a backtrace and
exits.  This should be caused only by a bug in XSB or included code.
The first predicate in the backtrace that is printed in these
circumstances may be incorrect or redundant.  This is because the
memory structures used to generate the backtrace are not always
completely consistent, and so an interrupt at an unexpected point may
result in the use of somewhat inconsistent information.

%---------------------------------------------------------------------------------------
\section{Predicates to Throw and Handle Errors}
\label{sec:errorpredicates}

\subsection{Predicates to Throw Errors}

XSB provides a variety of predicates that throw errors~\footnote{C
  functions for throwing terms and ISO-style errors are described in
  Volume 2, Chapter 3 {\em Foreign Language Interface}.}. Those likely
to be of interest to users are:
\begin{description}
\isoitem{throw(+ErrorTerm)}
%
Throws the error {\tt ErrorTerm}.  Execution traverses up the choice
point stack until a goal of the form {\tt catch(Goal,Term,Handler)} is
found such that {\tt Term} unifies with {\tt ErrorTerm}.  In this
case, {\tt Handler} is called.  If no catcher is found, the system
looks for a clause of {\tt default\_user\_error\_handler(Term)} such
that {\tt Term} unifies with {\tt ErrorTerm}.  Finally, if no such
clause is found the default system error handler is called.  {\tt
  throw/1} is most useful in conjunction with specialized handlers for
new types of errors not already supported in XSB.
%
\ourmoditem{domain\_error(+Valid\_type,-Culprit,+Predicate,+Arg)}{domain\_error/4}{error\_handler}
%
Throws a domain error.  Using the default system error handler, an
example (with {\tt backtrace\_on\_error} set to off) is {\small
\begin{verbatim}
domain_error(posInt,-1,checkPosInt/3,3).
++Error[XSB/Runtime/P]: [Domain (-1 not in domain posInt)] in arg 3 of predicate checkPosInt/3
\end{verbatim} }
%
\ourmoditem{evaluation\_error(+Flag,+Predicate,+Arg)}{evaluation\_error/3}{error\_handler}
%
Throws an evaluation error.  Using the default system error handler, an
example (with {\tt backtrace\_on\_error} set to off) is {\small
\begin{verbatim}
evaluation_error(zero_divisor,unidiv/1,2).
++Error[XSB/Runtime/P]: [Evaluation (zero_divisor)] in arg 2 of predicate unidiv/2
\end{verbatim} }
%
\ourmoditem{existence\_error(+Object\_type,?Culprit,+Predicate,+Arg)}{existence\_error/4}{error\_handler}
%
Throws an existence error.  Using the default system error
handler, an example (with {\tt backtrace\_on\_error} set to off) is 
{\small 
\begin{verbatim}
existence_error(file,'myfile.P','load_intensional_rules/2',2).
++Error[XSB/Runtime/P]: [Existence (No file myfile.P exists)]  in arg 2 of predicate load_intensional_rules/2
\end{verbatim}
}
%
\ourmoditem{instantiation\_error(+Predicate,+Arg,+State)}{instantiation\_error/4}{error\_handler}
%
Throws an instantiation error.  Using the default system error
handler, an example (with {\tt backtrace\_on\_error} set to off) is 
{\small 
\begin{verbatim}
?- instantiation_error(foo/1,1,nonvar).
++Error[XSB/Runtime/P]: [Instantiation]  in arg 1 of predicate foo/1: must be nonvar
\end{verbatim}
}
%
\ourmoditem{permission\_error(+Op,+Obj\_type,?Culprit,+Predicate)}{permission\_error/4}{error\_handler}
%
Throws a permission error.  Using the default system error
handler, an example (with {\tt backtrace\_on\_error} set to off) is 
{\small 
\begin{verbatim}
| ?- permission_error(write,file,'myfile.P',foo/1).
++Error[XSB/Runtime/P]: [Permission (Operation) write on file: myfile.P]  in foo/1
\end{verbatim}
}
%
\ourmoditem{representation\_error(+Flag,+Predicate,+Arg)}{representation\_error/3}{error\_handler}
% 
Throws a representation error.  Using the default system error handler, an
example (with {\tt backtrace\_on\_error} set to off) is {\small
\begin{verbatim}
representation_error(max_arity,assert/1,1).
++Error[XSB/Runtime/P]: [Representation (max_arity)] in arg 1 of predicate assert/1
\end{verbatim} }
%
\ourmoditem{resource\_error(+Flag,+Predicate)}{resource\_error/3}{error\_handler}
%
Throws a resource error.  Using the default system error handler, an
example (with {\tt backtrace\_on\_error} set to off) is {\small
\begin{verbatim}
resource_error(open_files,open/3)
++Error[XSB/Runtime/P]: [Resource (open_files)] in predicate open/3
\end{verbatim} }
%
\ourmoditem{type\_error(+Valid\_type,-Culprit,+Predicate,+Arg)}{type\_error/4}{error\_handler}
%
Throws a type error.  Using the default system error
handler, an example (with {\tt backtrace\_on\_error} set to off) is 
{\small 
\begin{verbatim}
| ?- type_error(atom,f(1),foo/1,1).
++Error[XSB/Runtime/P]: [Type (f(1) in place of atom)]  in arg 1 of predicate foo/1
\end{verbatim}
}
%
\ourmoditem{misc\_error(+Message)}{misc\_error/3}{error\_handler}
%
Throws a miscellaneous error that will
be caught by the default system handler.  For good programming
practice miscellaneous errors should only be thrown when the cases
above are not applicable, and the type of error is not of interest for
structured error handling.  Such situations occur can occur for
instance in debugging, during program development, or in small-special
purpose programs.  Note that this {\tt misc\_error/2} replaces the
obsolescent {\tt abort/1} and {\tt abort/2}.
%
\end{description}
%
% \ouritem{abort}\index{\texttt{abort/0}}
% Abandons the current execution and returns to the top level.  This
%     predicate should normally normally be used: 
% \begin{itemize} 
%\item when a non-ISO exception has occurred and the user wishes to
%abort the computation to the top-level of the interpreter.  
%
%\item {\em and} the type of the error is not of interest for
%structuring error handling.
%\end{itemize}
%
%Such situations occur can occur for instance in debugging, during
%program development, or in small-special purpose programs.
%
% \ouritem{abort(+Message)}\index{\texttt{abort/1}} \index{\texttt{STDERR}}
%    Acts as {\tt abort/0} but sents {\tt Message} to {\tt STDERR}
%    before aborting.

\subsection{Predicates to Handle Errors}~\label{sec:catch}

\begin{description}
\isoitem{catch(?Goal,?CatchTerm,+Handler)}{catch/3}
%
Calls {\tt Goal}, and sets up information so that future throws will
be able to access {\tt CatchTerm} under the mechanism mentioned
above. {\tt catch/3} does not attempt to clean up system level
resources.  Thus, it is left up to the handler to close open tables
(via {\tt close\_open\_tables/0}, close any open files, reset current
input and output, and so on \footnote{cf. the default system error
  handler, which performs these functions, if needed.}.
%
\standarditem{default\_user\_error\_handler(?CatchTerm)}{default\_user\_error\_handler/1}
%
Handles any error terms that unify with {\tt CatchTerm} that are not
caught by invocations of {\tt catch/3}.  This predicate {\em does}
close open tables and release mutexes held by the calling thread, but
does not attempt to clean up other system level resources, which is
left to the handler.
%
\ourrepeatmoditem{error\_write(?Message)}{error\_write/1}{standard}

\ourmoditem{error\_writeln(?Message)}{error\_writeln/1}{standard}

Utility routines for user-defined error catching.  These predicates
output {\tt Message} to XSB's {\tt STDERR} stream, rather than to
XSB's {\tt STDOUT} stream, as does {\tt write/1} and {\tt writeln/1}.
In addition, if {\tt Message} is a comma list, the elements in the
comma list are output as if they were concatenated together.  Each of
these predicates must be implicitly from the module {\tt standard}.

\ourmoditem{close\_open\_tables}{close\_open\_tables/0}{machine}
%
Removes table data structures for all incomplete tables, but does not
affect any incomplete tables.  In \version{} this predicate should
only be used to handle exceptions in {\tt
  default\_user\_error\_handler/1}.  In addition, for the
multi-threaded engine, this predicate unlocks any system mutexes held
by the thread calling this predicate.

\end{description}

%----------------------------------------------------------------------------
\section{Convenience Predicates}

The following convenience predicates are provided to make a commonly
used check and throw an ISO error if the check is not satisfied.  All
these predicates must be imported from the module {\tt
  error\_handler}.

\begin{description}
\ourmoditem{check\_atom(?Term,+Predicate,+Arg)}{check\_atom/3}{error\_handler}
%
Checks that {\tt Term} is an atom.  If so, the predicate succeeds;
if not it throws a type error.

\ourmoditem{check\_ground(?Term,+Predicate,+Arg)}{check\_ground/3}{error\_handler}
%
Checks that {\tt Term} is ground.  If so, the predicate succeeds;
if not it throws an instantiation error.

\ourmoditem{check\_integer(?Term,+Predicate,+Arg)}{check\_integern/3}{error\_handler}
%
Checks that {\tt Term} is an integer.  If so, the predicate succeeds;
if not it throws a type error.

\ourmoditem{check\_nonvar(?Term,+Predicate,+Arg)}{check\_nonvar/3}{error\_handler}
%
Checks that {\tt Term} is not a variable.  If not, the predicate succeeds;
if {\tt Term} is a variable,  it throws an instantiation error.

\ourmoditem{check\_var(?Term,+Predicate,+Arg)}{check\_var/3}{error\_handler}
%
Checks that {\tt Term} is a variable.  If so, the predicate succeeds;
if not it throws an instantiation error.

\ourmoditem{check\_nonvar\_list(?Term,+Predicate,+Arg)}{check\_nonvar\_list/3}{error\_handler}

Checks that {\tt Term} is a list, each of whose elements is ground.
If so, the predicate succeeds; if not it throws an instantiation
error.
	    
\ourmoditem{check\_stream(?Stream,+Predicate,+Arg)}{check\_stream/3}{error\_handler}
%
Checks that {\tt Stream} is a stream.  If so, the predicate succeeds;
if not it throws an instantiation error~\footnote{The representation
of streams in XSB is subject to change.}.

\ourmoditem{check\_one\_thread(+Operation,+Object\_Type,+Predicate)}{check\_one\_thread/3}{error\_handler}
%
In the Multi-Threaded Engine, checks that there is only one active
thread: if not, a miscellaneous error is thrown indicating that {\tt
  Operation} is not permitted on {\tt ObjectType} as called by {\tt
  Predicate}, when more than one thread is active.  This check
provides a convenient way to allow inclusion of certain operations
that are difficult to make thread-safe by other means.

In the single-threaded engine this predicate always succeeds.

\end{description}

%----------------------------------------------------------------------------
\section{Backtraces}
\label{sec:backtrace}
\index{permanent variables}

Displaying a backtrace of the calling context of an error in addition
to an error message can greatly expedite debugging.  For XSB's default
error handler, backtraces are printed out by default, a behavior that
can be overridden for a given thread by the command: {\tt
  set\_xsb\_flag(backtrace\_on\_error,off)}.  For users who write
their own error handlers, the following predicates can be used to
manipulate backtraces.

It is important to note that Prolog backtraces differ in a significant
manner from backtraces obtained from other languages, such as C
backtraces produced by GDB.  This is because a Prolog backtrace
obtains forward continuations from the local environment stack, and in
the WAM, local stack frames are only created when a given clause
requires permanent variables -- otherwise these stack frames are
optimized away.  The precise conditions for optimizing away a local
stack frame require an understanding of the WAM (and of a specific
compiler).  However in general, longer clauses with many variables
require a local stack frame and their forward continuations will be
displayed, while shorter clauses with fewer variables do not and their
forward continuations will not be displayed.

\begin{description}
\ourmoditem{xsb\_backtrace(-Backtrace)}{xsb\_backtrace/1}{machine}
%
Upon success {\tt Backtrace} is bound to a structure indicating the
forward continuations for a point of execution.  This structure should
be treated as opaque, and manipulated by one of the predicates below.
%
\ourmoditem{get\_backtrace\_list(+Backtrace,-PredicateList)}{get\_backtrace\_list/2}{error\_handler}
%
Given a backtrace structure, this predicate produces a list of
predicate identifiers or the form {\tt Module:Predicate/Arity}.  This
list can be manipulated as desired by error handling routines.
%
\ourmoditem{print\_backtrace(+Backtrace)}{print\_backtrace/1}{error\_handler}
%
 This predicate, which is used by XSB's default error handler, prints
 a backtrace structure to {\tt STDMSG}.
\end{description}
