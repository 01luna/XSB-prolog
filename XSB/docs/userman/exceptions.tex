\chapter{Exception Handling}\label{exception}
\index{exceptions}

We use the term {\em exceptions} to define errors in program execution
that are handled by a non-local change in execution state.  The
preferred mechanism for dealing with exceptions in XSB is to use the
predicates {\tt catch/3}, {\tt throw/1}, and {\tt
default\_user\_error\_handler/1} together.  These predicates are
ISO-compatable, and their use can give a great deal of control to
exception handling.  At a high level, when an exception is encountered
an error term $T$ is {\em thrown}.  Throwing an error term $T$ causes
XSB to examine its choice point stack until it finds a {\em catcher}
that unifies with $T$.  This catcher then calls a {\em handler}.  If
no explicit catcher for $T$ exists, a default handler is invoked,
which usually results in an abort, and returns execution to the
top-level of the interpreter.

More precisely, a handler is set up when {\tt
catch(Goal,Catcher,Handler)} is called.  At this point a continuation
is saved (i.e. a Prolog choice point), and {\tt Goal} is called.  If
no exceptions are encountered, answers for {\tt Goal} are obtained as
usual.  Within the execution of {\tt Goal}, an exception can be
signalled by a call to {\tt throw(Error)}.  This predicate searches
for an ancestor of the current environment called by {\tt catch/3} and
whose catcher (second argument) unifies with {\tt Error}.  If such an
ancestor is found, program execution reverts to the ancestor and all
intervening choice points are removed.  The ancestor's handler (third
argument) is called and the exception is thereby handled.  On the
other hand, if no ancestor was called using {\tt catch/3} the system
checks whether a clause with head {\tt
default\_user\_error\_handler(Term)} has been asserted, such that {\tt
Term} unifies with {\tt Error}.  If so, this handler is executed.  If
not, XSB's default system error handler in invoked an error message is
output and execution returns to the top level of the interpreter.

The following, somewhat fanciful example, helps clarify these
concepts~\footnote{Code for this example can be found in {\tt
\$XSBDIR/examples/exceptions.P}.}.  Consider the predicate {\tt
userdiv/2} (Figure~\ref{fig:userdiv}) which is designed to be called
with the first argument instantiated to a number.  A second number is
then read from a console, and the first number is divided by the
second, and unified with the second argument of {\tt userdiv/2}.  By
using {\tt catch/3} and {\tt throw/1} together the various types of
errors can be caught.

%------------------------------------------------------------------------------------------
\begin{figure}[hbtp]
\longline
\begin{small}
\begin{verbatim}
:- import error_writeln/1 from standard.

userdiv(X,Ans):- 
        catch(userdiv1(X,Ans),mydiv1(Y),handleUserdiv(Y,X)).

userdiv1(X,Ans):- 
        (number(X) -> true; throw(mydiv1(error1))),
        write('Enter a number: '),read(Y),
        (number(Y) -> true ; throw(mydiv1(error2(Y)))),
        (Y < 0 -> throw(mydiv1(error3(Y))); true),
        (Y =:= 0 -> throw(error(zerodivision,userdiv/1)); true),
        Ans is X/Y.

handleUserdiv(error1,X):- 
        error_writeln(['userdiv/1 called with non-numeric numerator: ',X]),fail.
handleUserdiv(error2(Y),_X):- 
        error_writeln(['a non-numeric denominator was entered in userdiv/1: ',Y]),fail.
handleUserdiv(error3(Y),_X):- 
        error_writeln(['a negative denominator was entered in userdiv/1: ',Y]),fail.
\end{verbatim}
\end{small}
\caption{The {\tt userdiv/1} program} \label{fig:userdiv}
\longline
\end{figure}
%------------------------------------------------------------------------------------------

The behavior of this program on some representative inputs is shown
below.

\begin{small}
\begin{verbatim}
| ?- userdiv(p(1),F).
userdiv/1 called with non-numeric numerator: p(1)

no
| ?- userdiv(3,F).
Enter a number: foo.
a non-numeric denominator was entered in userdiv/1: foo

no
|| ?- userdiv(3,F).
Enter a number: -1.
a negative denominator was entered in userdiv/1: -1

no
| ?- userdiv(3,Y).
Enter a number: 2.

Y = 1.5000

yes
\end{verbatim}
\end{small}

\noindent
Note, however the following behavior.

\begin{small}
\begin{verbatim}
| ?- userdiv(3,F).
Enter a number: 0.
++Error[XSB/Runtime/P] uncaught exception: error(zerodivision,userdiv / 1)
Aborting...
\end{verbatim}
\end{small}

\noindent
By examining the program above, it can be seen that if {\tt 0} is
entered, an error term of the form {\tt error(zerodivision,userdiv/1)}
is thrown, and that this term does not unify with the second argument
of the {\tt catch/3} literal in the body of {\tt userdiv/1}.  The
error is instead caught by XSB's default system error handler which
prints out the above message and aborts to the top level of the
interpreter.  XSB's default system error handler does recognize
certain error formats (see Section \ref{}), and even if it doesn't an
error message like the one above may be adequate in many situations.
On the other hand, there may be times when an application would like
special default handling: perhaps the application calls XSB from C, so
that aborts are not practical.  Alternately, perhaps XSB is being
called from a graphical user interface via Interprolog~\cite{Cale01}
or some other interface, and in addition to a special abort handling,
one would like to display an error window.  In these cases it is
convenient to make use of the dynamic predicate {\tt
default\_user\_error\_handler/1}.  {\tt
default\_user\_error\_handler/1} is called immediately before the
default system error handler, and after it is ascertained that no
catcher for an error term is available via a {\tt catch/3} ancestor.

Accordingly, suppose the following clause is asserted into {\tt
usermod}:
%
\begin{small}
\begin{verbatim}
?- assert((default_user_error_handler(error(zerodivision,Pred)):- 
        error_writeln(['Aborting: division by 0 in: ',Pred]))).
\end{verbatim}
\end{small}
%
The behavior will now be
\begin{small}
\begin{verbatim}
| ?- userdiv(4,F).
Enter a number: 0.
Aborting: division by 0 in: userdiv / 1
\end{verbatim}
\end{small}
The actions of {\tt catch/3} and {\tt throw/1} resemble that of the
Prolog cut in that they remove choice points that lie between a call
to {\tt throw/1} and the matching {\tt catch/3} that serves as its
ancestor.  However, if this process encounters a choice point for an
incomplete table, execution is aborted to the top user level.

\section{Representations of ISO Errors}

All exceptions that occur during the execution of an XSB program can
be caught.  However, by structuring error terms in a consistent
manner, different classes of errors can be handled much more easily by
user-defined handlers.  This philosophy partly underlies the ISO
Standard for defining classes of Prolog errors \cite{ISO-Prolog}.
While the ISO standard defines various types of errors and how they
should arise during execution of ISO Prolog predicates, it does not
define the actual error terms a system should use.  Accordingly, we
define the formats for various ISO errors~\footnote{We note that XSB's
system predicates are in the process of being updated to handle these
errors.}.  Below, in Section~\ref{} we provide convenience predicates
for throwing various ISO errors and performing various error checks.

\begin{description}
\item[error(instantiation\_error,Msg)] is the format of an ISO
instantiation error.  
%
\item[error(type\_error(Valid\_type,Culprit,Msg)] is the format of an ISO
type error, where {\tt Valid\_type} is the type expected and {\tt
Culprit} is the term observed.  This should be used for checks of
Prolog types only (i.e. integers, floats, atoms, etc.)
%
\item [error(domain\_error(Valid\_type,Culprit,Msg)] is the format of an ISO
type error, where {\tt Valid\_type} is the domain expected and {\tt
Culprit} is the term observed.  Unlike types, domains can be
user-defined.
%
\item [error(existence\_error(Type,Culprit,Msg)] is the format of an ISO
type error, where {\tt Type} is the type of a resource and {\tt
Culprit} is the term observed.  
%
\item [error(undefined\_predicate(Name,Arity,Module)] is the format of
an ISO undefined predicate error, for the call to {\tt Name/Arity} in
{\tt Module}.
%
\item [error(permission\_error(Op,Obj\_type,Culprit)] is the format of
an ISO permission error, for an operation {\tt Op} applied to an
object of type {\tt Obj\_type}, where {\tt Culprit} was observed.
%
\end{description}

%---------------------------------------------------------------------------------------------
\section{Error Handling Predicates}

\begin{description}
\ouritem{throw(ErrorTermt}\index{{\tt throw/1}} 
Throws the error {\tt ErrorTerm}.  Execution traverses up the choice
point stack until a goal of the form {\tt catch(Goal,Term,Handler)} is
found such that {\tt Term} unifies with {\tt ErrorTerm}.  In this
case, {\tt Handler} is called.  If no catcher is found, the system
looks for a clause of {\tt default\_user\_error\_handler(Term)} such
that {\tt Term} unifies with {\tt ErrorTerm}.  Finally, if no such
clause is found the default system error handler is called.
%
\ouritem{catch(Goal,CatchTerm,Handler)} \index{{\tt catch/3}} 
Calls {\tt Goal}, and sets up information so that future throws will
be able to access {\tt CatchTerm} under the mechanism mentioned
above. {\tt catch/3} does not attempt to clean up system level
resources.   Thus, it is left up to the handler to close open tables
(via {\tt close\_open\_tables/0}, close any open files, reset current
input and output, and so on.
%
\ouritem{default\_user\_error\_handler(CatchTerm)}
\index{{\tt default\_user\_error\_handler/1}}
Handles any error terms that unify with {\tt CatchTerm} that are not
caught by invocations of {\tt catch/3}.  This predicate {\em does}
close open tables, but does not attempt to clean up other system level
resources, which is left to the handler.
%
\ouritem{abort}\index{{\tt abort/0}}
    Abandons the current execution and returns to the top level.  This
    predicate should normally normally be used: 
 \begin{itemize} 
\item when a non-ISO exception has occurred and the user wishes to
abort the computation to the top-level of the interpreter.  
%
\item {\em and} the type of the error is not of interest for
structuring error handling.
\end{itemize}

Such situations occur can occur for instance in debugging, during
program development, or in small-special purpose programs.
    Aborting closes all incomplete tables (those which may not have a 
    complete set of answers).  Closed tables are unaffected, even if 
    the tables were created during the aborted computation.

{\tt abort/0} is implemented by throwing the term {\tt
'\_\$abort\_ball'/0}.  Implementations that wish to override the
default behavior of {\tt abort/0} must call a top-level goal via {\tt
catch/1} and ensure that {\tt '\_\$abort\_ball'/0} is handled.

\ouritem{abort(+Message)}\index{{\tt abort/1}} \index{{\tt STDERR}}
    Acts as {\tt abort/0} but sents {\tt Message} to {\tt STDERR}
    before aborting.

\ournewitem{error\_write(Message)}{standard}\index{{\tt errror\_write/1}}
\vspace{-0.3in}
\ournewitem{error\_writeln(Message)}{standard}\index{{\tt errror\_writeln/1}}

Utility routines for user-defined error catching.  These predicates
output {\tt Message} to XSB's {\tt STDERR} stream, rather than to
XSB's {\tt STDOUT} stream, as does {\tt write/1} and {\tt writeln/1}.
In addition, if {\tt Message} is a comma list, the elements in the
comma list are output as if they were concatenated together.  Each of
these predicates must be implicitly imported from the module {\tt
standard}.

\ournewitem{close\_open\_tables}{machine}\index{{\tt close\_open\_tables/0}}
%
Removes table data structures for all incomplete tables, but does not
affect any incomplete tables.  In \version{} this predicate should
only be used to handle exceptions in {\tt
default\_user\_error\_handler/1}.

\end{description}

\section{Convenience Predicates}

The following convenience predicates are provided to either throw an
ISO error, or to make a commonly used check and throw an ISO error if
the check is not satisfied.  All these predicates must be imported
from the module {\tt error\_handler}.

\begin{description}
\ournewitem{instantiation\_error(Predicate,Arg,State)}{error\_handler}
\index{{\tt instantiation\_error/4}}
Throws an instantiation error.  Using the default system error
handler, an example is 
{\small 
\begin{verbatim}
?- instantiation_error(foo/1,1,nonvar).
++Error[XSB/Runtime/P]: [Instantiation]  in arg 1 of predicate foo/1: must be nonvar
Aborting...
\end{verbatim}
}

\ournewitem{type\_error(Valid\_type,Culprit,Predicate,Arg)}{error\_handler}
\index{{\tt type\_error/4}}
Throws a type error.  Using the default system error
handler, an example is 
{\small 
\begin{verbatim}
| ?- type_error(atom,f(1),foo/1,1).
++Error[XSB/Runtime/P]: [Type (f(1) in place of atom)]  in arg 1 of predicate foo/1
Aborting...
\end{verbatim}
}

\ournewitem{domain\_error(Valid\_type,Culprit,Predicate,Arg)}{error\_handler}
\index{{\tt domain\_error/4}}
Throws a domain error.  Using the default system error
handler, an example is 
{\small 
\begin{verbatim}
domain_error(positive_integer,-1,cdfCheckPosint,3).
++Error[XSB/Runtime/P]: [Domain (-1 not in domain positive_integer)]  in arg 3 of predicate cdfCheckPosint
Aborting...
\end{verbatim}
}

\ournewitem{existence\_error(Object\_type,Culprit,Predicate,Arg)}{error\_handler}
\index{{\tt existence\_error/4}}
Throws an existence error.  Using the default system error
handler, an example is 
{\small 
\begin{verbatim}
existence_error(file,'myfile.P','load_intensional_rules/2',2).
++Error[XSB/Runtime/P]: [Existence (No file myfile.P exists)]  in arg 2 of predicate load_intensional_rules/2
\end{verbatim}
}

\ournewitem{permission\_error(Op,Obj\_type,Culprit,Predicate)}{error\_handler}
\index{{\tt permission\_error/4}}
Throws a permission error.  Using the default system error
handler, an example is 
{\small 
\begin{verbatim}
| ?- permission_error(write,file,'myfile.P',foo/1).
++Error[XSB/Runtime/P]: [Permission (Operation) write on file: myfile.P]  in foo/1
Aborting...
\end{verbatim}
}


\ournewitem{check\_var(Term,Predicate,Arg)}{error\_handler}
\index{{\tt check\_var/3}}
Checks that {\tt Term} is a variable.  If so, the predicate succeeds;
if not it throws an instantiation error.

\ournewitem{check\_nonvar(Term,Predicate,Arg)}{error\_handler}
\index{{\tt check\_nonvar/3}}
Checks that {\tt Term} is not a variable.  If not, the predicate succeeds;
if {\tt Term} is a variable,  it throws an instantiation error.

\ournewitem{check\_ground(Term,Predicate,Arg)}{error\_handler}
\index{{\tt check\_ground/3}}
Checks that {\tt Term} is ground.  If so, the predicate succeeds;
if not it throws an instantiation error.

\ournewitem{check\_nonvar\_list(Term,Predicate,Arg)}{error\_handler}
\index{{\tt check\_nonvar\_list/3}}
Checks that {\tt Term} is a list, each of whose elements is ground.
If so, the predicate succeeds; if not it throws an instantiation
error.
	    
\ournewitem{check\_atom(Term,Predicate,Arg)}{error\_handler}
\index{{\tt check\_atom/3}}
Checks that {\tt Term} is an atom.  If so, the predicate succeeds;
if not it throws an instantiation error.

\ournewitem{check\_stream(Stream,Predicate,Arg)}{error\_handler}
\index{{\tt check\_stream/3}}
Checks that {\tt Stream} is a stream.  If so, the predicate succeeds;
if not it throws an instantiation error~\footnote{The representation
of streams in XSB is subject to change.}.

\end{description}

