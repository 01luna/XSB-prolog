\section{Modification of the Database} \label{sec:assert}

XSB provides an array of features for modifying the dynamic database.
Using {\tt assert/1}, clauses can be asserted using first-argument
indexing in a manner that is now standard to Prolog implementations.
While this is the default behavior for XSB, other behaviour can be
specified using the (executable) directives {\tt index/3} and {\tt
index/2}.  For instance, dynamic clauses can be declared to have
multiple or joint indexes, while this indexing can be either
hash-based as is typical in Prolog systems or based on {\em tries}.
No matter what kind of indexing is used, space is dynamically
allocated when a new clause is asserted and, unless specified
otherwise, released when it is retracted.  Furthermore, the size of
any index table expands dynamically as clauses are asserted.

Consider first dynamic predicates that use traditional hash-based
indexing.  XSB asserts WAM code for such clauses, leading to execution
times similar to compiled code for unit and binary clauses.
Furthermore, tabling can be used with a dynamic predicate by
explicitly declaring a predicate to be both dynamic and tabled.  For
clauses that are asserted as WAM code, the {\em ``immediate
semantics''} of dynamic predicates is used, not the so-called {\em
``logical semantics''} of assert/retract \cite{LiOk87}. This means
that significant care must be taken when modifying the definition of a
predicate which is currently being executed. Notice that this makes
some operations difficult. For example, one might try to retract from
dynamically asserted predicates, {\tt p/1} and {\tt q/1}, exactly
their intersection, by issuing the following query:
\begin{center} 
{\tt :- p(X), q(X), retract(p(X)), retract(q(X)), fail.}
\end{center}
Neither {\tt retract/1} nor {\tt retractall/1} support this behavior,
due to their techniques for space reclamation.  One alternative is to
use {\tt findall/3} to collect the intersection first, before retracting.
Another is to use the predicates {\tt retract\_nr/1} and {\tt
reclaim\_space/1}, described below.  

Asserting clauses as WAM code might be considerably slow for some
applications.  To remedy this, XSB provides an alternative to {\tt
assert/1} which implements assert's functionality using the trie-based
tabling data structures \cite{RRSSW98}.  Though trie-based dynamic
code can be created (and usually executed) significantly faster than
using {\tt assert/1}, users of the following predicates should be
aware that trie-based assert can be used only for unit clauses where a
relation is viewed as a set, and where the order of the facts is not
important.

XSB does not at this time fully support dynamic predicates defined
within compiled code.  The only way to generate dynamic code is by
explicitly asserting it, or by using the standard predicate {\tt
load\_dyn/1} to read clauses from a file and assert them (see
the section {\it Asserting Dynamic Code} in Volume 2).  There is a
{\tt dynamic/1} predicate (see 
page~\pageref{dynamic/1}) that declares a predicate within the system
so that if the predicate is called when no clauses are presently
defining it, the call will quietly fail instead of issuing an {\sf
``Undefined predicate''} error message.

\begin{description}

\ouritem{assert(+Clause)}\index{{\tt assert/1}}
adds a dynamic clause, {\tt Clause}, to the database. {\tt Clause}
must be of one of the forms: {\tt Head} or {\tt Head :- Body}. Note
that because of the precedence of {\tt :-/2}, using the second form
requires an extra set of parentheses: {\tt assert((Head :- Body))}.
Default: first-argument indexing.

\ouritem{asserta(+Clause)}\index{{\tt asserta/1}}
If the index specification for the preicate is not {\tt tries}, this
predicate adds a dynamic clause, {\tt Clause}, to the database {\em
before} any other clauses for the same predicate currently in the
database.  If the index specification for the preicate is {\tt trie},
the clause is asserted arbitrarily within the trie, and a warning
message sent to {\tt stderr}.


\ouritem{assertz(+Clause)}\index{{\tt assertz/1}}
If the index specification for the preicate is not {\tt tries}, this
predicate adds a dynamic clause, {\tt Clause}, to the database {\em
after} any other clauses for the same predicate currently in the
database.  If the index specification for the preicate is {\tt trie},
the clause is asserted arbitrarily within the trie, and a warning
message sent to {\tt stderr}.

\ouritem{retract(+Clause)}\index{{\tt retract/1}}
removes through backtracking all clauses in the database that match with
{\tt Clause}.  {\tt Clause} must be of one of the forms:  {\tt Head} or
{\tt Head :- Body}.  Note, that because of the precedence of {\tt :-/2},
using the second form requires an extra set of parentheses:
{\tt retract((Head :- Body))}.  Space is reclaimed when a 
clause is retracted.

\ouritem{retractall(+Head)}\index{{\tt retractall/1}}
removes every clause in the database whose head matches with {\tt Head}.
The predicate whose clauses have been retracted retains the {\tt dynamic}
property (contrast this behaviour with that of predicates 
{\tt abolish/[1,2]} below).
Predicate {\tt retractall/1} is determinate and always succeeds.
The term {\tt Head} is not further instantiated by this call.

\ouritem{abolish(+PredSpec)}\index{{\tt abolish/1}}
Removes the definition of the specified predicate.  {\tt PredSpec}
is of the form {\tt Pred/Arity}.  Everything about the abolished
predicate is completely forgotten by the system (including the
{\tt dynamic} property).  There is also an {\tt abolish/2} which
takes {\tt Pred} and {\tt Arity} as its two arguments.

\ouritem{clause(+Head,?Body)}\index{{\tt clause/2}}
Returns through backtracking all dynamic clauses in the database whose head
matches {\tt Head} and Body matches {\tt Body}.  For facts the {\tt Body} is
{\tt true}.  

\ouritem{retract\_nr(+Clause)}\index{{\tt retract\_nr/1}}
Performs just as {\tt retract/1} does, except that it does not reclaim the
space used by the retracted clause. This is provided to allow programmers
to modify dynamic clauses while executing them (a practice that is 
discouraged.) For example, to retract an intersection, as described above,
one could do:
\begin{center}
{\tt :- p(X), q(X), retract\_nr(p(X)), retract\_nr(q(X)), fail.}
\end{center}
In order to reclaim space after using {\tt retract\_nr/1}, see 
{\tt reclaim\_space/1} below.  Predicate {\tt retract\_nr/1}
is not a standard predicate and must be imported from module {\tt assert}.
{\tt retract\_nr/1} is provided for (partial)
compatibility with the {\tt retract/1} predicate of SB-Prolog.

%In this case, the {\tt retract\_nr/1} deletes the clauses, but will not
%drastically modify the clause data structure,
%and this code will execute. Of course, space is not reclaimed for further
%use (but see {\tt reclaim\_space/1} below).  Predicate {\tt retract\_nr/1}
%is not a standard predicate but must be imported from module {\tt assert}.
%The use of this predicate is discouraged; it is provided for (partial)
%compatibility with the {\tt retract/1} predicate provided by SB-Prolog that
%did not reclaim space.

\ouritem{reclaim\_space(+Head)}\index{{\tt reclaim\_space/1}}
Runs through the dynamic code for the predicate indicated by {\tt Head}, and
reclaims space for any clauses that have been deleted from that predicate by
{\tt retract\_nr/1}.  This cannot safely be used when execution is still
within some invocation of the specified predicate, or will backtrack into
such a scope.  To complete our example of retracting the intersection of
dynamic predicates:
\begin{center}
{\tt :- p(X), q(X), retract\_nr(p(X)), retract\_nr(q(X)), fail\ ;\\
     reclaim\_space(p(\_)), reclaim\_space(q(\_)).}
\end{center}
would do the trick. Notice that the {\tt reclaim\_space} calls 
must be made after execution has completely failed
out of choice points for {\tt q(X)} and {\tt p(X)}.  Predicate 
{\tt reclaim\_space/1} is not
standard but must be imported from module {\tt assert}.
As with {\tt retract\_nr}, the use of this predicate is discouraged; 
it is provided for (partial) compatibility with SB-Prolog.

\ouritem{index(+PredSpec, +IndexSpec, +HashTableSize)}\index{{\tt index/3}}
\vspace{-.35in}
\ouritem{index(+PredSpec, +IndexSpec)}\index{{\tt index/2}}
\label{index_dynamic}
{\tt index/3} and {\tt index/2} are executable directives that
determine what form of indexing a dynamic predicate will take and to a
certain extent, what semantics the dynamic predicate will have.  

If {\tt index/3} is used, then the predicate indicated by {\tt
PredSpec} is declared to be indexed according to {\tt IndexSpec} using
initial hash table of sizes of {\tt HashTableSize}.  After this
directive is given, all clauses asserted to {\tt PredSpec} will be so
indexed.  The index directive does {\em not\/} re-index an already
existing predicate but takes effect only if the program store contains
no clauses for {\tt PredSpec} (a Database Error is raised
otherwise). Index directives can be given to the compiler as part of
source code or executed during program execution (analogously to {\tt
op/3}).

XSB supports different indexing for compiled and asserted code.  For a
compiled predicate {\tt IndexArgSpec} must be an integer which
indicates the argument on which an index is to be constructed.  If
{\tt IndexArgSpec} is~0 (or []), then no index is kept (an efficient
strategy for predicates with only one or two clauses.)  
%If {\tt IndexArgSpec} is a negative integer, then indexing information
%regarding the indicated predicate is deleted, and its handling reverts
%to the default.  
For a dynamic predicate, (to which clauses are asserted), {\tt
IndexArgSpec} may be a list of {\tt IndexArg}s.  An {\tt IndexArg} is
an integer that indicates the argument on which to index.  One may
specify, for example, {\tt index(p/3,[2,1],150)} to indicate that
clauses asserted to the predicate {\tt p/3} should be indexed on both
the second and the first argument, and that the initial hash-table
size has 150 buckets.  Subsequent calls to {\tt p/3} will first check
to see if the second argument is nonvariable, and if so use that
index. If the second argument is variable, it will check to see if the
first argument is nonvariable and if so, use that index.  One may have
any number of distinct indexes.  For asserted clauses an {\tt
IndexArg} may also be a sequence of up to three integers separated by
{\tt +}, e.g., {\tt 1+2+3}.  This indicates that an index is to be
built using those specified field values together. To use such an
index in a call, all the indicated fields must be nonvariable.

As an example, one could specify: {\tt index(p/5,[1+2,1,4],300)}.
After clauses are asserted to it, a call to {\tt p/5} would first
check to see if both the first and second arguments are nonvariable
and if so, use an index based on both those values. Otherwise, it
would see if the second argument is nonvariable and if so, use an
index based on it. Otherwise, it would see if the fourth argument is
nonvariable and if so use an index based on it. As a last resort, it would
use no index but backtrack through all the clauses in the predicate.
(Notice that it may well make sense to include an argument that 
appears in a joint specification later alone, as 1 in this example,
but it never makes sense foring the single argument to appear earlier. In
that case the joint index would never be used.)

The executable declaration {\tt index(Predspec,trie)} causes clauses
for {\tt Predspec} to be asserted using tries (see \cite{RRSSW98},
which is available through the XSB web page).  The name trie indexing
is something of a misnomer since the trie itself both indexes the term
and represents it.  In XSB, the above rie index is formed using a
left-to-right traversal of the unit clauses.  These indexes can be
very effective if discriminating information lies deep within a term,
and if there is sharing of left-prefixes of a term, can reduce the
space needed to represent terms.  Furthermore, asserting a unit as a
trie is much faster than asserting it using default WAM code.  Trie
indexing can be used with alternative or joint indexes.  For the
directive {\tt index(p/3,[2,1],trie)}, two trie indices would be
formed for {\tt p/3}: one that traversed arguments in order {\em
2,1,3} and another that traversed arguments in order {\em 1,2,3}.  The
actual implementation seeks to reduce redundant storage of code for
alternative indices.

Despite these advantages, representing terms as tries leads to
semantic differences from asserted code, of which the user should be
aware.  First, the order of clauses within a trie is arbitrary: using
{\tt asserta/1} or {\tt assertz} for a predicate currently using trie
indexing will give the same behavior as using {\tt assert}.  Also, the
current version of XSB only allows trie indexing for unit clauses.

\ouritem{dynamic(+PredSpec)}\index{{\tt dynamic/1}}\label{dynamic/1}
is an executable predicate which converts a predicate specified as
(Predicate/Arity) to a dynamic predicate. If {\tt Predicate} is not
previously defined, it will be initialized to empty (so that calls to
it quietly fail, instead of issuing {\sf ``Undefined predicate''}
error messages.) If the predicate is previously defined and dynamic,
{\tt dynamic/1} is a noop. If previously defined as compiled, {\tt
Predicate} will be converted to dynamic, which means that clauses can
be added, although the compiled portion cannot be manipulated.  Note
that {\tt dynamic/1} can be used like a compiler directive, since it
will be passed through to be executed when the module is loaded. Note,
however, that the semantics is different from that of the standard
\cite{ISO-Prolog} when the file contains clauses defining the
so-specified predicate.

\ouritem{table(+PredSpec)}\index{{\tt table/1}}
is an executable predicate, where PredSpec is a predicate
specification for a dynamic predicate. (This is also a compiler
directive when {\tt PredSpec} specifies a compiled predicate. See the
section of this manual on compiler directives.) This predicate
declares a dynamic predicate to be tabled. It simply saves information
to be used at the time of assert and so it must be called before any
clauses are asserted into the specified predicate in order for the
predicate to be tabled.

\end{description}

%----------------------------------------------------------------------
%\input{tr_assert}	% To be uncommented when they become standard.
%----------------------------------------------------------------------



\section{Execution State}\label{environmental}

\begin{description}

\ouritem{abort}\index{{\tt abort/0}} 
    Abandons the current execution and returns to the top level.  This
    predicate is normally used in one of the following two cases: 
    \begin{itemize} 
    \item when some error condition or exception has occurred and carrying
	  on the computation is of no further use.
    \item when using the debugger (see Chapter~\ref{debugging}).  
    \end{itemize}
    Currently, all exception handling routines terminate with a call to 
    predicate {\tt abort/0}, so an exception encountered at some break level 
    other than the top level will return the interpreter at the top level.

    The user should be aware of the fact that {\tt abort/0} does not close 
    any files which may have been opened.  If the program under execution is 
    doing file manipulation using {\tt see/1} and {\tt tell/1}, then strange 
    behaviour may occur after the program is aborted and restarted, unless
    the user manually closes the files.

    Aborting closes all incomplete tables (those which may not have a 
    complete set of answers).  Closed tables are unaffected, even if 
    the tables were created during the aborted computation.

\ouritem{break}\index{{\tt break/0}}
    Causes the current execution to be suspended at the beginning of the next 
    call.  The interpreter then enters break level 1 and is ready to accept
    input as if it were at top level.  If another call to {\tt break/0} is 
    encountered, it moves up to break level 2, and so on.  While execution 
    is done at break level $n>0$ the prompt changes to {\tt $n$: ?-}.

    To close a break level and resume the suspended execution, the user can 
    type the the atom {\tt end\_of\_file} or the end-of-file character 
    applicable on the system (usually {\tt CTRL-d} on UNIX systems).  
    Predicate {\tt break/0} 
    then succeeds (note in the following example that the calls to {\tt break/0}
    do not succeed), and the execution of the interrupted program is resumed.  
    Alternatively, the suspended execution can be abandoned by calling the 
    standard predicate {\tt abort/0}, which causes a return to the top level.

    An example of {\tt break/0} 's use is the following:

    \stuff{
	\>   \>     | ?- break. \\
	\>   \>     [ Break (level 1) ] \\
	\>   \>     1: ?- break. \\
	\>   \>     [ Break (level 2) ] \\
	\>   \>     2: ?- end\_of\_file. \\
	\>   \>     [ End break (level 2) ] \\
	\\
	\>   \>     yes \\
	\>   \>     1: ?-
    }

    Entering a break closes all incomplete tables (those which may not have a 
    complete set of answers).  Closed tables are unaffected, even if 
    the tables were created during the computation for which the break was
    entered.

\ouritem{halt}\index{{\tt halt/0}}
    Exits the \ourprolog\ session regardless of the break level.  On exiting
    the system cpu and elapsed time information is displayed.

\ouritem{prompt(+NewPrompt, ?OldPrompt)}\index{{\tt prompt/2}}
    Sets the prompt of the top level interpreter to {\tt NewPrompt} and 
    returns the old prompt in {\tt OldPrompt}.

    An example of {\tt prompt/2} 's use is the following:

    \stuff{
	\>   \>     | ?- prompt('Yes master > ', P). \\
	\\
	\>   \>     P = | ?- ; \\
	\\
	\>   \>     no \\
	\>   \>     Yes master > fail. \\
	\\
	\>   \>     no \\
	\>   \>     Yes master >
    }
 
\ouritem{cputime(-CPU\_Time)}\index{{\tt cputime/1}}
    Returns the {\tt CPU\_Time} at the time of the call in seconds. 
    The difference between results of successive calls to this
    predicate can measure the time spent in specific predicates.

\ouritem{statistics}\index{{\tt statistics/0}}
    Prints on the current output stream:
    \begin{itemize}
    \item Information about allocation of memory 
	  (see Section~\ref{emulator_options}) containing the 
          \begin{itemize}
          \item global stack (heap) and local (environment) stack
          \item trail and choice point stack
	  \item SLG subgoal space (tablestack)
          \item SLG unification stack
          \item SLG completion stack
	  \item the spacce occupied by subgoal and answer tables
		(in the form of tries).
          \end{itemize}
    \item Current use of the above specified memory areas
		(allocated/in use/free).
    \item Information about process cpu and clock time.
    \end{itemize}

    Additionally, if the emulator is invoked with the {\tt '-s'} option
    (see Section~\ref{emulator_options}), information is printed out about
    \begin{itemize}
    \item Maximum use of the memory areas.
    \end{itemize}
    The {\tt '-s'} option slows down the emulator by about 10\%.

    Example:
    {\footnotesize
     \begin{verbatim}
     | ?- statistics.

     memory (total)        1873737 bytes:     171317 in use,    1702420 free
        permanent space     169801 bytes
        glob/loc space      786432 bytes:       1080 in use,     785352 free
           global                                152 bytes
           local                                 928 bytes
        trail/cp space      786432 bytes:        436 in use,     785996 free
           trail                                 240 bytes
           choice point                          196 bytes
        SLG subgoal space        0 bytes:          0 in use,          0 free
        SLG unific. space    65536 bytes:          0 in use,      65536 free
        SLG completion       65536 bytes:          0 in use,      65536 free
        SLG trie space           0 bytes:          0 in use,          0 free
        (call+ret. trie          0 bytes,   trie hash tables          0 bytes)

        Maximum stack use: global 224, local 1384, trail 240, cp 492
        Maximum stack use: SLG completion 0.  Max level: 0

          0 Trail unwinds,       0 levels

      0.570 sec. cputime,  5.088 sec. elapsetime
     \end{verbatim}}
    shows how the emulator output looks if it is invoked with the 
    {\tt '-s'} option (without it the Maximum use line is not shown).
    Information about the allocation size is provided since the sizes
    can be changed through emulator options
    (see Section~\ref{emulator_options}).


%\ouritem{analyze\_table(+Pred\_indicator)}\index{{\tt analyze\_table/1}}
%    This predicate provides additional means for the user to profile
%    execution of tabled predicates.  Given a predicate indicator for a 
%    tabled predicate, {\tt analyze\_table/1} prints the following information.
%    \begin{itemize}
%    \item  "$\langle$Num$\rangle$ call hash buckets used out of 
%	   $\langle$Total$\rangle$".  The number
%           of call hash buckets containing at least one element.
%    \item  "avg call bucket length (of used) $\langle$Avg$\rangle$".  
%	   The average length of call hash buckets containing at least one
%	   element.
%    \item  "max call bucket length $\langle$Length$\rangle$".
%    \item  "$\langle$Num$\rangle$ ret hash buckets used out of 
%	   $\langle$Total$\rangle$".  The number
%           of ret hash buckets containing at least one element.
%    \item  "avg ret bucket length (of used) $\langle$Avg$\rangle$".
%	   The average length of ret hash buckets containing at least
%	   one element.
%    \item  "max ret bucket length $\langle$Length$\rangle$".
%    \end{itemize}

%    In \version, {\tt analyze\_table/1} is most readily used to 
%    determine proper call and return hash sizes.  The hash table sizes can
%    be reset using {\tt set\_table\_prop/3} from the interpreter level
%    (see Section~\ref{tabling_predicates}) or by recompiling the source code
%    with new declarations (see Section~\ref{compiler_directives}).

%    Example:
%    {\footnotesize
%     \begin{verbatim}
%     | ?- analyze_table(ancestor/2).
%                 1 hash buckets full out of 97
%                 avg call bucket length (of filled) 1.00
%                 max call bucket length 1
%                 ret buckets used 97 out of 97
%                 avg return bucket length 11.92
%                 max ret bucket length 17
%
%     yes
%     \end{verbatim}}
%    The output of {\tt analyze\_table} indicates that the return hash size for
%    ancestor is too small for the query.  Changing the return hash size will
%    speed up similar queries which add answers to tables, but the size of the
%    return hash table has no effect on the speed of queries which use completed
%    tables.

\ouritem{unix(+SystemCall)}\index{{\tt unix/1}}
    Calls the operating system with the atom {\tt SystemCall} as argument.
    It succeeds if {\tt SystemCall} is executed successfully, otherwise it
    fails.  As a notational convenience, the user can supply {\tt SystemCall} 
    in the form of a list (something currently not possible for {\tt unix/2}).

    For example, the call:

% JF
%    \demo{\verb+| ?- unix('echo \$HOME').+}
    \demo{{\tt $|$ ?- unix('echo \$HOME').}}

    \noindent
    will output in the current output stream of \ourprolog\ the name of
    the user's home directory; while the call:

% JF
%   \demo{\verb+| ?- File = 'test.c', unix(['cc -c ', File]).+}
    \demo{{\tt $|$ ?- File = 'test.c', unix(['cc -c ', File]).}}

    \noindent
    will call the C compiler to compile the file {\tt test.c}.

    Note that in UNIX systems, since {\tt unix/1} is executed by forking 
    off a shell process,
    it cannot be used, for example, to change the working directory of the 
    interpreter.  For that reason the standard predicate {\tt cd/1} described 
    below should be used.

%    {\em Note to DOS users.} \index{DOS}  The {\tt unix/1} command works
%    for DOS through the underlying GNU djgpp mechanism used to port \ourprolog.

\ouritem{unix(+SystemCall, -Result)}\index{{\tt unix/2}}
    Calls the operating system with the atom {\tt SystemCall} as argument
    and returns the result of the call in {\tt Result}.  In comparison with
    {\tt unix/1} this predicate always succeeds, even if the {\tt SystemCall} 
    cannot be successfully executed.

%    {\em Note to DOS users.} \index{DOS}  The {\tt unix/2} command works
%    for DOS through the underlying GNU djgpp mechanism used to port \ourprolog.

\ouritem{ls}\index{{\tt ls/0}} 
    Under UNIX, this command lists in the current output stream the files 
    in the system's current directory if it can do so.  If so, it succeeds.
    It is the same as {\tt unix('ls -F', 0)}.  

\ouritem{cd(+Dir)}\index{{\tt cd/1}}
    Under UNIX and Windows, this predicate changes the interpreter's
    working directory to {\tt Dir}.  If the directory specified does
    not exist or is not a directory, or the user does not have execute
    permission for that directory, predicate {\tt cd/1} simply fails
    raising no permission error.

    Exceptions:
    \begin{description}
    \item[{\tt instantiation\_error}]
	{\tt Dir} is not instantiated at the time of call.
    \item[{\tt type\_error}]
	{\tt Dir} is not an atom.
    \end{description}

\ouritem{edit(+Module)}\index{{\tt edit/1}}
    Provided that the environment variable {\tt EDITOR} has been set,
    and the system is executing under UNIX, 
    a call to {\tt edit(foo)} will call the default editor on the file
    named {\tt foo.P}.  The argument to {\tt edit/1}, should be instantiated,
    it can be an absolute or a relative file name, but it should {\em not}
    contain the suffix {\tt .P}.  Users can also set their preferred options
    of calling the default editor by setting an environment variable named
    {\tt EDITOR\_OPTIONS}.

    Examples of possible uses of predicate {\tt edit/1} are:
    \begin{enumerate}
    \item If the environment variables have been set as follows:
          \begin{verbatim}
	        setenv EDITOR /usr/ucb/vi
	        setenv EDITOR_OPTIONS -l
          \end{verbatim}
          a call like:
          \begin{verbatim}
	        | ?- edit(foo).
          \end{verbatim}
          will call the vi editor in the mode where left and right parentheses
	  and curly brackets are checked for balance for the file {\tt foo.P}
	  in the current working directory.
    \item If, on the other hand, they have been set as follows:
          \begin{verbatim}
	        setenv EDITOR /usr/local/bin/emacs
	        setenv EDITOR_OPTIONS -r
          \end{verbatim}
          a call like:
          \begin{verbatim}
	        | ?- edit('~/foo').
          \end{verbatim}
          will call the emacs editor in reverse video for the file {\tt foo.P}
	  in user's home directory.
    \end{enumerate}

\end{description}

%need op (just so we dont forget ???


\section{Tabling}\label{tabling_predicates}

In \ourprolog, tables are designed so that they can be used
transparently by computations.  However, it is often useful to be able
to explicitly inspect a table, or alter its state, especially during
program development.  The following predicates are supplied for these
purposes.

In the following predicates, {\tt Skeleton} refers to information
about the function and arity of a predicate.  If p/2 is a predicate,
its skeleton can be represented as {$p(\langle arg1 \rangle,\langle
arg2 \rangle)$} where {\em arg1} and {\em arg2} can be any
instantiation pattern.  Thus the information derived from the
skeletons {\em p(1,2)} and {\em p(A,B)} would be the same.

The user should be aware that skeletons which are dynamically created
(e.g. by {\tt functor/3}) are located in usermod.  In such a case, the
tabling predicates below may not behave in the desired manner if the
tabled predicates themselves have not been imported into usermod.

\begin{description}

\ournewitem{get\_calls(+Skeleton,-Subgoal\_Structure\_Pointer,-Return\_Skeleton)}{Tabling}
\index{{\tt get\_calls/3}}
Backtracks through the subgoal trie, instantiates the {\tt Skeleton}
to a call term in the {\tt call trie} by binding the variable in the skeleton,
binds {\tt Subgoal\_Structure\_Pointer} with the pointer to the 
{\tt subgoal structure} and binds {\tt Return\_Skeleton} to 
a term of the form ret/n where each of its arguments corresponds to
a free variable in the call.
Example:

For the following program and table
     \begin{center}
     \begin{tabular}{cc}
     \begin{minipage}{1.1in}
     {\tt
          :- table p/2.	\\
          p(1,2).       \\
          p(1,3).       \\
          p(1,X).       \\
          p(2,3). 
     }
     \end{minipage}
     &
     \begin{tabular}{||l|l||}   \hline
     {\em Call}			& {\em Returns} \\ \hline \hline
     p(1,X)			& p(1,2) \\ 
 	       			& p(1,3) \\
                                & p(1,X) \\ \hline
     p(X,3)			& p(1,3) \\ 
 	       			& p(2,3) \\ \hline
     \end{tabular}
     \end{tabular}
     \end{center}
calls to {\tt get\_calls/3} will act as follows
	{\footnotesize
	\begin{verbatim}
                   | ?- get_calls(p(X,Y),Cs,Ret).

                   X = _864816
                   Y = 3
                   Cs = 2927152
                   Ret = ret(_864816);

                   X = 1
                   Y = _864644
                   Cs = 2927104
                   Ret = ret(_864644);

                   no
                   | ?- get_calls(p(1,Y),Cs,Ret).

                   Y = 3
                   Cs = 2927152
                   Ret = ret(1);

                   Y = _864620
                   Cs = 2927104
                   Ret = ret(_864620);

                   no
                   | ?- get_calls(p(Y,3),Cs,Ret).

                   Y = _864792
                   Cs = 2927152
                   Ret = ret(_864792);

                   Y = 1
                   Cs = 2927104
                   Ret = ret(3);

                   no
                   | ?- get_calls(p(1,3),Cs,Ret).

                   Cs = 2927152
                   Ret = ret(1);

                   Cs = 2927104
                   Ret = ret(3);

                   no

	\end{verbatim}
	}

\ournewitem{get\_calls\_for\_table(+Skeleton,-Call,-Empty)}{Tabling}\index{{\tt get\_table\_pred/3}}
Succeeds whenever {\tt Skeleton} corresponds to a predicate that has been 
declared tabled.  For these predicates {\tt get\_calls\_for\_table/3} 
upon backtracking returns the calls of all current table entries.
Also returned with each call is an atom which is a member of the set
\begin{center}
\{ {\tt empty, notempty} \}.
\end{center}
indicating whether the table is empty or non-empty.  

This predicate does not provide any information about whether a table is
complete.  Use {\tt table\_state} to inquire about a table's state.

\noindent
Calls to {\tt get\_calls\_for\_table/3} for the 
program in {\tt get\_calls/3} would act as follows
    {\footnotesize
     \begin{verbatim}
                   | ?- get_calls_for_table(p(X,Y), Call, Empty).
                   X = _646608
                   Y = _646436
                   Call = p(1,_646724)
                   Empty = notempty;

                   X = _646608
                   Y = _646436
                   Call = p(_646720,3)
                   Empty = notempty

                   | ?- get_calls_for_table(p(1,2), Call, Empty).
                   Call = p(1,_646676)
                   Empty = notempty

                   Call = p(_646672,3)
                   Empty = notempty
     \end{verbatim}
     }
The second example backtracks through all entries in the table, since
only skeletal information is used from the first argument.

Exception:
    \begin{description}
    \item[{\tt table\_error}]
	First argument is not callable.
    \end{description}

% qtls this had been fixed, and will be again soon.
%{\bf Release Note:} In future versions, this predicate may be changed to 
%allow partial instantiation of the second argument. 

\ournewitem{get\_returns(+Subgoal\_Structure\_Pointer,-Return\_Skeleton)}{Tabling}\index{{\tt get\_returns/2}}
Backtracks through the answer trie for the subgoal whose {\tt subgoal structure}
is pointed to by {\tt Subgoal\_Structure\_Pointer}, and instantiates the 
{Return\_Skeleton} with the bindings corresponding to the return.


A way of accessing some subgoals and answers of the table in
in {\tt get\_calls/3}  would act as follows
    {\footnotesize
     \begin{verbatim}
                   | ?- get_calls(p(Y,3),Cs,Ret), get_returns(Cs,Ret).

                   Y = 2
                   Cs = 2561656
                   Ret = ret(2);

                   Y = 1
                   Cs = 2561656
                   Ret = ret(1);

                   Y = 1
                   Cs = 2559032
                   Ret = ret(3);

                   Y = 1
                   Cs = 2559032
                   Ret = ret(3);

                   no

     \end{verbatim}
     }


\ournewitem{get\_returns(+Subgoal\_Structure\_Pointer,-Return\_Skeleton,-LeafNode)}{Tabling}\index{{\tt get\_returns/3}}
Same as above, except for the fact that the third argument is bound
to the leaf node corresponding to the return in the return trie.



\ournewitem{get\_returns\_for\_call(+Call,-Return)}{Tabling}\index{{\tt get\_returns\_for\_call/2}}
Succeeds whenever there is a table entry for a variant of {\tt Call}.
If this is the case, {\tt Return} is instantiated with an
entry for the table corresponding to call, and successive entries are returned
by backtracking.  If there are no answers in the table, or no table entry
for {\tt Call}, the goal fails.

This predicate creates fresh variables for the return, rather than
unifying them with variables in the first argument.  An explicit
unification of a call with its return can be done if so desired.

Example:
Let us continue from the example in {\tt get\_calls\_for\_table/3}.

    {\footnotesize
     \begin{verbatim}
                   | ?- get_returns_for_call(p(1,X), Return).

                   X = _646412
                   Return = p(1,_646628);

                   X = _646412
                   Return = p(1,2);

                   X = _646412
                   Return = p(1,3);

                   no
                   | ?- get_returns_for_call(p(X,Y), Return).
                   no

                   | ?- get_returns_for_call(p(1,2), Return).
                   no
      \end{verbatim}
      }

Exception:
    \begin{description}
    \item[{\tt table\_error}]
	Argument 1 is not a callable predicate.
    \end{description}

% qtls this had been fixed, and will be again soon.
%{\bf Release Note:} In future versions, this predicate may be changed to 
%allow partial instantiation of the second argument. 


\ournewitem{delete\_return(+Subgoal\_Structure\_Pointer,+LeafNode)}{Tabling}
\index{{\tt delete\_returns/2}}
Deletes a return from the return trie. The LeafNode parameter should
be obtained from a call to get\_returns/3.


\ournewitem{abolish\_all\_tables}{Tabling}\index{{\tt abolish\_all\_tables/0}}
{\tt abolish\_all\_tables/0} abolishes all tables presently in the
system.  Predicates which have been declared tabled remain tabled, but
information about calls and returns is deleted.  In \version, {\tt
abolish\_all\_tablesl/0} reclaims used space.  The predicate always 
succeeds.

Note that incomplete tables are abolished automatically by the system on
exceptions and when the interpreter level is resumed.  In these cases,
the user does not need to abolish tables to maintain correctness.


\ournewitem{abolish\_table\_pred(?Skeleton)}{Tabling}\index{{\tt abolish\_table\_pred/1}}
{\tt abolish\_table\_pred/1} abolishes tables for all calls to the
predicate denoted by {\em Skeleton}.  The predicate remains
tabled, but information about its calls and their returns is removed
from the system.  In \version, {\tt abolish\_table\_pred/1}
reclaims used space.  Note that incomplete tables are abolished automatically
by the system on exceptions and when the interpreter level is resumed.
In these cases, the user does not need to abolish tables to maintain
correctness.

Exception:
    \begin{description}
    \item[{\tt table\_error}]
        Not a tabled predicate.
    \end{description}


\ournewitem{abolish\_table\_call(?Call)}{Tabling}
\index{{\tt abolish\_table\_call/1}}
{\tt abolish\_table\_call/1} abolishes tables for a particular call to
a predicate denoted by {\em Call}.  The predicate remains tabled, and
information about all other tables remains intact.  In \version, {\tt
abolish\_table\_call/1} does not reclaim used space.  Note that incomplete
tables are abolished automatically by the system on exceptions and
when the interpreter level is resumed.  In these cases, the user does
not need to abolish tables to maintain correctness.

Example:

    Continuing the example started in the description of predicate
    {\tt get\_calls\_for\_table/3} the call
    {\tt abolish\_table\_call(p(1,X))} would produce the table
    \begin{center}
    \begin{tabular}{||l|l||}               \hline
    {\em Call}			& {\em Returns} \\ \hline \hline
    p(X,3)			& p(1,3) \\ 
	       			& p(2,3) \\ \hline
    \end{tabular}
    \end{center}

Exception:
    \begin{description}
    \item[{\tt table\_error}]
	Argument 1 is not a callable predicate.
    \end{description}


\ournewitem{table\_state(?Call,-State)}{Tabling}
\index{{\tt table\_state/2}}
If the first argument is a valid Term indicator, {\tt table\_state/1}
unifies state with one of the set 
	\{{\tt not\_yet\_called, complete, incomplete, undef}\}.
The meaning of these atoms is defined as:
\begin{description}
\item[{\tt not\_yet\_called}] iff the predicate corresponding to
Call has been declared tabled, but there is no table entry for call.
\item[{\tt complete}] iff the table entry for {\tt Call} contains
all solutions.
\item[{\tt incomplete}] iff the table entry for {\tt Call} may not contain
all solutions.
\item[{\tt undef}] iff the predicate corresponding to {\tt Call} 
has not been declared tabled.
\end{description}

Exceptions:
    \begin{description}
    \item[{\tt type\_error}]
	Argument 1 is not a callable predicate.
    \end{description}


\comment{

\ournewitem{table\_prop(?Pred\_indicator,+Type,?Old\_val,+New\_val)}{Tabling}
\index{{\tt table\_prop/4}}
This predicate is described together with the following one.

\ournewitem{table\_prop(?Pred\_indicator,+Type,-Value)}{Tabling}\index{{\tt table\_prop/3}}
This predicate is used either to provide information about a table
property, or to reset one of properties to a particular value.  These
values can also be set by the table declarations in the source code.
{\tt table\_prop} provides a method for doing so without having to
recompile code.

In previous versions {\tt table\_prop} was used to reset hash sizes for
calls and returns.  It is not used in \version, but is maintained since
in future versions {\tt table\_prop} may be used for other table
properties such as subgoal subsumption.

}

%In \version\ the properties visible to the user are the call and return
%hash sizes.  Each tabled predicate contains two types of hash tables.
%A {\em call hash table} is used to find whether the variant of a call
%to a tabled predicate has occurred before in the course of a
%computation.  Each tabled call has a {\em return hash table} which is
%used during a computation to determine whether a given return needs to
%be added to the table.  Appropriate indexing for call and return
%lookup, including proper hash table sizes, is crucial to good
%performance.  

%Hash table sizes are non-negative integers denoting the number of
%buckets in the hash table.  When {\tt Pred\_indicator} denotes a
%tabled predicate, the predicate can take one of two modes depending on
%its arity.  In {\tt table\_prop/4}, the {\tt New\_value} is provided by
%the user and {\tt Old\_value} is unified with the precious value of
%the table property.  In {\tt table\_prop/3} the present value of the 
%property is returned.

%Presently, the hash table sizes can only be reset when there are no tabled
%calls to {\tt Pred\_indicator}.  This can be ensured by a call to 
%{\tt abolish\_table\_pred/1}
%
%Future versions of \ourprolog\ may include additional indexing
%strategies as configurable table properties.

\comment{

    Exceptions:
    \begin{description}
%    \item[{\tt table\_error}]
%	Attempt to reset hash table size for predicate with tabled calls.
    \item[{\tt table\_error}]
	Argument 1 is not a tabled predicate.
    \end{description}

}
%\ounewitem{table\_copy(+From,-To)} \index{{\tt table\_copy/2}}
%Table copy provides an explicit mechanism for the user to reduce the
%amount of copying into and out of a table.  {\tt table\_copy/2}
%succeeds when {\tt From} is ground, and returns a copy of that
%structure (or list) in the table space.  No copying of that structure
%will be required when calls or returns containing it are tabled, and
%no copying will be necessary when returns using it are resolved with
%goals. 
%
%For an example of the use of table copy, see Section~\ref{dcg_tabling}

\comment{

\ournewitem{breg\_retskel(+Breg,+Arity,-Return\_Skeleton,-SubgoalStructurePtr)}
\index{{\tt breg\_retskel/4}}
If Breg is bound to a pointer to the tabling choice point and 
Arity is bound to the arity of the predicate corresponding to 
the choice point, {\tt breg\_retskel} succeeds after binding 
{\tt Return\_Skeleton} to ret/n as in {\tt get\_calls/3} and 
{\tt SubgoalStructurePtr} to the pointer to the subgoal structure.

Warning:This builtin works for tabling predicates only.

}

\end{description}


