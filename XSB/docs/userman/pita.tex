%%% Local Variables: 
%%% mode: latex
%%% TeX-master: "manual2"
%%% End: 
\chapter{ PITA: Probabilistic  Inference}
\label{package:pita} 

  \begin{center}
    {\Large {\bf By Fabrizio Riguzzi}}
  \end{center}

\index{PITA}
\index{Logic Programs with Annotated Disjunction}
\index{CP-logic}
\index{LPADs}
\index{PRISM}
\index{Probabilistic Logic Programming}
\index{Possibilistic Logic Programming}

%%\ifnum\pdfoutput>0 % pdflatex compilation


``Probabilistic  Inference with Tabling and Answer subsumption'' (PITA) \cite{RigSwi10-ICLP10-IC} is a package  for reasoning with Logic Programs with Annotated Disjunctions (LPADs)\cite{VenVer03-TR,VenVer04-ICLP04-IC} and CP-logic programs \cite{VenDenBru-JELIA06,DBLP:journals/tplp/VennekensDB09}.

An example of LPAD/CP-logic program is 
\begin{eqnarray*}
(heads(Coin):0.5)\vee (tails(Coin):0.5)&\leftarrow&
toss(Coin),\neg biased(Coin).\\
(heads(Coin):0.6)\vee (tails(Coin):0.4)&\leftarrow&
toss(Coin), biased(Coin).\\
(fair(Coin):0.9) \vee (biased(Coin):0.1).&&\\
toss(Coin).&&
\end{eqnarray*}
The first clause states that if we toss a coin that is not biased it has equal probability of landing heads and tails. The second states that if the coin is biased it has a slightly higher probability of landing heads. The third states that the coin is fair with probability 0.9 and biased with probability 0.1 and the last clause states that we toss a coin with certainty.

PITA computes the probability of queries by tranforming the input program into a normal logic program and then calling a modified version of the query on the transformed programs.

\subsection{Installation}
PITA uses \href{http://www.gtk.org/}{GLib 2.0} and  \href{http://vlsi.colorado.edu/~fabio/CUDD/}{CUDD}.
GLib is a standard GNU package 
so it is easy to install it using the package management software of your Linux 
distribution.

To install CUDD, follow the instructions at  \url{http://vlsi.colorado.edu/~fabio/CUDD/} to get the package (or get directly from \url{ftp://vlsi.colorado.edu/pub/cudd-2.4.2.tar.gz}), for example \texttt{cudd-2.4.2.tar.gz}.
After decompressing, you will have a direcory \texttt{cudd-2.4.2} with various subdirectories.
Compile CUDD following the included instructions.

To install PITA with XSB, run XSB \texttt{configure} in the \texttt{build} directory with option \texttt{--with-pita=DIR} where \texttt{DIR} is the folder where CUDD is.


\subsubsection{Syntax}

Disjunction in the head is represented with a semicolon and atoms in the head are separated from probabilities by a colon. For the rest, the usual syntax of Prolog is used.
For example, the  CP-logic clause
$$h_1:p_1\vee \ldots \vee h_n:p_n\leftarrow b_1,\dots,b_m ,\neg c_1,\ldots,\neg c_l$$
is represented by
\begin{verbatim}
h1:p1 ; ... ; hn:pn :- b1,...,bm,\+ c1,....,\+ cl
\end{verbatim}
 No parentheses are necessary. The \texttt{pi} are numeric expressions. It is up to the user to ensure that the numeric expressions are legal, i.e. that they sum up to less than one.

If the clause has an empty body, it can be represented like this
\begin{verbatim}
h1:p1 ; ... ;hn:pn.
\end{verbatim}
If the clause has a single head with probability 1, the annotation can be omitted and the clause takes the form of a normal prolog clause, i.e. 
\begin{verbatim}
h1:- b1,...,bm,\+ c1,...,\+ cl.
\end{verbatim}
stands for 
\begin{verbatim}
h1:1 :- b1,...,bm,\+ c1,...,\+ cl.
\end{verbatim}



The body of clauses can contain a number of built-in predicates including:
\begin{verbatim}
is/2 >/2 </2 >=/2 =</2 =:=/2 =\=/2 true/0 false/0
=/2 ==/2 \=/2 \==/2 length/2 member/2
\end{verbatim}
The coin example above thus is represented as (see file \texttt{coin.cpl} in subdirecoty \texttt{examples})
\begin{verbatim}
heads(Coin):1/2 ; tails(Coin):1/2:- 
     toss(Coin),\+biased(Coin).
heads(Coin):0.6 ; tails(Coin):0.4:- 
     toss(Coin),biased(Coin).
fair(Coin):0.9 ; biased(Coin):0.1.
toss(coin).
\end{verbatim}

Subdirectory \texttt{examples} contains other example programs.

\subsection{Use}

First write your program in a file with extension \texttt{.cpl}, then load PITA in XSB with
\begin{verbatim}
:- [pita].
\end{verbatim}
load you program, say \texttt{coin.cpl}, with 
\begin{verbatim}
:- load(coin).
\end{verbatim}
and compute the probability of query atom \texttt{heads(coin)} by
\begin{verbatim}
:- prob(heads(coin),P).
\end{verbatim} 
%
\texttt{load(file)} reads \texttt{file.cpl}, translates it into a normal  program, writes the result in \texttt{file.P} and loads \texttt{file.P}.

PITA offers also the predicate \texttt{parse(infile,outfile)} which translates the LPAD in \texttt{infile} into a normal progam and writes it to  \texttt{outfile}.

Moreove, you can use \texttt{prob(goal,P,CPUTime,WallTime)} that returns the probability of goal \texttt{P} together with the CPU and wall time used.



%\ifnum\pdfoutput>0 % pdflatex compilation


``Probabilistic  Inference with Tabling and Answer subsumption'' (PITA) \cite{RigSwi10-ICLP10-IC} is a package for uncertain reasoning. In particular, it allowsvarious forms of Probabilistic Logic Programming and Possibilistic Logic Programming. It accepts the language of Logic Programs with Annotated Disjunctions (LPADs)\cite{VenVer03-TR,VenVer04-ICLP04-IC} and CP-logic programs \cite{VenDenBru-JELIA06,DBLP:journals/tplp/VennekensDB09}.

An example of LPAD/CP-logic program is 
\begin{eqnarray*}
(heads(Coin):0.5)\vee (tails(Coin):0.5)&\leftarrow&
toss(Coin),\neg biased(Coin).\\
(heads(Coin):0.6)\vee (tails(Coin):0.4)&\leftarrow&
toss(Coin), biased(Coin).\\
(fair(Coin):0.9) \vee (biased(Coin):0.1).&&\\
toss(Coin).&&
\end{eqnarray*}
The first clause states that if we toss a coin that is not biased it has equal probability of landing heads and tails. The second states that if the coin is biased it has a slightly higher probability of landing heads. The third states that the coin is fair with probability 0.9 and biased with probability 0.1 and the last clause states that we toss a coin with certainty.

PITA computes the probability of queries by tranforming the input program into a normal logic program and then calling a modified version of the query on the transformed programs.

\subsection{Installation}
PITA uses \href{http://www.gtk.org/}{GLib 2.0},
a standard GNU package 
so it is easy to install it using the package management software of your Linux 
distribution.


To install PITA with XSB, run XSB \texttt{configure} in the \texttt{build} directory with option \texttt{--with-pita} and then run \texttt{makexsb} as usual.

When compiling in cygwin, also build the cygwin dll with \texttt{makexsb cygdll}.


\subsubsection{Syntax}

Disjunction in the head is represented with a semicolon and atoms in the head are separated from probabilities by a colon. For the rest, the usual syntax of Prolog is used.
For example, the  CP-logic clause
$$h_1:p_1\vee \ldots \vee h_n:p_n\leftarrow b_1,\dots,b_m ,\neg c_1,\ldots,\neg c_l$$
is represented by
\begin{verbatim}
h1:p1 ; ... ; hn:pn :- b1,...,bm,\+ c1,....,\+ cl
\end{verbatim}
 No parentheses are necessary. The \texttt{pi} are numeric expressions. It is up to the user to ensure that the numeric expressions are legal, i.e. that they sum up to less than one.

If the clause has an empty body, it can be represented like this
\begin{verbatim}
h1:p1 ; ... ;hn:pn.
\end{verbatim}
If the clause has a single head with probability 1, the annotation can be omitted and the clause takes the form of a normal prolog clause, i.e. 
\begin{verbatim}
h1:- b1,...,bm,\+ c1,...,\+ cl.
\end{verbatim}
stands for 
\begin{verbatim}
h1:1 :- b1,...,bm,\+ c1,...,\+ cl.
\end{verbatim}



The body of clauses can contain a number of built-in predicates including:
\begin{verbatim}
is/2 >/2 </2 >=/2 =</2 =:=/2 =\=/2 true/0 false/0
=/2 ==/2 \=/2 \==/2 length/2 member/2
\end{verbatim}
The coin example above thus is represented as (see file \texttt{coin.cpl} in subdirecoty \texttt{examples})
\begin{verbatim}
heads(Coin):1/2 ; tails(Coin):1/2:- 
     toss(Coin),\+biased(Coin).
heads(Coin):0.6 ; tails(Coin):0.4:- 
     toss(Coin),biased(Coin).
fair(Coin):0.9 ; biased(Coin):0.1.
toss(coin).
\end{verbatim}

Subdirectory \texttt{examples} contains other example programs.

\subsection{Use}
\subsubsection{Probabilistic Logic Programming}
First write your program in a file with extension \texttt{.cpl}.
If you want to use inference on LPADs load PITA in XSB with
\begin{verbatim}
:- [pita].
\end{verbatim}
load you program, say \texttt{coin.cpl}, with 
\begin{verbatim}
:- load(coin).
\end{verbatim}
and compute the probability of query atom \texttt{heads(coin)} by
\begin{verbatim}
:- prob(heads(coin),P).
\end{verbatim} 
%
\texttt{load(file)} reads \texttt{file.cpl}, translates it into a normal  program, writes the result in \texttt{file.P} and loads \texttt{file.P}.

PITA offers also the predicate \texttt{parse(infile,outfile)} which translates the LPAD in \texttt{infile} into a normal progam and writes it to  \texttt{outfile}.

Moreove, you can use \texttt{prob(goal,P,CPUTime,WallTime)} that returns the probability of goal \texttt{P} together with the CPU and wall time used.

In case the modeling assumptions of PRISM hold, i.e.:
 \begin{itemize}
  \item the probability of a conjunction $(A,B)$ is
computed as the product of the probabilities of A and B (independence assumption),
\item
the probability of a disjunction $(A;B)$ is computed as the sum of
the probabilities of A and B
(exclusiveness assumption),
  \end{itemize}
you can perform faster inference with an optimized version of PITA in package \texttt{pitaindexc.P}. It accepts the same commands of \texttt{pita.P}.  \texttt{pitaindexc.P} simulates PRISM and does not need CUDD and GLib.


If you want to compute the Viterbi path and probability of a query (the Viterbi path is the explanation with the highest probability) as with the predicate \texttt{viterbif/3} of PRISM, you can use package \texttt{pitavitind.P}.

The package \texttt{pitacount.P} can be used to count the explanations for a query, provided that the independence assumption holds. To count the number of explanations for a query use
\begin{verbatim}
:- count(heads(coin),C).
\end{verbatim}
\texttt{pitacount.P} does not need CUDD and GLib.

\subsubsection{Possibilistic Logic Programming}
PITA can be used also for answering queries to possibilistic logic program \cite{DBLP:conf/iclp/DuboisLP91}, a form of logic progamming based on possibilistic logic \cite{DubLanPra-poss-94}. The package \texttt{pitaposs.P} provides possibilistic inference. 
You have to write the possibilistic program as an LPAD in which the rules have a single head whose annotation is the lower bound on the  necessity of the clauses. To compute the highest  lower bound on the necessity of a query
use
\begin{verbatim}
:- poss(heads(coin),P).
\end{verbatim}
\texttt{pitaposs.P} does not need CUDD and GLib.
