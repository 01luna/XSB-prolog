
\newcommand{\pxversion}{Version 1.0 (beta)}
\newcommand{\px}{{\tt px}}

\begin{center}
%\chapter[Python Calling XSB]{\px{}: The Python 3-XSB Interface} \label{chap:px}
\chapter{\px{}: The Python 3-XSB Interface} \label{chap:px}
\end{center}

\vspace*{-.30in} 
\begin{center}
{\Large {\bf  \pxversion}}
\end{center}

\begin{center}
  {\Large {\bf By Theresa Swift}}
\end{center}

\noindent
    {\large {\bf {\em px has been tested on MacOs and various versions of
        Linux. It is not currently working on Windows.}} See
      Section~\ref{sec:px-limits} for a list of currently unsupported
      features and known bugs.}

The \px{} module allows XSB to be used as an embedded subprocess
within a Python process.  \px{} currently works independently of {\tt
  xsbpy}, although they both use much of the same code and in
particular bi-translate Prolog terms and Python data structures in the
same manner.\footnote{Fully combining the two systems is a high
  priority for development.}

\px{} is only slightly slower than {\tt xsbpy}.  Nearly a million
calls to, and returns from a simple XSB predicate can be made per
second.  Data is transferred very quickly: for instance list elements
can be transferred at a few tens of nanoseconds per list element.

\section{Configuration, Loading and Start-up} \label{px:config}

%{\sc Working draft -- not complete.
%  \bi
%  \item Not mentioning LD\_LIBRARY or XSB\_ROOT issues.
%  \item  only describing Linux.
%  \item Once the above kinks are out, I need to integrate this
%    documentation it with the xsbpy configuration.
%    \ei
%    }

%Also, because \px{} is installed as a site
%package for a given version of Python, we strongly recommend using a
%Python virtual envonment, just as a virtual Python environment is
%usually necessary for installation via {\tt pip}.  Fortunately, once
%the appropriate Python packages are present and a user has set up a
%virtual Python environment, \px{} is easy to set up.

\subsection{Installation on Linux and macOS}

\subsubsection{System Prerequisites}

\begin{itemize}
\item {\bf Linux}

%To properly execute these steps the proper Python system packages must
%have been installed.

Beyond Linux packages like {\tt build-essential} that are needed to
build and make XSB itself, \px{} requires Python development packages
to have been installed for the Python version of interest.  (See
Section~\ref{sec:xsbpy-linux} for an overview of installing such
dependencies.) On Ubuntu, and other Debian-derived Linuxes
installation also requires installing {\tt \$\{PYTHON\}-distutils},
though this is not required for other Linuxes. For a given version of
Python {\tt \$PYTHON} on Ubuntu, the command to install these Python
packages would be:

{\tt sudo apt-get install \$\{PYTHON\}-dev \$\{PYTHON\}-distutils}

On Fedora-based Linuxes, the {\tt dnf} command must be
used.\footnote{\px{} has been tested on Ubuntu v. 18 and v. 20, and on
  Fedora 35; and Python versions 3.7, 3.8, 3.9 and 3.10 were tested.
  We belive that \px{} will work on other recent Unix distributions
  and Python versions as well.}

%Since \px{} requires {\tt distutils}, the {\tt python-distutils}
%package also needs to have been installed.

%On Linux, setting up a Python virtual environment requires the {\tt
%  python-dev} and {\tt python-venv} packages to be installed,

\item {\bf macOS}
\end{itemize}

\subsubsection{Configuring \px{} on Linux and macOS}

\paragraph{Quick Start}
\begin{enumerate}
\item cd to {\tt \$XSB\_ROOT/XSB/packages/xsbpy} and execute  {\tt bash px\_configure.sh}
\item {\tt source \$XSB\_HOME/packages/xsbpy/px\_activate} from anywhere
\item start using px!
\end{enumerate}

\paragraph{Discussion}
If the correct Python libraries have been installed, configuration of
{\tt px} should be simple, and nearly the same on Linux and
macOS.  When XSB is first configured, via
\begin{verbatim}
$XSB_ROOT/XSB/build/configure
\end{verbatim}
the {\tt xsbpy} package, which allows XSB to call Python, is also
configured and compiled.  Using files created during the {\tt xsbpy}
build, \px{} is configured and build by cd-ing to
\begin{verbatim}
$XSB_ROOT/XSB/packages/xsbpy
\end{verbatim}
\noindent
and executing
\begin{verbatim}
  bash px_configure.sh
\end{verbatim}
This second configuration step uses XSB to generate a {\tt setup.py}
file for the Python {\tt distutils} package, and then builds and
installs {\tt px}.  \footnote{At some point, this confuration will be
  changed to use Python's {\tt setuptools}.}  The compiled C
extensions for Python are build and put in the user's {\tt
  site-packages} directory, i.e.,
\begin{verbatim}
~/.local/lib/python3.10/site-packages/
\end{verbatim}
In addition the second configuration creates a file, {\tt
  px\_activate}, somewhat analogous to activation code in {\tt venv}
or other virtual environments.\footnote{This activation file updates
  the {\tt LD\_LIBRARY\_PATH} used by the {\tt ld} command, and adds
  the {\tt px} directory to {\tt PYTHONPATH}.  On MacOs {\tt
    LD\_LIBRARY\_PATH} is also updated.} In {\tt bash} or {\tt zsh},
simply type
\begin{verbatim}
source $XSB_HOME/packages/xsbpy/px_activate
\end{verbatim}
and you're ready to go.

%\begin{itemize}
%\item As a default, if the command
%
%  {\tt \$XSB\_ROOT/XSB/packages/xsbpy/configure}
%
%  is issued, \px{} is installed in the directory corresponding to {\tt
%    sys.USER\_SITE} in the version of Python {\tt \$Python} with which
%  {\tt xsbpy} was configured. In Linux this is under the users home
%  directory (\verb|~|):
%\begin{verbatim}
%~/.local/lib/$PYTHON/site-packages
%\end{verbatim}
%If this default directory is used, {\tt px} will only be available to
%the user who installed it, and {\tt px} will be available for all
%Python projects for that user.
%
%\item If \px{} is to be installed in a Python virtual environment
%  that uses {\tt venv}, the virtual environment will first need to be
%  activated in a given shell.  In {\tt bash} this is done as:
%
%\begin{verbatim}
%source $VENV_DIR/bin/activate
%\end{verbatim}
%
%Where {\tt \$VENV\_DIR} is the root directory for the virtual
%environment.  Once this is done, the command
%
%\begin{verbatim}  
%$XSB_ROOT/XSB/packages/xsbpy/configure ???
%\end{verbatim}
%
%is issued, \px{} will be installed in 
%
%\begin{verbatim}
%$VENV_DIR/lib/$PYTHON/site-packages
%\end{verbatim}
%
%\item For system administrators, if the command
%
%\begin{verbatim}  
%sudo $XSB_ROOT/XSB/packages/xsbpy/configure ???
%\end{verbatim}
%
%  is issued, \px{} will be installed in
%
%\begin{verbatim}  
%/usr/local/lib/$PYTHON/dist-packages
%\end{verbatim}  
%
%  If this option is used, \px{} will be avaiable to anyone who uses
%  the version {\tt \$PYTHON}.
%
%\item As a final alternative, the command 
%
%\begin{verbatim}  
%  $XSB_ROOT/XSB/packages/xsbpy/configure ???
%\end{verbatim}
%
%will install \px{} directly in the specified directory.
%\end{itemize}
%
%{\sc TES: If possible, I need to redo setup.py to include px.py in the
%  egg so that px.py does not need to be in the library path.}

Once \px has been built and activated, \px{} can be used just as any other package
installed in {\tt \$\{PYTHON\}}.

\begin{verbatim}
$PYTHON
>>> from px import *
[xsb_configuration loaded]
[sysinitrc loaded]
[xsbbrat loaded]
[xsbpy loaded, cpu time used: 0.001 seconds]
xsbpy_initted_with_python(auto(python3.9))
[px loaded]
[px_test loaded]
\end{verbatim}

\noindent
Note that XSB is initialized in the Python process when the \px{}
module is loaded.

As a final point, you can test {\tt px} by running the script {\tt
  \$XSB\_ROOT/xsbtests/python\_tests/test.sh} via the command {\tt
  test.sh <python>} using whatever version of Python was used to
configure {\tt px}.  This directory also contains a number of examples
you can try.

%------------------------------------------------------------

%\noindent
%Once the appropriate packages are present, a virtual environment can
%be set up, usually in a user's home directory via a command like:

%{\tt \$\{PYTHON\} -m venv {\tt \$VENV\_DIR}}

%Next, the virtual environment needs to be activated in a command
%shell,  In {\tt bash} this is done as:

%{\tt source \$VENV\_DIR/bin/activate  }

%In the same shell the \px{} configuration will be run.  If XSB has
%not already been configured and made, you'll need to do so as
%discussed in Volume 1 of this manual, which will configure \px{}
%(and {\tt xsbpy}).  Otherwise, {\tt cd} to the {\tt xsbpy}
%subdirectory and run the command
%
%{\tt ./configure}
%%

%\subsection{Configuring on Linux Systems}

%Currently {\tt setup.py} requires include and
%library directories as shown below.  The paths to these directories
%must be changed to include the proper {\tt \$XSB\_ROOT} and, in the
%case of the library directory the path of the XSB binary appropriate
%to your system.  An example is:

%\begin{footnotesize}
%  \begin{verbatim}
%module1 = Extension('xsbext',
%                    include_dirs = ['/usr/include/python3.8',
%                                    '/home/tswift/xsb-code/XSB/emu',
%                                    '/home/tswift/xsb-code/XSB/packages/xsbpy',
%                                    '/home/tswift/xsb-code/XSB/config/x86_64-unknown-linux-gnu'],
%                    libraries = ['xsb'],
%                    library_dirs = ['/home/tswift/xsb-code/XSB/config/x86_64-unknown-linux-gnu/bin'],
%                    sources = ['xsbext.c','xsbpy_common.c'])
%  \end{verbatim}
%\end{footnotesize}
%
%Once {\tt setup.py} is properly configured, the user need only enter a
%python comaand to build and install the module ({\tt sudo} privilege
%may be needed for this).  For the above file the command is:
%
%\begin{verbatim}
%%python3.8 setup.py install
%end{verbatim}
%\noindent
%This command compiles the px sources, builds a Python egg, and
%installs the egg in the Python libary for distributed packages
%(e.g. {\tt dist-packages})

\subsection{Configuring PX on Windows}

The {\tt px} configuration is not yet working for Windows.  If you
have experience in Windows configuration and would like to help,
please let us know.

\section{Using \px}

The intention of \px{} is to make essentially all of XSB's
functionality directly callable by Python.  As noted, many of these
features are still under development.

Although Python and Prolog have similarities at the data structure
level (Section~\ref{sec:bi-translation}) they differ substantially in
their execution.  The design of \px{} is to restrict the form of
Prolog goals so that they fit well with Python's function execution.
Fortunately, this can be done without sacrificing XSB's expressivity.
\px{} functions are variadic and pass to XSB a module, a predicate
name, zero or more input arguments and zero or more keyword arguments.
At a general level, XSB then constructs a call from these pieces,
executes the call and passes the answer substitutions back to Python
as a function return. Exactly how this is done differs for the various
\px{} functions, which we now introduce by a series of examples.  

\subsection{Deterministic Queries and Commands} \label{sec:pxdet}

In these examples, many features of \px{} are presented via its
support for commands and non-deterministic queries before turning to
general support of deterministic queries.

\begin{example} \rm
{\em Calling a deterministic query}
  
  As described in Section~\ref{px:config} \px{} is loaded like any
  other module.  Once loaded, a simple way to use \px{} is to
  execute a deterministic query to XSB.  The Python statement:

\begin{verbatim}  
>>> Answer,TV = px_qdet('basics','reverse',[1,2,3,('mytuple'),{'a':{'b':'c'}}])
\end{verbatim}

\noindent
  asks XSB to reverse a list using {\tt basics:reverse(+,-)} -- i.e.,
  with the first argument ground and the second argument free.  To
  execute this query the input list along with the tuple and
  dictionary it contains are translated to an XSB term as described in
  Section~\ref{sec:bi-translation}, the query is executed, and the
  answer translated back to Python.  The return of {\tt px\_qdet()} is
  in fact a {\em return tuple} containing an {\em answer} ({\tt
    Answer} above) and its {\em truth value} ({\tt TV} above).  In this
    case {\tt Answer} is

  {\tt [{'a':{'b':'c'}},('mytuple'),3,2,1]}

\noindent    
and the truth value {\tt TV} is {\tt 1} indicating that the answer is
true.  For deterministic queries and commands the truth value returned
can have the value:

\bi
\item {\tt 1} representing the truth value {\em true}.  This means the
  XSB query succeeded and that the answer is true in the Well-Founded
  Model of the program.\footnote{See Volume 1 Chapter 5 of this manual
    for an explanation of the well-founded model and XSB's
    three-valued logic.}
\item {\tt 0} representing the truth value {\em false}.  This means
  the XSB query failed and has no answers in the Well-Founded Model of
  the program.  If the return tuple has truth value {\em false}, the answer
  will always be the Python {\tt None} keyword.
\item {\tt 2} representing the truth value {\em undefined}. This means
  that the XSB query succeeded, but the answer is neither {\em true} nor
  {\em false} in the Well-Funded Model of the program.  \ei
\noindent
Although XSB's three-valued logic can be highly useful for many
purposes, it can be safely ignored in many applications, and most
queries will either succeed with true answers or will
fail.\footnote{As shown in Example~\ref{ex:px-comp-undef}, truth
  values can also be represented by delay lists in list and set
  comprehensions.}

For learning \px{} or for tutorials, a family of pretty printing calls
can be useful.  The {\tt pp\_px\_qdet()} function calls {\tt
  px\_qdet()} and upon return pretty prints both the call and return
in a style like that used in XSB''s command line interface.  For
example if the following call is made on the Python command line
interface:

\begin{verbatim}
>>> pp_px_qdet('basics','reverse',[1,2,3,('mytuple'),{'a':{'b':'c'}}])
\end{verbatim}
\noindent
the function will print out both the query and answer in Prolog syntax
as if it were executed in XSB.\footnote{Note that {\tt
    pp\_px\_query()} does not change the Python command line interface
  -- it simply prints out the query and the answer both in Prolog
  syntax.}

\begin{verbatim}
?- basics:reverse(([1,2,3,''('mytuple'), pyDict([('a',pyDict([('b','c')]))]),Answer).

   Answer  = [pyDict([('a', pyDict([('b', 'c')]))]), 'mytuple', 3, 2, 1]
   TV = True
\end{verbatim}
\end{example}

Note that the Python calls in the above example each had a module name
as its first position, a function name in its second position, and the
Prolog query argument in its third position.  Translation to XSB adds
an extra unbound variable as the last argument in the query so that
the query had two arguments.  The following example illustrates the
flexibility of {\tt px} in constructing Prolog queries.

\begin{example} \rm {\it Detour: Varying the number of arguments in queries and commands}

%  {\sc The {\tt vars} keyword is not yet implemented.}

  Suppose you wanted to make a ground XSB query, say {\tt ?- p(a)}:
  the information answered by this query would simply indicate whether
  the atom {\tt p(a)} was {\tt true}, {\tt false},,or {\tt undefined}
  in the well-founded model.In {\tt px} such a query could simply be
  made as

\begin{verbatim}  
>>> px_cmd('px_test','p','a')  
\end{verbatim}  

%Note that unlike (the default behavior of) {\tt px\_qdet()}, {\tt
%  px\_cmd()} does not add an extra return argument to the Prolog
%call.
Since {\tt px\_cmd} does not return any answer bindings, it returns
the truth value directly to Python, rather than as part of a tuple.
Note that unlike (the default behavior of) {\tt px\_qdet()}, {\tt
  px\_cmd()} does not add an extra return argument to the Prolog call.
However, {\tt px\_qdet()} and {\tt px\_cmd()} are both variadic
functions so that the number of input arguments can also vary as shown
in the table below.

In \pxversion{} of {\tt px}, {\tt px\_cmd} adds no return variables,
while {\tt px\_qdet()} adds a single return variable.  More generality
is allowed in {\tt px\_comp()} discussed more fully below.  In {\tt
  px\_comp()} the optional keyword argument {\tt vars} can be used to
  indicate the number of return arguments desired.  So, if {\tt
    vars=2} were added as a keyword argument, two arguments arguments
  would be added to the call, with each a free variable.  Combining
  both approaches, a variety of different Prolog queries can be made
  as shown in the following table.~\footnote{Support for the {\tt
      vars} keyword option in {\tt px\_qdet()} requires a partial
    reimplementation of XSB's C embeeding interface, but may be made
    available in future versions.  Of course in the meantime {\tt
      px\_qdet()} can always pass back multiple argument values via
    Python tuples or other means.}

\begin{tabular}{lll}
  {\tt px\_cmd('mod','cmd')}                & calls the goal & {\tt mod:cmd()}\\
  {\tt px\_cmd('mod','cmd','a')}            & calls the goal & {\tt mod:cmd(a)}\\
  {\tt px\_cmd('mod','cmd','a','b')}        & calls the goal & {\tt mod:cmd(a,b)}\\
   \\ 
  {\tt px\_qdet('mod','pred')}              & calls the goal & {\tt mod:pred(X1)} \\
  {\tt px\_qdet('mod','pred','a')}          & calls the goal & {\tt mod:pred(a,X1)} \\
  \\
  {\tt px\_comp('mod','pred'),vars=2}       & calls the goal & {\tt mod:pred(X1,X2)} \\
  {\tt px\_comp('mod','pred','a',vars=0)}   & calls the goal & {\tt mod:pred(a)}. \\
  {\tt px\_comp('mod','pred','a',vars=1)}   & calls the goal & {\tt mod:pred(a,X1)} \\
  {\tt px\_comp('mod','pred','a',vars=2)}   & calls the goal & {\tt mod:pred(a,X1,X2)} \\
  {\tt px\_comp('mod','pred','a','b',vars=2)} & calls the goal & {\tt mod:pred(a,b,X1,X2)} \\
\end{tabular}

Several points are worth noting here.  If the keyword argument {\tt
  vars=N} is included in a {\tt px\_comp()} call and $N$ is not {\tt
  1} the answer substitutions to the arguments (the answer part of the
return tuple) will itself be a tuple.  Next, although the number of
ground and free arguments can vary, when the XSB goal is constructed,
all ground arguments will precede all non-ground arguments.  In
addition \px{} does not provide a way to represent logical variables
in Python so that XSB will see any argument as fully ground.
(However, as shown in Example~\ref{ex:connstraints} it is possible to
pass logical variables to XSB via strings.)\footnote{Because {\tt
    px\_cmd()} is slightly simpler than {\tt px\_qdet()} it may be
  very slightly faster.}
% Finally, a call to {\tt
%  px\_qdet()} with {\tt vars=0} acts in much the same way as a call to
%{\tt px\_cmd}.
\end{example}

Using {\tt px\_qdet()} and {\tt px\_cmd()} the XSB session can be
controlled from Python.  Furthermore, any errors that arise in XSB can
also be handled in Python.

\begin{example} \rm {\it Session management and error handling in \px}

  Once {\tt px} has been imported (initializing XSB), any user XSB
  code can be loaded easily.  One can execute
  
\begin{verbatim}
>>>  px_cmd('consult','consult','xsb_file')
\end{verbatim}

\noindent
which loads the XSB file {\tt xsb\_file.\{P,pl\}}, compiling it if
necessary.  For convenience, {\tt px} also defines a shortcut for
consulting since it is a common operation:

\begin{verbatim}
>>>  px.consult('xsb_file')
\end{verbatim}

\noindent
{\tt px} also provides shortcuts for some other frequent Prolog calls
-- other desired shortcuts are easily implemented via Python
functions.

If a Prolog file {\tt xsb\_file.P}, is modified it can be reconsulted
in the same session just as if the XSB interpreter were being used.
Indeed, using {\tt px}, the Python interpreter can be used as a
command-line interface for writing and debugging XSB code (although
the XSB interpreter is recommended for most XSB development tasks).

If an exception occurs when XSB is executing a goal, the error can be
caught in XSB by {\tt catch/3} in the usual manner.  If the error is
not caught by user code, it will be caught by {\tt px}, translated to
a Python exception of the vanilla {\tt Exception} class,\footnote{A
  future implementation will use more precise mapping of XSB error
  types to Python error types.}  and can be caught as any other Python
exception of that type.  Precise information on the XSB exception is
available to Python through the \px{} function {\tt
  px\_get\_error\_message()},

Consider what happens when trying to consult a file that doesn't exist
in any of XSB library paths.  In this case, XSB's {\tt consult/1}
throws an exception, the {\tt px} sub-system catches it and raises a
Python error as mentioned above.  Returning to the pretty-printing {\tt
  pp\_px\_qdet()} the call

\begin{verbatim}
>>> pp_px_qdet('usermod','open','missing_file','read')
\end{verbatim}

\noindent
catches the exception and prints out

\begin{verbatim} 
Exception Caught from XSB: 
      ++Error[XSB/Runtime/P]: [Permission (Operation) open[mode=r,errno= ENOENT: 
No such file or directory] on file: missing_file]  in /(open,3)
\end{verbatim}

To obtain similar behavior in any other {\tt px} function, simply call
the function in the following block:

\begin{verbatim}
    try
      <some px_function>
    except Exception as err:
      display_xsb_error(err)
\end{verbatim}
\noindent
where {\tt display\_xsb\_error()} is a call to the function:

\begin{verbatim}
def display_xsb_error(err):    
        print('Exception Caught from XSB: ')
        print('      ' + px_get_error_message())
\end{verbatim}

\noindent
Here, {\tt px\_get\_error\_message()} is calls C to return the last
\px{} error text as a string.

As a final point if an exception arises during execution of a {\tt px}
function the function returns the value {\tt None} in addition to
setting a Python Error.
\end{example}

\subsection{Comprehension of Non-Deterministic Queries}

A declarative aspect of Python is its support for comprehension of
lists, sets and aggregates.  \px{} fits into non-deterministic queries
into this paradigm with {\em query comprehension}: calls to XSB that
return all solutions to an XSB query as a Python list or set.

\begin{example} \rm {\it List and Set Comprehension in \px{}} \label{ex:list-comp-1}

  Consider the XSB predicate, {\tt test\_comp/2} defined as

{\small  
\begin{verbatim}  
test_comp(a,1).
test_comp(b,2).
test_comp(c,3).
test_comp(d,4).
test_comp(e,5):- unk(something),p.
test_comp(e,5):- q,unk(something_else).
test_comp(e,5):- failing_goal,unk(something_else).

p.
q.
\end{verbatim}
}

\noindent
An XSB query to this predicate may be determinisitic (e.g., {\tt ?-
  test\_comp(a,X1)}) or non-deterministic (e.g., {\tt ?-
  test\_comp(X1,X2)}).  This second query will have multiple answers,
some of whose truth values are {\em undefined}.  The \px{} query

\begin{verbatim}
>>>  px_comp('px_test','test_comp',vars=2)
\end{verbatim}

\noindent
calls the XSB goal {\tt ?- px\_test:test\_comp(X1,X2)} in a manner
similar to that shown above for {\tt px\_qdet()}.  The difference is
that {\tt px\_comp()} returns multiple solutions as a list

{\small  
\begin{verbatim}
     [
      ((e,5),2),
      ((e,5),2),
      ((d,4),1),
      ((c,3),1),
      ((b,2),1),
      ((a,1),1) 
     ]
\end{verbatim}
}

\noindent
In the above list, two answers for {\tt test\_comp(e,5)} have truth
values of {\em undefined}: that there are two answers is expected
since there are two different ways in which {\tt test\_comp(e,5)} is
undefined, as will be discussed in Example~\ref{ex:px-comp-undef}.  To
remove this redundancy, the function call:

\begin{verbatim}
>>>  px_comp('px_test','test_comp',vars=2,set_collect=True)
\end{verbatim}

\noindent
returns a set rather than a list:\footnote{Due to how sets are
  implemented in Python, the order in which the set elements are
  returned is non-deterministic.}

{\small  
\begin{verbatim}
     {
      ((d),1),
      ((a),1),
      ((e),2),
      ((b),1),
      ((c),1) 
     }
\end{verbatim}
}

Whether it is a list or a set, the return of {\tt px\_comp()} will be
iterable and can be used as any other object of its type, for
example:

{\small
  \begin{verbatim}
>>> for answer,tv in px_comp('px_test','test_comp',vars=2):
...     print('answer = '+str(answer)+' ; tv = '+str(tv))
... 
answer = ('e', 5) ; tv = 2
answer = ('e', 5) ; tv = 2
answer = ('d', 4) ; tv = 1
answer = ('c', 3) ; tv = 1
answer = ('b', 2) ; tv = 1
answer = ('a', 1) ; tv = 1
\end{verbatim}
  }
\end{example}

With {\tt px\_comp()} the truth values can optionally be removed.  The
reason {\tt px\_comp()} can support this is that {\tt px\_comp()}
returns a list or set of {\em true} or {\em undefined} answers.  If
there are no answers that satisfy the query {\tt px\_comp()} returns
the empty list or set.  This behavior is different than {\tt
  px\_qdet()} and {\tt px\_cmd()} that may also fail in their
execution and require a {\em false} truth value for this.

\begin{example} \rm {\it Simplifying Comprehensions} \label{ex:simple-comp}

  Suppose you have a predicate {\tt simple\_comp/2} defined as:

{\small  
\begin{verbatim}  
simple_comp(a,1).
simple_comp(b,2).
simple_comp(c,3).
simple_comp(d,4).
\end{verbatim}
}

  The clauses in {\tt simple\_comp/2} are the same as the first four
  clauses in {\tt test\_comp/2}, and due to this restriction, a query
  to this predicate will produce only answers whose truth value is
  {\em true}.  In this case, the function call

\begin{verbatim}  
>>>  px_comp('px_test','test_comp',truth_vals=NO_TRUTHVALS)}
\end{verbatim}

will return only the answer substitution for each element of the list.

{\small  
\begin{verbatim}
     [
      (e,5),
      (e,5),
      (d,4),
      (c,3),
     ]
\end{verbatim}
}

The keyword argument value {\tt NO\_TRUTHVALS} is a (constant)
variable in the Python module {\tt px}.
\end{example}

\subsection{Callbacks from XSB to Python} \label{sec:callbacks}

When XSB is called from Python, {\tt xsbpy} can easily be used to make
callbacks to Python.  For instance, the query:

\begin{verbatim}
px_qdet('px_callbacks','test_json')
\end{verbatim}

\noindent
calls the XSB goal {\tt px\_callbacks:test\_json(X)} as usual.  The
file {\tt px\_callbacks.P} can be found in the directory

\verb|$XSB_ROOT/xsbtests/python_tests|

\noindent
This directory contains many other examples including those discussed
in this chapter.  In particular, {\tt px\_callbacks.P} contains the
predicate:

\begin{verbatim}
test_json(Ret):- 
   pyfunc(xp_json,
          prolog_loads('{"name":"Bob","languages": ["English","Fench","GERMAN"]}'),
          Ret).
\end{verbatim}

\noindent
that calls the Python Json loader as in Example~\ref{ex:xsbpy-json}, returning the tuple
\begin{verbatim}
({'name': 'Bob', 'languages': ['English', 'Fench', 'GERMAN']}, 1)
\end{verbatim}
to Python.  This example shows how easy it can be for XSB and Python
to work together: the Python call
\verb|px_qdet('px_callbacks','test_json')| causes the Json structure
to be read from a string into a Python dictionary, translated to a
Prolog dictionary term, and then back to a Python dictionary
associated with its truth value.

As another example of callbacks consider the goal:

\begin{verbatim}
NewClass,TV = px_qdet('px_callbacks','test_class','joe')
\end{verbatim}
\noindent
that calls the XSB rule:
\begin{verbatim}
test_class(Name,Obj):-
   px_qdet('px_callbacks','test_class','joe')
\end{verbatim}
that in turn creates an instance of the class {\tt Person} via:
\begin{verbatim}
class Person:
  def __init__(self, name, age, favorite_ice_cream=None):
    self.name = name
    self.age = age
    if favorite_ice_cream is None:
      favorite_ice_cream = 'chocolate'
    self.favorite_ice_cream = favorite_ice_cream
\end{verbatim}
The reference to the new {\tt Person} instance is returned to Prolog,
then back to Python and assigned to the variable {\tt NewClass}.
Afterwards, accessing {\tt NewClass} properties from the original
Python command line:
\begin{verbatim}
>>> NewClass.name
\end{verbatim}
is evaluated to {\tt 'joe'} as expected.  In other words, the Python
environment calling XSB and that called by XSB are one and the same.
The coupling between Python and XSB is both implementationally tight
and semantically transparent; micro-computations can be shifted between
XSB and Python depending on which language is more useful for a given
purpose.

\subsection{Constraints and Residual Programs} 

The material in this section is not necessary to understand for a
basic use of \px{}, but shows how \px{} can be used for
constraint-based reasoning, and how it can interface to the results of
non-monotonic reasoning.

\begin{example} \rm {\it Evaluating queries with constraints} \label{ex:connstraints}

  XSB provides support for constraint-based reasoning via
  CLPR~\cite{Holz95} and other packages, both for Prolog-style and for
  tabled resolution.  However, using constraint-based reasoners like
  CLPR requires explicit use of logical variables
  (cf. Chapter~\ref{chap:constraints}), and as mentioned in
  Section~\ref{sec:pxdet}, \px{} does not provide a direct way to
  represent logical variables since logical variables do not naturally
  correspond to a Python type.  Fortunately, it is not difficult to
  pass constraint expressions containing logical variables to XSB
  within Python strings.

  Consider a query to find whether 

  \[X  > 3*Y + 2,Y>0 \models X > Y\]

\noindent  
  In CLPR this is done by writing a clause to assert the two
  constraints -- in Prolog syntax as calls to the literals {\tt \{X
    $>$ 3*Y + 2\}} and {\tt \{Y$>$0\}} -- and then calling the CLPR
  goal {\tt entailed(Y$>$0)}.  Within XSB, one way to generalize the
  entailment relation into a predicate would be to see if one set of
  constraints, represented as a list, implies a given constraint:

{\small  
\begin{verbatim}  
:- import {}/1,entailed/1 from clpr.

  check_entailed(Constraints,Entailed):- 
     set_up_constraints(Constraints),
     entailed(Entailed).

  set_up_constraints([]).
  set_up_constraints([Constr,Rest]):- 
     {Constr},
     set_up_constraints(Rest).
\end{verbatim}
}
\noindent
Using our example, a query to this predicate would have the form

\begin{verbatim}
 ?- check_entailed([X  > 3*Y + 2,Y>0],X > Y).
\end{verbatim}
\noindent
This formulation requires the logical variables {\tt X} and {\tt Y} to
be passed into the call.  Checking constraint entailment via \px{}
only requires writing the constraints as a string in Python and later
having XSB read the string.  A predicate to do this from Python is
similar to {\tt check\_entailed/2} above, but unpacks the constraints
from the input atom (i.e., the XSB translation of the Python string).

{\small
\begin{verbatim}
:- import read_atom_to_term/3 from string.

px_check_entailed(Atom):-
    read_atom_to_term(Atom,Cterm,_Cvars),
    Cterm = [Constraints,Entailed],
    set_up_constraints(Constraints),
    entailed(Entailed).
\end{verbatim}
}

The function call from Python is simply:

\begin{verbatim}
>>>  px_cmd('px_constraints','check_entailed','[[X  > 3*Y + 2,Y>0],[X > Y]]')
\end{verbatim}
\noindent
Note that the only difference when calling from Python is to put the
two arguments together into a single Python string, so that XSB's
reader treats the {\tt Y} variable in the input constraints and the
entailment constraint as the same~\footnote{Code for this is contained
  in the file {\tt px\_clpr.P}.}
\end{example}

\begin{example} \rm {\it Accessing the XSB's Residual Program} \label{ex:px-comp-undef}

As discussed, XSB can evaluate queries to programs whose well-founded
model is 3-valued.  In fact, XSB's evaluation can be seen as a program
transformation that forms the three-valued reduct of each clause used
in an evaluation~\cite{ChWa96}.  This reduct is called the {\em
  residual program} for a given query to a given program, and
non-empty bodies in the resudual program are called {\tt delay lists}.
Recall from Example~\ref{ex:list-comp-1} that {\tt test\_comp(e,5)} is
{\em undefined} in the well-founded model.  The residual program after
evaluating the query {\tt test\_comp(e,X)} is (in rule form):

\begin{verbatim}
test_comp(e,5):- unk(something).
test_comp(e,5):- unk(something_else).
\end{verbatim}

Note that compared to the source program in
Example~\ref{ex:list-comp-1}, the true literals {\tt p} and {\tt q}
have been removed from the rule bodies in which they occurred, while
the rule body containing {\tt failing\_goal} has been removed.

Residual programs have many uses.  They can be sent to an ASP engine
to evaluate partial stable models, or the residual program can be
examined to determine why a given literal is undefined. The reason
could be a negative loop somewhere in the program, due to floundering
negation, or due to a type of bounded rationality supported in XSB
called {\tt restraint} (cf. e.g. \cite{GroS13}).

In our running example, the function call

\begin{verbatim}  
>>>  px_comp('px_test','test_comp',truth_vals=DELAY_LISTS)}
\end{verbatim}

\noindent
returns the delay lists of answers.  

\begin{verbatim}
     [
      ((e,5),[unk(something_else)]),
      ((e,5),[unk(something)]),
      ((d,4),[]),
      ((c,3),[]),
      ((b,2),[]),
      ((a,1),[]) 
     ]
\end{verbatim}

Any answer whose delay list is empty ({\tt []}) is {\em true}, and any
other delay list indicates that an answer is {\em undefined}.  The two
delay lists for {\tt test\_comp(e,5)} show two instances of this
answer were returned in previous queries where {\tt sec\_collection}
was not used.
  \end{example}

\subsection{Other \px{} Resources and Examples}

Many of the examples from this chapter have been saved into Jupyter
notebooks in {\tt \$XSB\_ROOT/XSB/examples}, with associated PDF files in
{\tt \$XSB\_ROOT/docs/JupyterNotebooks}.

In addition, the directory {\tt \$XSB\_ROOT/xsbtests/python\_tests}
contains a series of tests for the examples in this chapter and many
others.  It is invoked by {\tt test.sh} and makes use of several
different Python and XSB files.

%----------------------------------------------------------------------
\section{The \px{} API}

\begin{description}

  \pymoditem{px\_cmd(module,pred,*args)}{px\_cmd()}{px}
%
  Allows Python to execute a Prolog goal {\tt Goal} containing no
  variables.  Each argument in {\tt Goal} corresponds to an element
  in {\tt args}, i.e., the input is translated to
  $module.pred(\vec{args})$, where $\vec{args}$ is an argument vector.
  For instance the Python call

  {\tt px\_cmd('consult','ensure\_loaded','px\_test')}

  calls {\tt consult:ensure\_loaded(px\_test)}.  Calls to XSB
  predicates that are not in a module may be made with {\tt module}
  set to {\tt usermod}.

  {\tt px\_cmd()} returns the truth value of the goal as explained
  above, or {\tt None} if an error occurred.
  
  \pymoditem{px\_qdet(module,pred,*args)}{px\_qdet()}{px}
%
  Allows Python to execute a Prolog query, the last argument of which
  is a variable.  If the number of {\tt args} is {\tt n}, a call will
  be made to {\tt pred/(n+1)} in which the first {\tt n} arguments
  correspond to the arguments in {\tt args} and the binding of the 
  final argument is returned to Python as a Python object along with
  its truth value, i.e. a call $module.pred(\vec{args},Ret)$ is
  created, where the binding of $Ret$ forms part of the return.  For
  example: the call

  {\tt px\_qdet('basics','reverse',[1,2,3,{'a':{'b':'c'}}]))}

  \noindent
  executes the prolog goal {\tt
    basics.reverse([1,2,3,{'a':{'b':'c'}}],Ret} and passes back the tuple

  {\tt ( [{'a':{'b':'c'}},3,2,1], 1)}

  \noindent
  to Python, where the first argument of the {\em return tuple} is the
  {\em answer} and the second is the truth value of that answer.  See
  Section~\ref{sec:pxdet} for precise information on truth values.

%  \bi
%  \item If {\tt truthValue} is {\tt 1} (true) or {\tt 2} (undefined),
%    {\tt Goal} succeeded.  The return of {\tt px\_qdet()} is a tuple
%    {\tt (dataStructure,truthValue)} in which {\tt dataStructure}
%    corresponds to the binding of the last argument of {\tt Goal}, and
%    {\tt truthValue} is the truth value of the answer;
%   \item If {\tt truthValue} is {\tt 0} (false), {\tt Goal} failed,
%     and the return of {\tt px\_qdet()} is a tuple {\tt (None,0)};
%   \item If the execution of {\tt Goal} caused a Prolog error the
%     Python {\tt None} object is returned.
%     \ei
%
%     If {\tt Goal} is non-deterministic, only the first answer will be
%     produced; multiple answers for a goal can be returned by {\tt
%       px\_list} and {\tt px\_set}.
%
%     Note that {\tt px\_cmd()} and {\tt px\_query()} will always call
%     different Prolog goals.  For instance {\tt
%       px\_cmd(mod,pred,arg1,arg2)} will call the Prolog goal {\tt
%       mod:pred(arg1,arg2)} while {\tt px\_query(mod,pred,arg1,arg2)}
%     will call {\tt mod:pred(arg1,arg2,Ret)}.
%     
%     Calls to XSB predicates that are not in a module may be made with
%     {\tt module} set to {\tt usermod}.

  \pymoditem{px\_comp(module,pred,*args,**kwargs)}{px\_com()}{px}
%
  Allows Python to call XSB to perform the equivalent of list or set
  comprehension.  {\tt px\_comp()} allows zero or more input arguments
  each containing a Python term ($\overrightarrow{input}$) and zero or
  more output arguments ($\overrightarrow{output}$) to call an XSB goal

\[module:pred(\overrightarrow{inputs},\overrightarrow{outputs}) \]

It then returns to Python a list or set of tuples representing all
bindings (or all unique bindings) to $\overrightarrow{outputs}$ for
which the above goal is true.  See Examples \ref{ex:list-comp-1},
\ref{ex:simple-comp} and \ref{ex:px-comp-undef} for elaboration on
this.

The actual behavior of {\tt px\_comp()} depends on the keyword
arguments passed to it.

{\tt kwargs} can take the following values:
\begin{itemize}
  \item {\tt vars=N} where {\tt N} is a non-negative integer,
    determines the number {\tt N} of output variables for the call.
    For instance
    
\begin{tabular}{lll}
    {\tt px\_comp(mod,pred)}            & calls the goal & {\tt mod:pred(X1)} \\
    {\tt px\_comp(mod,pred),vars=2}     & calls the goal & {\tt mod:pred(X1,X2)} \\
    {\tt px\_comp(mod,pred,a,vars=0)}   & calls the goal & {\tt mod:pred(a)}. \\
    {\tt px\_comp(mod,pred,a)}          & calls the goal & {\tt mod:pred(a,X1)} \\
    {\tt px\_comp(mod,pred,a,vars=1)}   & calls the goal & {\tt mod:pred(a,X1)} \\
    {\tt px\_comp(mod,pred,a,vars=2)}   & calls the goal & {\tt mod:pred(a,X1,X2)} \\
    {\tt px\_comp(mod,pred,a,b,vars=2)} & calls the goal & {\tt mod:pred(a,b,X1,X2)} \\
\end{tabular}    
    
    \noindent
    The default is {\tt 1}.
    
    \item {\tt set\_collect=True/False} determines the type of
      collection in which the bindings are returned: if the keyword
      argument is {\tt True}, the answers are collected as a set, and
      if {\tt False} the answers are collected as a list.  Default is
      {\tt False}. \footnote{The reason for making lists the default
        collection type is that Python sets can only contain
        non-mutable objects, and so cannot contain lists or
        dictionaries.}
      
    \item {\tt truth\_vals} determines whether how each answer in the
      collection is associated with its truth value.  Values can be:
      
      \bi

      \item {\tt PLAIN\_TRUTHVALS} which associates each answer with
        its truth value true (represented as {\tt 1} or {\em
          undefined} represented as {\tt 2}. (Unlike {\tt px\_cmd()}
        or {\tt px\_qdet()}, {\em false} answers are never included in
        the collection returned by {\tt px\_comp()}.)  Using {\tt
          PLAIN\_TRUTHVALS}, each element of the collection is a 2-ary
        tuple consisting of an answer and its truth value.  This is
        the default behavior for {\tt px\_comp()}.
      
      \item {\tt DELAY\_LISTS} which associates each answer with its
        SLG delay list (cf. Example~\ref{ex:px-comp-undef} or for more
        information, \cite{ChWa96} or the chapter {\em Using Tabling
          in XSB: A Tutorial Introduction} in Volume 1 of this
        manual).  {\tt DELAY\_LISTS} is specified, each element of the
        collection is a 2-ary tuple consisting of an answer and its
        delay list.

      \item {\tt NO\_TRUTHVALS} does not associate an answer with any
        truth value.  This option should only be used in situations
        where it is know that no answers will be {\em undefined}.
        \ei
  \end{itemize}

  \pymoditem{px\_get\_error\_message()}{px\_get\_error\_message()}{px}
  If an XSB exception was raised by the previous call to XSB, {\tt
    px\_get\_error\_message()} returns the XSB exception message as a
  Python Unicode string.
  
\repeatpymoditem{pp\_px\_cmd(Module,Pred,*args)}{pp\_px\_cmd()}{px}
  \repeatpymoditem{pp\_px\_qdet(Module,Pred,*args)}{pp\_px\_qdet()}{px}
  \pymoditem{pp\_px\_comp(Module,Pred,*args,**kwargs)}{pp\_px\_comp()}{px}
  
\end{description}

\subsection{\px{} API Convenience Predicates}

These predicates for managing the XSB session are defined in terms of
other predicates in the \px{} API, and are included only for
convenience.

\begin{description}

  \repeatpymoditem{consult(File)}{consult()}{px}
  \pymoditem{ensure\_loaded(File)}{ensure\_loaded()}{px}
%
  Convenience functions for loading and/or compiling Prolog files.
  They are defined as {\tt px\_cmd('consult','consult',File)} and {\tt
    px\_cmd('consult','ensure\_loaded',File)}.\footnote{On-demand
    loading, available in XSB, is not yet available within \px{}.}

  Note that a given XSB file can be compiled and/or loaded into the
  running Python-XSB session (via {\tt consult()} or {\tt
    ensure\_loaded()}), edited and then repeatedly recompiled and
    reloaded without problems.

  \pymoditem{prolog\_paths()}{prolog\_paths()}{px}
%  
  Convenience function to return a list of all current XSB library
  paths (XSB's equivalent of Python's {\tt sys.path}).

  \pymoditem{add\_prolog\_path(Paths)}{add\_prolog\_path()}{px}
%  
    Convenience function to add one or more XSB library paths
    designated as a list of strings.  This function calls XSB's
    equivalent of Python's {\tt sys.path.append()}) and is defined as:
    {\tt px\_cmd('consult','add\_lib\_dir',Paths)}.
  
\end{description}

\section{Performance} \label{sec:px-perf}

{\tt px} performance is similar to that of {\tt xsbpy} as described in
Section \ref{sec:xsbpy-perf}.  Simple calls to {\tt px\_cmd()} and
{\tt px\_qdet()} take roughly a microsecond on a reasonably fast
processor, while {\tt px}/{\tt xsbpy} data transfer is exactly the
same.  (The latter fact is unsurprising since both libraries use the
same data transfer code.)  A simple call to {\tt px\_comp} take
roughly 5-6 times that of a similar call to {\tt px\_qdet}.

The directory {\tt \$XSB\_ROOT/xsbtests/python\_tests/px\_bench.py}
contains the script {\tt memtest.py} that can be run to provide
benchmark times for on a given platform.  The script includes a
variety of benchmarks, including those summarized above.  Importantly,
{\tt px\_bench.py} also uses the Python {\tt guppy.heapy} module to
examine whether executing millions of {\tt px} calls creates any
memory leaks in Python.  Running the script on recent version of
Python show that {\tt px} calling and {\tt px}/{\tt xsbpy} data
transfer creates virtually no memory leaks for Python.

\section{Current Issues and Limitations} \label{sec:px-limits}

\px{} is under active development and its issues and limitations
change on a weekly or even daily basis.

\bi
\item \px{} has not yet been tested on any platform other than Ubuntu
  Linux.
\item XSB's heap garbage collection is currently disabled when XSB is
  called from Python, although expansion is allowed for all stacks.
\item \pxversion{} of \px{} the Python session that calls \px{} must
  not be itself embedded in another process.
\ei  
  
%\item Python sometimes core dumps on exit (via {\tt quit()} or {\tt
%  ctrl-D} after using \px.
 
