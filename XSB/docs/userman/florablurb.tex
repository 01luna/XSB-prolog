


\newcommand{\FLIP}{{\mbox{\sc Flip}}\xspace}
\newcommand{\FLORA}{{\mbox{${\cal F}${\sc lora}\rm\emph{-2}}}\xspace}
\newcommand{\FLORAone}{{\mbox{${\cal F}${\sc lora}}}\xspace}
\newcommand{\FLORID}{{\mbox{\sc Florid}}\xspace}
\newcommand{\fl}{\mbox{F-logic}\xspace}



%\pagenumbering{arabic}
%\setcounter{page}{1}


%\section{FLORA-2}

\FLORA is a sophisticated object-oriented knowledge base language and
application development platform. It is implemented as a set of run-time
libraries and a compiler that translates a unified language of \fl
\cite{KLW95}, HiLog \cite{hilog-jlp}, and Transaction Logic
\cite{trans-chapter-98,trans-tcs94} into tabled Prolog code.

Applications of \FLORA include intelligent agents, Semantic Web, ontology
management, integration of information, and others. 

\index{FLIP}
\index{FLORID}
%%
The programming language supported by \FLORA is a dialect of \fl with
numerous extensions, which include a natural way to do meta-programming in
the style of HiLog and logical updates in the style of Transaction
Logic. \FLORA was designed with extensibility and flexibility in mind, and
it provides strong support for modular software design through its unique
feature of dynamic modules.
Other extensions, such as the versatile syntax of \FLORID path
expressions, are borrowed from
\FLORID, a C++-based \fl system developed at
Freiburg University.\footnote{
  %%
  See {\tt http://www.informatik.uni-freiburg.de/$\sim$dbis/florid/} for more
  details.
  %%
}
%%
Extensions aside, the syntax of \FLORA differs in many
important ways from \FLORID, from the original version of \fl, as described
in \cite{KLW95}, and from an earlier implementation of \FLORAone. These
syntactic changes were needed in order to bring the syntax of \FLORA closer
to that of Prolog and make it possible to include simple Prolog programs
into \FLORA programs without choking the compiler.  Other syntactic
deviations from the original F-logic syntax are a direct consequence of the
added support for HiLog, which obviates the need for the ``@'' sign in
method invocations (this sign is now used to denote calls to \FLORA
modules).

\FLORA is distributed in two ways. First, it is part of the official
distribution of XSB and thus is installed together with XSB.  Second, a
more up-to-date version of the system is available on \FLORA's Web site at
%%
\begin{quote}
  {\tt http://flora.sourceforge.net}
\end{quote}
%%
These two versions can be
installed at the same time and used independently ({\it e.g.}, if you want
to keep abreast with the development of \FLORA or if a newer version was
released in-between the releases of XSB). The installation instructions are
somewhat different in these two cases. Here we only describe the process of
configuring the version \FLORA included with XSB.

\paragraph{Installing \FLORA under UNIX.}
To configure a version of \FLORA that was downloaded as part of the
distribution of XSB, simply configure XSB as usual:
%%
\begin{verbatim}
 cd XSB/build
 configure
 makexsb
\end{verbatim}
%%
and then run
%%
\begin{verbatim}
 makexsb packages 
\end{verbatim}
%%

If you downloaded XSB from its CVS repository earlier and are updating your
copy using the {\tt cvs update} command, then it might be a good idea to
also do the following: 
%%
\begin{verbatim}
 cd packages/flora2
 makeflora clean
 makeflora
\end{verbatim}
%%

\paragraph{Installing \FLORA in Windows.}
First, you need Microsoft's {\tt nmake}.
Then use the following commands to configure
\FLORA (assuming that XSB is already installed and configured):
%%
\begin{verbatim}
   cd flora2
   makeflora clean
   makeflora path-to-prolog-executable
\end{verbatim}
%%
Also make sure that the {\tt packages} directory contains a
shortcut called {\tt flora2.P} to the file
{\tt packages$\backslash$flora2$\backslash$flora2.P}.


\paragraph{Running \FLORA.}
\FLORA is fully integrated into the underlying XSB engine, including its
module system. In particular, \FLORA modules can invoke predicates defined in
other Prolog modules, and Prolog modules can query the objects defined in
\FLORA modules. 

\index{local scheduling in XSB}
%%
Due to certain problems with XSB, \FLORA runs best when XSB is configured
with \emph{local} scheduling, which is the default XSB configuration.
However, with this type of scheduling, many Prolog intuitions that relate
to the operational semantics do not work. Thus, the programmer must think
``more declaratively'' and, in particular, to not rely on the 
order in which answers are returned.


\index{runflora script}
\label{runflora-page}
%%
The easiest way to get a feel of the system
is to start \FLORA shell and begin to enter queries interactively.
The simplest way to do this is to use the shell script
%%
\begin{verbatim}
 .../flora2/runflora  
\end{verbatim}
%%
where ``...'' is the directory where \FLORA is downloaded. For instance, to
invoke the version supplied with XSB, you would type something like
%%
\begin{verbatim}
   ~/XSB/packages/flora2/runflora
\end{verbatim}
%%

At this point, \FLORA takes over and \fl syntax becomes the
norm. To get back to the Prolog command loop, type {\tt Control-D} 
(Unix) or Control-Z (Windows), or 
%%
\begin{quote}
  \tt
| ?- \_end.  
\end{quote}
%%

\noindent
If you are using \FLORA shell frequently, it pays to define an alias, say
(in Bash):
%%
\begin{verbatim}
 alias runflora='~/XSB/packages/flora2/runflora'
\end{verbatim}
%%
\FLORA can then be invoked directly from the shell prompt by typing
{\tt runflora}. 
%%
It is even possible to tell \FLORA to execute commands on start-up.
For instance, 
%%
\begin{verbatim}
 foo>  runflora -e "_help."
\end{verbatim}
%%
will cause the system to execute the help command right after after the
initialization. Then the usual \FLORA shell prompt is displayed.

\noindent
\FLORA comes with a number of demo programs that live in
%%
\begin{quote}
 \verb|.../flora2/demos/|  
\end{quote}
%%
The demos can be run issuing the command
``\verb|_demo(demo-filename).|''
at the \FLORA prompt, {\it e.g.},
%%
\begin{quote}
 \verb|flora2 ?- _demo(flogic_basics).|
\end{quote}
%%
There is no need to change to the demo directory, as {\tt flDemo} knows
where to find these programs.


%%% Local Variables: 
%%% mode: latex
%%% TeX-master: "manual2"
%%% End: 
