\chapter{Foreign Language Interface}
%===================================
\label{foreign}

When \ourprolog\ is used to build real-world systems, a foreign-language
interface may be necessary to:
\begin{itemize}
\item combine \ourprolog\ with existing programs and libraries, thereby
      forming composite systems;
\item interface \ourprolog\ with the operating system, graphical user 
      interfaces or other system level programs;
\item speed up certain critical operations.
\end{itemize}

\ourprolog\ provides a simple yet powerful foreign language interface.
We have tried to keep the interface part of a foreign predicate similar
to that of a normal Prolog predicate.  Currently we only provide a C 
interface for SUN3/SPARC machines (running SunOS 4.x.x or Solaris 2.x),
and 80386/80486-based machines running Linux, to which the discussion
below is restricted.  In the future releases we intend to provide the
same foreign language interface for the other platforms where \ourprolog\
can be used.

Foreign predicates must always appear in modules, which can contain
only foreign predicates.  The main difference between a normal module
and a foreign module is the very natural one: the source file of the
module implementation, which is in C, must appear in a {\tt *.c} file
rather than a {\tt *.P} file.  This {\tt *.c} file cannot contain a
{\tt main()} function.  Furthermore, a {\tt *.P} file with the same
name {\em must not} be present or else the {\tt *.c} file is ignored
and the module is compiled as a regular Prolog module.  The interface
part of a foreign module, which has the same syntax as that of a
normal module, is written in Prolog and hence must appear in a {\tt
*.H} file.  This {\tt *.H} file contains {\tt export} declarations
for each and every one of the foreign predicates that are to be used
by other modules.  Here is an example of a {\tt .H} file for a foreign
module:

\begin{verbatim}
/*======================================================================
  File                  :  XSB/examples/simple_foreign.H
  Author(s)             :  Kostis F. Sagonas
  Last modification     :  December 8, 1994
========================================================================*/

:- export minus_one/2, my_sqrt/2, change_char/4.

:- ldoption('-lm').     % link together with the math library
\end{verbatim}

Directives such as {\tt index}, {\tt hilog}, {\tt table}, {\tt auto\_table} 
or even {\tt import} make no sense in the case of a foreign module and thus 
are ignored by the compiler.  However, another directive, namely 
{\tt ldoption}, is recognized in a foreign module and is used to instruct 
the dynamic loading and linking of the module.  
The syntax of the {\tt ldoption} directive is simply:
\begin{center}
{\tt  :- ldoption(Option).    }
\end{center}
where {\tt Option} should either be an atom (currently of length less
or equal to 255 characters) or a list of atoms.  Multiple {\tt ldoption}
directives may appear in the same {\tt .H} file of a foreign module.

%(Explain more about the directive.....).
The foreign language interface of \ourprolog\ uses the Unix command 
{\tt ld} that combines object programs to create an executable file  
or another object program suitable for further {\tt ld} processing.
\version\ of \ourprolog\ assumes that the {\tt ld} command resides
in the file {\tt /usr/bin/ld}.

C functions that implement foreign predicates must return values of type
{\tt int}. If a non-zero is returned, the foreign predicate succeeds; a
zero return value means failure.

A well-designed foreign predicate must check that its arguments are of the
correct types and modes. However, such checks can also be done using
Prolog-side wrappers that invoke a foreign predicate.

At the C level, the procedure that implements the foreign predicate
must have the same name as the predicate (that is declared in the 
% again the example. ???
{\tt *.H} file), and it must be {\em parameterless}.  The Prolog level 
arguments are converted to C data structures through several 
predefined functions rather than through direct parameter passing.

Our foreign language interface currently supports exchange of Prolog's 
atomic data types (atoms, integers, and floating-point numbers).  The data 
has to be converted between Prolog's internal representation and the internal 
representation of the foreign language.  For that reason, the following 
C functions are used to convert Prolog and C data types to each other.
% again the example ???
\begin{description}
\desc{int ptoc\_int(int N)}
        Argument {\tt N} is assumed to hold a Prolog integer, and this
        function returns its integer value in C format.
\desc{float ptoc\_float(int N)}
        Argument {\tt N} is assumed to hold a Prolog floating point number,
        and this function returns its floating point value in C format.
	(Precision is less than single word floating point).
\desc{char *ptoc\_string(int N)}
        Argument {\tt N} is assumed to hold a Prolog atom, and this
        function returns the C string (of type {\tt char *}) that 
	corresponds to this Prolog atom.
\desc{void ctop\_int(int N, int V)}
        Argument {\tt N} is assumed to hold a Prolog free variable, and
        this function binds that variable to an integer of value {\tt V}.
\desc{void ctop\_float(int N, float V)}
        Argument {\tt N} is assumed to hold a Prolog free variable, and
        this function binds that variable to a floating point number of 
	value {\tt V}.
\desc{void ctop\_string(int N, char * V)}
        Argument {\tt N} is assumed to hold a Prolog free variable, and
        this function binds that variable to a Prolog atom of value {\tt V}.
	In C, {\tt V} is of type {\tt char *}.

	Note that the atom of value {\tt V} is not interned, i.e. it is 
	not inserted into the Prolog atom table.  For that reason, the
	{\tt string\_find(char *V, int Insert)} function should be used.
        Function {\tt string\_find()} searches the symbol table for
	the symbol, and if the symbol does not appear there and the
	value of {\tt Insert} is non-zero, it inserts it.  Thus, the most
	common use of this function is as follows:
	\begin{center}
	{\tt	ctop\_string(N, string\_find(V, 1))	}
	\end{center}
	Refer to the example {\tt simple\_foreign} in the {\tt examples}
	directory to see a use of this function.
\desc{void c2p\_string(int N, char *V)}
	Same as {\tt ctop\_string(N, V)} but it interns the atom of
	value {\tt V}.  So, it is equivalent to the following:
	\begin{center}
	{\tt	ctop\_string(N, string\_find(V, 1))	}
	\end{center}
\end{description}
A limitation of the current implementation of the foreign language
interface is that the above procedures assume correct modes and types
being passed between C and the Prolog level.  So, output unification
should be explicitly performed in the Prolog level.  The function
prototypes should be declared before the corresponding functions are
used.  This is done by including the {\tt "cinterf.h"} header
file (located in the {\tt XSB/emu} directory).

In the current implementation, the Prolog procedures that are
attached to foreign predicates are deterministic, in the sense that
they succeed at most once for a given call and are not re-entered on
backtracking.  Note that this requirement imposes no serious
limitation, since it is always possible to divide a foreign predicate
into the part to be done on the first call and the part to be redone
on backtracking.  Backtracking can then take place at the Prolog
level where it is more naturally expressed.

A foreign module can be {\tt compile}d or {\tt consult}ed just like a
normal Prolog module.  Currently, predicates {\tt consult/[1,2]}
recompile both the {\tt *.c} and the {\tt *.H} files of a foreign
module when at least one of them has been changed from the time the
corresponding object files have been created (see the section {\it
Compiling and Consulting} in Volume 1. 
The C compiler used to compile the {\tt *.c} files can be set as a
compilation option or defaults to that used for the configuration of
XSB (refer to the section {\it Getting Started with XSB} in Volume 1.
Moreover, the user can control the compiler options that can be passed
to the C compiler.  To give an example, the following command will
compile file {\tt file.c} using the Gnu C Compiler with optimization
and by including {\tt /usr/local/X11/R6/include} to the directories that
will be searched for header files.
\begin{center}
{\tt  :- consult(file,
		 [cc(gcc), cc\_opts('-O2 -I/usr/local/X11/R6/include')]). }
\end{center}
If no C compiler options are specified, the compilation of the C-file
defaults to $CC$~{\tt -c~file.c} where $CC$ is the name of the C compiler
used to install \ourprolog.
Any Prolog compiler options are ignored when compiling a foreign module.

We end by a very simple example of using the foreign language
interface of \ourprolog.  The programs above and below are programs
{\tt simple\_foreign.\{H,c\}} in the {\tt examples} directory.

\begin{small}
\begin{verbatim}
/*======================================================================
  File                  :  XSB/examples/simple_foreign.c
  Author(s)             :  Kostis F. Sagonas
  Last modification     :  December 8, 1994
========================================================================*/

#include <math.h>
#include <stdio.h>
#include <string.h>
#include <alloca.h>

/*----- Make sure your C compiler finds the following header file ------*/

#include "../emu/cinterf.h"

/*----------------------------------------------------------------------*/

int minus_one(void)
{
   int  i = ptoc_int(1);

   ctop_int(2, i-1);
   return TRUE;
}

/*----------------------------------------------------------------------*/

int my_sqrt(void)
{
   int i = ptoc_int(1);

   ctop_float(2, (float) pow((double)i, 0.5));
   return TRUE;
}

/*----------------------------------------------------------------------*/

int change_char(void)
{
   char *str_in;
   int  pos;
   int c;
   char *str_out;

   str_in = (char *) ptoc_string(1);
   str_out = (char *) alloca(strlen(str_in)+1);
   strcpy(str_out, str_in);
   pos = ptoc_int(2);
   c = ptoc_int(3);
   if (c < 0 || c > 255) /* not a character */
     return FALSE; /* this predicate will fail on the Prolog side */

   str_out[pos-1] = c;

   /* Now that we have constructed a new symbol, we must ensure that it
      appears in the symbol table.  This can be done using function
      string_find() that searches the symbol table for the symbol, and
      if the symbol does not appear there, it inserts it.  If we are
      sure that the symbol already appeared in the symbol table there
      is no need to use string_find().
    */

   ctop_string(4, (char *) string_find(str_out,1));  /* 1 = INSERT */
   return TRUE;
}

/*----------------------------------------------------------------------*/
\end{verbatim}
\end{small}

Here is a sample session of using these files.
\begin{small}
\begin{verbatim}
XSB Version 1.9 (Mouton Cadet) of March 31, 1999
[i686-pc-linux-gnu; mode: optimal; engine: slg-wam; scheduling: batched]
| ?- [simple_foreign].
[Compiling C file ./simple_foreign.c using gcc]
[Compiling Foreign Module ./simple_foreign]
[simple_foreign compiled, cpu time used: 0.0099993 seconds]
[simple_foreign loaded]

yes
| ?- change_char('Kostis', 2, 119, TempStr), % 119 is w
     change_char(TempStr, 5, 104, GrkName).  % 104 is h

TempStr = Kwstis
GrkName = Kwsths;

no
| ?- minus_one(43, X).

X = 42;

no
| ?- minus_one(43, 42).	           % No output unification is allowed
Wrong arg in ctop_int 2a2 (Reg = 2)

yes
| ?- my_sqrt(4,X). 

X = 2

yes
| ?- my_sqrt(23,X).

X = 4.7958;

no
\end{verbatim}
\end{small}

There are some more sample programs in the {\tt examples} directory that
exhibit most of the features of the foreign language interface, and
will probably make clear many questions that users may have.


%%% Local Variables: 
%%% mode: latex
%%% TeX-master: "manual2"
%%% End: 
