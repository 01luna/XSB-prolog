\begin{center}
{\bf {\Large 
		Credits
            %=============%
}}
\end{center}

% Apologies to anyone I left out...  It wasn't intentional! TLS.

\begin{quote}
Day-to-day care and feeding of XSB including bug fixes, ports, and
configuration management is currently done by David Warren and
Terrance Swift with the help of Michael Kifer and others.  In the past
Kostis Sagonas, Prasad Rao, Steve Dawson, Juliana Freire, Ernie
Johnson, Baoqiu Cui, Bart Demoen and Luis F.  Castro have provided
tremendous help.

In \version, the core engine development of the SLG-WAM has been
mainly implemented by Terrance Swift, David Warren, Kostis Sagonas,
Prasad Rao, Juliana Freire, Ernie Johnson, Luis Castro and Rui
Marques.  The breakdown, very roughly, was that Terrance Swift wrote
the initial tabling engine, the SLG-WAM, and its built-ins; and leads
the current development of the tabling subsystem.  Prasad Rao
reimplemented the engine's tabling subsystem to use tries for
variant-based table access and Ernie Johnson extended and refactored
these routines in a number of ways, including adding call subsumption.
Kostis Sagonas implemented most of tabled negation.  Juliana Freire
revised the table scheduling mechanism starting from Version~1.5.0 to
create the batched and local scheduling that is currently used.
Baoqiu Cui revised the data structures used to maintain delay lists,
and added attributed variables to the engine.  Luis Castro rewrote the
emulator to use jump tables and wrote a heap-garbage collector for the
SLG-WAM.  Rui Marques was responsible for the concurrency control
algorithms used for shared tables, and mainly responsible for making
the XSB engine multi-threaded.  The incremental table maintenance
subsystem was designed and first implemented by Diptikalyan Saha, and
its design and development has been continued by Terrance Swift.
Answer subsumption was written by David Warren and Terrance Swift.
David Warren implemented hash-consed, or ``interned'' tables.  Call
abstraction and answer abstraction (restraint) were written by
Terrance Swift.

Other engine work includes the following.  Memory expansion code for
WAM stacks was written by Ernie Johnson, Bart Demoen and David
S. Warren.  Heap garbage collection was written by Luis de Castro,
Kostis Sagonas and Bart Demoen.  Atom space garbage collection was
written by David Warren; table garbage collection was written by
Terrance Swift based in part on space reclamation code written by
Prasad Rao.  Rui Marques rewrote much of the engine to make it
compliant with 64-bit architectures.  Assert and retract code was
based on code written by Jiyang Xu; it significantly revised by David
S. Warren, who added alternative, multiple, and star indexing and by
Terrance Swift who implemented dynamic clause garbage collection. Trie
assert/retract code, and trie interning code was written by Prasad
Rao.  Neng-fa Zhou, Terrance Swift and David Warren upgraded XSB from
ASCII to the character sets UTF-8, C1253, and LATIN-1.  The current
version of {\tt findall/3} was re-written from scratch by Bart Demoen,
as was XSB's original throw and catch mechanism.  64-bit floats were
added by Charles Rojo.  The interface from C to Prolog and DLL
interface were implemented by David Warren and extended to
multi-threading by Terrance Swift; the interface from Prolog to C
(foreign language interface) was developed by Jiyang Xu, Kostis
Sagonas, Steve Dawson and David Warren.

In terms of core system Prolog code, Kostis Sagonas was responsible
for HiLog compilation and associated built-ins as well as coding or
revising many standard predicates.  Steve Dawson implemented
Unification Factoring.  The revision of XSB's I/O into ISO-compatible
streams was done by Michael Kifer and Terrance Swift.  The {\tt
  auto\_table} and {\tt suppl\_table} directives were written by
Kostis Sagonas.  The DCG expansion module was written by Kostis
Sagonas for non-tabled code and by Baoqiu Cui, David Warren and
Terrance Swift for tabled code.  The handling of the {\tt multifile}
directive was written by Baoqiu Cui and David
Warren. C.R. Ramakrishnan wrote the mode analyzer for XSB.  Michael
Kifer implemented the {\tt storage} module.  The multi-threaded API
was written by Terrance Swift and Rui Marques.  Walter Wilson has
written several of XSB's library predicates for tabling.  Paulo Moura
has added several predicates to make XSB more consistent with other
Prologs.  

Michael Kifer has been in charge of XSB's installation procedures,
rewriting parts of the XSB code to make XSB configurable with GNU's
Autoconf, implementing XSB's package system, and integrated GPP with
XSB's compiler.  GPP, the source code preprocessor used by XSB, was
written by Denis Auroux, who also wrote the GPP manual reproduced in
Appendix A.

The starting point of XSB (in 1990) was PSB-Prolog 2.0 by Jiyang Xu
and David Warren.  PSB-Prolog in its turn was based on SB-Prolog,
primarily designed and written by Saumya Debray, David S. Warren, and
Jiyang Xu.  Thanks are also due to Weidong Chen for his work on Prolog
clause indexing for SB-Prolog, to Richard O'Keefe, who contributed the
Prolog code for the Prolog reader and the C code for the tokenizer, to
Ciao Prolog whose {\tt write\_term/[2,3]} we use, and to SWI Prolog
for their CLP(R) package.

... Now what did I forget this time ?

\end{quote}

%%% Local Variables: 
%%% mode: latex
%%% TeX-master: "manual1"
%%% End: 
