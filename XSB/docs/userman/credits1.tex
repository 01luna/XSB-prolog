\begin{center}
{\bf {\Large 
		Credits
            %=============%
}}
\end{center}

% Apologies to anyone I left out...  It wasn't intentional! TLS.

\begin{quote}
Day-to-day care and feeding of XSB including bug fixes, ports, and
configuration management has been done by Kostis Sagonas, David
Warren, Terrance Swift, Prasad Rao, Steve Dawson, Juliana Freire, 
Baoqiu Cui and Michael Kifer.

In \version, the core engine development of the SLG-WAM has been
mainly implemented by Terrance Swift, Kostis Sagonas, Prasad Rao and
Juliana Freire.  The breakdown, roughly, was that Terrance Swift wrote
the initial tabling engine and builtins.  Prasad Rao wrote the
trie-based table access routines, and Kostis Sagonas implemented most
of tabled negation.  Juliana Freire revised the table scheduling
mechanism starting from Version 1.5.0 to create a more efficient
engine, and implemented the engine for local evaluation.  Starting
from XSB Version 2.0, XSB includes another tabling engine, CHAT, which
was designed and developed by Kostis Sagonas and Bart Demoen.  CHAT
supports heap garbage collection (both based on a mark\&slide and on a
mark\&copy algorithm) which was developed and implemented by Bart
Demoen and Kostis Sagonas.  Memory expansion code for WAM stacks was
written by Ernie Johnson and Bart Demoen, while memory management code
for CHAT areas was written by Bart Demoen and Kostis Sagonas.  Rui
Marques improved the trailing of the SLG-WAM and rewrote much of the
engine to make it compliant with 64-bit architectures.  Assert and
retract code was based on code written by Jiyang Xu and significantly
revised by David S. Warren and Rui Marques.  Trie assert and retract
code was written by Prasad Rao.  The current version of {\tt
findall/3} was re-written from scratch by Bart Demoen.

In the XSB complier, Kostis Sagonas was responsible for HiLog
compilation and associated builtins.  Steve Dawson implemented
Unification Factoring.  The {\tt auto\_table} and {\tt suppl\_table}
directives were written by Kostis Sagonas.  The DCG expansion module
was written by Kostis Sagonas.  The handling of the {\tt multifile}
directive was written by Baoqiu Cui.  C.R. Ramakrishnan wrote the mode
analyzer for XSB.  The safety check for tabling within the scope of
cuts was written by Kate Dvortsova.

Michael Kifer rewrote parts of the XSB code to make XSB configurable
with GNU's Autoconf.  Harald Schroepfer helped the XSB group with the
Solaris port, and Yiorgos Adamopoulos suggested the bits to use for
the HP-700 series port.  Steven Dawson, Larry B. Daniel and Franklin
Chen were responsible for the MkLinux and Solaris x86 ports.

GPP, the source code preprocessor used by XSB, was written by Denis Auroux.
He also wrote the GPP manual reproduced in Appendix A.

The starting point of XSB (in 1990) was PSB-Prolog 2.0 by Jiyang Xu.
PSB-Prolog in its turn was based on SB-Prolog, primarily designed and
written by Saumya Debray, David S. Warren, and Jiyang Xu.  Thanks are
also due to Weidong Chen for his work on Prolog clause indexing for
SB-Prolog and to Richard O'Keefe, who contributed the Prolog code for
the Prolog reader and the C code for the tokenizer.  

\end{quote}

%%% Local Variables: 
%%% mode: latex
%%% TeX-master: "manual1"
%%% End: 
