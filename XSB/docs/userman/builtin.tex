\chapter{Standard Predicates} \label{standard}
%=============================================

Standard predicates are always available to the Prolog interpreter, and do
not need to be imported or loaded explicitly as do other Prolog predicates.
Our standard predicates are listed below.  Standard predicates whose 
semantics depend on HiLog terms or on SLG evaluation are marked as 
{\sf HiLog} or {\sf Tabling}.

It is possible for the user to add standard predicates not provided
in the standard release. See the section on Customizing XSB.

\section{Input and Output}

XSB has two levels of I/O: a higher level that supports DEC-10 and
ISO-like file and stream handling; and a lower level that is based in
XSB I/O ports.  This section describes higher-level I/O, while the
lower-level is described in Volume 2 Section 1.5 of this manual.
Presently, higher-level input and output operations bind a file
identifier to the current input or output stream.  The semantics of
the operations here is basically that of Edinbourough Prolog's {\tt
see/1} and {\tt tell/1}.  However, this interface has been extended to
provide portability for programs written using the ISO standard, and
includes predicates such as {\tt open/3}, {\tt close/1}, {\tt write/2}
and so on.

The default input and output streams are internally denoted by userin
and userout 
and they refer to the process'es standard input and standard output
streams).


\subsection{File Handling}
\begin{description}
\ournewitem{set\_input(+F)}{ISO}\index{\texttt{set\_input/1}}
\vspace{-7mm} 
\ouritem{see(+F)}\index{\texttt{see/1}}
    Makes file {\tt F} the current input stream. 
    \begin{itemize}
    \item If there is an open input stream associated with the file that 
          has {\tt F} as its file name, and that stream was opened previously
	  by {\tt see/1}, then it is made the current input stream.
    \item Otherwise, the specified file is opened for input and made the
          current input stream. If the file does not exist, {\tt see/1} 
	  fails.
    \end{itemize}
    Also note that different file names (that is, names which do not unify) 
    represent different input streams (even if these different file names 
    correspond to the same file).

    The implementation of the ISO preducate {\tt set\_input/1}, is
    essentially that of {\tt see/1}.

    Exceptions:
    \begin{description}
    \item[{\tt permission\_error}]
    	File {\tt F} is directory or file is not readable. 
    \item[{\tt instantiation\_error}]
    	{\tt F} is not instantiated at the time of call. 
    \item[{\tt existence\_error}]
    	File {\tt F} does not exist. 
    \item[{\tt domain\_error}]
    	{\tt F} is not an atomic file identifier
    \end{description}

\ouritem{seeing(?F)}\index{\texttt{seeing/1}}
    {\tt F} is unified with the name of the current input stream.
    This is exactly the same with predicate {\tt current\_input/1}
    described in Section~\ref{State}, and it is only provided for
    upwards compatibility reasons.

\ouritem{seen}\index{\texttt{seen/0}}
    Closes the current input stream. 
    Current input reverts to {\tt ``userin''} (the standard input stream).

\ournewitem{set\_output(+F)}{ISO}\index{\texttt{set\_output/1}}
\vspace{-7mm} 
\ouritem{tell(+F)}\index{\texttt{tell/1}}
    Makes file {\tt F} the current output stream. 
    \begin{itemize}
    \item If there is an open output stream associated with {\em F}  
          and that was opened previously 
          by {\tt tell/1}, then that stream is made the current output 
	  stream. 
    \item Otherwise, the specified file is opened for output and made the
          current output stream. If the file does not exist, it is created.
    \end{itemize}

    Also note that different file names (that is, names which do not unify) 
    represent different output streams (even if these different file names 
    correspond to the same file).

    The implementation of the ISO preducate {\tt set\_output/1}, is
    essentially that of {\tt tell/1}.

    Exceptions:
    \begin {description}
    \item[{\tt permission\_error}]
	File {\tt F} does not have write permission, or is a directory.
    \item[{\tt instantiation\_error}]
	{\tt F} is uninstantiated.
    \item[{\tt domain\_error}]
    	{\tt F} is not an atomic file identifier
    \end{description}

\ouritem{telling(?F)}\index{\texttt{telling/1}}
    {\tt F} is unified with the name of the current output stream.
    This predicate is exactly the same with predicate {\tt current\_output/1}
    described in Section~\ref{State}, and it is only provided for
    upwards compatibility reasons.

\ouritem{told}\index{\texttt{told/0}}
    Closes the current output stream. 
    Current output stream reverts to ``userout'' (the standard output stream).

\ournewitem{open(+File,+Mode,-Stream)}{ISO}\index{\texttt{open/3}}
    {\tt open/1} creates a stream for the file designated in {\em
    File}, and binds {\em Stream} to a structure representing that
    stream.  {\em Mode} can be one of either {\tt read} to create an
    input stream or {\tt write} or {\tt append} to create an output
    stream.  If the mode is {\tt write}, the contents of {\em File}
    are removed and {\em File} becomes a record of the output stream.
    If the mode is {\tt append} the output stream is appended to the
    contents of {\em File}.

    Exceptions ({\tt read} mode)
    \begin{description}
    \item[{\tt permission\_error}]
    	File {\tt F} is directory or file is not readable. 
    \item[{\tt instantiation\_error}]
    	{\tt F} is not instantiated at the time of call. 
    \item[{\tt existence\_error}]
    	File {\tt F} does not exist. 
    \end{description}

    Exceptions ({\tt write} mode)
    \begin {description}
    \item[{\tt permission\_error}]
	File {\tt F} does not have write permission, or is a directory.
    \item[{\tt instantiation\_error}]
	{\tt F} is uninstantiated.
    \end{description}

\ournewitem{open(+File,+Mode,-Stream.+Options)}{ISO}\index{\texttt{open/4}}

	This predicate is implemented in XSB for partial ISO
	compatability.  In the case where {\tt Options} is equal to {\tt
	[]}, {\tt open/4} is defined as {\tt open/3}, otherwise it
	aborts.


\ournewitem{close(+Stream)}{ISO}\index{\texttt{close/1}}
	{\tt close/1} closes the stream {\em Stream}.

\ournewitem{close(+Stream.+Options)}{ISO}\index{\texttt{close/2}}

	This predicate is implemented in XSB for partial ISO
	compatability.  In the case where {\tt Options} is equal to
	{\tt []}, {\tt close/2} is defined as {\tt close/1}, otherwise
	it aborts.

\ouritem{file\_exists(+F)}\index{\texttt{file\_exists/1}}
    Succeeds if file {\tt F} exists. {\tt F} must be instantiated to
    an atom at the time of the call, or an error message is displayed on
    the standard error stream and the predicate aborts.

    Exceptions:
    \begin {description}
    \item[{\tt instantiation\_error}]
	{\tt F} is uninstantiated.
    \end{description}

\end{description}


\subsection{Character I/O}
\begin{description}
\ouritem{nl}\index{\texttt{nl/0}}
    A new line character is sent to the current output stream.

\ouritem{nl(+Stream)}\index{\texttt{nl/1}}
    A new line character is sent to the designated output stream.

\ouritem{get0(?N)}\index{\texttt{get0/1}} 
    {\tt N} is the ASCII code of the next character read from the
    current input stream (regarded as a text stream). If the current
    input stream reaches its end of file, a {\tt -1} is returned.

    Compatibility Note:  Unlike other Prologs, such as C-Prolog, the input 
    stream is not closed on encountering the end-of-file character.

\ouritem{get(?N)}\index{\texttt{get/1}} 
    {\tt N} is the ASCII code of the next non-blank printable
    character from the current input stream (regarded as a text
    stream).  If the current input stream reaches its end of file, a
    {\tt -1} is returned.

    Compatibility Note:  Unlike other Prologs, such as C-Prolog, the input 
    stream is not closed on encountering the end-of-file character.

\ouritem{get\_code(?N)}\index{\texttt{get\_code/1}}
\vspace{-7mm} 
\ouritem{get\_code(+Stream,?N)}\index{\texttt{get\_code/2}}
   {\tt N} is the ASCII code of the next character from the current
   input stream or from {\tt Stream}.  The semantics of these
   predicates are based on that of {\tt get0/1} and do not conform in
   all cases to the ISO standard.

\ouritem{get\_char(?Char)}\index{\texttt{get\_char/1}}
\vspace{-7mm} 
\ouritem{get\_char(+Stream,?Char)}\index{\texttt{get\_char/2}}
   {\tt Char} is the next ASCII character from the current input stream
   or from {\tt Stream}.  The semantics of these predicates are based
   on that of {\tt get0/1} and do not conform in all cases to the ISO
   standard.

\ouritem{put(+N)}\index{\texttt{put/1}}
    Puts the ASCII character code {\tt N} to the current output stream.

    Exceptions:
    \begin{description}
    \item[{\tt instantiation\_error}]
	{\tt N} is not instantiated at the time of the call.
    \item[{\tt type\_error}]
	{\tt N} is not an integer at the time of the call.
    \end{description}

\ouritem{put\_code(+N)}\index{\texttt{put\_code/1}}
\vspace{-7mm} 
\ouritem{put\_code(+Stream,+N)}\index{\texttt{put\_code/2}}
    Puts the ASCII character code {\tt N} to the current output stream
    or to {\tt Stream}.  The semantics of these predicates are based
    on that of {\tt put/1} and do not conform in all cases to the ISO
    standard.

\ouritem{put\_char(+Char)}\index{\texttt{put\_char/1}}
\vspace{-7mm} 
\ouritem{put\_char(+Stream,+Char)}\index{\texttt{put\_char/2}}
    Puts the ASCII code of the character {\tt Char} to the current
    output stream or to {\tt Stream}.  The semantics of these
    predicates are based on that of {\tt put/1} and do not conform in
    all cases to the ISO standard.


\ouritem{tab(+N)}\index{\texttt{tab/1}}
    Puts {\tt N} spaces to the current output stream. 

    Exceptions:
    \begin{description}
    \item[{\tt instantiation\_error}]
	{\tt N} is not instantiated at the time of the call.
    \item[{\tt type\_error}]
	{\tt N} is not an integer at the time of the call.
    \end{description}
\end{description}

\subsection{Term I/O}
\begin{description}
\ouritem{read(?Term)}\index{\texttt{read/1}}
    A HiLog term is read from the current or designated input stream,
    and unified with {\tt Term} according to the operator declarations
    in force.  (See Section~\ref{TermSyntax} for the definition and
    syntax of HiLog terms). The term must be delimited by a full stop
    (i.e. a ``.'' followed by a carriage-return, space or tab).
    Predicate {\tt read/1} does not return until a valid HiLog term is
    successfully read; that is, in the presense of syntax errors {\tt
    read/1} does not fail but continues reading terms until a term
    with no syntax errors is encountered.  If a call to {\tt
    read(Term)} causes the end of the current input stream to be
    reached, variable {\tt Term} is unified with the term {\tt
    end\_of\_file}.  In that case, further calls to {\tt read/1} for
    the same input stream will cause an error failure.

    Exceptions:
    \begin{description}
    \item[{\tt existence\_error}]
	{\tt end\_of\_file} is reached before the current term is read.
    \end{description}

%--------

\ouritem{read(+Stream, ?Term)}\index{\texttt{read/2}}
	{\tt read/2} has the same behavior as {\tt read/1} but the
	input stream is explicitly designated using the first argument.

%--------
\ouritem{read\_term(?Term,?OptionsList)}\index{\texttt{read\_term/2}}
%
A term is read from the current input stream as in {\tt read/1}; but
{\tt OptionsList} is a (possibly empty) list of {\em read options}
that specifie additional behavior.  The read options include
\begin{itemize}
\item {\tt variables(Vars)}: once a term has been read, {\tt Vars} is a
list of the variables in the term, in left-to-right order. 
\item {\tt variable\_names(VN\_List)}: once a term has been read {\tt
VN\_List} is a list of non-anonymous variables in the term.  The
elements of the list have the form {\tt A = V} where {\tt V} is a
non-anonymous variable of the term, and {\tt A} is the string used to
denote the variable in the input stream.
\item {\tt singletons(VS\_List)}: once a term has been read {\tt
VN\_List} is a list of the non-anonymous {\tt singleton} varables in
the term.  The elements of the list have the form {\tt A = V} where
{\tt V} is a non-anonymous variable of the term, and {\tt A} is the
string used to denote the variable in the input stream.
\end{itemize}

    Exceptions:
    \begin{description}
    \item[{\tt instantiation\_error}]
	{\tt OptionsList} is a variable, or is a list containing a
			variable element.
    \item[{\tt type\_error}]
	{\tt OptionsList} is neither a list nor a variable
    \item[{\tt domain\_error}]
	{\tt OptionsList} contains a non-variable element that is not
	a read option.
    \item[{\tt existence\_error}]
	{\tt end\_of\_file} is reached before the current term is read.
    \end{description}

%--------

\ouritem{read\_term(+Stream, ?Term,?OptionsList)}\index{\texttt{read\_term/3}}
%
{\tt read\_term/3} has the same behavior as {\tt read\_term/2} but
the input stream is explicitly designated using the first argument.

Exceptions are the same as {\tt read\_term/2}, but with the addition:
    \begin{description}
    \item[{\tt instantiation\_error}]
	{\tt Strean} is a variable.
    \item[{\tt domain\_error}]
	{\tt Stream} is not an XSB stream (currently an atom).
    \end{description}

\ouritem{write(?Term)}\index{\texttt{write/1}}
    The HiLog term {\tt Term} is written to the current output stream, 
    according to the operator declarations in force.  Any uninstantiated 
    subterm of term {\tt Term} is written as an anonymous variable (an 
    underscore followed by a non-negative integer).  

    All {\em proper HiLog terms} (HiLog terms which are not also Prolog terms) 
    are not written in their internal Prolog representation.  Predicate 
    {\tt write/1} always succeeds without producing an error.

    The HiLog terms that are output by {\tt write/1} cannot in general be 
    read back using {\tt read/1}.  This happens for two reasons:
    \begin{itemize}
    \item The atoms appearing in term {\tt Term} are not quoted. In that case 
          the user must use {\tt writeq/1} or 
          {\tt write\_canonical/1} described below, which quote around atoms 
          whenever necessary.
    \item The output of {\tt write/1} is not terminated by a full-stop;
          therefore, if the user wants the term to be accepted as input to
          {\tt read/1}, the terminating full-stop must be explicitly sent 
          to the current output stream. 
    \end{itemize}

    Predicate {\tt write/1} treats terms of the form \verb|'$VAR'(N)|
    specially: it writes {\tt 'A'} if {\tt N}=0, {\tt 'B'} if {\tt
    N}=1, $\ldots$, {\tt 'Z'} if {\tt N}=25, {\tt 'A1'} if {\tt N}=26,
    etc.  Terms of this form are generated by {\tt numbervars/[1,3]}
    described in the section {\it Library Utilities} in Volume 2.
    \verb|'$VAR'(-1)| is written as the anonymous variable \verb|'_'|.

\ouritem{write(+Stream, ?Term)}\index{\texttt{write/2}}
	{\tt write/2} has the same behavior as {\tt write/1} but the
	output stream is explicitly designated using the first argument.

\ouritem{write\_term(?Term,+Options)}\index{\texttt{write\_term/2}}
\vspace{-7mm} 
\ouritem{write\_term(+Stream, ?Term,+Options)}\index{\texttt{write\_term/3}}

    Outputs {\tt +Term} to the current output ({\tt write\_term/2}) or
    to {\tt Stream} ({\tt write\_term/3}) according to the list of
    write options, {\tt Options}.  The current set of write options
    which form a superset of the ISO-standard write options, are as
    follows:
%
\begin{itemize}
%
\item {\tt quoted(+Bool)}.  If {\tt Bool = true}, then atoms and
    functors that can't be read back by {\tt read/1} are quoted, if
    {\tt Bool = false}, each atom and functor is written as its
    unquoted name. Default value is {\tt false}.
%
\item {\tt ignore\_ops(+Bool)}. If {\tt Bool = true} each compound term
is output in functional notation; curly brackets and list braces are
ignored, as are all explicitly defined operators.  If {\tt Bool =
false}, curly bracketed notation and list notation is enabled when
outputing compound terms, and all other operator notation is
enabed.  Default value is {\tt false}.
%
 \item {\tt numbervars(+Bool)}.  If {\tt Bool = true}, a term of the
form {\tt '\$VAR'(N)} where {\tt N} is an integer, is output as a
variable name consisting of a capital letter possibly followed by an
integer.  A term of the form {\tt '\$VAR'(Atom)} where {\tt Atom} is an
atom, is output as itself (without quotes).  Finally, a term of the
form {\tt '\$VAR'(String)} where {\tt String} is a character string, is
output as the atom corresponding to this character string.  If
{\tt bool} is {\tt false} this cases are not treated in any special
way.  Default value is {\tt false}.
%
% See predicates {\tt numbervars/3} and {\tt prettyvars/1}.  
%
% TLS: need predicate portray attribute, if we dont have it.
%
\comment{
\item {\tt portrayed(+Bool)}. If {\tt Bool = true}, then a call is made
to the predicate @pred{portray/1}, to
provide the user handlers for pretty printing some terms.  {\tt
portray_attribute/1} is called whenever an attributed variable is to
be printed, {\tt portray/1} is called whenever a non-variable term is
to be printed.  If either call succeeds, then it is assumed that the
term has been output, else it is printed as usual.  If {\tt bool} is
{\tt false}, these predicates are not called. Default value is {\tt
false}.  This option is set by the toplevel when writting the final
values of variables, and by the debugging package when writting the
goals in the tracing messages.  Thus you can vary the forms of these
messages if you wish.
}
\item {\tt max\_depth(+Depth)}. {\tt Depth} is a positive integer or
zero. If positive, it denotes the depth limit on printing compound
terms. If {\tt Depth} is zero, there is no limit. Default value is
{\tt 0} (no limit).
%
\item {\tt priority(+Prio)} {\tt Prio} is an integer between 1 and
1200.  If the term to be printed has higher priority than {\tt Prio},
it will be printed parenthesized.  Default value is 1200 (no term
parenthesized).
\end{itemize}

From the following examples it can be seen that {\tt
write\_term/\{2,3\}} can duplicate the behavior of a number of other
I/O predicates such as {\tt writeq/\{1,2\}}, {\tt
write\_canonical/\{1,2\}}, etc.
{\small
\begin{verbatim}

| ?- write_term(f(1+2,'A',"string",'$VAR'(3),'$VAR'('Temp'),(multifile foo)),[]).
f(1 + 2,A,"string",$VAR(3),$VAR(Temp),(multifile foo))
yes

| ?- write_term(f(1+2,'A',"string",'$VAR'(3),'$VAR'('Temp'),(multifile foo)),
                [quoted(true)]).
f(1 + 2,'A',"string",'$VAR'(3),'$VAR'('Temp'),(multifile foo))
yes

| ?- write_term(f(1+2,'A',"string",'$VAR'(3),'$VAR'('Temp'),(multifile foo)),
                [quoted(true),ignore_ops(true),numbervars(true)]).
f(+(1,2),'A','.'(115,'.'(116,'.'(114,'.'(105,'.'(110,'.'(103,[])))))),D,Temp,(multifile foo))
yes

| ?- write_term(f(1+2,'A',"string",'$VAR'(3),'$VAR'('Temp'),(multifile foo)),
                [quoted(true),ignore_ops(true),numbervars(true),priority(1000)]).
f(+(1,2),'A','.'(115,'.'(116,'.'(114,'.'(105,'.'(110,'.'(103,[])))))),D,Temp,multifile(foo))
yes
\end{verbatim}
}

\ouritem{writeln(?Term)}\index{\texttt{writeln/1}}
    {\tt writeln(Term)} can be defined as {\tt write(Term), nl}.

\ouritem{writeln(+Stream,?Term)}\index{\texttt{writeln/2}}
    {\tt writeln(Term)} can be defined as {\tt write(Stream,Term),
    nl(Stream)}.

\ouritem{display(?Term)}\index{\texttt{display/1}}
    The HiLog term {\tt Term} is displayed on the terminal (standard output 
    stream), according to the operator declarations in force. In other words,
    {\tt display/1} is similar to {\tt write/1} but the result is always
    written on {\tt ``userout''}\@.  Like {\tt write/1}, {\tt display/1} 
    always succeeds without producing an error. After returning from a call 
    to this predicate, the current output stream remains unchanged.

\ournewitem{write\_prolog(?Term)}{HiLog}\index{\texttt{write\_prolog/1}}
    This predicate acts as does {\tt write/1} except that
    any HiLog term {\tt Term} is written as a Prolog term. 
    {\tt write\_prolog/1} outputs {\tt Term} according to the operator 
    declarations
    in force.  Because of this, it differs from {\tt write\_canonical/1} 
    described below, despite the fact that both predicates write HiLog terms
    as Prolog terms.

\ournewitem{write\_prolog(+Stream,?Term)}{HiLog}\index{\texttt{write\_prolog/2}}
	{\tt write\_prolog/2} has the same behavior as {\tt
	write\_prolog/1} but the output stream is explicitly
	designated using the first argument.

\ouritem{writeq(?Term)}\index{\texttt{writeq/1}}
    Acts as {\tt write(Term)}, but atoms and functors are quoted
    whenever necessary to make the result acceptable as input to 
    {\tt read/1}\@. {\tt writeq/1} treats treats terms of the form 
    \verb|'\VAR'(N)| the same way as {\tt write/1}, writing {\tt A} if
    {\tt N}= 0, etc. 
%% $

    {\tt writeq/1} always succeeds without producing an error.

\ouritem{writeq(+Stream, ?Term)}\index{\texttt{writeq/2}}
	{\tt writeq/2} has the same behavior as {\tt writeq/1} but the
	output stream is explicitly designated using the first argument.

\ouritem{write\_canonical(?Term)}\index{\texttt{write\_canonical/1}}
    This predicate is provided so that the HiLog term {\tt Term}, 
    if written to a file, can be read back using {\tt read/1} regardless of 
    special characters appearing in {\tt Term} or prevailing operator 
    declarations. Like {\tt write\_prolog/1}, {\tt write\_canonical/1} 
    writes all proper HiLog terms to the current output stream using the 
    standard Prolog syntax (see Section~\ref{TermSyntax} on the standard 
    syntax of HiLog terms). {\tt write\_canonical/1} also quotes atoms and 
    functors as {\tt writeq/1} does, to make them acceptable as input of 
    {\tt read/1}\@.  Operator declarations are not taken into consideration,
    and compound terms are therefore always written in the form:

		\[ \langle predicate\ name \rangle
			(\langle arg_1 \rangle, \ldots,
			 \langle arg_n \rangle) \]

    Unlike {\tt writeq/1}, {\tt write\_canonical/1} does not treat terms 
    of the form \verb|'$VAR'(N)| specially. It writes square bracket lists 
    using {\tt '.'/2} and {\tt []} (that is, {\tt [foo, bar]} is written 
    as \verb|'.'(foo,'.'(bar,[]))|).
%% $

\ournewitem{read\_canonical(-Term)}{basics} 
\index{\texttt{read\_canonical/1}} 
Reads a term that is in canonical format from the
current input stream and returns it in {\tt Term}. On end-of-file, it
returns the atom {\tt end\_of\_file}.  If it encounters an error, it
prints an error message on stderr and returns the atom {\tt
read\_canonical\_error}. This is significantly faster than {\tt
read/1}, but requires the input to be in canonical form.

\ouritem{write\_canonical(+Stream, ?Term)}\index{\texttt{write\_canonical/2}}
	{\tt write\_canonical/2} has the same behavior as {\tt
	write\_canonical/1} but the output stream is explicitly
	designated using the first argument.

\ouritem{print(?Term)}\index{\texttt{print/1}}
    This predicate is intended to provide a handle for user-defined 
    pretty-printing.  Currently it is defined as {\em write/1}.

\end{description}

\comment{ TLS: Taken out because this appears to be superceded by
documentation in Manual 2

\section{Special I/O}

\begin{description}
\ouritem{fmt\_read(+Format, types(+T1,+T2,...), args(-A1,-A2,...), -RetCode)}\index{\texttt{fmt\_read/4}} 
 This predicate implements C-style formatted input. It reads the current
 input according to the {\tt Format} string.  {\tt Format} has the
 same syntax as the input format in C. The term {\tt types(...)} lists the
 types of the arguments; they must match the types specified in {\tt
 Format}. Here, 1 means string, 2 means integer, and 3 means float.
 The term {\tt args()} specifies the variables for the input. {\tt RetCode}
 specifies the return code: 0 -- ok; -1 -- end of file.

\ouritem{read\_line(-Line, -Status)}\index{\texttt{read\_line/2}} 
 Reads the next line from the current input and puts it in {\tt Line}.
 If the line is larger than the available buffer, then {\tt Status} is 0.
 If the line was read in full, up to and including the newline character,
 then {\tt Status} is 1. 

\ouritem{fmt\_write(+Format, args(-A1,-A2,...))}\index{\texttt{fmt\_write/2}} 
 Similar to formatted write in C. The semantics of the arguments is the
 same as for {\tt fmt\_read/4}.

\ouritem{fmt\_write\_string(-String, +Format, args(-A1,-A2,...))}\index{\texttt{fmt\_write/2}} 
 Like {\tt fmt\_write/2}, but the output string is placed in {\tt String}.

\end{description}
}

\section{Evaluating Arithmetic Expressions through {\tt is/2}}
\label{Arithmetic} 

As do most Prologs, XSB supports evaluation of arithmetic expressions
in two ways.  First, XSB supports evaluation of ground arithmetic
expressions throught the {\tt is/2} operator.  Support of comparisons
of non-ground arithmetic expressions is provided through the port to
XSB of the CLPQR constraint handling interface.  In this section we
describe how arithmetic expressions can be evaluated through {\tt
is/2}, while the CLPQR interface is described in Volume 2 of this
manual.

Before describing {\tt is/2} and the expressions that it can evaluate,
we note that in \version{} of XSB, integers and floats in XSB are
represented using a single word of 32 or 64 bits, depending on the
machine architecture.  Since type tags constitute part of this word,
this means that there may be as little as 28 bits available to
represent a floating point number.  In addition, evaluation of
arithmetic expressions through {\tt is/2} does not check for overflow
or underflow.  As a result, XSB's floating point operations do not
conform to IEEE floating point standards, and deviates in this regard
from the ISO Prolog standard (see \cite{ISO-Prolog} Section 9).  We
hope to fix these problems in a future release \footnote{We also note
that the ISO Prolog evaluable functors {\tt sign/1}, {\tt
float\_round/1}, {\tt float\_truncate/1}, {\tt float\_integer\_part/1}
and {\tt float\_fractional\_part/1} are not implemented in \version .}

All of the evaluable functors describe below throw a type error if
their evaluated input is not numeric.  We describe below only their
behavior on correctly typed input.

\begin{description}
\ouritem{is(?Result,+Expression)}\index{\texttt{is/2}} 
{\tt is(Result,Expression)} is true iff the result of evaluating {\tt
Expression} as a sequence of evaluable functors unifies with {\tt
Result}.  As mentioned in Section~\ref{inline_predicates}, {\tt is/2}
is an inline predicate, so calls to {\tt is/2} within compiled code
will not be visible during a trace of program execution.
\end{description}

\subsection{Evaluable Functors for Arithmetic Expressions}

\begin{description}
\ournewitem{+(+Expr1,+Expr2)}{Evaluable Functor}\index{\texttt{+/2}} 
If {\tt +Expr1} evaluates to {\tt Number1}, and {\tt Expr2} evaluates
to {\tt Number2}, returns {\tt Number1 + Number2}, performing any
necessary type conversions.

\ournewitem{-(+Expr1,+Expr2)}{Evaluable Functor}\index{\texttt{-/2}} 
If {\tt +Expr1} evaluates to {\tt Number1}, and {\tt Expr2} evaluates
to {\tt Number2}, returns {\tt Number1 - Number2}, performing any
necessary type conversions.

\ournewitem{*(+Expr1,+Expr2)}{Evaluable Functor}\index{\texttt{*/2}} 
If {\tt +Expr1} evaluates to {\tt Number1}, and {\tt Expr2} evaluates
to {\tt Number2}, returns {\tt Number1 * Number2} (i.e. multiplies
them), performing any necessary type conversions.

\ournewitem{/(+Expr1,Expr2)}{Evaluable Functor}\index{\texttt{'/'/2}}
If {\tt +Expr1} evaluates to {\tt Number1}, and {\tt Expr2} evaluates
to {\tt Number2}, returns {\tt Number1 / Number2} (i.e. divides
them), performing any necessary type conversions.

\ournewitem{//(+Expr1,Expr2)}{Evaluable Functor}\index{\texttt{'//'/2}}
If {\tt +Expr1} evaluates to {\tt Number1}, and {\tt Expr2} evaluates
to {\tt Number2}, returns {\tt Number1 // Number2} (i.e. integer
division), performing any necessary type conversions, and rounding to
0 if necessary.  

    Example:
    {\footnotesize
     \begin{verbatim}
                | ?- X is 3/2.

                X = 1.5000

                yes
                | ?- X is 3 // 2.

                X = 1

                yes
                | ?- X is -3 // 2.

                X = -1

                yes
  \end{verbatim}}

\ournewitem{-(+Expr1)}{Evaluable Functor}\index{\texttt{-/1}} If {\tt
+Expr} evaluates to {\tt Number}, returns {\tt -Number1},
performing any necessary type conversions.

\ournewitem{'/\'(+Expr1,+Expr2)}{Evaluable Functor}
\index{\texttt{'/\'/2}} If {\tt +Expr1} evaluates to {\tt Number1}, and {\tt Expr2}
evaluates to {\tt Number2}, returns the bitwise conjunction of {\tt
Number1} and {\tt Number2}.

\ournewitem{'\/'(+Expr1,+Expr2)}{Evaluable Functor}
\index{\texttt{'\/'/2}} If {\tt +Expr1} evaluates to {\tt Number1}, and {\tt Expr2}
evaluates to {\tt Number2}, returns the bitwise disjunction {\tt
Number1} and {\tt Number2}.

\ournewitem{'>>'(+Expr1,+Expr2)}{Evaluable Functor}
\index{\texttt{'>>'/2}} If {\tt +Expr1} evaluates to {\tt Number1}, and {\tt Expr2}
evaluates to {\tt Number2}, returns the logical shift right of 
{\tt Number1}, {\tt Number2} places.

\ournewitem{'<<'(+Expr1,+Expr2)}{Evaluable Functor}
\index{\texttt{'>>'/2}} If {\tt +Expr1} evaluates to {\tt Number1}, and {\tt Expr2}
evaluates to {\tt Number2}, returns the logical shift left of 
{\tt Number1}, {\tt Number2} places.

\ournewitem{ceiling(+Expr)}{Evaluable Functor}\index{\texttt{ceiling/1}}
If {\tt +Expr} evaluates to {\tt Number}, {\tt ceiling(Number)}
returns the integer ceiling of {\tt Number} if {\tt Number} is a
float, and {\tt Number} itself if {\tt Number} is an integer.

\ournewitem{float(+Expr)}{Evaluable Functor}\index{\texttt{float/1}} If
{\tt +Expr} evaluates to {\tt Number}, {\tt float(Number)} converts 
{\tt Number} to a float if {\tt Number} is an integer, and returns {\tt
Number} itself if {\tt Number} is a float.

\ournewitem{floor(+Expr)}{Evaluable Functor}\index{\texttt{floor/1}} If
{\tt +Expr} evaluates to {\tt Number}, {\tt floor(Number)} returns the
integer floor of {\tt Number} if {\tt Number} is a float, and {\tt
Number} itself if {\tt Number} is an integer.

\ournewitem{mod(+Expr1,+Expr2)}{Evaluable Functor}\index{\texttt{mod/2}}
If {\tt +Expr1} evaluates to {\tt Number1} and {\tt Expr2} evaluates
to {\tt Number2} where {\tt Number2} is not 0, {\tt
mod(Number1,Number2)} returns
\[
	Number1 - (\lfloor (Number1 / Number2) \rfloor) \times Number2)
\]

\ournewitem{rem(+Expr1,+Expr2)}{Evaluable Functor}\index{\texttt{rem/2}}
If {\tt +Expr1} evaluates to {\tt Number1} and {\tt Expr2} evaluates
to {\tt Number2} where {\tt Number2} is not 0, {\tt
rem(Number1,Number2)} returns
\[
	Number1 - ({Number1} // {Number2}) \times Number2)
\]

    Example:
    {\footnotesize
     \begin{verbatim}
                | ?- X is 5 mod 2.

                X = 1

                yes
                | ?- X is 5 rem 2.

                X = 1

                yes
                | ?- X is 5 mod -2.

                X = -1

                yes
                | ?- X is 5 rem -2.

                X = 1

                yes
  \end{verbatim}}

\ournewitem{round(+Expr)}{Evaluable Functor}\index{\texttt{round/1}} If
{\tt +Expr} evaluates to {\tt Number}, {\tt round(Number)} returns the
nearest integer to {\tt Number} if {\tt Number} is a float, and {\tt
Number} itself if {\tt Number} is an integer.

\ournewitem{sqrt(+Expr)}{Evaluable Functor}\index{\texttt{sqrt/1}}
If {\tt +Expr} evaluates to {\tt Number}, {\tt sqrt(Number)}
returns the square root of {\tt Number}.

\ournewitem{truncate(+Expr)}{Evaluable Functor}
\index{\texttt{truncate/1}} If {\tt +Expr} evaluates to {\tt Number}, {\tt
truncate(Number)} truncates {\tt Number} if {\tt Number} is a float,
and returns {\tt Number} itself if {\tt Number} is an integer.
\end{description}

\subsubsection{Mathematical Functions from {\tt math.h}}
\index{\texttt{cos/1}} \index{\texttt{sin/1}} \index{\texttt{tan/1}}
\index{\texttt{acos/1}} \index{\texttt{asin/1}} \index{\texttt{atan/1}}
\index{\texttt{log/1}} \index{\texttt{log10/1}}

XSB also allows as evaluable functors, many of the functions form the
C library {\tt math.h}.  Functions included in XSB \version{} are {\tt
cos/1}, {\tt sin/1}, {\tt tan/1}, {\tt acos/1}, {\tt asin/1}, {\tt
atan/1}. {\tt log/1}, and {\tt log10/1}.  For their semantics, see
documentation to {\tt math.h}.

\section{Convenience} \label{Convenience}
These predicates are standard and often self-explanatory, so they are 
described only briefly.
\begin{description}

\ouritem{true}\index{\texttt{true/0}} 
%\predindex{true/0~(I)}
    Always succeeds.

\ouritem{otherwise}\index{\texttt{otherwise/0}} 
%\predindex{otherwise/0~(B)}
    Same as {\tt true/0}.

\ouritem{fail}\index{\texttt{fail/0}}
%\predindex{fail/0~(I)}
    Always fails.

\ouritem{X = Y}\index{\texttt{=/2}} 
%\predindex{=/2~(I)}
    Defined as if by the clause ``Z=Z'', i.e.~{\tt X} and~{\tt Y} are unified.

\ouritem{X $\backslash$\,= Y}\index{\texttt{$\backslash$=/2}}
    Succeeds if~{\tt X} and~{\tt Y} are not unifiable,
    fails if~{\tt X} and~{\tt Y} are unifiable.
    It is thus equivalent to {\tt $\backslash$+}\/{\tt (X = Y)}.
\end{description}


\section{Negation and Control}\label{sec:control}
\index{control}

\begin{description}
\ouritem{\cut}\index{\texttt{"!/0}}  \index{cut} 
    Cut (discard) all choice points made since the parent goal
    started execution.
    Cuts across tabled predicates are not valid.  The compiler checks for
    such cuts, although whether the scope of a cut includes a tabled 
    predicate is undecidable in the presence of meta-predicates like
    {\tt call/1}.
    Further discussion of conditions allowing cuts and of their actions 
    can be found in Section~\ref{tabling_env}.

\ouritem{fail\_if(+P)}\index{\texttt{fail\_if/1}}
    If the goal {\tt P} has a solution, fails, otherwise it succeeds.
    Equivalently, it is true iff {\tt call(P)} 
    (see Section~\ref{meta_predicates}) is false. Argument {\tt P} 
    must be ground for sound negation as failure, although no runtime 
    checks are made by the system.

    The standard predicate {\tt fail\_if/1} is compiled by the 
    XSB compiler.

    Exceptions:
    \begin{description}
    \item[{\tt instantiation\_error}]
	{\tt P} is not instantiated.
    \item[{\tt type\_error}]
	{\tt P} is not a callable term.
    \end{description}


\ouritem{$\backslash$+ +P}\index{$\backslash$\texttt{+/1}}
    Exactly the same as {\tt fail\_if/1}.  Its existence is only 
    for compatibility with other Prolog systems.

\ouritem{not +P}\index{\texttt{not/1}} 
    If the goal {\tt P} has a solution, fails, otherwise it succeeds.
    It is defined by:
    \begin{center}
    \begin{minipage}{2.40in}
    \begin{verbatim}
	not(P) :- call(P), !, fail.
	not(_).
    \end{verbatim}
    \end{minipage}
    \end{center}

    Argument {\tt P} must be ground for sound negation, although no 
    runtime checks are made by the system.

    Note that in contrast to the other two kinds of negation as failure
    (\not\ and {\tt fail\_if/1}), predicate {\tt not/1} is not compiled
    by the compiler but the above definition is used.

    Exceptions: The same as {\tt call/1} 
		(see Section~\ref{meta_predicates}).

\ournewitem{tnot(+P)}{Tabling}\index{\texttt{tnot/1}}\label{tnot/1}
    The semantics of {\tt tnot/1} allows for correct execution of
    programs with according to the well-founded semantics.  {\tt P}
    must be a tabled predicate, 
% tls: need to provide an example.
%and the actions of {\tt tnot/1} are
%presented in Figure~\ref{fig:tnot}. 
%
%%----------------------------------------------------------------------------
\begin{figure}[hbt]
\longline
\begin{sf}
\begin{tabbing}
foooo\=foooo\=foooo\=foooo\=fooooooooooooooooooooo\=ooo\=\kill
\> tnot($S$) {\tt :-} \\
\> \> ( ground($S$) $\rightarrow$ \\
\> \> \> ( subgoal\_not\_in\_system($S$), call($S$), fail \\
\> \> \> ;   ( is\_complete($S$) $\rightarrow$ has\_no\_answers($S$) \\
\> \> \> \ \ ; negation\_suspend($S$), true \ \ \ 
			{\em /* if execution reaches here, $S$ */} \\
\> \> \> \ \ ) \> \>		{\em /* is completed with no answers */}\\
\> \> \> ) \\
\> \> ; error("Flounder: subgoal $S$ is not ground") \\
\> \> ).
\end{tabbing}
\longline
\end{sf}
\caption{The Prolog implementation of tabled negation ({\tt tnot/1})
	 for LRD-stratified programs.}
\label{fig:tnot}
\end{figure}
%----------------------------------------------------------------------------
 
%
    For a detailed description of the actions of tabled negation for
    in XSB \version\ see~\cite{SaSw98, SaSW96}.
    Chapter~\ref{chap:TablingOverview} contains further discussion of the
    functionality of {\tt tnot/1}.

    Exceptions:
    \begin{description}
    \item[{\tt instantiation\_error}]
	{\tt P} is not ground (floundering occurs).
    \item[{\tt type\_error}]
	{\tt P} is not a callable term.
    \item[{\tt table\_error}]
	{\tt P} is not a call to a tabled predicate.
    \end{description}


    \ournewitem{sk\_not(+P)}{Tabling}\index{\texttt{sk\_not/1}} 
%% 
If {\tt +P} is a tabled predicate, {\tt sk\_not/1} acts as {\tt tnot/1}
    but permits variables in its subgoal argument.  This replaces the
    {\tt 't not'/1} predicate of earlier XSB versions whose
    implementation and semantics were dubious. The semantics in the
    case of unbound variables is as follows: 
%% 
\begin{quote} 
\tt ... :- ...,~sk\_not(p(X)),~...  \end{quote} 
%% 
is equivalent to 
%%
\begin{quote}
 \tt ... :- ..., tnot(pp),~...\\ pp :- p(X).
\end{quote} 
%% 
where {\tt pp} is a new proposition. Thus, the unbound variable $X$ is
treated as $\tt tnot(\exists X (p(X)))$.

If {\tt +P} is a non-tabled predicate ensures that {\tt +P} is ground
and called via a tabled predicate so that {\tt sk\_not/1} can be used
with non-tabled predicates as well, regardless of whether {\tt +P} is
ground or not.

\ouritem{P -> Q ; R}\index{\texttt{->/2}} 
    Analogous to if {\tt P} then {\tt Q} else {\tt R}, i.e.\ defined as 
    if by
	\begin{center}
	\begin{minipage}{2.10in}
	\begin{verbatim}
	(P -> Q ; R) :- P, !, Q.
	(P -> Q ; R) :- R.
	\end{verbatim}
	\end{minipage}
	\end{center}

\ouritem{P -> Q} \index{\texttt{->/2}} 
    When occurring other than as one of the alternatives of a disjunction,
    is equivalent to:
	\begin{center}
	{\tt P -> Q ; fail.}
	\end{center}

\ouritem{repeat} \index{\texttt{repeat/2}} 
    Generates an infinite sequence of choice points (in other words it 
    provides a very convenient way of executing a loop). It is defined 
    by the clauses:
    \begin{center}
    \begin{minipage}{1.5in}
    \begin{verbatim}
	repeat.
	repeat :- repeat.
    \end{verbatim}
    \end{minipage}
    \end{center}

\end{description}


\section{Meta-Logical}\label{MetaLogical}

To facilitate manipulation of terms as objects in themselves,
XSB provides a number meta-logical predicates.  These
predicates include the standard meta-logical predicates of Prolog,
along with their usual semantics.  In addition are provided predicates
which provide special operations on HiLog terms.  For a full
discussion of Prolog and HiLog terms see Section~\ref{TermSyntax}.

\begin{description}
\ouritem{var(?X)}\index{\texttt{var/1}} 
    Succeeds if {\tt X} is currently uninstantiated (i.e.\ is still a 
    variable); otherwise it fails.  

    Term {\tt X} is uninstantiated if it has not been bound to anything, 
    except possibly another uninstantiated variable. Note in particular,
    that the HiLog term X(Y,Z) is considered to be instantiated.  There 
    is no distinction between a Prolog and a HiLog variable.

    Examples:
    {\footnotesize
     \begin{verbatim}
                | ?- var(X).
                yes
                | ?- var([X]).
                no
                | ?- var(X(Y,Z)).
                no
                | ?- var((X)).
                yes
                | ?- var((X)(Y)).
                no
     \end{verbatim}}


\ouritem{nonvar(?X)}\index{\texttt{nonvar/1}} 
    Succeeds if {\tt X} is currently instantiated to a non-variable term;
    otherwise it fails. This has exactly the opposite behaviour of 
    {\tt var/1}\@.

\ouritem{atom(?X)}\index{\texttt{atom/1}} 
    Succeeds only if the {\tt X} is currently instantiated to an atom, that
    is to a Prolog or HiLog non-numeric constant.

    Examples:
    {\footnotesize
     \begin{verbatim}
                | ?- atom(HiLog).
                no
                | ?- atom(10).
                no
                | ?- atom('HiLog').
                yes
                | ?- atom(X(a,b)).
                no
                | ?- atom(h).
                yes
                | ?- atom(+).
                yes
                | ?- atom([]).
                yes
     \end{verbatim}}

\ouritem{integer(?X)}\index{\texttt{integer/1}}
    Succeeds if {\tt X} is currently instantiated to an integer; 
    otherwise it fails. 

\ouritem{real(?X)}\index{\texttt{real/1}} 
    Succeeds if {\tt X} is currently instantiated to a floating point number;
    otherwise it fails. 

\ouritem{float(?X)}\index{\texttt{float/1}} 
    Same as {\tt real/1}. Succeeds if {\tt X} is currently instantiated 
    to a floating point number; otherwise it fails.  This predicate is 
    included for compatibility with earlier versions of SBProlog.
	
\ouritem{number(?X)}\index{\texttt{number/1}} 
    Succeeds if {\tt X} is currently instantiated to either an integer or 
    a floating point number (real); otherwise it fails.

\ouritem{atomic(?X)}\index{\texttt{atomic/1}} 
    Succeeds if {\tt X} is currently instantiated to an atom or a number;
    otherwise it fails.

    Examples:
    {\footnotesize
     \begin{verbatim}
                | ?- atomic(10).
                yes
                | ?- atomic(p).
                yes
                | ?- atomic(h).
                yes
                | ?- atomic(h(X)).
                no
                | ?- atomic("foo").
                no
                | ?- atomic('foo').
                yes
                | ?- atomic(X).
                no
                | ?- atomic(X((Y))).
                no
     \end{verbatim}}

\ouritem{compound(?X)}\index{\texttt{compound/1}} 
    Succeeds if {\tt X} is currently instantiated to a compound term (with 
    arity greater that zero), i.e.\ to a nonvariable term that is not atomic;
    otherwise it fails.

    Examples:
    {\footnotesize
     \begin{verbatim}
                | ?- compound(1).
                no
                | ?- compound(foo(1,2,3)).
                yes
                | ?- compound([foo, bar]).
                yes
                | ?- compound("foo").
                yes
                | ?- compound('foo').
                no
                | ?- compound(X(a,b)).
                yes
                | ?- compound((a,b)).
                yes	
     \end{verbatim}}

\ouritem{structure(?X)}\index{\texttt{structure/1}} 
    Same as {\tt compound/1}\@. Its existence is only for compatibility 
    with SB-Prolog version 3.1.

\ouritem{is\_list(?X)}\index{\texttt{is\_list/1}}
    Succeeds if {\tt X} is a {\em proper list}. In other words if it is 
    either the atom {\tt []} or {\tt [H|T]} where H is any Prolog or HiLog
    term and T is a proper list; otherwise it fails.

    Examples:
    {\footnotesize
     \begin{verbatim}
                | ?- is_list([p(a,b,c), h(a,b)]).
                yes
                | ?- is_list([_,_]).
                yes
                | ?- is_list([a,b|X]).
                no
                | ?- is_list([a|b]).
                no
     \end{verbatim}}

\ouritem{is\_charlist(+X)}\index{\texttt{is\_charlist/1}}
    Succeeds if {\tt X} is a Prolog string, {\it i.e.}, a list of
    characters.
    Examples:
    {\footnotesize
     \begin{verbatim}
                | ?- is_charlist("abc").
                yes
                | ?- is_charlist(abc).
                no
     \end{verbatim}}

\ouritem{is\_charlist(+X,-Size)}\index{\texttt{is\_charlist/2}}
    Works as above, but also returns the length of that string in the second
    argument, which must be a variable.

\ouritem{is\_attv(+Term)}\index{\texttt{is\_attv/1}}
    Succeeds is {\tt Term} is an attributed variable, and fails otherwise.

\ouritem{is\_most\_general\_term(?X)}
\index{\texttt{is\_most\_general\_term/1}}
    Succeeds if {\tt X} is compound term with all distinct variables
as arguments, or if {\tt X} is an atom. (It fails if X is a cons node.)
    {\footnotesize
     \begin{verbatim}
                | ?- is_most_general_term(f(_,_,_,_)).
                yes
                | ?- is_most_general_term(abc).
                yes
                | ?- is_most_general_term(f(X,Y,Z,X)).
                no
                | ?- is_most_general_term(f(X,Y,Z,a)).
                no
                | ?- is_most_general_term([_|_]).
                no
     \end{verbatim}}

\ouritem{callable(?X)}\index{\texttt{callable/1}}
    Succeeds if {\tt X} is currently instantiated to a term that standard
    predicate {\tt call/1} could take as an argument and not give an 
    instantiation or type error.  Note that it only checks for errors of
    predicate {\tt call/1}.  In other words it succeeds if {\tt X}
    is an atom or a compound term; otherwise it fails.  Predicate
    {\tt callable/1} has no associated error conditions.

    Examples:
    {\footnotesize
     \begin{verbatim}
                | ?- callable(p).
                yes
                | ?- callable(p(1,2,3)).
                yes
                | ?- callable([_,_]).
                yes
                | ?- callable(_(a)).
                yes
                | ?- callable(3.14).
                no
     \end{verbatim}}

\ournewitem{proper\_hilog(?X)}{HiLog}\index{\texttt{proper\_hilog/1}}
    Succeeds if {\tt X} is a proper HiLog term; otherwise it fails.

    Examples:
    (In this example and the rest of the examples of this section we assume
     that {\tt h} is the only parameter symbol that has been declared a HiLog
     symbol).

    {\footnotesize
     \begin{verbatim}
                | ?- proper_hilog(X).
                no
                | ?- proper_hilog(foo(a,f(b),[A])).
                no
                | ?- proper_hilog(X(a,b,c)).
                yes
                | ?- proper_hilog(3.6(2,4)).
                yes
                | ?- proper_hilog(h).
                no
                | ?- proper_hilog([a, [d, e, X(a)], c]).
                yes
                | ?- proper_hilog(a(a(X(a)))).
                yes
     \end{verbatim}}

\ouritem{functor(?Term, ?Functor, ?Arity)}\index{\texttt{functor/3}}
    Succeeds if the {\em functor} of the Prolog term {\tt Term} is 
    {\tt Functor} and the {\em arity} (number of arguments) of {\tt Term} is
    {\tt Arity}\@.  {\tt Functor} can be used in either the following two 
    ways:
    \begin{enumerate}
    \item If {\tt Term} is initially instantiated, then
          \begin{itemize}
          \item If {\tt Term} is a compound term, {\tt Functor} and 
                {\tt Arity} are unified with the name and arity of 
                its principal functor, respectively. 
          \item If {\tt Term} is an atom or a number, {\tt Functor} is 
                unified with {\tt Term}, and {\tt Arity} is unified with 0.
          \end{itemize}
    \item If {\tt Term} is initially uninstantiated, then either both 
          {\tt Functor} and {\tt Arity} must be instantiated, or {\tt Functor}
          is instantiated to a number, and
          \begin{itemize}
          \item If {\tt Arity} is an integer in the range 1..255, then
                {\tt Term} becomes instantiated to {\em the most general 
                Prolog term} having the specified {\tt Functor} and 
                {\tt Arity} as principal functor and number of arguments,
                respectively. The variables appearing as arguments of 
                {\tt Term} are all distinct.
          \item If {\tt Arity} is 0, then {\tt Functor} must be either an 
                atom or a number and it is unified with {\tt Term}. 
          \item If {\tt Arity} is anything else, then {\tt functor/3} aborts.
          \end{itemize}
    \end{enumerate}

    Exceptions:
    \begin{description}
    \item[{\tt domain\_error}]
	{\tt Functor} is instantiated to a compound term.
    \item[{\tt instantiation\_error}]
	Both {\tt Term}, and either {\tt Functor}, or {\tt Arity} are 
	uninstantiated.
    \end{description}

    Examples:
    {\footnotesize
     \begin{verbatim}
                | ?- functor(p(f(a),b,t), F, A).
                F = p
                A = 3

                | ?- functor(T, foo, 3).
                T = foo(_595708,_595712,_595716)

                | ?- functor(T, 1.3, A).
                T = 1.3
                A = 0

                | ?- functor(foo, F, 0).
                F = foo

                | ?- functor("foo", F, A).
                F = .
                A = 2

                | ?- functor([], [], A).
                A = 0

                | ?- functor([2,3,4], F, A).
                F = .
                A = 2

                | ?- functor(a+b, F, A).
                F = +
                A = 2

                | ?- functor(f(a,b,c), F, A).
                F = f
                A = 3

                | ?- functor(X(a,b,c), F, A).
                F = apply
                A = 4

                | ?- functor(map(P)(a,b), F, A).
                F = apply
                A = 3

                | ?- functor(T, foo(a), 1).
                ++Error: Wrong type in argument 2 of functor/3
                Aborting...

                | ?- functor(T, F, 3).
                ++Error: Uninstantiated argument 2 of functor/3
                Aborting...

                | ?- functor(T, foo, A).
                ++Error: Uninstantiated argument 3 of functor/3
                Aborting...
     \end{verbatim}}

\ournewitem{hilog\_functor(?Term, ?F, ?Arity)}{HiLog}
\index{\texttt{hilog\_functor/3}}
    The XSB standard predicate {\tt hilog\_functor/3} succeeds 
    \begin{itemize}
    \item when {\tt Term} is a Prolog term and the principal function 
          symbol ({\em functor}) of {\tt Term} is {\tt F} and the
          {\em arity} (number of arguments) of {\tt Term} is 
          {\tt Arity}, or
    \item when {\tt Term} is a HiLog term, having {\em name} {\tt F} 
          and the number of arguments {\tt F} is applied to, in the 
          HiLog term, is {\tt Arity}.
    \end{itemize}
    The first of these cases corresponds to the ``usual'' behaviour of
    Prolog's {\tt functor/3}, while the second is the extension of
    {\tt functor/3} to handle HiLog terms. Like the Prolog's {\tt
    functor/3} predicate, {\tt hilog\_functor/3} can be used in either
    of the following two ways:
    \begin{enumerate}
    \item If {\tt Term} is initially instantiated, then
          \begin{itemize}
          \item If {\tt Term} is a Prolog compound term, {\tt F} and 
                {\tt Arity} are unified with the name and arity of 
                its principal functor, respectively.
          \item If {\tt Term} is an atom or a number, {\tt F} is unified 
                with {\tt Term}, and {\tt Arity} is unified with 0.
          \item If {\tt Term} is any other HiLog term, {\tt F} and {\tt Arity} 
                are unified with the name and the number of arguments 
                that {\tt F} is applied to. Note that in this case {\tt F}
                may still be uninstantiated.
          \end{itemize}
    \item If {\tt Term} is initially uninstantiated, then at least
          {\tt Arity} must be instantiated, and
          \begin{itemize}
          \item If {\tt Arity} is an integer in the range 1..255, then
                {\tt Term} becomes instantiated to {\em the most general 
                Prolog or HiLog term} having the specified {\tt F} and 
                {\tt Arity} as name and number of arguments {\tt F} is 
                applied to, respectively. The variables appearing as 
                arguments are all unique.
          \item If {\tt Arity} is 0, then {\tt F} must be a Prolog or
                HiLog constant, and it is unified with {\tt Term}\@. Note
                that in this case {\tt F} cannot be a compound term.
          \item If {\tt Arity} is anything else, then {\tt hilog\_functor/3}
                aborts.
          \end{itemize}
    \end{enumerate}
    In other words, the standard predicate {\tt hilog\_functor/3} either
    decomposes a given HiLog term into its {\em name} and {\em arity}, or
    given an arity ---and possibly a name--- constructs the corresponding 
    HiLog term creating new uninstantiated variables for its arguments. 
    As happens with {\tt functor/3} all constants can be their own 
    principal function symbols.

    Examples:
    {\footnotesize
     \begin{verbatim}
               | ?- hilog_functor(f(a,b,c), F, A).
               F = f
               A = 3

               | ?- hilog_functor(X(a,b,c), F, A).
               X = _595836
               F = _595836
               A = 3

               | ?- hilog_functor(map(P)(a,b), F, A).
               P = _595828
               F = map(_595828)
               A = 2

               | ?- hilog_functor(T, p, 2).
               T = p(_595708,_595712)

               | ?- hilog_functor(T, h, 2).
               T = apply(h,_595712,_595716)

               | ?- hilog_functor(T, X, 3).
               T = apply(_595592,_595736,_595740,_595744)
               X = _595592

               | ?- hilog_functor(T, p(f(a)), 2).
               T = apply(p(f(a)),_595792,_595796)

               | ?- hilog_functor(T, h(p(a))(L1,L2), 1).
               T = apply(apply(apply(h,p(a)),_595984,_595776),_596128)
               L1 = _595984
               L2 = _595776

               | ?- hilog_functor(T, a+b, 3).
               T = apply(a+b,_595820,_595824,_595828)
     \end{verbatim}}

\ouritem{arg(+Index, +Term, ?Argument)}\index{\texttt{arg/3}}
    Unifies {\tt Argument} with the ${\tt Index}^{th}$ argument of 
    {\tt Term}\@, where the index is taken to start at $1$.
    Initially, {\tt Index} must be instantiated to any integer and {\tt Term} 
    to any non-variable Prolog or HiLog term.
    The arguments of the {\tt Term} are numbered from~1 upwards. An atomic term
    has $0$ arguments. If the initial conditions are not satisfied or~$I$ is 
    out of range, the call quietly fails.

    Examples:
    {\footnotesize
     \begin{verbatim}
                   | ?- arg(2, p(a,b), A).
                   A = b

                   | ?- arg(2, h(a,b), A).
                   A = a

                   | ?- arg(0, foo, A).
                   no

                   | ?- arg(2, [a,b,c], A).
                   A = [b,c]

                   | ?- arg(2, "HiLog", A).
                   A = [105,108,111,103]

                   | ?- arg(2, a+b+c, A).
                   A = c

                   | ?- arg(3, X(a,b,c), A).
                   X = _595820 
                   A = b 

                   | ?- arg(2, map(f)(a,b), A).
                   A = a

                   | ?- arg(1, map(f)(a,b), A). 
                   A = map(f)

                   | ?- arg(1, (a+b)(foo,bar), A).
                   A = a+b
     \end{verbatim}}

\ouritem{arg0(+Index, +Term, ?Argument)}\index{\texttt{arg0/3}}
    Unifies {\tt Argument} with the ${\tt Index}^{th}$ argument of {\tt Term}
    if {\tt Index} $>$ 0, or with the functor of {\tt Term} if {\tt Index} = 0.

\ournewitem{hilog\_arg(+Index, +Term, ?Argument)}{HiLog}\index{\texttt{hilog\_arg/3}}
    If {\tt Term} is a Prolog term, it has the same behaviour as {\tt arg/3},
    but if {\tt Term} is a proper HiLog term, {\tt hilog\_arg/3} unifies 
    {\tt Argument} with the 
    $({\tt Index}+1)^{th}$ argument of the Prolog representation of 
    {\tt Term}\@.  Semantically, {\tt Argument} is the ${\tt Index}^{th}$ 
    argument to which the {\em HiLog functor} of {\tt Term} is applied.
    The arguments of the {\tt Term} are numbered from~1 upwards. An atomic term 
    is taken to have $0$ arguments.  
    
    Initially, {\tt Index} must be instantiated to a positive integer and 
    {\tt Term} to any non-variable Prolog or HiLog term.
    If the initial conditions are not satisfied or~$I$ is 
    out of range, the call quietly fails. Note that like {\tt arg/3}
    this predicate does not succeed for {\tt Index}=0.

    Examples:
    {\footnotesize
     \begin{verbatim}
                   | ?- hilog_arg(2, p(a,b), A).
                   A = b

                   | ?- hilog_arg(2, h(a,b), A).
                   A = b

                   | ?- hilog_arg(3, X(a,b,c), A).
                   X = _595820
                   A = c

                   | ?- hilog_arg(1, map(f)(a,b), A).
                   A = a

                   | ?- hilog_arg(2, map(f)(a,b), A).
                   A = b

                   | ?- hilog_arg(1, (a+b)(foo,bar), A).
                   A = foo

                   | ?- hilog_arg(1, apply(foo), A). 
                   A = foo

                   | ?- hilog_arg(1, apply(foo,bar), A).
                   A = bar
     \end{verbatim}}

    Note the difference between the last two examples. The difference is 
    due to the fact that {\tt apply/1} is a Prolog term, while 
    {\tt apply/2} is a proper HiLog term.

\ouritem{?Term =.. [?Functor |?ArgList]}\index{\texttt{=../2}} 
    Succeeds when {\tt Term} is any (Prolog or) HiLog term, {\tt Functor} is 
    its Prolog functor and {\tt ArgList} is the list of its arguments. 
    The use of {\tt =../2} (pronounced {\em univ}) although convenient, 
    can nearly always be avoided.
    Whenever efficiency is critical, it is advisable to use the
    predicates {\tt functor/3} and {\tt arg/3}, since {\tt =../2} is 
    implemented by calls to these predicates.  The behaviour of 
    {\tt =../2} is as follows:
    \begin{itemize}
    \item If initially {\tt Term} is uninstantiated, then the list in the 
          second argument of {\tt =../2} must be instantiated either to a
          {\em proper list} (list of determinate length) whose head is an 
          atom, or to a list of length 1 whose head is a number.
    \item If the arguments of {\tt =../2} are both uninstantiated, or if 
          either of them is not what is expected, {\tt =../2} aborts, 
          producing an appropriate error message.
    \end{itemize}

    Examples:
    {\footnotesize
     \begin{verbatim}
                   | ?- X - 1 =.. L.
                   X = _595692
                   L = [-,_595692,1]

                   | ?- p(a,b,c) =.. L.
                   L = [p,a,b,c]

                   | ?- h(a,b,c) =.. L.
                   L = [apply,h,a,b,c]

                   | ?- map(p)(a,b) =.. L.
                   L = [apply,map(p),a,b]

                   | ?- T =.. [foo].
                   T = foo

                   | ?- T =.. [3|X].
                   T = 3
                   X = []

                   | ?- T =.. [apply,X,a,b].
                   T = apply(X,a,b)

                   | ?- T =.. [1,2].
                   ++Error: Wrong type(s) in argument 2 of =../2
                   Aborting...

                   | ?- T =.. [a+b,2].
                   ++Error: Wrong type(s) in argument 2 of =../2
                   Aborting..

                   | ?- X =.. [foo|Y].
                   ++Error: Argument 2 of =../2 is not a proper list
                   Aborting...
     \end{verbatim}}

     Exceptions:
     \begin{description}
     \item[{\tt instantiation\_error}]
	Argument 2 of {\tt =../2} is not a proper list.
     \item[{\tt type\_error}]
	Head of argument 2 of {\tt =../2} is not an atom or number.
     \end{description}

%in the following line I used a very ``dirty'' LaTeX hack! -- Kostis.

\ournewitem{?Term \^{ }=.. [?F |?ArgList]}{HiLog}\index{\texttt{\^{ }=../2}} 
    When {\tt Term} is a Prolog term, this predicate behaves exactly like
    the Prolog {\tt =../2}. However when {\tt Term} is a proper HiLog term, 
    {\tt \verb|^|=../2} 
    succeeds unifying {\tt F} to its HiLog functor and {\tt ArgList} to the 
    list of the arguments to which this HiLog functor is applied. Like 
    {\tt =../2}, the use of {\tt \verb|^|=../2} can nearly always be avoided
    by using the more efficient predicates {\tt hilog\_functor/3} and 
    {\tt hilog\_arg/3}. The behaviour of {\tt \verb|^|=../2}, on HiLog terms 
    is as follows:
    \begin{itemize}
    \item If initially {\tt Term} is uninstantiated, then the list in the 
          second argument of {\tt \verb|^|=../2} must be instantiated to 
          a {\em proper list} (list of determinate length) whose head can 
          be any Prolog or HiLog term.
    \item If the arguments of {\tt \verb|^|=../2} are both uninstantiated, 
          or if the second of them is not what is expected, 
          {\tt \verb|^|=../2} aborts, producing an appropriate error message.
    \end{itemize}
    Examples:
    {\footnotesize
     \begin{verbatim}
                   | ?- p(a,b,c) ^=.. L.
                   L = [p,a,b,c]

                   | ?- h(a,b,c) ^=.. L.
                   L = [h,a,b,c]

                   | ?- map(p)(a,b) ^=.. L.
                   L = [map(p),a,b]

                   | ?- T ^=.. [X,a,b].
                   T = apply(X,a,b)

                   | ?- T ^=.. [2,2].
                   T = apply(2,2)

                   | ?- T ^=.. [a+b,2].
                   T = apply(a+b,2)

                   | ?- T ^=.. [3|X].
                   ++Error: Argument 2 of ^=../2 is not a proper list
                   Aborting...
     \end{verbatim}}

     Exceptions:
     \begin{description}
     \item[{\tt instantiation\_error}]
	Argument 2 of {\tt \verb|^|=../2} is not a proper list.
     \end{description}

\ouritem{copy\_term(+Term, -Copy)}\index{\texttt{copy\_term/2}}
    Makes a {\tt Copy} of {\tt Term} in which all variables have been
    replaced by brand new variables which occur nowhere else. It can
    be very handy when writing (meta-)interpreters for logic-based
    languages.  The version of {\tt copy\_term/2} provided is {\em space
    efficient} in the sense that it never copies ground terms. 
    Predicate {\tt copy\_term/2} has no associated errors or exceptions.

    Examples:
    {\footnotesize
     \begin{verbatim}
                   | ?- copy_term(X, Y).

                   X = _598948
                   Y = _598904

                   | ?- copy_term(f(a,X), Y).

                   X = _598892
                   Y = f(a,_599112)
     \end{verbatim}}

\end{description}

\subsection{Atom Manipulation}

\begin{description}

\ouritem{name(?Constant, ?CharList)}\index{\texttt{name/2}} 
    The standard predicate {\tt name/2} performs the conversion 
    between a constant and its character list representation. 
    If {\tt Constant} is supplied (and is any atom or number), {\tt CharList} 
    is unified with a list of ASCII codes representing the {\em ``name''} 
    of the constant.  In that case, {\tt CharList} is exactly the list of 
    ASCII character codes that appear in the printed representation of 
    {\tt Constant}\@.  If on the other hand {\tt Constant} is a variable, 
    then {\tt CharList} must be a proper list of ASCII character codes. 
    In that case, {\tt name/2} will convert a list of ASCII characters 
    that can represent a number to a number rather than to a character string.
    As a consequence of this, there are some atoms (for example \verb|'18'|)
    which cannot be constructed by using {\tt name/2}\@. 
    If conversion to an atom is preferred in these cases, the 
    standard predicate {\tt atom\_codes/2} should be used instead. The 
    syntax for numbers that is accepted by {\tt name/2} is exactly the one 
    which {\tt read/1} accepts.  Predicate {\tt name/2} is provided for 
    backwards compatibility.  It is advisable that new programs use
    the predicates {\tt atom\_codes/2} and {\tt number\_codes/2} described
    below.

    In \version\ predicate {\tt name/2} is not yet implemented for 
    converting from a real number to its character list representation, 
    and if the representation of a real is provided as {\tt CharList}, 
    it will be converted to an atom.

    If both of the arguments of {\tt name/2} are uninstantiated or 
    {\tt CharList} is not a proper list of ASCII characters, {\tt name/2} 
    will abort and an error message will be sent to the standard error stream.

    Examples:
    {\footnotesize
     \begin{verbatim}
                   | ?- name('Foo', L).
                   L = [70,111,111]

                   | ?- name([], L).
                   L = [91,93]

                   | ?- name(431, L).
                   L = [52,51,49]

                   | ?- name(X, [102,111,111]).
                   X = foo
 
                   | ?- name(X, []).
                   X = ''

                   | ?- name(X, "Foo").
                   X = 'Foo'

                   | ?- name(X, [52,51,49]).
                   X = 431

                   | ?- name(X, [45,48,50,49,51]), integer(X).
                   X = -213

                   | ?- name(3.14, L).
                   ++Error: Predicate name/2 for reals is not implemented yet
                   Aborting...
     \end{verbatim}}

     Exceptions:
     \begin{description}
     \item[{\tt instantiation\_error}]
	Both arguments are uninstantiated, or argument 2 of {\tt name/2} 
	contains a variable or is not a proper list.
     \item[{\tt type\_error}]
	{\tt Constant} is not a variable, an atom or a number.
     \item[{\tt range\_error}]
	{\tt CharList} is not a list of ASCII characters.
     \item[{\tt implementation\_error}]
	{\tt Constant} is a real number (conversion from a real to its 
	character list representation is not implemented yet).
     \end{description}

\ouritem{atom\_codes(?Atom, ?CharCodeList)}\index{\texttt{atom\_codes/2}} 
    The standard predicate {\tt atom\_codes/2} performs the conversion 
    between an atom and its character list representation. 
    If {\tt Atom} is supplied (and is an atom), {\tt CharList} 
    is unified with a list of ASCII codes representing the {\em ``name''} 
    of that atom.  In that case, {\tt CharList} is exactly the list of 
    ASCII character codes that appear in the printed representation of 
    {\tt Atom}.  If on the other hand {\tt Atom} is a variable, 
    then {\tt CharList} must be a proper list of ASCII character codes. 
    In that case, {\tt Atom} is instantiated to an atom containing
    exactly those characters, even if the characters look like the
    printed representation of a number.

    If both of the arguments of {\tt atom\_codes/2} are uninstantiated or
    {\tt CharList} is not a proper list of ASCII characters, {\tt
    atom\_codes/2} aborts, and an error message will be sent to
    the standard error stream.

    Examples:
    {\footnotesize
     \begin{verbatim}
                   | ?- atom_codes('Foo', L).
                   L = [70,111,111]

                   | ?- atom_codes([], L).
                   L = [91,93]

                   | ?- atom_codes(X, [102,111,111]).
                   X = foo
 
                   | ?- atom_codes(X, []).
                   X = ''

                   | ?- atom_codes(X, "Foo").
                   X = 'Foo'

                   | ?- atom_codes(X, [52,51,49]).
                   X = '431'

                   | ?- atom_codes(X, [52,51,49]), integer(X).
                   no

                   | ?- atom_codes(X, [52,Y,49]).
                   ! Instantiation error in argument 2 of atom_codes/2
                   ! Aborting...

                   | ?- atom_codes(431, L).
                   ! Type error: in argument 1 of atom_codes/2
                   ! atom expected, but something else found
                   ! Aborting...

                   | ?- atom_codes(X, [52,300,49]).
                   ! Range error: in argument 2 of atom_codes/2
                   ! ASCII code expected, but 300 found
                   ! Aborting...
     \end{verbatim}}

    Exceptions:
    \begin{description}
    \item[{\tt instantiation\_error}]
	Both arguments are uninstantiated, or argument 2
	is not a proper list, or it contains a variable.
    \item[{\tt type\_error}]
	{\tt Atom} is not a variable or an atom.
    \item[{\tt range\_error}]
	{\tt CharList} is not a list of ASCII characters.
    \end{description}

\ouritem{atom\_chars(?Number, ?CharAtomList)}\index{\texttt{atom\_chars/2}} 
    Like \verb|atom_codes|, but the list returned (or input) is a list of
    characters \emph{as atoms} rather than ASCII codes. For instance, 
    \verb|atom_chars(abc,X)| binds {\tt X} to the list {\tt [a,b,c]}
    instead of {\tt [97,98,99]}.

\ouritem{number\_codes(?Number, ?CharCodeList)}\index{\texttt{number\_codes/2}} 
    The standard predicate {\tt number\_codes/2} performs the conversion 
    between a number and its character list representation. 
    If {\tt Number} is supplied (and is a number), {\tt CharList} is
    unified with a list of ASCII codes comprising the printed representation
    of that {\tt Number}.  If on the other hand {\tt Number} is a variable, 
    then {\tt CharList} must be a proper list of ASCII character codes that
    corresponds to the correct syntax of a number (either integer or float)
    In that case, {\tt Number} is instantiated to that number, otherwise
    {\tt number\_codes/2} will simply fail.

    If both of the arguments of {\tt number\_codes/2} are uninstantiated or
    {\tt CharList} is not a proper list of ASCII characters, {\tt
    number\_codes/2} aborts, and an error message will be sent to
    the standard error stream.

    Examples:
    {\footnotesize
     \begin{verbatim}
                   | ?- number_codes(123, L).
                   L = [49,50,51];

                   | ?- number_codes(N, [49,50,51]), integer(N).
                   N = 123

                   | ?- number_codes(31.4e+10, L).
                   L = [51,46,49,51,57,57,57,55,69,43,49,48]

                   | ?- number_codes(N, "314e+8").
                   N = 3.14e+10

                   | ?- number_codes(foo, L).
                   ! Type error: in argument 1 of number_codes/2
                   ! number expected, but something else found
                   ! Aborting...
     \end{verbatim}}

    Exceptions:
    \begin{description}
    \item[{\tt instantiation\_error}]
	Both arguments are uninstantiated, or argument 2
	is not a proper list, or it contains a variable.
    \item[{\tt type\_error}]
	{\tt Number} is not a variable or a number.
    \item[{\tt range\_error}]
	{\tt CharList} is not a list of ASCII characters.
    \end{description}

\ouritem{number\_chars(?Number, ?CharAtomList)}\index{\texttt{number\_chars/2}} 
    Like \verb|number_codes|, but the list returned (or input) is a list of
    characters \emph{as atoms} rather than ASCII codes. For instance, 
    \verb|number_chars(123,X)| binds {\tt X} to the list {\tt ['1','2','3']}
    instead of {\tt [49,50,51]}.

\ouritem{number\_digits(?Number, ?DigitList)}\index{\texttt{number\_digits/2}} 
    Like \verb|number_chars|, but the list returned (or input) is a list of
    digits \emph{as numbers} rather than ASCII codes (for floats, the atom
    '.', '+' or '-', and 'e' will also be present in the list). For instance, 
    \verb|number_digits(123,X)| binds {\tt X} to the list {\tt [1,2,3]}
    instead of {\tt ['1','2','3']}, and \verb|number_digits(123.45,X)|
    binds {\tt X} to {\tt [1,.,2,3,4,5,0,0,e,+,0,2]}.

\ourisousageitem{atom\_concat(Atom1,Atom2,Atom3)}\index{\texttt{atom\_concat/3}}
\ourusage{atom\_concat(?Atom,?Atom,+Atom)}
\ourusage{atom\_concat(+Atom,+Atom,-Atom)}


\end{description}


\section{All Solutions and Aggregate Predicates}
\index{sets, bags} \index{aggregate predicates!prolog}
%-----------------------------------------------------
Often there are many solutions to a problem and it is necessary
somehow to compare these solutions with one another.  The most general
way of doing this is to collect all the solutions into a list, which
may then be processed in any way desired.  So XSB provides
ISO-standard predicates such as {\tt setof/3}, {\tt bagof/3}, and {\tt
findall/3} to collect solutions into lists.  Sometimes however, one
wants simply to perform some aggregate operation over the set of
solutions, for example to find the maximum or minimum of the set of
solutions.  XSB uses tabling and HiLog to provide a general and
powerful aggregation facility as discussed in Section
\ref{tabling_aggregate_predicates}.

\begin{description}
\ouritem{setof(?X, +Goal, ?Set)}\index{\texttt{setof/3}}
    This predicate may be read as ``{\tt Set} is the set of all instances 
    of {\tt X} such that {\tt Goal} is provable''.
    If~{\tt Goal} is not provable, {\tt setof/3} fails.
    The term {\tt Goal} specifies a goal or goals as in {\tt call(Goal)}.
    {\tt Set} is a set of terms represented as a list of those terms,
    without duplicates, in the standard order for terms 
    (see Section~\ref{Comparison}).
    If there are uninstantiated variables in {\tt Goal} which do not also 
    appear in {\tt X}, then a call to this evaluable predicate may backtrack,
    generating alternative values for~{\tt Set} corresponding to different
    instantiations of the free variables of~{\tt Goal}.
    Variables occurring in {\tt Goal} will not be treated as free if they 
    are explicitly bound within~{\tt Goal} by an existential quantifier.
    An existential quantification can be specified as:
    \begin{center}
    {\tt Y \^\ G}\index{\^}
    \end{center}
    meaning there exists a {\tt Y} such that {\tt G} is true,
    where {\tt Y} is some Prolog term (usually, a variable).
  
    Exceptions: Same as predicate {\tt call/1}
		(see Section~\ref{meta_predicates}).

\ouritem{bagof(?X, +Goal, ?Bag)} \index{\texttt{bagof/3}}
    This predicate has the same semantics as {\tt setof/3} except that the 
    third argument returns an unordered list that may contain duplicates.

    Exceptions: Same as predicate {\tt call/1}
		(see Section~\ref{meta_predicates}).

\ouritem{findall(?X, +Goal, ?List)} \index{\texttt{findall/3}}
    Similar to predicate {\tt bagof/3}, except that variables in {\tt Goal}
    that do not occur in {\tt X} are treated as existential, and alternative
    lists are not returned for different bindings of such variables.  This
    makes {\tt findall/3} deterministic (non-backtrackable).  Unlike
    {\tt setof/3} and {\tt bagof/3}, if {\tt Goal} is unsatisfiable,
    {\tt findall/3} succeeds binding {\tt List} to the empty list.

    Exceptions: Same as predicate {\tt call/1}
		(see Section~\ref{meta_predicates}).

\ournewitem{tfindall(?X, +Goal, ?List)}{Tabling}
\index{\texttt{tfindall/3}}
%\predindex{tfindall/3~(L)}
\label{tfindall/3}

    {\em Note: {\tt tfindall/3} may be deprecated in current versions.
    Please use the predicates described in
    Section~\ref{tabling_aggregate_predicates} if possible}.

    Like {\tt findall/3}, {\tt tfindall/3} treats all variables in
    {\tt Goal} that do not occur in {\tt X} as existential.  However,
    in {\tt tfindall/3}, the {\tt Goal} must be a call to a single
    tabled predicate.
	
    {\tt tfindall/3} allows the user to build programs that use
    stratified aggregation.  If the table to {\tt Goal} is incomplete,
    {\tt tfindall/3} suspends until the table has been completed, and
    only then computes {\tt List}.  See Chapter~\ref{chap:TablingOverview}
    for further discussion of {\tt tfindall/3}.  Like {\tt findall/3},
    if {\tt Goal} is unsatisfiable, {\tt tfindall/3} succeeds binding
    {\tt List} to the empty list.

    Some of the differences between predicates {\tt findall/3} and
    {\tt tfindall/3} can be seen from the following example:

    {\footnotesize
    \begin{verbatim}
            | ?- [user].
            [Compiling user]
            :- table p/1.
            p(a).
            p(b).
            [user compiled, cpu time used: 0.639 seconds]
            [user loaded]

            yes
            | ?- p(X), findall(Y, p(Y), L).

            X = a
            Y = _922928
            L = [a];

            X = b
            Y = _922820
            L = [a,b];

            no
            | ?- abolish_all_tables.

            yes
            | ?- p(X), tfindall(Y, p(Y), L).

            X = b
            Y = _922820
            L = [b,a];

            X = a
            Y = _922820
            L = [b,a];

            no
    \end{verbatim}
    }

    Exceptions: Same as predicate {\tt findall/3} (see above).  Also:
    \begin{description}
    \item[{\tt table\_error}]
	Upon execution {\tt Goal} is not a subgoal of a tabled predicate.
    \end{description}

\ournewitem{tbagof(?X, +Goal, ?List) / tsetof(?X, +Goal, ?List)}{Tabling}
\index{\texttt{tsetof/3}}
\index{\texttt{tbagof/3}}

    {\em Note: {\tt tbagof/3} and {\tt tsetof/3} may be deprecated in
    current versions.  Please use the predicates described in
    Section~\ref{tabling_aggregate_predicates} if possible.}

The standard predicates {\tt tbagof/3} and {\tt tsetof/3} provide
tabled versions of {\tt bagof/3} and {\tt setof/3} in a similar manner
to the way in which {\tt tfindall/3} provides a tabled version of {\tt
findall/3}.

\ouritem{X \^\ Goal}
%\predindex{$\wedge/2$~(L)}\index{$\wedge/2$|bold}
    The system recognises this as meaning there exists an {\tt X} such
    that {\tt Goal} is true, and treats it as equivalent to {\tt call(Goal)}.
    The use of this explicit existential quantifier outside predicates
    {\tt setof/3} and {\tt bagof/3} constructs is superfluous.
\end{description}

\subsection{Tabling Aggregate Predicates}\label{tabling_aggregate_predicates}
%-------------------------------------------------------------------------
\index{aggregate predicates!tabling} \index{tabling!aggregate predicates}
%-------------------------------------------------------------------------

HiLog provides an elegant way to introduce aggregate operations into
XSB.  HiLog allows a user to define named (and parameterized) sets (or
bags).  For example, say we have a simple database-like predicate,
\verb|employee(Name,Dept,Sal)|, which contains a tuple for each
employee in our concern and contains the employee's name, department,
and salary.  From this predicate we can construct a set, or bag
really, that contains all the salaries of employees in the relation:
\begin{verbatim}
    :- hilog salaries.
    salaries(Sal) :- employee(_Name,_Dept,Sal).
\end{verbatim}
So \verb|salaries| is the name of a unary predicate that is true of
all salaries, or rather is the name of a {\em bag} of all salaries.
It is a bag since it may contain the same salary multiple times.
XSB provides a predicate \verb|bagSum| which can be used to
sum up the elements in a named bag.  So given the definition of the
HiLog predicate \verb|salaries/1| above, we can get the sum of all the
salaries with:
\begin{verbatim}
    :- bagSum(salaries,TotalSals).
\end{verbatim}
The first argument to \verb|bagSum| is the name of a bag, and the
second is bound to the sum of the elements in the bag.

We can also do a ``group by'' to get total salaries within departments
as follows.  We define a parameterized predicate, \verb|sals(Dept)|,
to be the bag of salaries of employees in department \verb|Dept|, as
follows:
\begin{verbatim}
    sals(Dept)(Sal) :- employee(_Name,Dept,Sal).
\end{verbatim}
This rule says that \verb|Sal| is in the bag named \verb|sals(Dept)|
if there is an employee with some name who works in department
\verb|Dept| and has salary \verb|Sal|.

Now with this definition, we can define a predicate,
\verb|deptPayroll/2|, that associates with each department the sum of
all the salaries of employees in that department:
\begin{verbatim}
    deptPayroll(Dept,Payroll) :- bagSum(sals(Dept),Payroll).
\end{verbatim}

XSB provides analogous aggregate operators, described below, to
compute the minimum, maximum, count, and average, of a bag,
respectively.  These predicates are all defined using a more basic
predicate \verb|bagReduce/4|.

\begin{description}

\ournewitem{bagReduce(?SetPred,?Arg,+Op,+Id)}{HiLog,Tabling}
\ournewitem{filterReduce(?SetPred,?Arg,+Op,+Id)}{Tabling}
\index{\texttt{bagReduce/4}} 
\index{\texttt{filterReduce/4}} 
{\tt SetPred} must be a HiLog set specification, i.e., a unary HiLog
predicate.  {\tt Op} must be a Hilog operation, i.e., a 3-ary HiLog
predicate that defines an associative operator.  The predicate must
define a binary function in which the first two arguments determine
the third.  {\tt Id} must be the identity of the operator.  {\tt
bagReduce} returns with {\tt Arg} bound to the ``reduce'' of the
elements of the bag determined by {\tt SetPred} under the operation
{\tt Op}.  I.e., {\tt Arg} becomes the result of applying the operator
to all the elements in the bag that unify with {\tt SetPred}.  See the
{\tt bagSum} operator below to see an example of {\tt bagReduce}'s
use.

{\tt filterReduce/4} acts as {\tt bagReduce/4} with two differences.
First, it does not depend on HiLog, so that {\tt filterReduce/4} will
be more robust especially when XSB's module system is used.  In
addition, {\tt filterReduce/4} aggregates solutions to {\tt Pred}
using a variance rather than unification.  An example of the use of
{\tt filterReduce/4} is given in Chapter \ref{chap:TablingOverview}.

\ournewitem{bagPO(?SetPred,?Arg,+Order)}{HiLog,Tabling}
\ournewitem{filterPO(?SetPred,?Arg,+Order)}{Tabling}
\index{\texttt{bagPO/3}}
\index{\texttt{filterPO/3}}
    {\tt SetPred} must be a HiLog set specification, i.e., a unary
    HiLog predicate.  {\tt Order} must be a binary Hilog relation that
    defines a partial order.  {\tt bagPO} returns nondeterministically
    with {\tt Arg} bound to the maximal elements, under {\tt Order}, of
    the bag {\tt SetPred}.  {\tt bagPO/3} can be used with {\tt Order}
    being subsumption to reduce a set of answers and keep only the most
    general answers.

    See the {\tt bagMax} operator below to see an example of {\tt
    bagPO}'s use.

{\tt filterPO/3} acts as {\tt bagPO/3} with the single difference that
it does not depend on HiLog, so that {\tt filterPO/3} will be more
robust especially when XSB's module system is used.

\ournewitem{filterPO(\#Pred,+Order)}{Tabling} 
\index{\texttt{filterPO/2}} 

{\tt filterPO(\#Pred,+Order)} succeds only for a solution $Pred\theta$
of {\tt Pred} for which there is no solution $Pred\eta$ to {\tt Pred}
such that {\tt Order($Pred\eta$,$Pred\theta$)}.

Example:

For the following program
     \begin{center}
     {\tt
     \begin{tabular}{l}
          :- table p/2.	\\
          b(1,2).       \\
          p(1,3).       \\
          b(1,1).       \\
\\
	  prefer(b(X,X),b(X,Y)):- X \== Y. 
     \end{tabular}
     }
     \end{center}
the query 
\begin{center}
{\tt ?- filterPO(b(X,Y)}
\end{center}
will succeed only with the binding {\em X = 1,Y = 1}.

\ournewitem{bagMax(?SetPred,?Arg)}{HiLog,Tabling}
\index{\texttt{bagMax/2}}
    {\tt SetPred} must be a HiLog set specification, i.e., a unary
    HiLog predicate.  {\tt bagMax} returns with {\tt Arg} bound to the
    maximum element (under the Prolog term ordering) of the set {\tt
    SetPred}.  To use this predicate, it must be imported from aggregs,
    and you must give the following definitions in the main module {\tt
    usermod}:
\begin{verbatim}
:- hilog maximum.
maximum(X,Y,Z) :- X @< Y -> Z=Y ; Z=X.
\end{verbatim}
    (These decarations are necessary because of a current limitation in
    how HiLog predicates can be used.  This requirement will be lifted in
    a future release.)  With this definition, {\tt bagMax/2} can be (and
    is) defined as follows:
\begin{verbatim}
bagMax(Call,Var) :- bagReduce(Call,Var,maximum,_).
\end{verbatim}
    (Where variables are minimal in the term ordering.)

Another possible definition of {\tt bagMax/2} would be:
\begin{verbatim}
:- hilog lt.
lt(X,Y) :- X @< Y.

bagMax(Call,Var) :- bagPO(Call,Var,lt).
\end{verbatim}
This definition would work, but it is slightly less efficient than the
previous definition since it is known that {\tt bagMax} is
deterministic.

\ournewitem{bagMin(?SetPred,?Arg)}{HiLog,Tabling}
\index{\texttt{bagMin/2}}

    {\tt SetPred} must be a HiLog set specification, i.e., a unary
    HiLog predicate.  {\tt bagMin} returns with {\tt Arg} bound to the
    minimum element (under the Prolog term ordering) of the set {\tt
    SetPred}.  To use this predicate, it must be imported from aggregs,
    and you must give the following definitions in the main module {\tt
    usermod}:
\begin{verbatim}
:- hilog minimum.  
minimum(X,Y,Z) :- X @< Y -> Z=X ; Z=Y.
\end{verbatim}
    (These decarations are necessary because of a current limitation in
    how HiLog predicates can be used.  This requirement will be lifted in
    a future release.)  With this definition, {\tt bagMin/2} can be (and
    is) defined as:
\begin{verbatim}
bagMin(Call,Var) :- bagReduce(Call,Var,minimum,zz(zz)).
\end{verbatim}
    (where structures are the largest elements in the term ordering.)

\ournewitem{bagSum(?SetPred,?Arg)}{HiLog,Tabling}
\index{\texttt{bagSum/2}}
    {\tt SetPred} must be a HiLog set specification, i.e., a unary
    HiLog predicate.  {\tt bagSum} returns with {\tt Arg} bound to the sum
    of the elements of the set {\tt SetPred}.  To use this predicate, it
    must be imported from aggregs, and you must give the following
    definitions in the main module {\tt usermod}:
\begin{verbatim}
:- hilog sum.
sum(X,Y,Z) :- Z is X+Y.
\end{verbatim}
    (These decarations are necessary because of a current limitation in
    how HiLog predicates can be used.  This requirement will be lifted in
    a future release.)  With this definition, {\tt bagSum/2} can be (and
    is) defined as:
\begin{verbatim}
bagSum(Call,Var) :- bagReduce(Call,Var,sum,0).
\end{verbatim}

\ouritem{bagCount(?SetPred,?Arg)}{HiLog,Tabling}
\index{\texttt{bagCount/2}}
    {\tt SetPred} must be a HiLog set specification, i.e., a unary
    HiLog predicate.  {\tt bagCount} returns with {\tt Arg} bound to the
    count (i.e., number) of elements of the set {\tt SetPred}.  To use
    this predicate, it must be imported from aggregs, and you must give
    the following definitions in the main module {\tt usermod}:
\begin{verbatim}
:- hilog successor.
successor(X,_Y,Z) :- Z is X+1.
\end{verbatim}
    (These decarations are necessary because of a current limitation in
    how HiLog predicates can be used.  This requirement will be lifted in
    a future release.)  With this definition, {\tt bagCount/2} can be (and
    is) defined as:
\begin{verbatim}
bagCount(Call,Var) :- bagReduce(Call,Var,successor,0).
\end{verbatim}

\ournewitem{bagAvg(?SetPred,?Arg)}{HiLog,Tabling}
\index{\texttt{bagAvg/2}}
    {\tt SetPred} must be a HiLog set specification, i.e., a unary
    HiLog predicate.  {\tt bagAvg} returns with {\tt Arg} bound to the
    average (i.e., mean) of elements of the set {\tt SetPred}.  To use
    this predicate, it must be imported from aggregs, and you must give
    the following definitions in the main module {\tt usermod}:
\begin{verbatim}
:- hilog sumcount.
sumcount([S|C],X,[S1|C1]) :- S1 is S+X, C1 is C+1.
\end{verbatim}
    (These decarations are necessary because of a current limitation in
    how HiLog predicates can be used.  This requirement will be lifted in
    a future release.)  With this definition, {\tt bagAvg/2} can be (and
    is) defined as:
\begin{verbatim}
bagAvg(Call,Avg) :- 
    bagReduce(Call,[Sum|Count],sumcount,[0|0]),
    Avg is Sum/Count.
\end{verbatim}

\end{description}

%-------------------------------------------------------------------------
\section{Comparison} \label{Comparison}
\index{comparison of terms}
\index{terms!comparison of}
The evaluable predicates described in this section are meta-logical.
They are used to compare and order terms, rather than to evaluate or 
process them.  They treat uninstantiated variables as objects with values
which may be compared, and they never instantiate those variables.
Each of these predicates simply succeeds or fails; there is no 
side-effect, substitution or error condition associated with them.
The predicates described in this section should {\em not} be used when 
what the user really wants is arithmetic comparison predicates
%(see section~\ref{Arithmetic}) 
or unification predicates (see section~\ref{Convenience}).

The predicates described take into account a standard total ordering 
of terms, which has as follows:
	\[		variables 
	   \ {\tt @<} \ floating \ point \ numbers
	   \ {\tt @<} \ integers 
	   \ {\tt @<} \ atoms 
	   \ {\tt @<} \ compound \ terms \]
Within each one of the categories, the ordering is as follows:
\begin{itemize}
\item	variables are put in a standard order (roughly, the oldest first
	--- the order is {\em not\/} related to the names of
	variables).  Also, note that two anonymous variables are not
	identical terms.  As with most WAM-based Prologs, the order of
	variables may change as variables become bound to one another.
	If the order is expected to be invariant across variable
	bindings, other mechanisms, such as attributed variables,
	should be used.
%	Unfortunately in the current implementation of our system (\version) 
%	variables ``tend to move'' rather quickly as a result of unification,
%	and thus the ordering may not continue to hold if the variables get
%	unified to some other variables.  We intend to ameliorate this bug in
%	future releases.
\item	floating point numbers and integers are put in numeric order, 
	from $-\infty$ to $+\infty$.  Note that a floating point number is
	always less than an integer, regardless of their numerical values.
\item	atoms are put in alphabetical (i.e.\ ASCII) order;
\item	compound terms are ordered first by arity, then by the name of their
	principal functor and then by their arguments (in a left-to-right 
	order).
\item	lists are compared as ordinary compound terms having arity 2 and 
	functor {\tt '.'}.
\end{itemize}
For example, here is a list of terms sorted in increasing standard order:
\begin{center}
	{\tt [ X, 3.14, -9, fie, foe, fum(X), [X], X = Y, fie(0,2), fie(1,1) ]}
\end{center}
The basic predicates for comparison of arbitrary terms are:
\begin{description}
\ouritem{T1 == T2}
\index{\texttt{==/2}}
    Tests if the terms currently instantiating {\tt T1} and {\tt T2}
    are literally identical (in particular, variables in equivalent positions
    in the two terms must be identical).
    For example, the question:

% JF
%    \demo{\verb+|+ ?- X == Y.}
    \demo{$|$ ?- X == Y.}

    \noindent
    fails (answers no) because {\tt X} and {\tt Y} are distinct variables.
    However, the question

% JF
%    \demo{\verb+|+ ?- X = Y, X == Y.}
    \demo{$|$ ?- X = Y, X == Y.}

    \noindent
    succeeds because the first goal unifies the two variables 
    (see section~\ref{Convenience}).

\ouritem{T1 {$\backslash$==} T2}
\index{\texttt{$\backslash$==/2}}
    Tests if the terms currently instantiating {\tt T1} and {\tt T2}
    are not literally identical.

% TLS: for some reason these @ comparison operators don't work.
\ouritem{T1 @$<$ T2}
\index{@$</2$}
    Succeeds if term {\tt T1} is before term {\tt T2} in the standard order.

\ouritem{T1 @$>$ T2}
\index{@$>/2$}
    Succeeds if term {\tt T1} is after term {\tt T2} in the standard order.

\ouritem{T1 @$=<$ T2}
\index{@$=</2$}
    Succeeds if term {\tt T1} is not after term {\tt T2} in the standard order.

\ouritem{T1 @$>=$ T2}
\index{@$>=/2$}
    Succeeds if term {\tt T1} is not before term {\tt T2} in the standard order.

\ouritem{T1 @$=$ T2}
\index{@$=/2$}
    Succeeds if {\tt T1} and {\tt T2} are identical variables, or if
    the main strucure symbols of {\tt T1} and {\tt T2} are identical.

\end{description}

Some further predicates involving comparison of terms are:

\begin{description}
\ouritem{compare(?Op, +T1, +T2)}
\index{\texttt{compare/3}}
    Succeeds if the result of comparing terms {\tt T1} and {\tt T2} 
    is {\tt Op}, where the possible values for {\tt Op} are:
    \begin{description}
    \item[`='] if {\tt T1} is identical to {\tt T2},
    \item[`$<$'] if {\tt T1} is before {\tt T2} in the standard order,
    \item[`$>$'] if {\tt T1} is after {\tt T2} in the standard order.
    \end{description}
    Thus {\tt compare(=, T1, T2)} is equivalent to {\tt T1==T2}.
    Predicate {\tt compare/3} has no associated error conditions.

\ouritem{sort(+L1, ?L2)} 
\index{\texttt{sort/2}}
    The elements of the list {\tt L1} are sorted into the standard order,
    and any identical (i.e.\ `==') elements are merged, yielding the 
    list~{\tt L2}.  The time to perform the sorting is $O(n log n)$ where 
    $n$ is the length of list {\tt L1}.  

    Examples:
    {\footnotesize
     \begin{verbatim}
                | ?- sort([3.14,X,a(X),a,2,a,X,a], L).

                L = [X,3.14,2,a,a(X)];

                no
     \end{verbatim}}
    Exceptions:
    \begin{description}
    \item[{\tt instantiation\_error}]
	Argument 1 of {\tt sort/2} is a variable or is not a proper list.
    \end{description}


\ouritem{keysort(+L1, ?L2)}
\index{\texttt{keysort/2}}
    The list {\tt L1} must consist of elements of the form \verb'Key-Value'.
    These elements are sorted into order according to the value of {\tt Key},
    yielding the list~{\tt L2}.  The elements of list {\tt L1} are scanned
    from left to right.  Unlike {\tt sort/2}, in {\tt keysort/2} no
    merging of multiple occurring elements takes place.  The time to perform
    the sorting is $O(n log n)$ where $n$ is the length of list {\tt L1}.  
    Note that the elements of {\tt L1} are sorted only according to the
    value of {\tt Key}, not according to the value of {\tt Value}.  The 
    sorting of elements in {\tt L1} is not guaranteed to be stable.

    Examples:
    {\footnotesize
     \begin{verbatim}
                | ?- keysort([3-a,1-b,2-c,1-a,3-a], L).

                L = [1-b,1-a,2-c,3-a,3-a];

                no \end{verbatim}}
     Exceptions: 
\begin{description} 
\item[{\tt instantiation\_error}]
     {\tt L1} {\tt keysort/2} is a variable or is not a proper
     list.  
\item[{\tt domain\_error(key\_value\_pair,Element)}] {\tt
     L1} contains an element {\tt Element} that is not of the
     form \verb'Key-Value'.  
\end{description}

\ournewitem{parsort(+L1, +SortSpec, +ElimDupl, ?L2)}{machine}
\index{\texttt{parsort/4}}

    {\tt parsort/4} is a very general sorting routine.  The list {\tt
    L1} may consist of elements of any form.  {\tt SortSpec} is the
    atom {\tt asc}, the atom {\tt desc}, or a list of terms of the
    form {\tt asc(I)} or {\tt desc(I)} where {\tt I} is an integer
    indicating a sort argument position.  The elements of list {\tt
    L1} are sorted into order according to the sort specification.
    {\tt asc} indicates ascending order based on the entire term; {\tt
    desc} indicates descending order.  For a sort specification that
    is a list, the individual elements indicate subfields of the
    source terms on which to sort.  For example, a specification of
    {\tt [asc(1)]} sorts the list in ascending order on the first
    subfields of the terms in the list.  {\tt [desc(1),asc(2)]} sorts
    into descending order on the first subfield and within equal first
    subfields into ascending order on the second subfield.  The order
    is determined by the standard predicate {\tt compare}.  If {\tt
    ElimDupl} is nonzero, merging of multiple occurring elements takes
    place (i.e., duplicate (whole) terms are eliminated in the
    output).  If {\tt ElimDupl} is zero, then no merging takes place.
    A {\tt SortSpec} of {\tt []} is equivalent to ``asc''.  The time
    to perform the sorting is $O(n log n)$ where $n$ is the length of
    list {\tt L1}.  The sorting of elements in {\tt L1} is not
    guaranteed to be stable. {\tt parsort/4} must be imported from
    module {\tt machine}.

    Examples:
    {\footnotesize
     \begin{verbatim}
            | ?- parsort([f(3,1),f(3,2),f(2,1),f(2,2),f(1,3),f(1,4),f(3,1)],
                 [asc(1),desc(2)],1,L). 

            L = [f(1,4),f(1,3),f(2,2),f(2,1),f(3,2),f(3,1)];

            no \end{verbatim}}

{\bf Error Cases:}
\begin{description} 
\item[{\tt instantiation\_error}]
     {\tt L1} is a variable or not a proper list.  
% TLS: not sure how to make sense of this...
%\item[{\tt type\_error}] The elements of {\tt L1} are not terms with at
%     least as many arguments as required by {\tt SortSpec}, or {\tt
%     SortSpec} is not of an allowed form.  
\end{description}
\end{description}


%-------------------------------------------------------------------------

\section{Meta-Predicates} \label{meta_predicates}
\begin{description}
\ouritem{call(\#X)}\index{\texttt{call/1}} 
%\predindex{call/1~(P)}
    If {\tt X} is a nonvariable term in the program text, then it is 
    executed exactly as if {\tt X} appeared in the program text instead 
    of {\tt call(X)},
    e.g.
    \begin{center}
        {\tt $\ldots$, p(a), call( (q(X), r(Y)) ), s(X), $\ldots$}
    \end{center}
    is equivalent to
    \begin{center}
        {\tt $\ldots$, p(a), q(X), r(Y), s(X), $\ldots$}
    \end{center}
    However, if {\tt X} is a variable in the program text,
    then if at runtime {\tt X} is instantiated to a term which 
    would be acceptable as the body of a clause, the goal 
    {\tt call(X)} is executed as if that
    term appeared textually in place of the {\tt call(X)},
    {\em except that} any cut (`!')\index{\texttt{"!/0}}\index{cut}
    occurring in {\tt X} will remove only those choice points in~{\tt X}.
    If~{\tt X} is not instantiated as described above,
    an error message is printed and {\tt call/1} fails.

    Exceptions:
    \begin{description}
    \item[{\tt instantiation\_error}]
	Argument 1 of {\tt call/1} is not instantiated.
     \item[{\tt type\_error}]
	Argument 1 of {\tt call/1} is not a callable term.
    \end{description}

\ouritem{\#X}					      
    (where {\tt X} is a variable) executes exactly the same as 
    {\tt call(X)}.
    However, the explicit use of {\tt call/1} is considered better
    programming practice.  The use of a top level variable subgoal
    elicits a warning from the compiler.

\ouritem{once(\#X)}
\index{\texttt{once/1}} 
    {\tt once/1} is defined as {\tt once(X):- call(X),!.}  {\tt
    once/1} should be used with care in tabled programs.  The compiler
    can not determine whether a tabled predicate is called in the
    scope of {\tt once/1}, and such a call may lead to runtime errors.
    If a tabled predicate may occur in the scope of {\tt once/1}, use
    {\tt table\_once/1} instead.

    Exceptions: The same as {\tt call/1}.

\ouritem{table\_once(\#X)}
\index{\texttt{table\_once/1}} 
    {\tt table\_once/1} is a weaker form of {\tt once/1}, suitable for
    situations in which a single solution is desired for a
    subcomputation that may involve a call to a tabled predicate.  {\tt
    table\_once(?Pred)} succeeds only once even if there are many
    solutions to the subgoal {\tt Pred}.  However, it does not ``cut
    over'' the subcomputation started by the subgoal {\tt Pred},
    thereby ensuring the correct evaluation of tabled subgoals.

\end{description}


%%% Local Variables: 
%%% mode: latex
%%% TeX-master: "manual1"
%%% End: 
