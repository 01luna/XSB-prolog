\chapter{Embedding XSB in a Process}
%===================================
\label{ccallingxsb}

There are many situations in which it is desirable to use XSB as a
rule- or constraint- processing subcomponent of a larger system that
is written in another language.  Depending on the intended
architecture, it may be appropriate for XSB to reside in its own
process, separate from other components of an application, and
communicating through sockets, a database, or some other mechanism.
However it is often useful for XSB to reside in the same process as
other components.  To do this, one wants to be able to {\em call}\ XSB
from the host language, providing queries for XSB to evaluate.  An
interface for calling XSB from C is provided for this purpose and is
described in this chapter.  Based on this C interface, XSB can also be
called from Java either through a JNI or a socket-based interface, as
described in the documentation for InterProlog, available through {\tt
  xsb.sourceforge.net}.  To call XSB from Visual Basic, a DLL is
created as described in this chapter, and additional declarations must
be made in visual basic as described in the web page ``How to use XSB
DLL from Visual Basic'' \url{http://xsb.sourceforge.net/vbdll.html}.
In addition, the interface described in this chapter has also been
extended to allow XSB to be called from Delphi and Ruby.  However,
since all of these interfaces -- Java, Ruby, Delphi and Visual Basic
-- depend on XSB's C API, we refer in this chapter to C programs or
threads calling XSB, although each of the examples suitably modified
can be extended to other calling languages.

%New to Version 3.1 are extensions to the C API to allow multiple XSB
%threads to be called from multiple C threads~\footnote{XSB's threading
%  model is based on POSIX threads, which can be called in Windows
%  through a variety of POSIX APIs -- see Volume 1 chapter 8 {\em
%    Multi-threaded Programming in XSB}.}.
%In this Chapter, we provide an overview of XSB's C API, and then
%elaborate its use through a series of examples, beginning with a
%single XSB thread called by a single C thread, then showing how a C
%thread can interact with multiple XSB threads, and finally discuss how
%multiple XSB threads can interact with multiple POSIX threads.
%Finally, Section~\ref{sec:CAPI} describes each C function in the API.

%
\section{Calling XSB from C}

XSB provides several C functions (declared in {\tt
  \$XSBDIR/emu/cinterf.h} and defined in {\tt
  \$XSBDIR/emu/cinterf.c}), which can be called from C to interact
with XSB as a subroutine. These functions allow a C program to
interact with XSB in a number of ways.
\begin{itemize}
\item XSB may be initialized, using most of the parameters available
  from the command-line. 
%
\item XSB may then execute a series of {\em commands} or {\em
  queries}.  A command is a deterministic query which simply succeeds
  or fails without performing any unification on the query term.  On
  the other hand, a non-deterministic query can be evaluated so that
  its answer substitutions are retrieved one at a time, as they are
  produced, just as if XSB were called on a command line.  Alternately
  a non-deterministic query can be closed in the case where not every
  answer to the query is needed.  Only one query per thread can be
  active at a time.  I.e., an application must completely finish
  processing one query to a given thread $T$ (either by retrieving all
  the answers for it, or by issuing a call to {\tt
    xsb\_close\_query()}, before trying to evaluate another using $T$.
%
\item Finally, XSB can be closed, so that no more queries can be made
%  to any XSB threads.
  to XSB.
\end{itemize}

In general, while any functions in the C API to XSB can be intermixed,
the functions can be classified as belonging to three different
levels.
%
\begin{itemize}
\item {\em A VarString level} which uses an XSB-specific C-type
  definition for variable-length strings
  (Section~\ref{sec-varstring}), to return answers.
%
\item {\em A fixed-string level} provides routines that return answers
  in fixed-length strings.
%
\item {\em A register-oriented level} that requires users to set up
  queries by setting registers for XSB which are made globally
  available to calling functions.  The mechanisms for this resemble
  the lower-level C interface discussed in Chapter~\ref{foreign}.
%  This level of interface should only be used for the single-threaded
%  applications, as it is difficult to prevent race-conditions at this
%  level of interface when multiple C threads are used to call XSB.

\end{itemize}
%
The appropriate level to use depends on the nature of the calling
program, the speed desired, and the expertise of the programmer.  By
and large, functions in the {\tt VarString} level are the the easiest
and safest to use, but they depend on a C type definition that may not
be available to all calling programs (e.g. it may be difficult to use
if the calling program is not directly based on C, such as Visual
Basic or Delphi).  For such applications functions from the
fixed-string level would need to be used instead.  In general, most
applications should use either functions from the {\tt VarString} or
the fixed-string level, rather than the register-oriented level.  This
latter level should only be used by programmers who are willing to
work at a low interface level, when the utmost speed is needed by an
application.
%, and when multiple threads do not need to interact with XSB.


\section{Examples of Calling XSB}

We introduce a series of examples of how XSB would be called using the
string-level interfaces.  Simple examples of the register-level
interface are given in the {\tt XSB/examples/c\_calling\_XSB}
subdirectory, in files {\tt cmain.c}, {\tt cmain2.c}, {\tt ctest.P},
and {\tt Makefile}, but are not discussed in this section.

We structure out discussion by first showing how to construct a C
program to call the single-threaded engine alone in
Section~\ref{sec:CXSBSeq}.
%This example is mostly pedagogic: with a
%small amount of extra coding a C program can be constructed to call
%both the single- and the multi-threaded engine, and these extensions
%are discussed in Section~\ref{sec:CXSB}.
%Next, we show how to a C
%program can call and manage multiple XSB threads in Section~\ref{sec:CXSBMT}.
%Finally, we show how multiple XSB threads can interact with multiple C
%threads in Section~\ref{sec:CXSBMT}.

%\subsection{The XSB API for the Sequential Engine Only} \label{sec:CXSBSeq}
\subsection{The XSB API for the Sequential Engine Only} \label{sec:CXSBSeq}

We start with a simple program shown, in Figure~\ref{fig:varstringex},
that will call the following XSB predicate
%
\begin{small}
\begin{verbatim}
  p(a,b,c).
  p(1,2,3).
  p([1,2],[3,4],[5,6]).
  p(A,B,A).

r(c,b,a).
r(3,2,1).
r([5,6],[3,4],[1,2]).
r(_A,B,B).

\end{verbatim}
\end{small}
%
and backtrack through unifying answers (cf. {\tt
  \$XSBDIR/examples/c\_calling\_xsb/edb.P}).  .  This example will
only compile properly if the sequential engine is used, and its style
is {\tt not} recommended: it will be shown in Section~\ref{sec:CXSB}
how to extend the style.
%
\begin{figure}[hbtp]
\begin{small}
\begin{verbatim}
#include <stdio.h>
#include <string.h>

/* cinterf.h is necessary for the XSB API, as well as the path manipulation routines*/
#include "cinterf.h"

extern char *xsb_executable_full_path(char *);
extern char *strip_names_from_path(char*, int);

int main(int argc, char *argv[]) { 

  char init_string[1024];
  int rc;
  XSB_StrDefine(return_string);

  /* xsb_init_string() relies on the calling program to pass the absolute or relative
     path name of the XSB installation directory. We assume that the current
     program is sitting in the directory ../examples/c_calling_xsb/
     To get the installation directory, we strip 3 file names from the path. */

  strcpy(init_string,strip_names_from_path(xsb_executable_full_path(argv[0]),3));

  if (xsb_init_string(init_string) == XSB_ERROR) {
    fprintf(stderr,"++initializing XSB: %s/%s\n",xsb_get_init_error_type(),
            xsb_get_init_error_message());
    exit(XSB_ERROR);
  }

  /* Create command to consult a file: edb.P, and send it. */
  if (xsb_command_string("consult('edb.P').") == XSB_ERROR)
    fprintf(stderr,"++Error consulting edb.P: %s/%s\n",xsb_get_error_type(),xsb_get_error_message());

  rc = xsb_query_string_string("p(X,Y,Z).",&return_string,"|");
  while (rc == XSB_SUCCESS) {
    printf("Return %s\n",(return_string.string));
    rc = xsb_next_string(&return_string,"|");
  }
 
 if (rc == XSB_ERROR) 
    fprintf(stderr,"++Query Error: %s/%s\n"xsb_get_error_type(),xsb_get_error_message());

  xsb_close();     
}
\end{verbatim}
\end{small}
\caption{Calling the Sequential Engine Using the {\tt VarString} Interface} \label{fig:varstringex}
\end{figure}

We discuss the program in Figure~\ref{fig:varstringex} in detail.
%This program, slightly modified so that it compiles with the
%multi-threaded engine is in {\tt
%  \$XSBDIR/examples/c\_calling\_xsb/cvartest.c}.
An executable for this program can be make most easily by calling {\tt
  \$XSBDIR/examples/c\_calling\_xsb/make.P}, which makes the
executable {\tt cvstest}.

The program begins by including some standard C headers: note that
{\tt string.h} is needed for string manipulation routines such as {\tt
  strcpy}.  In addition, the XSB library header {\tt cinterf.h} is
necessary for the XSB C API.  Since the program in
Figure~\ref{fig:varstringex} uses functions in the {\tt VarString}
interface, within {\tt main()} the routine {\tt
  XSB\_StrDefine(return\_string)} declares and initializes a structure
of type {\tt VarString}, named {\tt return\_string}.

The next order of business is to initialize XSB.  In order to do this,
{\tt xsb\_init\_string()} needs to know the installation directory for
XSB, which must be passed as part of the initialization string.  In
Figure~\ref{fig:varstringex} this is done by manipulating the path of
the executable ({\tt cvstest}) that calls XSB.  In fact any other
approach would also work as long as the XSB installation directory
were passed.  Within the initialization string, other command line
arguments can be passed to XSB if desired with the following
exceptions: the arguments {\tt -B} (boot module), {\tt -D} (command
loop driver), {\tt -i} (interpreter) and {\tt -d} (disassembler)
cannot be used when calling XSB from a foreign language~\footnote{In
  previous versions of XSB, initialization from the C level required a
  {\tt -n} option to be passed.  This is no longer required.}.  As a
final point on initialization, note that the function {\tt
  xsb\_init()} can also be used to initialize XSB based on an argument
vector and count (see Section~\ref{sec:CAPI}).

Note that the calling program checks for any errors returned by {\tt
  xsb\_init\_string()} and other API commands.  In general, {\tt
  xsb\_init\_string()} may throw an error if the XSB's installation
directory has become corrupted, or for similar reasons.  This
mechanism for error handling is different than that used if XSB is
called in its usual stand-alone mode, in which case such an error
would cause XSB to exit).  An error returned by XSB's API are similar
to an error ball described in Volume 1 {\em Exception Handling} in
that it has both a {\em type} and a {\em message}.  For normal Prolog
exceptions, XSB's API will throw the same kinds of errors as XSB
called in a stand-alone (or server) mode, i.e. instantiation errors,
type errors, etc.  However XSB's API adds two new error types:
%
\begin{itemize}
\item {\tt init\_error} is used as the type of an error discovered
  upon initialization of XSB, before query and command processing has
  begun.  If an {\tt init\_error} is raised, XSB has not been properly
  initialized and will not run.
%
\item {\tt unrecoverable\_error} is used to indicate that XSB has
  encountered an error, (such as a memory allocation error), during
  command or query processing from which it cannot recover.  Such an
  error would cause XSB to immediately exit if it were called in a
  stand-alone mode.  In general the calling program should handle
  unrecoverable errors as fatal since there is a good chance that the
  error conditions will affect the calling program as well as XSB.
\end{itemize}
%
Errors raised by {\tt xsb\_init\_string()} usually have type {\tt
  init\_type}.  

 and a string pointer to the associated message can be
found by the function {\tt xsb\_get\_init\_error\_message()}.  

As can be seen from the example, handling errors from commands is done
in manner similar to that of initialization.  For non-initialization
errors, a string pointer to the type can be obtained by {\tt
  xsb\_get\_error\_type()}, while a string pointer to the message can
be obtained by {\tt xsb\_get\_error\_message()}.  
%The reason that
%initialization errors are handled separately is discussed in
%Section~\ref{}.  

Next in Figure~\ref{fig:varstringex} the file {\tt edb.P} is consulted
(containing the {\tt p/3} and {\tt r/3} predicates shown above).
Note, that the argument to {\tt xsb\_command\_string} must be a
syntactically valid Prolog term ending with a period, otherwise a
syntax error will be thrown, which may be displayed through {\tt
  xsb\_get\_error\_type()} and {\tt
  xsb\_get\_error\_message()}~\footnote{Most XSB errors are handled in
  this manner when XSB is called through its API.  A few errors will
  print directly to {\tt stderr} and some XSB warnings will print to
  {\tt stdwarn} which upon startup is dup-ed to {\tt stderr}.}.

Queries to XSB are a little more complicated than commands.  Since a
query may return multiple solutions, a query should usually be called
from inside a loop.  In Figure~\ref{fig:varstringex}, the query is
opened with {\tt xsb\_query\_string()}.  If the query has at least one
answer, {\tt xsb\_query\_string()} will return {\tt XSB\_SUCCESS}; if
the query fails, it will return {\tt XSB\_FAILURE}, and if there is an
exception it will return {\tt XSB\_ERROR} as usual.  Any answer will
be returned as a string in the {\tt VarString} {\tt return\_string},
and each argument of the query will be separated by the character {\tt
  |}.  Thus, in our example, the first answer will write the string
%
\begin{small}
\begin{verbatim}
a|b|c
\end{verbatim}
\end{small}
% 
Once a query has been opened, subsequent answers can be obtained via
{\tt xsb\_next\_string()}.  These answers are written to {\tt
  return\_string} in the same manner as {\tt
  xsb\_query\_string\_string()}.
%
\begin{small}
\begin{verbatim}
1|2|3
[1,2]|[3,4]|[5,6]
_h102|_h116|_h102
\end{verbatim}
\end{small}
%
A query is automatically closed when no more answers can be derived
from it.  Alternately, a query that may have answers remaining can be
closed using the command {\tt xsb\_close\_query()}.  If the calling
application will need to pass more queries or commands to XSB nothing
need be done at this point: a new queries or commands can be invoked
using one of the functions just discussed.  However if the calling
process is finished with XSB and will never need it again during the
life of the process, it can call {\tt xsb\_close()}.

\subsubsection{An Example using Fixed Strings}

\begin{figure}[hbtp]
\begin{small}
\begin{verbatim}
  int retsize = 15;
  char *return_string;
  int anslen;

  return_string = malloc(retsize);

  rc = xsb_query_string_string_b(CTXTc "p(X,Y,Z).",return_string,retsize,&anslen,"|");

  while (rc == XSB_SUCCESS || rc == XSB_OVERFLOW) {
  
    if (rc == XSB_OVERFLOW) {
      return_string = (char *) realloc(return_string,anslen);
      return_size = anslen;
      rc = xsb_get_last_answer_string(CTXTc return_string,retsize,&anslen);
    }    

    printf("Return %s %d\n",return_string,anslen);
    rc = xsb_next_string_b(CTXTc return_string,15,&anslen,"|");
  }
\end{verbatim}
\end{small}
\caption{Calling XSB using the Fixed String Interface} \label{fig:fixedstringex}
\end{figure}

Figure~\ref{fig:fixedstringex} shows a fragment of code indicating how
the previous example would be modified if the fixed-string interface
were used.  Note that {\tt return\_string} now becomes a pointer to
explicitly malloc-ed memory.  To open the query {\tt p(X,Y,Z)} the
function {\tt xsb\_query\_string\_string\_b()} is called, with the
{\tt \_b} indicating that a fixed buffer is being used rather than a
{\tt VarString}.  The call is similar to {\tt
  xsb\_query\_string\_string()}, except that the length {\tt anslen}
of the buffer pointed to by {\tt return\_string} is now also required.
If the answer to be returned (including separators) is longer than
{\tt anslen}, {\tt xsb\_query\_string\_string\_b()} will return {\tt
  XSB\_OVERFLOW}.  If this happens, a new answer buffer can be used
(here the old one is realloc-ed) and the answer retrieved via {\tt
  xsb\_get\_last\_answer\_string}.  Similarly, further answers are
obtained via {\tt xsb\_next\_string\_b()} whose length must be
checked.  Thus the only difference between the fixed-string level and
the {\tt VarString} level is that the length of each answer should be
checked and {\tt xsb\_get\_last\_answer\_string()} called if
necessary.

%\subsection{The General XSB API} \label{sec:CXSB}
%
%The previous section showed how to use the XSB API with both the {\tt
%  VarString} type and without.
%, but did not consider the multi-threaded
%engine.  In fact, there are different ways to use XSB's
%multi-threading that can have advantages for various situations.  In
%the first mode, threads are managed from Prolog, with a single XSB
%thread called from the API; that XSB thread can then create another
%XSB thread that does work, and the first thread can return almost
%immediately to handle more requests from the API's caller.  A second
%model allows the caller to manipulate a pool of several XSB threads,
%so that different XSB threads may be called from different threads
%over the API.  In this model each C, Java, Ruby, or other thread could
%a number of different Prolog threads.  In this section we sketch how
%to use the API to illustrate the first model, and sketch the second
%model in the next section.

%\begin{figure}[hbtp]
%\begin{small}
%\begin{verbatim}
%.....
%
%/* context.h is necessary for the type of a thread context. */
%#include "context.h"
%
%int main(int argc, char *argv[])
%{ 
%
%  char init_string[MAXPATHLEN];
%  int rc;
%  XSB_StrDefine(return_string);
%
%  strcpy(init_string,strip_names_from_path(xsb_executable_full_path(argv[0]),3));
%  if (xsb_init_string(init_string) == XSB_ERROR) {
%    fprintf(stderr,"++initializing XSB: %s/%s\n",xsb_get_init_error_type(),
%            xsb_get_init_error_message());
%    exit(XSB_ERROR);
%  }
%
%#ifdef MULTI_THREAD
%  th_context *th = xsb_get_main_thread();
%#endif
%
%  /* Create command to consult a file: edb.P, and send it. */
%  if (xsb_command_string(CTXTc "consult('edb.P').") == XSB_ERROR)
%    fprintf(stderr,"++Error consulting edb.P: %s/%s\n",xsb_get_error_type(CTXT),
%            xsb_get_error_message(CTXT));
%
%  rc = xsb_query_string_string(CTXTc "p(X,Y,Z).",&return_string,"|");
%  while (rc == XSB_SUCCESS) {
%    printf("Return %s\n",(return_string.string));
%    rc = xsb_next_string(CTXTc &return_string,"|");
%  }
% 
% if (rc == XSB_ERROR) 
%    fprintf(stderr,"++Query Error: %s/%s\n",xsb_get_error_type(CTXT),xsb_get_error_message(CTXT));
%
%  xsb_close();     
%}
%\end{verbatim}
%\end{small}
%\caption{Calling the Single- or Multi-Threaded Engine Using the {\tt VarString} Interface} 
%\label{fig:varstringex1}
%\end{figure}

%Figure~\ref{fig:varstringex1} shows how relevant portions of the
%previous {\tt VarString} example can be adapted to use the
%multi-threaded engine.  The main change is that a new variable is
%introduced on the C side that points to the context of the main
%thread.  As pointed out in Chapter~\ref{foreign}, each thread in the
%multi-threaded engine has a {\em context} in which is kept much of its
%thread-specific data (excluding tables and dynamic code).  Of the
%threads running in the multi-threaded engine the thread created upon
%the call to {\tt xsb\_init()} is designated as the {\tt main thread},
%and is closed only upon calling {\tt xsb\_close()}.
%
%Within the multi-threaded engine, a call to an API function such as
%{\tt xsb\_query\_string\_string()} is actually a call to a specific
%thread to do some work (using a thread context pointer).  Accordingly,
%since any errors produced will be specific to a given thread, all
%calls to error reporting functions are also thread-specific.  If no
%specific thread is needed, it may be best just to use the main thread,
%which is what is done in Figure~\ref{fig:varstringex1}.  The thread
%context pointer {\tt th} is initialized to the main thread using the
%API macro {\tt xsb\_get\_main\_thread()}.  Afterwards, this pointer is
%passed into the various interface functions by making use of XSB
%macros defined in {\tt context.h} In the multi-threaded engine, these
%macros are defined as
%%
%\begin{verbatim}
%#define CTXT th
%#define CTXTc th,
%\end{verbatim}
%%
%while in the single-threaded engine they are defined as empty strings,
%as is {\tt xsb\_get\_main\_thread()}.  As a result the code in
%Figure~\ref{fig:varstringex1} will compile and run properly both for
%the single-threaded and the multi-threaded engines.
%
%At this stage, suppose one wanted a new thread to execute a specific
%command, say {\tt do\_foo}.  In this case, a C call such as %
%%
%\begin{verbatim}
%xsb_query_string_string(CTXTc "thread_create(do_foo,Id).",&return_string,"|")
%\end{verbatim}
%creates a thread to execute the command, and returns the thread id of
%the newly created thread in {\tt return\_string}.  The behavior of
%this newly created thread is exactly the same as if it were created
%from the XSB command line: in particular the newly created thread will
%automatically exit upon completion of its command.  As a somewhat
%technical point, there are two different ways of referring to XSB
%threads.  The foreign language interfaces described in
%Chapter~\ref{foreign} and here use pointers to thread contexts so that
%the interfaces use much of the same code as the XSB engine.  However
%Prolog refers to threads using thread identifiers.  The two different
%forms can be converted into each other by the functions {\tt
%  xsb\_thread\_id\_to\_context()} and {\tt
%  xsb\_thread\_context\_to\_id()}.
%
%\subsection{Managing Multiple XSB Threads through the API} \label{sec:CXSBMT}
%
%The ability to pass thread contexts into query and command functions
%allows a great deal of flexibility~\footnote{For the sake of brevity,
%  we sometimes abuse notation and do not always distinguish between
%%  thread-contexts and their pointers.}.  Once XSB is initialized, XSB
%threads can be created from C and can execute independently of each
%other, effectively giving the ability for different calling threads to
%query XSB in a mechanism reminiscent of database cursors.
%
%\begin{figure}[hbtp]
%\begin{small}
%\begin{verbatim}
%.....
%/* context.h is necessary for the type of a thread context. */
%#include "context.h"
%
%int main(int argc, char *argv[])
%{ 
%
%  static th_context *p_th, *r_th;
%  char init_string[MAXPATHLEN];
%  int rcp, rcr;
%  XSB_StrDefine(p_return_string);
%  XSB_StrDefine(r_return_string);
%
%  strcpy(init_string,strip_names_from_path(xsb_executable_full_path(argv[0]),3));
%%
%  if (xsb_init_string(init_string)) {
%    fprintf(stderr,"%s initializing XSB: %s/%s\n",xsb_get_init_error_type(),
%            xsb_get_init_error_message());cin
%    exit(XSB_ERROR);
%  }
%
%  p_th = xsb_get_main_thread();
%
%  /* Create command to consult a file: edb.P, and send it. */
%  if (xsb_command_string(p_th, "consult('edb.P').") == XSB_ERROR)
%    fprintf(stderr,"++Error consulting edb.P: %s/%s\n",xsb_get_error_type(p_th),
%	    xsb_get_error_message(p_th));
%
%  xsb_ccall_thread_create(p_th,&r_th);
%
%  rcp = xsb_query_string_string(p_th,"p(X,Y,Z).",&p_return_string,"|");
%  rcr = xsb_query_string_string(r_th,"r(X,Y,Z).",&r_return_string,"|");
%
%  while (rcp == XSB_SUCCESS && rcr == XSB_SUCCESS) {
%
%    printf("Return p %s\n",(p_return_string.string));
%    rcp = xsb_next_string(p_th, &p_return_string,"|");
%
%    printf("Return r %s\n",(r_return_string.string));
%    rcr = xsb_next_string(r_th, &r_return_string,"|");
%  }
%
% if (rcp == XSB_ERROR) 
%   fprintf(stderr,"++Query Error p: %s/%s\n",xsb_get_error_type(p_th),xsb_get_error_message(p_th));
% if (rcr == XSB_ERROR) 
%   fprintf(stderr,"++Query Error r: %s/%s\n",xsb_get_error_type(r_th),xsb_get_error_message(r_th));
%
% xsb_close();      
%}
%
%\end{verbatim}\end{small}
%\caption{Manipulating Multiple Threads Using the {\tt VarString} Interface} 
%\label{fig:varstringex2}
%\end{figure}
%
%Figure~\ref{fig:varstringex2} illustrates a very simple example of
%this.  XSB is initialized and the file {\tt edb.P} consulted exactly as
%in Figure~\ref{fig:varstringex2}.  However, the function {\tt
%  xsb\_ccall\_thread\_create()} causes the XSB thread {\tt p\_th} to
%create a new thread, causes the new thread to call the same command
%loop as the main thread, and sets {\tt r\_th} to point to the context
%of the new thread.  The new thread {\tt r\_th} can be used for
%commands or queries just as {\tt p\_th}.
%Figure~\ref{fig:varstringex2} shows that queries to the two threads
%\end{verbatim}can be interleaved, and errors for both threads can be checked and
%reported independently.
 
%It is important to note that since each thread created by {\tt
%  xsb\_ccall\_thread\_create()} goes into a command-loop similar to
%the command loop, it will stay around until it is explicitly killed or
%until XSB is closed.  The call 
%\begin{verbatim}
%  xsb_kill_thread(r_th);
%\end{verbatim}
%is needed to make {\tt r\_th} to exit.  Once a thread is exited, all
%of its data structures will be freed, including those that support
%{\tt xsb\_get\_error\_type()} and {\tt
%  xsb\_get\_error\_message()}~\footnote{Note that causing XSB's main
%  thread to exit will cause the entire process to exit -- not just
%  XSB.}.

%\subsection{Calling Multiple XSB Threads using Multiple C Threads} \label{sec:CXSBMTMT}

%Figure~\ref{fig:varstringex2} shows how two XSB threads can be
%created, can receive different queries and can interleave their
%backtracking and answer return.  Although
%Figure~\ref{fig:varstringex2} demonstrated only backtracking through
%simple predicates, the mechanism employed works for complicated
%examples using tabling, dynamic code, and other features.  All this
%provides a sophisticated interface, but it is not ``fully''
%multi-threaded in the following sense.  When a C thread $T$ causes XSB
%to execute a command or query the thread must wait until the calling
%function returns before proceeding.  In certain applications it may be
%useful, for example, for $T$ to create a C thread $T_{new}$ which runs
%asynchronously from $T$, executing the XSB command or query and then
%exiting.  Alternately, an application may want to have a pool of C
%threads that can interact with a pool of XSB threads.
%
%XSB's C API has been designed to support these features.
%Figure~\ref{fig:varstringex3} shows fragments of
%Figure~\ref{fig:varstringex2} rewritten so that the routines to print
%out the answers to the queries {\tt p(X,Y,Z)} and {\tt r(X,Y,Z)} can
%be called from C threads specially designed for this purpose.  More
%specifically, the routine {\tt query\_ps()} calls {\tt p\_th} to query
%{\tt p(X,Y,Z)} and backtrack through its answers -- its use of a
%single {\tt void *} argument and a {\tt void *} return reflect the
%requirements of functions that are to be called using {\tt
%  pthread\_create()}.

%We note several points about this example.  First the XSB API is a
%low-level API that can be used to build application specific
%interfaces, and some experience with pthread programming is useful if
%multiple XSB threads are called from multiple C threads.  For
%instance, one issue is fairness.  When called from the C API each XSB
%thread $X_T$ makes use of mutexes to ensure that it answers only one
%query or command at a time.  If multiple C threads are are waiting for
%$X_T$ to respond to requests or queries, there is no guarantee that
%the requests will be processed in any sort of order, or even that a
%request will eventually be handled (In order to ensure this, the
%calling program would have to use a queue or some other scheduling
%mechanism to send requests to the XSB thread).  In addition, it is
%important to note that, {\em the main XSB thread should only be called
%  from the C thread that initialized XSB.}.  This restriction is due
%to the current design of synchronizing an XSB thread with calling
%threads, and may be lifted in the future.

\%begin{figure}[hbtp]
%\begin{small}
%\begin{verbatim}
%.....
%
%void *query_ps(void * arg) {
%  int rc;
%  th_context *p_th;
%%  XSB_StrDefine(p_return_string);
%  p_th = (th_context *)arg;
%  
%  rc = xsb_query_string_string(p_th,"p(X,Y,Z).",&p_return_string,"|");
%  while (rc == XSB_SUCCESS) {
%    printf("Return p %s\n",(p_return_string.string));
%    rc = xsb_next_string(p_th, &p_return_string,"|");
%  }
%
% if (rc == XSB_ERROR) 
%   fprintf(stderr,"++Query Error p: %s/%s\n",xsb_get_error_type(p_th),xsb_get_error_message(p_th));
% return NULL;
%}

%
%int main(int argc, char *argv[]) { 
%%
%  char init_string[MAXPATHLEN];
%  static th_context *p_th, *r_th;
%  int pstatus, rstatus;
%  pthread_t pthread_id,rthread_id;
%  XSB_StrDefine(p_return_string);
%  XSB_StrDefine(r_return_string);
%
%.....
%
%  main_th = xsb_get_main_thread();
%
%  /* Create command to consult a file: edb.P, and send it. */
%  if (xsb_command_string(xsb_get_main_thread(), "consult('edb.P').") == XSB_ERROR)
%    fprintf(stderr,"++Error consulting edb.P: %s/%s\n",xsb_get_error_type(main_th),
%	    xsb_get_error_message(main_th));
%
%  xsb_ccall_thread_create(main_th,&r_th);
%  xsb_ccall_thread_create(main_th,&p_th);
%
%  pthread_create(&rthread_id,NULL,command_rs,r_th);
%  pthread_create(&pthread_id,NULL,command_ps,p_th);
%  pthread_create(&rthread_id,NULL,command_rs,r_th);
%  pthread_create(&pthread_id,NULL,command_ps,p_th);
%
%  rstatus = pthread_join(rthread_id,&rreturn);
%  if (rstatus != 0) fprintf(stderr,"R join returns status %d\n",rstatus);
%  pstatus = pthread_join(pthread_id,&preturn);
%  if (pstatus != 0) fprintf(stderr,"P join returns status %d\n",pstatus);
%
% xsb_kill_thread(r_th);
% xsb_close();      
%}
%\end{verbatim}
%\end{small}
%\caption{Manipulating Multiple XSB Threads Using Multiple C Threads} 
%\label{fig:varstringex3}
%\end{figure}
%
%\subsubsection{Protected and Non-Protected API Functions} 
%
%Example~\ref{fig:varstringex3} shows that, when the {\tt Varstring}
%functions are used, if a single calling thread opens a query to an XSB
%thread $X_T$, $X_T$ will be protected from queries and commands posed
%by other C threads until the query is closed, failed out of, or exits
%via an error.  In fact, queries (and commands) are protected when the
%{\tt Varstring} or fixed string interfaces are used.  However,
%consider what may happen when the register level interface is used.
%In this case, a calling thread may call one or more API functions to
%set up the registers, execute a command or query, call several more
%API functions to obtain the output, and so on.  For this reason, if an
%application uses API commands that depend on user manipulation of
%registers ({\tt xsb\_command()},{\tt xsb\_query()},{\tt
%  xsb\_query\_string()}, and {\tt xsb\_next()}) the user must ensure
%that only one calling thread interacts with an XSB thread when that
%thread in the course of executing a command or query.  See {\tt
%  \$XSB\_DIR/examples/c\_calling\_xsb/cregs\_thread2.c} for an example
%of how mutexes can be used to protect XSB threads.
%
%When writing multi-threaded applications in XSB, be sure to be aware
%of how multiple threads share (and do not share) dynamic data and
%tables.  By default dynamic predicates (and tables) are unique to a
%given thread.  For data to be shared by multiple threads, a predicate
%must be declared to be {\tt shared}.  See section \ref{sec:shared} for
%details.

\section{A C API for XSB} \label{sec:CAPI}
%
\subsection{Initializing and Closing XSB}

\begin{description}

\ourCitem{int xsb\_init\_string(char *options)}{xsb\_init\_string}
%
This function is used to initialize XSB via an initialization string
{\tt *options}, and must be called before any other calls can be made.
The initialization string must include the path to the XSB directory
installation directory {\tt \$XSB\_DIR}, which is expanded to an
absolute path by XSB.  Any other command line options may be included
just as in a command line except {\tt -D}, {\tt -d}, {\tt -B} and {\tt
  -i}.  For example, a call from an executable in a sibling directory
of XSB might have the form
%%
\begin{verbatim}
   xsb_init_string("../XSB -e startup.");
\end{verbatim}
%%
which initializes XSB with the goal {\tt ?- startup.}

{\bf Return Codes}  
\bi
\item {\tt XSB\_SUCCESS} indicates that initialization returned
  successfully.
%
\item {\tt XSB\_ERROR} 
\bi
\item {\tt init\_error} if any error occurred during initialization. 

\item {\tt permission\_error} if {\tt xsb\_init\_string()} is called
  after XSB has already been correctly initialized.  
\ei
%
\ei

\ourCitem{int xsb\_init(int argc, char *argv[])}{xsb\_init}
%
This function is a variant of {\tt xsb\_init\_string()} which passes
initialization arguments as an argument vector: {\tt argc} is the
count of the number of arguments in the {\tt argv} vector.  The {\tt
  argv} vector is exactly as would be passed from the command line to
XSB.
%%
\begin{itemize}
\item $\tt argv[0]$ must be an absolute or relative path name of the XSB
  installation directory ({\it i.e.}, {\tt \$XSB\_DIR}).  Here is an
  example, which assumes that we invoke the C program from the XSB
  installation directory.
    %%
    \begin{verbatim}
int main(int argc, char *argv[])
{ 
  int myargc = 1;
  char *myargv[1];

  /* XSB_init relies on the calling program to pass the addr of the XSB
     installation directory. From here, it will find all the libraries */
  myargv[0] = ".";

  /* Initialize xsb */
  xsb_init(myargc,myargv);
}
    \end{verbatim}
    %%
\end{itemize}
%%
The return codes for {\tt xsb\_init()} are the same as those for {\tt
  xsb\_init\_string()}.

\ourCitem{int xsb\_close()}{xsb\_close}
%
This routine closes the entire connection to XSB .  After this, no
more calls can be made (not even calls to {\tt xsb\_init\_string()} or
{\tt xsb\_init()}).  In \version{}, no guarantee is made that all
space used by XSB will be restored to the process (even when the
process has dynamically linked to XSB), but space for any XSB tables
is freed.

{\bf Return Codes} 
\bi
\item {\tt XSB\_SUCCESS} indicates that XSB was closed successfully.
%
\item {\tt XSB\_ERROR} 
\bi
\item {\tt permission\_error} if {\tt xsb\_closed()} when XSB has not been
(correctly) initialized.  
\ei
%
\ei

\end{description}

\subsection{Passing Commands to XSB}
           {\bf {\em Unforunately, in the commands below, calling the multi-threaded XSB engine from C is no longer supported.}}
             
\begin{description}
\ourCitem{int xsb\_command\_string(th\_context *th, char *cmd)}{xsb\_command\_string}
%
This function passes a command to the XSB thread designated by {\tt
  th} (the first argument is not used in the single-threaded engine).
No query can be active in {\tt th} when the command is called.  The
command is a string consisting of a Prolog (or HiLog) term terminated
by a period (.).

When used in the multi-threaded engine, {\tt xsb\_command\_string}
protects the called thread from API calls from other pthreads until
the command is finished.

{\bf Return Codes}  
\bi
\item {\tt XSB\_SUCCESS} indicates that the command succeeded.
%
\item {\tt XSB\_FAILURE} indicates that the command failed.
%
\item {\tt XSB\_ERROR} 
\bi
\item {\tt permission\_error} if {\tt xsb\_command\_string()} is
  called while a query is open in {\tt th}.
%
\item Otherwise, any queries thrown during execution of the command
  are accessable through {\tt xsb\_get\_error\_type(th)} and {\tt
    xsb\_get\_error\_message(th)}.
\ei
%
\ei

\ourCitem{int xsb\_command(th\_context *th)}{xsb\_command}
%
This function passes a command to the XSB thread designated by {\tt
  th} (the first argument is not used in the single-threaded engine).
Any previous query must have already been closed.  Before calling {\tt
  xsb\_command()}, the calling program must construct the term
representing the command in register 1 in the XSB thread's space.
This can be done by using the {\tt c2p\_*} (and {\tt p2p\_*})
routines, which are described in Section \ref{c2p_p2p_p2c} below.
Register 2 may also be set before the call to {\tt xsb\_query()}
(using {\tt xsb\_make\_vars(int)} and {\tt xsb\_set\_var\_*()}) in
which case any variables set to values in the {\tt ret/n} term will be
so bound in the call to the command goal.  {\tt xsb\_command} invokes
the command represented in register 1 and returns {\tt XSB\_SUCCESS}
if the command succeeds, {\tt XSB\_FAILURE} if it fails, and {\tt
  XSB\_ERROR} if an error is thrown while executing the command.

When used in the multi-threaded engine, {\tt xsb\_command\_string}
{\em does not protect} the called thread from API calls from other
pthreads until the command is finished.  It is the user's
responsibility to protect the XSB thread, using a mutex or other
concurrency control, from the time the goal begins to be constructed
in the register 1 until the command has completed.

Apart from the steps necessary to formulate the query and the lack of
protection of the XSB thread, the behavior of {\tt xsb\_command()} is
similar to that of {\tt xsb\_command\_string()}, including its return
codes.
\end{description}

\subsection{Querying XSB}
           {\bf {\em Unforunately, in the commands below, calling the multi-threaded XSB engine from C is no longer supported.}}

\begin{description}
\ourCitem{int xsb\_query\_string\_string(th\_context *th, char *query, VarString *buff,char *sep)}{xsb\_query\_string\_string}
%
This function opens a query to the XSB thread designated by {\tt th}
(the first argument is not used in the single-threaded engine); it
returns the first answer (if there is one) as a {\tt VarString}.  Any
previous query to {\tt th} must have already been closed.  Any query
may return multiple data answers.  The first is found and made
available to the caller as a result of this call.  To get any
subsequent answers, {\tt xsb\_next\_string()} must be called.  An
example call is:
\begin{verbatim}
rc = xsb_query_string_string(th, "append(X,Y,[a,b,c]).",buff,";");
\end{verbatim}
The second argument is the period-terminated query string.  The third
argument is a pointer to a variable string buffer in which the
subroutine returns the answer (if any.) The variable string data type
{\tt VarString} is explained in Section~\ref{sec-varstring}. (Use {\tt
  xsb\_query\_string\_string\_b()} if you cannot declare a parameter
of this type in your programming language.)  The last argument is a
string provided by the caller, which is used to separate arguments in
the returned answer.  For the example query, {\tt buff} would be set
to the string:
\begin{verbatim}
        [];[a,b,c]
\end{verbatim}
which is the first answer to the append query.  There are two fields
of this answer, corresponding to the two variables in the query,
\verb|X| and \verb|Y|.  The bindings of those variables make up the
answer and the individual fields are separated by the \verb|sep|
string, here the semicolon (\verb|;|).  In the answer string, XSB
atoms are printed without quotes.  Complex terms are printed in a
canonical form, with atoms quoted if necessary, and lists produced in
the normal list notation.

When used in the multi-threaded engine, {\tt
  xsb\_query\_string\_string} protects the called thread from API
calls from other pthreads until the entire query is finished.

{\bf Return Codes}  
\bi
\item {\tt XSB\_SUCCESS} indicates that the query succeeded.
%
\item {\tt XSB\_FAILURE} indicates that the query failed.
%
\item {\tt XSB\_ERROR} 
\bi
\item {\tt permission\_error} if {\tt xsb\_query\_string\_string()} is
  called while a query to {\tt th} is open.
%
\item Otherwise, any errors thrown during execution of the query are
  accessable through {\tt xsb\_get\_error\_type()} and {\tt
    xsb\_get\_error\_message()}.  
\ei 
\ei

\item[{\mbox{\tt {\small int xsb\_query\_string\_string\_b(th\_context *th,char *query,char *buff,int bufflen,int *anslen,char *sep)}}}]
\predref{xsb\_query\_string\_string\_b}
%
This function provides a lower-level alternative to {\tt
  xsb\_query\_string\_string} (not using the {\tt VarString} type),
which makes it easier for non-C callers (such as Visual Basic or
Delphi) to access XSB functionality.  Any previous query to {\tt th}
must have already been closed.  Any query may return possibly multiple
data answers.  The first is found and made available to the caller as
a result of this call.  To get any subsequent answers, {\tt
  xsb\_next\_string\_b()} or a similar function must be called.  The
first and last arguments are the same as in {\tt
  xsb\_query\_string\_string()}.  The \verb|buff|, \verb|bufflen|, and
\verb|anslen| parameters are used to pass the answer (if any) back to
the caller.  \verb|buff| is a character array provided by the caller
in which the answer is returned.  \verb|bufflen| is the length of the
buffer (\verb|buff|) and is provided by the caller.  \verb|anslen| is
returned by this routine and is the length of the computed answer.  If
that length is less than \verb|bufflen|, then the answer is put in
\verb|buff| (and null-terminated).  If the answer is longer than will
fit in the buffer (including the null terminator), then the answer is
not copied to the buffer and {\tt XSB\_OVERFLOW} is returned.  In this
case the caller can retrieve the answer by providing a bigger buffer
(of size greater than the returned \verb|anslen|) in a call to {\tt
  xsb\_get\_last\_answer\_string()}.

When used in the multi-threaded engine, {\tt
  xsb\_query\_string\_string\_b} protects the called thread from API
calls from other pthreads until the entire query is finished.

{\bf Return Codes}  
\bi
\item {\tt XSB\_SUCCESS} indicates that the query succeeded.
%
\item {\tt XSB\_FAILURE} indicates that the query failed.
%
\item {\tt XSB\_ERROR} 
\bi
\item {\tt permission\_error} if {\tt xsb\_query\_string\_string\_b()}
  is called while a query to {\tt th} is open.
%
\item Otherwise, any queries thrown during execution of the command
  are accessable through {\tt xsb\_get\_error\_type()} and {\tt
    xsb\_get\_error\_message()}.
\ei

\item {\tt XSB\_OVERFLOW} indicates that the query succeeded, but the
  answer was too long for the buffer.
%
\ei

\ourCitem{int xsb\_query(th\_context *th)}{xsb\_query}
%
This function passes a query to the XSB thread {\tt th}.  Any previous
query to {\tt th} must have already been closed.  Any query may return
possibly multiple data answers.  The first is found and made available
to the caller as a result of this call.  To get any subsequent
answers, {\tt xsb\_next()} or a similar function must be called.
Before calling {\tt xsb\_query()} the caller must construct the term
representing the query in the XSB thread's register 1 (using routines
described in Section \ref{c2p_p2p_p2c} below.)  If the query has no
answers (i.e., just fails), register 1 is set back to a free variable
and {\tt xsb\_query()} returns {\tt XSB\_FAILURE}.  If the query has
at least one answer, the variables in the query term in register 1 are
bound to those answers and {\tt xsb\_query()} returns {\tt
  XSB\_SUCCESS}.  In addition, register 2 is bound to a term whose
main functor symbol is {\tt ret/n}, where n is the number of variables
in the query. The main subfields of this term are set to the variable
values for the first answer. (These fields can be accessed by the
functions {\tt p2c\_*}, or the functions {\tt xsb\_var\_*}, described
in Section \ref{c2p_p2p_p2c} below.)  Thus there are two places the
answers are returned. Register 2 is used to make it easier to access
them.  Register 2 may also be set before the call to {\tt xsb\_query()}
(using {\tt xsb\_make\_vars(int)} and {\tt xsb\_set\_var\_*()}) in
which case any variables set to values in the {\tt ret/n} term will be
so bound in the call to the goal.

When used in the multi-threaded engine, {\tt xsb\_query} {\em does not
  protect} the called thread from API calls from other pthreads until
the query is finished, or even when the registers are being accessed.
It is the user's responsibility to protect the XSB thread, using a
mutex or other concurrency control, from the time the goal begins to
be constructed in the register 1 until the query is closed, failed, or
exited upon error.

\ourCitem{int xsb\_get\_last\_answer\_string(th\_context *th, char
  *buff, int bufflen, int *anslen)}{xsb\_get\_last\_answer\_string\_b}
%
This function is used only when a call {\tt
  xsb\_query\_string\_string\_b()} or {\tt xsb\_next\_string\_b()} to
{\tt th} returns {\tt XSB\_OVERFLOW}, indicating that the buffer
provided was not big enough to contain the computed answer.  In that
case the user may allocate a larger buffer and then call this routine
to retrieve the answer (that had been saved.)  Only one answer is
saved per thread, so this routine must called immediately after the
failing call in order to get the right answer.  The parameters are the
same as the 2nd through 4th parameters of {\tt
  xsb\_query\_string\_string\_b()}.

{\bf Return Codes}  
\bi
\item {\tt XSB\_OVERFLOW} indicates that the answer was {\tt still}
  too long for the buffer.  
\ei

\ourCitem{int xsb\_query\_string(th\_context *th,char *query)}{xsb\_query\_string}
%
This function passes a query to the XSB thread {\tt th}.  The query is
a string consisting of a term that can be read by the XSB reader.  The
string must be terminated with a period (.).  Any previous query must
have already been closed.  In all other respects, {\tt
  xsb\_query\_string()} is similar to {\tt xsb\_query()}, except the
only way to retrieve answers is through Register 2.  The ability to
create the return structure and bind variables in it is particularly
useful in this function.

When used in the multi-threaded engine, {\tt xsb\_query\_string} {\em
  does not protect} the called thread from API calls from other
pthreads until the query is finished, or even when the registers are
being accessed.  It is the user's responsibility to protect the XSB
thread, using a mutex or other concurrency control, from the time the
goal begins to be constructed in the register 1 until the query is
closed, failed, or exited upon error.

{\bf Return Codes}  
\bi
\item {\tt XSB\_SUCCESS} indicates that the query succeeded.
%
\item {\tt XSB\_FAILURE} indicates that the query failed.
%
\item {\tt XSB\_ERROR} indicates that an error occurred while
  executing the query.  
%
\ei

\ourCitem{int xsb\_next\_string(th\_context *th,VarString *buff,char *sep)}{xsb\_next\_string}
%
This routine is called after {\tt xsb\_query\_string()} to retrieve a
subsequent answer in {\tt buff}.  If a query is not open in {\tt th},
an error is returned.  This function treats answers just as {\tt
  xsb\_query\_string\_string()}.  For example after the example call
\begin{verbatim}
rc = xsb_query_string_string(th,"append(X,Y,[a,b,c]).",buff,";");
\end{verbatim}
which returns with buff set to 
\begin{verbatim}
        [];[a,b,c]
\end{verbatim}
Then a call:
\begin{verbatim}
rc = xsb_next_string(th,buff,";");
\end{verbatim}
returns with buff set to 
\begin{verbatim}
        [a];[b,c]
\end{verbatim}
the second answer to the indicated query. 

In the multi-threaded engine, {\tt xsb\_next\_string()} protects the
XSB thread from concurrent access by other threads as long as the
query was invoked by {\tt xsb\_query\_string\_string(\_b)}.

{\bf Return Codes}  
\bi
\item {\tt XSB\_SUCCESS} indicates that the query succeeded.
%
\item {\tt XSB\_FAILURE} indicates that the query failed.
%
\item {\tt XSB\_ERROR} indicates that an error occurred while
  executing the query.  
%
\ei

\ourCitem{int xsb\_next\_string\_b(th\_context *th, char *buff, int bufflen, int *anslen, char *sep)}{xsb\_next\_string}
%
This function is a variant of {\tt xsb\_next\_string()} that does not
use the {\tt VarString} type.  Its parameters are the same as the 3rd
through 6th parameters of {\tt xsb\_query\_string\_string\_b()}.  The
next answer to the current query is returned in \verb|buff|, if there
is enough space.  If the buffer would overflow, this routine returns
{\tt XSB\_OVERFLOW}, and the answer can be retrieved by providing a
larger buffer in a call to {\tt xsb\_get\_last\_answer\_string\_b()}.
In any case, the length of the answer is returned in \verb|anslen|.

In the multi-threaded engine, {\tt xsb\_next\_string()} protects the
XSB thread from concurrent access by other threads as long as the
query was invoked by {\tt xsb\_query\_string\_string(\_b)}.

{\bf Return Codes}  
\bi
\item {\tt XSB\_SUCCESS} indicates that backtracking into the query succeeded.
%
\item {\tt XSB\_FAILURE} indicates that backtracking into the query failed.
%
\item {\tt XSB\_ERROR} indicates that an error occurred while further
  executing the query.

\item {\tt XSB\_OVERFLOW} indicates that backtracking into the query succeeded, but the
  new answer was too long for the buffer.
%
\ei

\ourCitem{int xsb\_next(th\_context *)}{xsb\_next}
%
This function is called after {\tt xsb\_query()} (which must have
returned {\tt XSB\_SUCCESS}) to retrieve more answers.  It rebinds the
query variables in the term in register 1 and rebinds the argument
fields of the {\tt ret/n} answer term in register 2 to reflect the
next answer to the query.  Its return codes are as with {\tt
  xsb\_next\_string()}.

When used in the multi-threaded engine, {\tt xsb\_next} {\em does not
  protect} the called thread from API calls from other pthreads until
the query is finished, or even when the registers are being accessed.
It is the user's responsibility to protect the XSB thread, using a
mutex or other concurrency control, through the time that registers
are accessed by the calling program.

\ourCitem{int xsb\_close\_query(th\_context *th)}{xsb\_close\_query}
%
This function allows a user to close a query to {\tt th} before all
its answers have been retrieved.  Since XSB is (usually) a
tuple-at-a-time system, answers that are not retrieved are not
computed so that closing a query may save time.  If a given query $Q$
is open, it is an error to open a new query without closing $Q$ either
by retrieving all its answers or explicitly calling {\tt
  xsb\_close\_query()} to close $Q$.  Calling {\tt
  xsb\_close\_query()} when no query is open gives an error message,
but otherwise has no effect.

{\bf Return Codes}  
\bi
\item {\tt XSB\_SUCCESS} indicates that the current query was closed.
%
\item {\tt XSB\_ERROR} 
\bi
\item {\tt permission\_error} if {\tt xsb\_close\_query()} is
  called while no query is open.
\ei
\ei
  
\ourCitem{int xsb\_add\_c\_predicate(th\_context *,char *,char *,int,int(*)())}{xsb\_add\_c\_predicate}
%
This function, called for example as: \\
{\tt xsb\_add\_c\_predicate(th, modname, predname, arity, cfunc)}\\
registers a C function (defined by the caller, here named {\tt cfunc})
as a foreign function to be invoked by a Prolog predicate.  The
arguments are: the name of the module of the Prolog predicate being
defined (NULL indicates usermod), the name of the Prolog predicate,
the arity of the Prolog predicate, and the function pointer of the
function defining the foreign routine.  That function must get (and
return) its arguments using the ctop (and ptoc) functions of the {\tt
  cinterf} Foreign Language Interface (\ref{foreign}).  This {\tt
  xsb\_add\_c\_predicate} function always returns 0.  After this
function has been called, the predicate {\tt modname:predname/arity}
can be called in Prolog and will result in {\tt cfunc} being invoked
as a foreign language function.

The {\tt th\_context} argument is used only for the multi-threaded XSB engine.

\end{description}

\subsection{Obtaining Information about Errors}
           {\bf {\em Unforunately, in the commands below, calling the multi-threaded XSB engine from C is no longer supported.}}

\begin{description}
\ourCitem{char * xsb\_get\_init\_error\_message()}{xsb\_get\_error\_message}
%
Used to find error messages if {\tt xsb\_init\_string()} or {\tt
  xsb\_init()} returns {\tt XSB\_ERROR}.  Any errors returned by these
functions have type {\tt init\_error}.  Because initialization errors
occur before XSB or any of its threads have been initialized,
initialization errors do not require a thread context for input.

\ourCitem{char * xsb\_get\_error\_type(th\_context *th)}{xsb\_get\_error\_type}
%
If a function called for {\tt th} returned {\tt XSB\_ERROR} this
function provides a pointer to a string representing the type of the
error.  Types are as in Volume 1 {\em Exception Handling} with the
addition of {\tt init\_error} for errors that occur during
initialization of XSB, and {\tt unrecoverable\_error} for errors from
which no recovery is possible for XSB (e.g. inability to allocate new
memory).  

\ourCitem{char *xsb\_get\_error\_message(th\_context *th)}{xsb\_get\_error\_message}
%
If a function called for {\tt th} returned {\tt XSB\_ERROR} this
function provides a pointer to a string representing a message
associated with the error.  For errors raised within the Prolog
portion of execution, messages are as in Volume 1 {\em Exception
  Handling}.

\end{description}

%\subsection{Thread Management from Calling Programs}
%
%\begin{description}
%
%\ourCitem{int xsb\_ccall\_thread\_create(th\_context *callingThread, th\_context **newThread)}{xsb\_ccall\_thread\_create}
%%
%Causes {\tt callingThread} to create a thread pointed to by {\tt
%  newThread}.  {\tt newThread} runs exactly the same interpreter loop
%as {\tt callingThread} and all API functions will work on {\tt
%  newThread} just as on the main thread, or any other thread.  {\tt
%  newThread} will be non-detached, and will inherit any private
%parameters from {\tt callingThread}.  To create a thread to do a
%specific task or a detached thread, rather than one that executes a
%command loop, simply call the query {\tt thread\_create/[2,3]} from
%one of the query functions.
%
%\ourCitem{th\_context *xsb\_get\_main\_thread()}{xsb\_get\_main\_thread}
%%
%Returns a pointer to the thread context of XSB's main thread.  If XSB
%has not been initialized or has been closed this function returns 0.
%
%\ourCitem{xsb\_tid xsb\_thread\_id\_to\_context(th\_context *th)}{xsb\_thread\_id\_to\_context}
%
%\ourCitem{th\_context *xsb\_thread\_context\_to\_id(xsb\_tid id)}{xsb\_thread\_context\_to\_id}
%
%\end{description}

\section{The Variable-length String Data Type}\label{sec-varstring}

\index{VarString} XSB uses variable-length strings to communicate with
certain C subroutines when the size of the output that needs to be passed
from the Prolog side to the C side is not known. Variable-length strings
adjust themselves depending on the size of the data they must hold and are
ideal for this situation. For instance, as we have seem the two subroutines
{\tt xsb\_query\_string\_string(query,buff,sep)} and {\tt
  xsb\_next\_string(buff,sep)} use the variable string data type, {\tt
  VarString}, for their second argument.  To use this data type, make sure
that
%%
\begin{verbatim}
#include "cinterf.h"  
\end{verbatim}
%%
appears at the top of the program file.  Variables of the {\tt VarString}
type are declared using a macro that must appear in the declaration section
of the program:
%%
\begin{verbatim}
XSB_StrDefine(buf);  
\end{verbatim}
%%
There is one important consideration concerning VarString with the
\emph{automatic} storage class: they must be
\emph{destroyed} on exit (see {\tt XSB\_StrDestroy}, below) from the procedure
that defines them, or else there will be a memory leak. 
It is not necessary to destroy static {\tt VarString}'s.

The public attributes of the type are {\tt int length} and {\tt char *string}.
Thus, {\tt buf.string} represents the actual contents of the buffer and
{\tt buf.length} is the length of that data. Although the length and the
contents of a {\tt VarString} string is readily accessible, the user {\bf must
not} modify these items directly. Instead, he should use the macros
provided for that purpose:
%%
\begin{itemize}
  \item {\tt XSB\_StrSet(VarString *vstr, char *str)}:~
    Assign the value of the regular null-terminated C string to the
    {\tt VarString} {\tt vstr}. The size of {\tt vstr} is adjusted
    automatically.
  \item {\tt XSB\_StrSetV(VarString *vstr1, VarString *vstr2)}:~
    Like {\tt XSB\_StrSet}, but the second argument is a variable-length
    string, not a regular C string.
  \item {\tt XSB\_StrAppend(VarString *vstr, char *str)}:~
    Append the null-terminated string {\tt str} to the {\tt VarString} {\tt
      vstr}. The size of {\tt vstr} is adjusted.
  \item {\tt XSB\_StrPrepend(VarString *vstr, char *str)}:~
    Like {\tt XSB\_StrAppend}, except that {\tt str} is prepended.
  \item {\tt XSB\_StrAppendV(VarString *vstr1, VarString *vstr2)}:~
    Like {\tt XSB\_StrAppend}, except that the second string is also a
    {\tt VarString}.
  \item {\tt XSB\_StrPrependV(VarString *vstr1, VarString *vstr2)}:~
    Like {\tt XSB\_StrAppendV}, except that the second string is prepended.
  \item {\tt XSB\_StrCompare(VarString *vstr1, VarString *vstr2)}:~
    Compares two {\tt VarString}. If the first one is lexicographically
    larger, then the result is positive; if the first string is smaller,
    than the result is negative; if the two strings have the same content
    ({\it i.e.}, {\tt vstr1->string} equals {\tt vstr2->string} then the
    result is zero.
  \item {\tt XSB\_StrCmp(VarString *vstr, char *str)}:~
    Like {\tt XSB\_StrCompare} but the second argument is a regular,
    null-terminated string.
  \item {\tt XSB\_StrAppendBlk(VarString *vstr, char *blk, int size)}:~
    This is like {\tt XSB\_StrAppend}, but the second argument is not assumed
    to be null-terminated. Instead, {\tt size} characters pointed to by
    {\tt blk} are appended to {\tt vstr}. The size of {\tt vstr} is
    adjusted, but the content is \emph{not} null terminated.
  \item {\tt XSB\_StrPrependBlk(VarString *vstr, char *blk, int size)}:~
    Like {\tt XSB\_StrPrepend}, but {\tt blk} is not assumed to point to a
    null-terminated string. Instead, {\tt size} characters from the region
    pointed to by {\tt blk} are prepended to {\tt vstr}.
  \item {\tt XSB\_StrNullTerminate(VarString *vstr)}:~
    Null-terminates the {\tt VarString}  string {\tt vstr}. This is used in
    conjunction with {\tt XSB\_StrAppendBlk}, because the latter does not
    null-terminate variable-length strings.
  \item {\tt XSB\_StrEnsureSize(VarString *vstr, int minsize)}:~
    Ensure that the string has room for at least {\tt minsize} bytes.
    This is a low-level routine, which is used to interface to procedures
    that do not use {\tt VarString} internally. If the string is larger
    than {\tt minsize}, the size might actually shrink to the nearest
    increment that is larger {\tt minsize}.
  \item {\tt XSB\_StrShrink(VarString *vstr, int increment)}:~ Shrink the
    size of {\tt vstr} to the minimum necessary to hold the data. {\tt
      increment} becomes the new increment by which {\tt vstr} is adjusted.
    Since {\tt VarString} is automatically shrunk by {\tt XSB\_StrSet}, it
    is rarely necessary to shrink a {\tt VarString} explicitly.  However,
    one might want to change the adjustment increment using this macro (the
    default increment is 128).
  \item {\tt XSB\_StrDestroy(VarString *vstr)}:~
    Destroys a {\tt VarString}.  Explicit destruction is necessary for
    {\tt VarString}'s with the automatic storage class. Otherwise, memory
    leak is possible.
\end{itemize}
%%


\section{Passing Data into an XSB Module}

The previous chapter described the low-level XSB/C interface that supports
passing the data of arbitrary complexity between XSB and C. However, in
cases when data needs to be passed into an executable XSB module by the
main C program, the following higher-level interface should suffice.  (This
interface is actually implemented using macros that call the lower level
functions.)  These routines can be used to construct commands and queries
into XSB 's register 1, which is necessary before calling {\tt
  xsb\_query()} or {\tt xsb\_command()}.


\begin{description}
\ourCitem{void xsb\_make\_vars((int) N)}{xsb\_make\_vars}
    {\tt xsb\_make\_vars} creates a return structure of arity {\tt N}
in Register 2.  So this routine may called before calling any of {\tt
xsb\_query}, {\tt xsb\_query\_string}, {\tt xsb\_command}, or {\tt
xsb\_command\_string} if parameters are to be set to be sent to the
goal.  It must be called before calling one of the {\tt
xsb\_set\_var\_*} routines can be called. {\tt N} must be the number
of variables in the query that is to be evaluated.

\ourCitem{void xsb\_set\_var\_int((int) Val, (int) N)}{xsb\_set\_var\_int}
    {\tt set\_and\_int} sets the {\tt N}$^{th}$ field in the return
structure to the integer value {\tt Val}.  It is used to set the value of
the {\tt N}$^{th}$ variable in a query before calling {\tt xsb\_query} or
{\tt xsb\_query\_string}.  When called in XSB, the query will
have the {\tt N}$^{th}$ variable set to this value.

\ourCitem{void xsb\_set\_var\_string((char *) Val, (int) N)}{xsb\_set\_var\_string}
    {\tt set\_and\_string} sets the {\tt N}$^{th}$ field in the return
structure to the atom with name {\tt Val}.  It is used to set the
value of the {\tt N}$^{th}$ variable in a query before calling {\tt
xsb\_query} or {\tt xsb\_query\_string}.  When called in XSB,
the query will have the {\tt N}$^{th}$ variable set to this value.

\ourCitem{void xsb\_set\_var\_float((float) Val, (int) N)}{xsb\_set\_var\_float}
    {\tt set\_and\_float} sets the {\tt N}$^{th}$ field in the return
structure to the floating point number with value {\tt Val}.  It is
used to set the value of the {\tt N}$^{th}$ variable in a query before
calling {\tt xsb\_query} or {\tt xsb\_query\_string}.  When called in
XSB, the query will have the {\tt N}$^{th}$ variable set to this
value.

\ourCitem{prolog\_int xsb\_var\_int((int) N)}{xsb\_var\_int}
{\tt xsb\_var\_int} is called after {\tt xsb\_query} or {\tt
xsb\_query\_string} returns an answer.  It returns the value of the
{\tt N}$^{th}$ variable in the query as set in the returned answer.
This variable must have an integer value (which is cast to {\tt
long} in a 64-bit architecture).

\ourCitem{char* xsb\_var\_string((int) N)}{xsb\_var\_string}
    {\tt xsb\_var\_string} is called after {\tt xsb\_query} or {\tt
xsb\_query\_string} returns an answer.  It returns the value of the
{\tt N}$^{th}$ variable in the query as set in the returned answer.
This variable must have an atom value.

\ourCitem{prolog\_float xsb\_var\_float((int) N)}{xsb\_var\_float}
    {\tt xsb\_var\_float} is called after {\tt xsb\_query} or {\tt
    xsb\_query\_string} returns an answer.  It returns the value of
    the {\tt N}$^{th}$ variable in the query as set in the returned
    answer.  This variable must have a floating point value (which is
    cast to {\tt double} in a 64-bit architecture).


\end{description}


\section{Creating an XSB Module that Can be Called from C}

To create an executable that includes calls to the above C functions,
these routines, and the XSB\ routines that they call, must be
included in the link ({\tt ld}) step.

\paragraph{Unix instructions:}
You must link your C program, which should include the main procedure, with
the XSB object file located in
%%
\begin{verbatim}
 $XSBDIR/config/<your-system-architecture>/saved.o/xsb.o  
\end{verbatim}
%%$
Your program should include the file {\tt cinterf.h} located in the {\tt
  XSB/emu} subdirectory, which defines the routines described earlier,
which you will need to use in order to talk to XSB.  It is therefore
recommended to compile your program with the option
\verb|-I$XSB_DIR/XSB/emu|.
%%$

The file {\tt \$XSB\_DIR/config/your-system-architecture/modMakefile} is a
makefile you can use to build your programs and link them with XSB.  It is
generated automatically and contains all the right settings for your
architecture, but you will have to fill in the name of your program, etc.

\index{64-bit architectures}
It is also possible to compile and link your program with XSB using XSB
itself as follows:
%%
\begin{verbatim}
:- xsb_configuration(compiler_flags,CFLAGS),
        xsb_configuration(loader_flags,LDFLAGS),
        xsb_configuration(config_dir,CONFDIR),
        xsb_configuration(emudir,EMUDIR),
        xsb_configuration(compiler,Compiler),
        str_cat(CONFDIR, '/saved.o/', ObjDir),
        write('Compiling myprog.c ... '),
        shell([Compiler, ' -I', EMUDIR, ' -c ', CFLAGS, ' myprog.c ']),
        shell([Compiler, CFLAGE, ' -o ', './myprog ',
               ObjDir, 'xsb.o ', ' myprog.o ', LDFLAGS]),
        writeln(done).  
\end{verbatim}
%%
This works for every architecture and is often more convenient than
using the make files~\footnote{The variable {\tt CFLAGS} is needed in
  the linking stage in order to ensure that the appropriate memory
  option is passed if XSB is configured {\tt --with-bits32} or {\tt
    --with-bits64} to override the default on a 64-bit platform.}.
There are simple examples of C programs calling XSB\ in the
{\tt \$XSB\_DIR/examples/c\_calling\_XSB} directory, in files {\tt cmain.c},
{\tt ctest.P}, {\tt cmain2.c}.

\paragraph{Windows instructions:}
To call XSB from C, you must build it as a DLL, which is done as follows:
%%
\begin{verbatim}
  cd $XSB_DIR\XSB\build
  makexsb_wind DLL="yes"
\end{verbatim}
%%$
The DLL, which you can call dynamically from your program is then found in 
%%
\[
 \tt
 \$XSB\_DIR\backslash config\backslash \mbox{\tt x86-pc-windows}\backslash
 bin \backslash xsb.dll
\]
%%
Since your program must include the file {\tt cinterf.h}, it is recommended
to compile it with the option \verb|/I$XSB_DIR\XSB\emu|.
%%$


%%% Local Variables: 
%%% mode: latex
%%% TeX-master: "manual2"
%%% End: 
