\chapter{System Description}
\label{system}

Throughout this chapter, we use \verb'$XSB_DIR' to refer to the
directory in which XSB was installed.

\section{Entering and Exiting XSB}
%=================================

After the system has been installed, the emulator's executable code appears 
in the file:
\begin{verbatim}
                     $XSB_DIR/bin/xsb
\end{verbatim}
or, if, after being built, XSB is later installed  at a central location,
\verb'$SHARED_XSB'.
\begin{verbatim}
                     $SHARED_XSB/bin/xsb
\end{verbatim}
and indeed, using this command, invokes XSB's top level interpreter
which is the usual way of
using XSB

\version{} of XSB can also directly execute object code files from the
command line interface.  Suppose you have a top-level routine {\tt go}
in a file {\tt foo.P} that you would like to run from the UNIX or
Windows command line.  As long as {\tt foo.P} contains a directive
{\tt :- go.}, and {\tt foo} has been compiled to an object file ({\tt
foo.O}), then
\begin{verbatim}
                       $XSB_DIR/bin/xsb -B foo.O
\end{verbatim}
will execute {\tt go}, loading the appropriate files as needed.  In
fact the command \verb'$XSB_DIR/bin/xsb' is equivalent to the
command:
\begin{verbatim}
             $XSB_DIR/bin/xsb -B $XSB_DIR/syslib/loader.O
\end{verbatim}
%%$

There are several ways to exit XSB.  A user may issue the
command \verb'halt.' or \verb'end_of_file', or simply type
\verb'CTRL-d' at the XSB prompt.  To interrupt XSB
while it is executing a query, strike \verb'CTRL-c'.


\section{The System and its Directories}
%=======================================
The XSB system, when installed, resides in a single directory that
contains the following subdirectories:
\begin{enumerate}
\item {\tt build}
\item {\tt docs}
\item {\tt emu}
\item {\tt etc}
\item {\tt examples}
\item {\tt cmplib}
\item {\tt lib}
\item {\tt packages}
\item {\tt syslib}
\end{enumerate}

The directory {\tt emu} contains the source and object code for
the XSB emulator, which is written in C.

The directories {\tt syslib}, {\tt cmplib} and {\tt lib} contain source
and object code for the basic Prolog libraries, the compiler, and the
extended Prolog libraries, respectively.  All the source programs
are written in \ourprolog, and all object (byte code) files contain SLG-WAM
instructions that can be executed by the emulator.  These byte-coded 
instructions are machine-independent, so usually no installation 
procedure is needed for the byte code files.

The directory {\tt packages} contains the various applications, written in
XSB, which are not part of the system per se.

You must already be familiar with the {\tt build} directory, which is what
you must have used to build XSB. This directory contains XSB configuration
scripts.  The directory {\tt etc} contains miscellaneous files used by XSB.

The directory {\tt docs} contains this manual in \LaTeX, dvi and Postscipt
format, and the directory {\tt examples} contains sample programs to 
demonstrate various features of \ourprolog.


\section{The Module system of \ourprolog} \label{Modules}
%========================================================
XSB has been designed as a module-oriented Prolog system.  Modules
provide a small step towards {\em logic programming ``in the large''}
that facilitates large programs or projects to be put together from
components which can be developed, compiled and tested separately.
Also module systems enforce the {\em principle of information hiding}
and can provide a basis for {\em data abstraction}.

The module system of \ourprolog, unlike the module systems of most
other Prolog systems is {\em atom-based}.  Briefly, the main
difference between atom-based module systems and predicate-based ones
is that in an atom-based module system {\em any symbol in a module can
be imported, exported or be a local symbol} as opposed to the
predicate-based ones where this can be done only for predicate
symbols~\footnote{Operator symbols can be exported as any other
symbols, but their precedence must be redeclared in the importing
module.}.

Usually the following three files are associated with a particular
module:
\begin{itemize}
\item A single {\it source} file, whose name is the module name plus 
      the suffix ``{\tt .P}''.  
\item An optional {\it header} file, whose name is the module name plus 
      the suffix ``{\tt .H}''.  
\item An {\it object (byte-code)} file, whose name consists of the module 
      name plus the suffix ``{\tt .O}''.
\end{itemize}
The header file is normally used to contain declarations and
directives while the source file usually contains the actual
definitions of the predicates defined in that module.  The module
hierarchy of XSB is therefore {\em flat} --- nested modules are not
possible.

In order for a file to be a module, it should contain one or more 
{\it export declarations}, which specify that a set of symbols appearing
in that module is visible and therefore can be used by any other module.
A module can also contain {\it local declarations}, which specify that
a set of symbols are {\it visible by this module only}, and therefore
cannot be accessed by any other module.
Any file (either module or not) may also contain {\it import declarations},
which allow symbols defined in and exported by other modules to be used
in the current module.  We note that only exported symbols can be
imported; for example importing a local symbol will cause an {\tt
environment conflict error}.

Export, local, and import declarations can appear anywhere in the source
or header files and have the following forms:

\demo{:- export $sym_1$, \ldots, $sym_l$. }

\demo{:- local $sym_1$, \ldots, $sym_m$. }

\demo{:- import $sym_1$, \ldots, $sym_n$ from $module$. }

\noindent
where $sym_i$ has the form $functor/arity$.

If the user does not want to use modules, he can simply
bypass the module system by not supplying any export declarations.
Such exportless files (non-modules) will be loaded into the module 
{\tt usermod}, which is the working module of the XSB interpreter.

Currently the module name is stored in its byte code file, which means
that if the byte code file is renamed, the module name is not altered,
and hence may cause confusion to the user and/or the system.  So, it
is advisable that the user not rename byte code files generated for
modules by the XSB compiler.  However, byte code files generated for
non-modules can be safely renamed.  We will try to fix the problem
described above in future releases.

In order to understand the semantics of modules, the user should keep in
mind that in a module oriented system, the name of each symbol is identified
as if it were prefixed with its module name, hence two symbols of the same
$functor/arity$ but different module prefixes are distinct symbols.  
Currently the following set of rules is used to determine the module 
prefix of a symbol:
\begin{itemize}
\item	Every predicate symbol appearing in a module (i.e. that appears as
	the head of some clause) is assumed to be {\em local} to that
	module unless it is declared otherwise (via an export or
	import declaration).  Symbols that are local to a given module
	are not visible to other modules.
\item	Every other symbol (essentially function symbols) in a module is
	assumed to be {\em global} (its module prefix is {\tt
	usermod}) unless declared otherwise.
\item	If a symbol is imported from another module (via an explicit import 
	declaration), the module prefix of the symbol is the module it is 
	imported from; any other symbol takes the module where the symbol 
	occurs as its module prefix.
\item	The XSB interpreter is entered with {\tt usermod} as its 
	working module.
\item	Symbols that are either defined in non-modules loaded into the
	system or that are dynamically created (by the use of standard
	predicates such as {\tt read/1, functor/3, '=..'/2}, etc) are
	contained in {\tt usermod}.
\end{itemize}
%
The following facts about the module system of XSB may not be
immediately obvious:
\begin{itemize}
\item   If users want to use a symbol from another module, they must
        explicitly import it otherwise the two symbols are different
        even if they are of the same $functor/arity$ form.
\item	A module can only export predicate symbols that are defined in
        that module.  As a consequence, a module {\em cannot} export
        predicate symbols that are imported from other modules.
        This happens because an {\tt import} declaration is just a
        request for permission to use a symbol from a module where
        its definition and an {\tt export} declaration appear.
\item   The implicit module for a particular symbol appearing in a
        module must be uniquely determined.  As a consequence, a
        symbol of a specific $functor/arity$ {\em cannot} be declared
        as both exported and local, or both exported and imported from
        another module, or declared to be imported from more than one
        module, etc.  These types of environment conflicts are
        detected at compile-time and abort the compilation.
\item   It is an error to import a symbol from a module that does not
        export it.  This error is {\em not\/} detected at compile-time
        but at run-time when a call to that symbol is made.  If the
        symbol is defined in, but not exported from the module that
	defines it, an environment conflict error will take place.
	If the symbol is not defined in that module an undefined
	predicate/function error will be be reported to the user.
	\index{errors!undefined predicate}
\item   In the current implementation, at any time only one symbol of
        a specific $functor/arity$ form can appear in a module.  As an
        immediate consequence of this fact, only one $functor/arity$
        symbol can be loaded into the current working module ({\tt
        usermod}).  An attempt to load a module that redefines that
        symbol results in a warning to the user and the newly loaded
        symbol {\em overrides} the definition of the previously loaded
        one.
\end{itemize}

\section{The Dynamic Loader and its Search Path} \label{LibPath}
\index{load search path}
%===============================================================
The dynamic (or automatic) loader comprises one of \ourprolog's 
differences from other Prolog systems.
In \ourprolog, the loading of user modules Prolog libraries (including
the XSB compiler itself) is delayed until predicates in them
are actually needed, saving program space for large Prolog
applications.  The delay in the loading is done automatically, unlike
other systems where it must be explicitly specified for non-system
libraries.

When a predicate imported from another module (see section~\ref{Modules})
is called during execution, the dynamic loader is invoked automatically
if the module is not yet loaded into the system, 
The default action of the dynamic loader is to search for the
byte code file of the module 
first in the system library directories (in the order {\tt lib, syslib}, 
and then {\tt cmplib}), and finally in the current working directory.
If the module is found in one of these directories, then it will 
be loaded ({\em on a first-found basis}). Otherwise, an error 
message will be displayed on the current output stream
reporting that the module was not found.

In fact, XSB loads the compiler and most system modules this way.
Because of dynamic loading, the time it takes to compile a file 
is slightly longer than usual the first time the compiler is 
invoked in a session.


\subsection{Changing the Default Search Path and the Packaging System}
%=====================================================================
Users are allowed to supply their own library directories and also to
override the default search path of the dynamic loader. 
User-supplied library directories are searched by the dynamic loader 
{\em before} searching the default library directories.

The default search path of the dynamic loader can easily be changed
by having a file named {\verb|.xsb/xsbrc.P|} in the user's home directory.  
The {\verb|.xsb/xsbrc.P|} file, which is automatically consulted by the
 XSB interpreter, might look like the following:
\begin{verbatim}
             :- assert(library_directory('./')).
             :- assert(library_directory('~/')).
             :- assert(library_directory('~my_friend')).
             :- assert(library_directory('/usr/lib/sbprolog')).
\end{verbatim}
After loading the module of the above example, the current working
directory is searched first (as opposed to the default action of
searching it last).  Also, \ourprolog's system library directories
({\tt lib, syslib}, and {\tt cmplib}), will now be searched {\em
after} searching the user's, {\tt my\_friend}'s and the {\tt
"/usr/lib/sbprolog/"} directory.

 In fact, XSB also uses {\tt library\_directory/1} for internal purposes.
 For instance, before the user's {\verb|.xsb/xsbrc.P|} is consulted,
 XSB puts the {\tt packages} directory and the directory 
 %%
 \begin{verbatim}
      .xsb/config/$CONFIGURATION
 \end{verbatim}
 %%$
 on the library search path.
 The directory \verb'.xsb/config/$CONFIGURATION' is used to store user
 libraries that are machine or OS dependent. (\verb'$CONFIGURATION' for a
 machine is something that looks like {\tt sparc-sun-solaris2.6} or
 {\tt pc-linux-gnu}, and is selected by XSB automatically at run time).  
 %%$

Note that the file {\verb|.xsb/xsbrc.P|} is not limited to setting the
library search path.  In fact, arbitrary Prolog code can go there.

We emphasize that in the presense of a {\verb|.xsb/xsbrc.P|} file
{\em it is the user's responsibility to avoid module name clashes 
with modules in \ourprolog's system library directories}.
Such name clashes can cause the system to behave strangely since these
modules will probably have different semantics from that expected
by the XSB system code.  The list of module names in
\ourprolog's system library directories can be found in
appendix~\ref{module_names}.

Apart from the user libraries, XSB now has a simple packaging system.
A {\em package\/} is an application consisting of one or more files that
are organized in a subdirectory of one of the XSB system or user libraries.
The system directory \verb|$XSB_DIR/packages| has several examples
%%$
of such packages. Packages are convenient as a means of organizing
large XSB applications, and for simplifying user interaction with such
applications. User-level packaging is implemented through the predicate
%%
\begin{verbatim}
     bootstrap_userpackage(+LibraryDir, +PackageDir, +PackageName).
\end{verbatim}
%%
\index{{\tt bootstrap\_userpackage/3}}
which must be imported from the {\tt packaging} module. 

To illustrate, suppose you wanted to create a package, {\tt foobar}, inside
your own library, {\tt my\_lib}. Here is a sequence of steps you can
follow:  
%%
\begin{enumerate}
\item Make sure that {\tt my\_lib} is on the library search path by putting
  an appropriate assert statement in your {\tt xsbrc.P}.
\item Make subdirectory \verb|~/my_lib/foobar| and organize all the
  package files there. Designate one file, say, {\tt foo.P}, as the
  entry point, {\it i.e.}, the application file that must be loaded first.
\item Create the interface program \verb|~/my_lib/foobar.P| with the
  following content:
    %%
    \begin{verbatim}
           :- bootstrap_userpackage('~/my_lib', 'foobar', foobar), [foo].
    \end{verbatim}
    %%
  The interface program and the package directory do not need to have the
  same name, but it is convenient to follow the above naming schema.
\item Now, if you need to invoke the {\tt foobar} application, you can
  simply type \verb|[foobar].| at the XSB prompt. This is because both and
  \verb|~/my_lib/foobar| have already been automatically added to the
  library search path.
\item If your application files export many predicates, you can simplify
  the use of your package by having \verb|~/my_lib/foobar.P| import all
  these predicates, renaming them, and then exporting them. This provides a
  uniform interface to the {\tt foobar} module, since all the package
  predicates are can now be imported from just one module, {\tt foobar}.
\end{enumerate}
%%
In addition to adding the appropriate directory to the library search path,
the predicate \verb|bootstrap_userpackage/3| also adds information to the
predicate \verb|package_configuration/3|, so that other applications could
query the information about loaded packages.

Packages can also be unloaded using the predicate
\verb|unload_package/1|. For instance, 

%%
\begin{verbatim}
       :- unload_package(foobar).  
\end{verbatim}
%%
removes the directory \verb|~/my_lib/foobar| from the library search path
and deletes the associated information from \verb|package_configuration/3|.
\index{{\tt unload\_package/1}}
\index{{\tt package\_configuration/2}}


\subsection{Dynamically loading predicates in the interpreter}
%=============================================================
Modules are usually loaded into an environment when they are consulted
(see section~\ref{Consulting}).  Specific predicates from a module can
also be imported into the run-time environment through the standard 
predicate {\tt import PredList from Module}\index{{\tt import/1}}.
Here, {\tt PredList} can either be a Prolog list or a comma list.  (The
{\tt import/1} can also be used as a directive in a source module 
(see section~\ref{Modules}).

We provide a sample session for compiling, dynamically loading, and 
querying a user-defined module named {\tt quick\_sort}.
For this example we assume that {\tt quick\_sort} is a file in the 
current working directory, and contains the definitions of the
predicates {\tt concat/3} and {\tt qsort/2}, both of which are exported.

{\footnotesize
\begin{verbatim}
             | ?- compile(quick_sort).
             [Compiling ./quick_sort]
             [quick_sort compiled, cpu time used: 1.439 seconds]

             yes
             | ?- import concat/3, qsort/2 from quick_sort. 

             yes
             | ?- concat([1,3], [2], L), qsort(L, S).

             L = [1,3,2]
             S = [1,2,3]

             yes.
\end{verbatim}
}

The standard predicate {\tt import/1} does not load the module 
containing the imported predicates, but simply informs the system 
where it can find the definition of the predicate when (and if) the
predicate is called.


\section{Command Line Arguments} \label{emulator_options}
%========================================================
\index{emulator!command line options}
\index{options!command line arguments}
\index{stacks!default sizes}
\index{stacks!expanding}
%========================================================

There are several command line options for the emulator. The general 
synopsis is:
\begin{verbatim}
      	xsb [flags] [-l] [-i]
	xsb [flags] -n
	xsb [flags] module
	xsb [flags] -B boot_module [-D cmd_loop_driver] [-t] [-e goal]
	xsb [flags] -B module_to_disassemble -d
	xsb -[h | v]
	xsb --help | --version | --nobanner | --quietload | --noprompt

	memory management flags:
	    -c tcpsize | -m glsize | -o complsize | -u pdlsize | -r | -g gc_type
	miscellaneous flags:
	    -s | -T

	module:
	    Module to execute after XSB starts up.
	    Module should have no suffixes, no directory part, and
	    the file module.O must be on the library search path.
      	boot_module:
	    This is a developer's option.
	    The -B flags tells XSB which bootstraping module to use instead
	    of the standard loader. The loader must be specified using its
	    full pathname, and boot_module.O must exist.
    	module_to_disassemble:
	    This is a developer's option.
	    The -d flag tells XSB to act as a disassembler.
	    The -B flag specifies the module to disassemble.
	cmd_loop_driver:
	    The top-level command loop driver to be used instead of the
	    standard one. Usually needed when XSB is run as a server.

      	    -i : bring up the XSB interpreter
       -e goal : evaluate goal when XSB starts up
      	    -l : the interpreter prints unbound variables using letters
      	    -n : used when calling XSB from C
	    -B : specify the boot module to use in lieu of the standard loader
	    -D : Sets top-level command loop driver to replace the default.
      	    -t : trace execution at the SLG-WAM instruction level
		 (for this to work, build XSB with the --debug option)
	    -d : disassemble the loader and exit
      	  -c N : allocate N KB for the trail/choice-point stack
      	  -m N : allocate N KB for the local/global stack
      	  -o N : allocate N KB for the SLG completion stackof
      	  -u N : allocate N KB for the SLG unification (table copy) stack
      	    -r : turn off automatic stack expansion
    -g gc_type : choose the garbage collection ("none", "sliding", or "copying")
      	    -s : maintains more detailed statistical information
      	    -T : print a trace of each called predicate
 -v, --version : print the version and configuration information about XSB.
    -h, --help : print this help message
    --nobanner : don't show the XSB banner on startup
   --quietload : don't show the `module loaded' messages
    --noprompt : don't show prompt (for non-interactive use)

\end{verbatim}
The order in which these options appear makes no difference.

\begin{description}
\item[{\tt -i}] Brings up the XSB interpreter.  This is the normal use
        and because of this, use of this option is optional and is
        only kept for backwards compatibility.
\item[{\tt -l}] Forces the interpreter to print unbound variables as
	letters, as opposed to the default setting which prints
	variables as memory locations prefixed with an underscore.
	For example, starting XSB's interpreter with this option will
	print the following:
        \begin{verbatim}
                  | ?- Y = X, Z = 3, W = foo(X,Z).

                  Y = A
                  X = A
                  Z = 3
                  W = foo(A,3)
	\end{verbatim}
	as opposed to something like the following:
	\begin{verbatim}
                  | ?- Y = X, Z = 3, W = foo(X,Z).

                  Y = _10073976
                  X = _10073976
                  Z = 3
                  W = foo(_10073976,3);
	\end{verbatim}
\item[{\tt -n}] used in conjunction with the {\tt -i} option, to
    indicate that the usual read-eval-print top-loop is not to be
    entered, but instead will interface to a calling C program.  See
    the chapter {\it Calling XSB from C} in Volume 2  for details.
\item[{\tt -d}] Produces a disassembled dump of {\tt byte\_code\_file} to 
    {\tt stdout} and exits.
\item[{\tt -c} {\em size}] Allocates {\em initial  size\/} Kbytes of space
    to the trail/choice-point stack area.  The trail stack grows
    upward from the bottom of the region, and the choice point stack
    grows downward from the top of the region.  Because this region is
    expanded automatically from Version 1.6.0 onward, this option
    should rarely need to be used.  Default initial size: 768 Kbytes.
\item[{\tt -m} {\em size}] Allocates {\em size\/} Kbytes
    of space to the local/global stack area.  The global stack grows 
    upward from the bottom of the region, and the local stack grows 
    downward from the top of the region.  Default: 768 Kbytes.
\item[{\tt -o} {\em size}] Allocates {\em size\/} Kbytes of space
    to thecompletion stack area.  Because this region is expanded
    automatically from Version 1.6.0 onward, this option should rarely
    need to be used. Default initial size 64 Kbytes.
\item[{\tt -u} {\em size}] Allocates {\em size} Kbytes of space
    to the unification (and table copy) stack.  Default 64 Kbytes.
    (This option should rarely need to be used).
\item[{\tt -D}] Tells XSB to use a top-level command loop driver specified
  here instead of the standard XSB interpreter. This is most useful when
  XSB is used as a server.
\item[{\tt -r}] Turns off automatic stack expansion.
\item[{\tt -g gc\_type}] Chooses the garbage collection strategy that
    is employed; choice of the strategy is between {\tt "none"} (meaning
    perform no garbage collection), or garbage collection based on
    {\tt "sliding"} or on {\tt "copying"}.  Since garbage collection is
    only available when the emulator is based on a CHAT model (see also
    the installation options), this option only makes sense in this
    context; it is ineffective when the emulator is SLG-WAM based.
\item[{\tt -s}] Maintains information on the size of program stacks 
    for the predicate {\tt statistics/0}.  This option may be expected
    to slow execution by around 10\%.  Default: off.
\item[{\tt -T}] Generates a trace at entry to each called predicate
    (both system and user-defined).  This option is available mainly
    for people who want to modify and/or extend \ourprolog, and it is
    {\em not\/} the normal way to trace XSB programs.  For the
    latter, the builtin predicates {\tt trace/0} or {\tt debug/0}
    should be used (see Chapter~\ref{debugging}).

    Note: This option is not available when the system is being used
    at the non-tracing mode (see Section~\ref{debugging}).
\item[{\tt -t}] Traces through code at SLG-WAM instruction level.  This
    option is for internal debugging and is not fully supported.
    It is also not available when the system is being used at the non-debug
    mode (see Section~\ref{debugging}).
\item[{\tt -e goal}] Pass {\tt goal}  to XSB at starup. This goal is evaluated
    right before the first prompt is issued. For instance, 
    \verb'xsb -e "write('Hello!'), nl."'
    will print a heart-warming message when XSB starts up.
  \item[{\tt --nobanner}] Start XSb without showing the startup banner.
    Useful in batch scripts and for interprocess communication (when XSB is
    launched as a subprocess).
  \item[{\tt --quietload}] Do not tell when a new module gets loaded. Again, is
    useful in non-interactive activities and for interprocess communication.
  \item[{\tt --noprompt}] Do not show te XSB prompt. This is useful only in batch
    mode and in interprocess communication when you do not want the prompt
    to clutter the picture.
\end{description}

As an example, a program which uses more heap and local stack than the
default configuration of XSB might be run by invoking XSB with the command.
\begin{verbatim}
         xsb -m 2000
\end{verbatim}


\section{Memory Management}\label{memory_management}
\index{memory management}
\index{garbage collection}
%===================================================
All execution stacks are automatically expenaded in \version{}
including the local stack/heap region, the trail/choice point region
and the completion stack region.  Each of these regions begin with an
initial value set by the user (or the default stated in
Section~\ref{emulator_options}), and double their size until it is not
possible to do so with available system memory.  At that point XSB
tries to find the maximal amount of space that will still fit in
system memory.  Garbage collection is automatically performed for
retracted clauses.  In addition, heap garbage collection is
automatically included when the {\tt --enable-chat} configuration
option is used.  (It can be turned off; see the predicate {\tt
garbage\_collection/1}.) 

The program area (the area into which the code is loaded) is also
dynamically expanded as needed, and the area occupied by dynamic code
(created using {\tt assert/1}, or the standard predicate {\tt
load\_dyn/1}) is reclaimed when that code is retracted.  Version 1.8
improves memory management for retracted dynamic code.

\version{} provides memory management for table space as well.  Space for
tables is dynamically allocated as needed and reclaimed through use of
the predicate {\tt abolish\_all\_tables/0}
%, {\tt, abolish\_table\_pred/1} and {\tt abolish\_table\_call/1}.  
(see Section~\ref{tabling_predicates}).


\section{Compiling and Consulting} \label{Consulting}
%====================================================
In XSB, both compiled and interpreted code are transformed into
SLG-WAM instructions.  The main differences are that compiled code may
be more optimized than interpreted code, and that compilation produces
an object code file.

\index{{\tt consult/[1,2]}}
This section describes the actions of the standard predicate {\tt
consult/[1,2]} (and of {\tt reconsult/[1,2]} which is defined to have
the same actions as {\tt consult/[1,2]}).  {\tt consult/[1,2]} is the
most convenient method for entering rules into XSB's database.  Though
{\tt consult} comes in many flavors, the most general form is:
\begin{center}{\tt
	consult(+FileList, +CompilerOptionList) }
\end{center} 
At the time of the call both of its arguments should be instantiated
(ground).  {\tt FileList} is a list of filenames or module names (see
section~\ref{Modules}) and {\tt CompilerOptionList} is a list of
options that are to be passed to the compiler when (and if) it should
be invoked.  For a detailed description of the format and the options
that can appear in this list see Section~\ref{the_compiler}.

If the user wants to consult one module (file) only, she can provide an
atom instead of a list for the first argument of {\tt consult/2}.
Furthermore, if there isn't any need for special compilation options the
following two forms:
\begin{center}{\tt
	    [FileName]. \\
	consult(FileName).
}\end{center}
are just notational shorthands for:
\begin{center}{\tt
	consult(FileName, []).
}\end{center}

Consulting a module (file) generally consists of the following five
steps which are described in detail in the next paragraphs.
\begin{description}
\item[Name Resolution]: determine the module to be consulted.
\item[Compilation]:  if necessary (and the source file is not too big),
      compile the module using predicate {\tt compile/2} with the options
      specified.
\item[Loading] load the object code of the module into memory.
\item[Importing] import all the exported predicates of that module to
      the current working module ({\tt usermod}).
\item[Query Execution]: execute any queries that the module may contain.
\end{description}

There are two steps to name resolution: determination of the proper
directory prefix and determination of the proper extension.  When {\tt
FileName} is absolute (i.e. in UNIX contains a slash {\tt '/'})
determination of the proper directory prefix is straightforward.
However, the user may also enter a name without any directory
prefix. In this case, the directory prefix is a directory in the
dynamic loader path (see section~\ref{LibPath}) where the source file
exists.  Once the directory prefix is determined, the file name is
checked for an extension.  If there is no extension the loader first
checks for a file in the directory with the {\tt .P} extension, (or
{\tt .c} for foreign modules) before searching for a file without the
extension.  Note that since directories in the dynamic loader path are
searched in a predetermined order (see section~\ref{LibPath}), if the
same file name appears in more than one of these directories, the
compiler will consult the first one it encounters.

Compilation is performed if the update date of the the source file
({\tt *.P}) is later than that of the the object file ({\tt *.O}),
{\em and} if the source file is not larger than the default compile
size.  This default compile is set to be 20,000 bytes (in {\tt
cmplib/config.P}), but can be reset by the user.  If the source file
is larger than the default compile size, the file will be loaded using
{\tt load\_dyn/1}, and otherwise it will be compiled ({\tt
load\_dyn/1} can also be called separately, see the section {\it
Asserting Dynamic Code} for details.
While {\tt load\_dyn} gives reasonibly good execution times,
compilation can always be done by using {\tt compile/[1,2]}
explicitly.  Currently (\version), a foreign language module is
compiled when at least one of files {\tt *.c} or {\tt *.H} has been
changed from the time the corresponding object files have been
created.

\index{{\tt multifile/2}}
Whether the file is compiled or dynamically loaded, the byte-code for
the file is loaded into XSB's database.  The default action upon
loading is to delete any previous byte-code for predicates defined in
the file.  If this is not the desired behavior, the user may add to
the file a declaration 
\begin{center}
{\tt :- multifile <Predicate\_List> .} \\
\end{center}
where {\tt Predicate\_List} is a list of predicates in {\em
functor/arity\/} form.  The effect of this declaration is to delete {\em
only\/} those clauses of {\tt predicate/arity} that were defined in the
file itself.

After loading the module, all exported predicates of that module are
imported into the current environment (the current working
module {\tt usermod}).  For non-modules (see Section~\ref{Modules}),
all predicates are imported into the current working module.

Finally any queries --- that is, any terms with principal functor {\tt
':-'/1} that are not directives like the ones described in
Section~\ref{the_compiler} --- are executed in the order that they are
encountered.

\section{The Compiler} \label{the_compiler} \index{Compiler}
%===========================================================

The XSB compiler translates XSB source files into
byte-code object files.  It is written entirely in Prolog.
Both the sources and the byte code\index{byte code!files!compiler}
for the compiler can be found in the XSB system directory
{\tt cmplib}\index{cmplib@{\tt cmplib}}.

\index{GPP}
Prior to compiling, XSB filters the programs through \emph{GPP}, a 
preprocessor written by Denis Auroux (auroux@math.polytechnique.fr).
This preprocessor maintains high degree of compatibility with the C
preprocessor, but is more suitable for processing Prolog programs.
The preprocessor is invoked with the compiler option \verb|xpp_on|
as described below. The various features of GPP are described in
Appendix~\ref{gpp-man}.

XSB also allows the programmer to use preprocessors other than GPP.
However, the modules that come with XSB distribution require GPP.
This is explained below (see \verb|xpp_on| compiler option).

The following sections describe the various aspects of the compiler 
in more detail.


\subsection{Invoking the Compiler} \label{compiler_invoking}
\index{invoking the Compiler}\index{Compiler!invoking}
%=====================================================

The compiler is invoked directly at the interpreter level (or in a 
program) through the Prolog predicates {\tt compile/[1,2]}.  

The general forms of predicate {\tt compile/2} are:
\begin{center}{\tt	
	compile(+File, +OptionList) \\
	compile(+FileList, +OptionList)
}
\end{center}
and at the time of the call both of its arguments should be ground.

The second form allows the user to supply a proper list of file names as
the parameter for {\tt compile/[1,2]}.  In this case the compiler will
compile all the files in {\tt FileList} with the compiler
options specified in {\tt OptionList} (but see
Section~\ref{compiler_options} below for the precise details.)

\demo{$|$ ?- compile(Files).} 

\noindent
is just a notational shorthand for the query:

\demo{$|$ ?- compile(Files, []).}

The standard predicates {\tt consult/[1,2]} call {\tt compile/1} (if
necessary).  Argument {\tt File} can be any syntactically valid UNIX
or Windows file name (in the form of a Prolog atom), but the user can also
supply a module name.

The list of compiler options {\tt OptionList}, if specified, 
should be a proper Prolog list, i.e.\ a term of the form:
\begin{center}
	{\tt [ $option_1$, $option_2$, $\ldots$, $option_n$ ].}
\end{center}
where $option_i$ is one of the options described in
Section~\ref{compiler_options}.

The source file name corresponding to a given module is obtained by 
concatenating a directory prefix and the extension {\tt .P} (or {\tt .c}) 
to the module name.  The directory prefix must be in the
dynamic loader path (see Section~\ref{LibPath}).
Note that these directories are searched in a predetermined
order (see Section~\ref{LibPath}), so if a module with the same name
appears in more than one of the directories searched, the compiler 
will compile the first one it encounters.  In such a case, the user can 
override the search order by providing an absolute path name.

If {\tt File} contains no extension, an attempt is made to compile the 
file {\tt File.P} (or {\tt File.c}) before trying compiling the file 
with name {\tt File}.  

We recommend use of the extension {\tt .P} for Prolog source file to
avoid ambiguity.  Optionally, users can also provide a header file for
a module (denoted by the module name suffixed by {\tt .H}).  In such a
case, the XSB compiler will first read the header file (if it
exists), and then the source file.  Currently the compiler makes no
special treatment of header files.  They are simply included in the
beginning of the corresponding source files, and code can, in
principle, be placed in either.  In future versions of XSB the
header files may be used to check interfaces across modules, hence it
is a good programming practice to restrict header files to
declarations alone.
 
The result of the compilation (an SLG-WAM object code file) is stored
in a ($\langle$filename$\rangle$.O), but {\tt compile/[1,2]} does {\em
not\/} load the object file it creates.  (The standard predicates {\tt
consult/[1,2]} and {\tt reconsult/[1,2]} both recompile the source
file, if needed, and load the object file into the system.)  The
object file created is always written into the directory where the
source file resides (the user should therefore have write permission
in that directory).
 
If desired, when compiling a module (file), clauses and directives can be
transformed as they are read.  This is indeed the case for definite clause
grammar rules (see Chapter~\ref{DCGs}), but it can also be done for clauses
of any form by providing a definition for predicate {\tt term\_expansion/2}
(see Section~\ref{DCG_builtins}).

Predicates {\tt compile/[1,2]} can also be used to compile foreign
language modules.  In this case, the names of the source files should
have the extension {\tt .c} and a {\tt .P} file must {\em not\/}
exist.  A header file (with extension {\tt .H}) {\em must} be present
for a foreign language module (see the chapter {\it Foreign Language
Interface} in Volume 2.


\subsection{Compiler Options}\label{compiler_options}
\index{Compiler!options}\index{options!Compiler}
%=================================================

Compiler options can be set in three ways: from a global list of
options (see {\tt set\_global\_compiler\_options/1}), from the
compilation command (see {\tt compile/2} and {\tt consult/2}), and
from a directive in the file to be compiled (see compiler directive
{\tt compiler\_options/1}).

\begin{description}
\ouritem{set\_global\_compiler\_options(+OptionsList)}\index{{\tt
set\_global\_compiler\_options/1}} 
    {\tt OptionsList} is a list of compiler options (described below).
    Each can optionally be prefixed by \verb|+| or \verb|-|,
    indicating that the option is to be turned on, or off,
    respectively.  (No prefix turns the option on.)  This evaluable
    predicate sets the global compiler options in the way indicated.
    These options will be used in any subsequent compilation, unless
    reset by another call to this predicate, or overridden by options
    provided in the compile invocation, or overridden by options in
    the file to be compiled.
\end{description}

The following options are currently recognized by the compiler:
\begin{description}
\item[{\tt optimize}]\index{{\tt optimize}}
	When specified, the compiler tries to optimize the object code.
	In \version, this option optimizes predicate calls, among other
	features, so execution may be considerably faster for recursive
	loops.  However,
	due to the nature of the optimizations, the user may not be able to
	trace all calls to predicates in the program.  Also the Prolog code
	should be {\em static}.  In other words, the user is {\em not} allowed
	to alter the entry point of these compiled predicates by asserting new
	clauses.  As expected, the compilation phase will also be slightly
	longer.  For these reasons, the use of the {\tt optimize} option may
	not be suitable for the development phase, but is
	recommended once the code has been debugged.


\item[{\tt xpp\_on}]\index{{\tt xpp\_on}}\index{gpp}
  Filter the program through a preprocessor  before sending it to the XSB 
  compiler. By default (and for the XSB code itself), XSB uses GPP, a
  preprocessor developed by Denis Auroux (auroux@math.polytechnique.fr)
  that has high degree of compatibility with the C preprocessor, but is
  more suitable for Prolog syntax. In this case, the source code can
  include the usual C 
  preprocessor directives, such as \verb|#define|, \verb|#ifdef|, and
  \verb|#include|. This option can be specified both as a parameter to
  compile/2 and as part of the {\tt compiler\_options/1} directive inside
  the source file. See Appendix~\ref{gpp-man} for more details on GPP.

  When an \verb|#include "file"| statement is encountered,
  XSB directs GPP preprocessor to search for the files to include in the
  directories \verb|$XSB_DIR/emu| 
  %%$
  and \verb|$XSB_DIR/prolog_includes|.
  %%$
  \index{\tt xpp\_include\_dir}
  %%
  However, additional directories can be added to this search path by
  asserting into the predicate \verb|xpp_include_dir/1|, {\bf which should
    be imported from module} {\tt parse}.
  
  Note that when compiling XSB programs, GPP {\bf does not} search the
  current directory or the directory of the parent file that contains the
  include-directive. If you want the current directory to be searched, then
  the following statements must be executed:
%%
\begin{verbatim}
    :- import xpp_include_dir/1 from parse.
    :- assert(xpp_include_dir('.')).
\end{verbatim}
%%
  If you want Gpp to search the directory of the parent file, then
  {\tt xpp\_options/1} can be used (see below).

   Note: if you assert something into this predicate then you must also {\tt
    retractall(xpp\_include\_dir(\_))} after that or else subsequent Prolog
  compilations might not work correctly.

  XSB predefines the constant {\tt XSB\_PROLOG}, which can be used for
  conditional compilation. For instance, you can write portable program
  to run under XSB and and other prologs that support C-style
  preprocessing and use conditional compilation to account for the
  differences: 
  %%
  \begin{samepage}
  \begin{verbatim}
#ifdef XSB_PROLOG
    XSB-specific stuff
#else
    other Prolog's stuff
#endif
    common stuff
  \end{verbatim}
  \end{samepage}
  %%

  \index{xpp\_program}
  %%
  However, as mentioned earlier, XSB lets the user filter programs (except
  the programs that belong to XSB distribution) through any preprocessor
  the user wants. To this end, one only needs to assert the appropriate
  command into the predicate \verb|xpp_program|, which should be imported
  from module {\tt parse}. The command should not include the file
  name---XSB appends the name of the file to be compiled to the command
  supplied by the user. For instance, executing
  %%
  \begin{verbatim}
   :- assert(xpp_program('/usr/bin/m4 -E -G')).
  \end{verbatim}
  %%
  before calling the compiler will have the effect that the next XSB
  program passed to the compiler will be first preprocessed by the M4 macro
  package. Note that the XSB compiler automatically clears out the
  {\tt xpp\_program} predicate, so there is no need to tidy up each time.
  But this also means that if you need to compile several programs with a
  non-standard preprocessor then you must specify that non-standard
  preprocessor each time the program is compiled.
  
\item[{\tt xpp\_options}] \index{{\tt xpp\_options}} This dynamic predicate
  must be imported from module {\tt parse}.  If some atom is asserted into
  {\tt xpp\_options} then this atom is assumed to be the list of command
  line options to be used by the preprocessor (only the first asserted atom
  is ever considered). If this predicate is empty, then the default list of
  options is used (which is {\tt '-P -m -nostdinc -nocurinc'}, meaning: use
  Prolog mode and do not search the standard C directories and the
  directory of the parent file that contains the include-instruction).
  
  As mentioned earlier, when XSB invokes Gpp, it uses the option {\tt
    -nocurinc} so that Gpp will not search the directory of the parent file. 
  If a particular application requires that the parent file directory
  must be searched, then this can be accomplished by executing 
  {\tt assert(xpp\_options('-P -m -nostdinc'))}.
  
  Note: if you assert something into this predicate then you must also {\tt
    retractall(xpp\_options(\_))} after that or else subsequent Prolog
  compilations might not work correctly.
  
\item[{\tt xpp\_dump}] \index{{\tt xpp\_dump}}
  %%
  This causes XSB to dump the output from the GPP preprocessor into a file.
  If the file being compiled is named {\tt file.P} then the dump file is
  named {\tt file.P\_gpp}. This option can be included in the list of
  options in the {\tt compiler\_options/1} directive, but usually it is
  used for debugging, as part of the {\tt compile/2} predicate. If {\tt
    gpp\_dump} is specified directly in the file using {\tt
    compiler\_options/1} directive, then it should \emph{not} follow the
  {\tt gpp\_on} option in the list (or else it will be ignored).

\item[{\tt quit\_on\_error}] \index{{\tt quit\_on\_error}}
  This causes XSB to exit if compilation of a program end with an error.
  This option is useful when running XSB from a makefile, when it is
  necessary to stop the build process after an error has been detected. For
  instance, XSB uses this option during its own build process.


%-------------------------------
\index{tabling!compiler options}
%-------------------------------
\item[{\tt auto\_table}]\index{{\tt auto\_table}} When specified as a
  compiler option, the effect is as described in
  Section~\ref{tabling_directives}.  Briefly, a static analysis is made to
  determine which predicates may loop under Prolog's SLD evaluation.  These
  predicates are compiled as tabled predicates, and SLG evaluation is used
  instead.
%qkostis
\item[{\tt suppl\_table}]\index{{\tt suppl\_table}} The intention of this
  option is to direct the system to table for efficiency rather than
  termination.  When specified, the compiler uses tabling to ensure that no
  predicate will depend on more than three tables or EDB facts (as
  specified by the declaration {\tt edb} of
  Section~\ref{tabling_directives}).  The action of {\tt suppl\_table} is
  independent of that of {\tt auto\_table}, in that a predicate tabled by
  one will not necessarily be tabled by the other.  During compilation,
  {\tt suppl\_table} occurs after {\tt auto\_table}, and uses table
  declarations generated by it, if any.
%--------------------------------------
\index{specialisation!compiler options}
%--------------------------------------
\item[{\tt spec\_repr}]\index{{\tt spec\_repr}}
	When specified, the compiler performs specialisation of partially
	instantiated calls by replacing their selected clauses with the
	representative of these clauses, i.e. it performs {\em folding\/}
	whenever possible.  We note in general, the code replacement
	operation is not always sound; i.e. there
	are cases when the original and the residual program are not
	computationally equivalent.  The compiler checks for sufficient (but
	not necessary) conditions that guarantee computational equivalence.
	If these conditions are not met, specialisation is not performed
	for the violating calls.
\item[{\tt spec\_off}]\index{{\tt spec\_off}}
	When specified, the compiler does not perform specialisation of
	partially instantiated calls.
\item[{\tt unfold\_off}]\index{{\tt unfold\_off}}
	When specified, singleton sets optimisations are not performed
	during specialisation.  This option is necessary in \version\
	for the specialisation of {\tt table} declarations that select
	only a single chain rule of the predicate.
%qtls ??
\item[{\tt spec\_dump}]\index{{\tt spec\_dump}}
	Generates a {\tt module.spec} file, containing the result of
	specialising partially instantiated calls to predicates defined
	in the {\tt module} under compilation.  The result is in Prolog
	source code form.

%---------------------------------------------
\index{unification factoring!compiler options}
%---------------------------------------------
\item[{\tt ti\_dump}]\index{{\tt ti\_dump}}
	Generates a {\tt module.ti} file containing the result of applying
	unification factoring to predicates defined in the {\tt module}
	under compilation.  The result is in Prolog source code form.
	See page~\pageref{transformational_indexing} for more information
	on unification factoring.
\item[{\tt ti\_long\_names}]\index{{\tt ti\_long\_names}}
	Used in conjunction with {\tt ti\_dump}, generates names for
	predicates created by unification factoring that reflect the
	clause head factoring done by the transformation.
\comment{
\item[{\tt init\_var\_off}]\index{{\tt init\_var\_off}}
	When specified, the compiler will give a warning (instead of an
	error) upon finding that a potentially uninitialized variable is
	being used.  {\em Potentially uninitialized variables\/} are
	variables that appear in only one branch of an {\sf or} or an
	{\sf if-then-else} goal in the body, and, furthermore, are used
	after that goal.
	In certain clauses, the variable may always be initialized after
	the {\sf or} or the {\sf if-then-else} goal, because the execution 
	cannot continue through the path of the branch that does not initialize
	the variable.  In these cases, the {\tt init\_var\_off} option can be
	useful, though the user is cautioned against careless use of this
	option.

	{\sc Warning:} The object-file generated by the compiler when this
		option is used may not execute correctly (or even cause
		XSB to core dump!) if the variable is indeed
		uninitialized when used.
}
%---------------------------------------------
\index{mode analysis!compiler options}
%---------------------------------------------
\item[{\tt modeinfer}]\index{{\tt modeinfer}}
	This option is used to trigger mode analysis. For each module
	compiled,  the mode analyzer creates a  {\tt {\em module}.D} file
	that contains the mode information.

	{\sc Warning:}
	Occasionally, the analysis itself may take a long time. 
	As far as we have seen,
	the analysis times are longer than the rest of the compilation time
	only when the module contains recursive predicates of arity $\geq 10$.
	If the analysis takes an unusually long time
	(say, more than 4 times as long as the rest of the compilation)
	you may want to abort and restart compilation without {\tt modeinfer}.
	
\item[{\tt mi\_warn}]\index{{\tt mi\_warn}}
	During mode analysis, the {\tt .D} files corresponding to the
	imported modules are read in. The option {\tt mi\_warn} is used
	to generate warning messages if these {\tt .D} files are 
	outdated --- {\em i.e.}, older
	than the last modification time of the source files.

\item[{\tt mi\_foreign}] This option is used {\em only\/} when mode analysis
	is performed on XSB system modules. This option is
	needed when analyzing {\tt standard} and {\tt machine} in
	{\tt syslib}.


% tls mention that the following are for hackers ...
\item[{\tt sysmod}] Mainly used by developers when compiling system
	modules. If specified, standard predicates (listed in
	Appendix~\ref{standard_predicates}) are automatically
	available for use only if they are primitive predicates (see
	the file {\tt syslib/machine.P} for a current listing of such
	predicates. When compiling in this
	mode, non primitive standard predicates must be explicitly
	imported from the appropriate system module.
\item[{\tt verbo}] Compiles the files (modules) specified in ``verbose'' mode, 
	printing out information about the progress of the compilation of each 
	predicate.
\item[{\tt profile}] This option is usually used when modifying the
	XSB compiler.  When specified, the compiler prints out
	information about the time spent in each phase of the
	compilation process.

\item[{\tt asm\_dump, compile\_off}] Generates a textual representation of 
	the SLG-WAM assembly code and writes it into the file {\tt module.A}
	where {\tt module} is the name of the module (file) being compiled.  
	
	{\sc Warning:} This option was created for compiler debugging and is
		not intended for general use.  There might be cases where
		compiling a module with these options may cause generation
		of an incorrect {\tt .A} and {\tt .O} file.  In such cases,
		the user can see the SLG-WAM instructions that are
		generated for a module by compiling the module as usual and
		then using the {\tt -d module.O} command-line option of the
		XSB emulator (see Section~\ref{emulator_options}).
\item[{\tt index\_off}] When specified, the compiler does not generate indices
	for the predicates compiled.  
\end{description}


\subsection{Specialisation}\label{specialisation}
\index{Compiler!specialisation}\index{specialisation!Compiler}
%=============================================================

From Version 1.4.0 on, the XSB compiler automatically performs
specialisation of partially instantiated calls.  Specialisation can be
thought as a source-level program transformation of a program to a
residual program in which partially instantiated calls to predicates
in the original program are replaced with calls to specialised versions
of these predicates.  The expectation from this process is that the
calls in the residual program can be executed more efficiently that
their non-specialised counterparts.  This expectation is justified
mainly because of the following two basic properties of the
specialisation algorithm:
\begin{description}
\item[Compile-time Clause Selection] The specialised calls of the
	residual program  directly select (at compile time) a subset
	containing only the clauses that the corresponding calls of the
	original program would otherwise have to examine during their
	execution (at run time).  By doing so, laying down unnecessary
	choice points is at least partly avoided, and so is the need to
	select clauses through some sort of indexing.
\item[Factoring of Common Subterms] Non-variable subterms of partially
	instantiated calls that are common with subterms in the heads
	of the selected clauses are factored out from these terms
	during the specialisation process.  As a result, some head
	unification ({\tt get\_*} or {\tt unify\_*}) and some argument
	register ({\tt put\_*}) WAM instructions of the original
	program become unnecessary.  These instructions are eliminated
	from both the specialised calls as well as from the specialised
	versions of the predicates.
\end{description}
Though these properties are sufficient to get the idea behind
specialisation, the actual specialisation performed by the XSB
compiler can be better understood by the following example.  The
example shows the specialisation of a predicate that checks if a list
of HiLog terms is ordered:
\begin{center}
\tt
\begin{tabular}{ccc}
\begin{tabular}{l}
ordered([]). \\
ordered([X]). \\
ordered([X,Y|Z]) :- \\
\ \ \ \ X @=< Y, ordered([Y|Z]). 
\end{tabular}
& $\longrightarrow$ &
\begin{tabular}{l}
ordered([]). \\
ordered([X]). \\
ordered([X,Y|Z]) :- \\
\ \ \ \ X @=< Y, \_\$ordered(Y, Z). \\
\\
:- index \_\$ordered/2-2. \\
\_\$ordered(X, []). \\
\_\$ordered(X, [Y|Z]) :- \\
\ \ \ \ X @=< Y, \_\$ordered(Y, Z).
\end{tabular}
\end{tabular}
\end{center}
The transformation (driven by the partially instantiated call
{\tt ordered([Y|Z])}) effectively allows predicate {\tt ordered/2}
to be completely deterministic (when used with a proper list as its
argument), and to not use any unnecessary heap-space for its
execution.  We note that appropriate {\tt :- index} directives are
automatically generated by the XSB compiler for all specialised
versions of predicates.

The default specialisation of partially instantiated calls is without
any folding of the clauses that the calls select.  Using the {\tt
spec\_repr} compiler option (see Section~\ref{compiler_options})
specialisation with replacement of the selected clauses with the
representative of these clauses is performed.  Using this compiler
option, predicate {\tt ordered/2} above would be specialised as follows:
%%
\begin{center}
\begin{minipage}{4.1in}
\begin{verbatim}
ordered([]).
ordered([X|Y]) :- _$ordered(X, Y).

:- index _$ordered/2-2.
_$ordered(X, []).
_$ordered(X, [Y|Z]) :- X @=< Y, _$ordered(Y, Z).
\end{verbatim}
\end{minipage}
\end{center}
%%$
%%
We note that in the presense of cuts or side-effects, the code
replacement operation is not always sound, i.e.  there are cases when
the original and the residual program are not computationally equivalent
(with respect to the answer substitution semantics).  The compiler
checks for sufficient (but not necessary) conditions that guarantee
computational equivalence, and if these conditions are not met,
specialisation is not performed for the violating calls.

The XSB compiler prints out messages whenever it specialises
calls to some predicate.  For example, while compiling a file
containing predicate {\tt ordered/1} above, the compiler would print
out the following message:
\begin{center}
{\tt	\% Specialising partially instantiated calls to ordered/1}
\end{center}
The user may examine the result of the specialisation transformation
by using the {\tt spec\_dump} compiler option
(see Section~\ref{compiler_options}).

Finally, we have to mention that for technical reasons beyond the scope of
this document, specialisation cannot be transparent to the user; predicates
created by the transformation do appear during tracing.


\subsection{Compiler Directives}\label{compiler_directives}
\index{Compiler!directives}\index{directives!Compiler}
%=====================================================

The following compiler directives are recognized in \version\ of XSB
\footnote{Any parallelisation directives ({\tt parallel}) are simply
ignored by the compiler, but do not result in syntax errors to enhance
compatibility with various other earlier versions of PSB-Prolog.}.

\subsubsection{Mode Declarations}\label{mode_declarations}
\index{modes!directives}\index{directives!modes}
%-----------------------------------------------------

The XSB compiler accepts {\tt mode} declarations of the form:

\demo{:- mode $ModeAnnot_1, \ldots, ModeAnnot_n$.}

\noindent
where each $ModeAnnot$ is a {\em mode annotation\/} (a {\em term
indicator\/} whose arguments are elements of the set {\tt
$\{$+,-,\#,?$\}$}).  From Version 1.4.1 on, {\tt mode} directives are
used by the compiler for tabling directives, a use which differs from
the standard use of modes in Prolog systems\footnote{The most common
uses of {\tt mode} declarations in Prolog systems are to reduce the
size of compiled code, or to speed up a predicate's execution.}.  See
Section~\ref{tabling_directives} for detailed examples.

Mode annotations have the following meaning:
\begin{description}
\item[{\tt +}]
	This argument is an input to the predicate.  In every invocation
	of the predicate, the argument position must contain a non-variable
	term.  This term may not necessisarily be ground, but the 
	predicate is guaranteed not to alter this argument).

	\demo{:- mode see(+), assert(+).}
\item[{\tt -}]
	This argument is an output of the predicate.  In every
	invocation of the predicate the argument
	position {\em will always be a variable\/} (as opposed to 
	the {\tt \#} annotation below).
	This variable is unified with the value returned by the predicate.
	We note that Prolog does not enforce the requirement that output
	arguments should be variables; however, output unification is not
	very common in practice.

	\demo{:- mode cputime(-).}
\item[{\tt \#}]
	This argument is either:
	\begin{itemize}
	\item	An output argument of the predicate for which a non-variable
		value may be supplied for this argument position.  If such a
		value is supplied, the result in this position is unified with
		the supplied supplied value.  The predicate fails if this
		unification fails.  If a variable term is supplied, the
		predicate succeeds, and the output variable is unified with
		the return value.

		\demo{:- mode '='(\#,\#).}
	\item	An input/output argument position of a predicate that has
		only side-effects (usually by further instantiating that
		argument).  The {\tt \#} symbol is used to denote the $\pm$
		symbol that cannot be entered from the keyboard.
	\end{itemize}
\item[{\tt ?}]
	This argument does not fall into any of the above categories. 
        Typical cases would be the following:
	\begin{itemize}
	\item	An argument that can be used both as input and as output
		(but usually not with both uses at the same time).

		\demo{:- mode functor(?,?,?).}
	\item	An input argument where the term supplied can be a variable
		(so that the argument cannot be annotated as {\tt +}), or is
		instantiated to a term which itself contains uninstantiated
		variables, but the predicate is guaranteed {\em not\/} to
		bind any of these variables.

		\demo{:- mode var(?), write(?).}
	\end{itemize}
\end{description}
We try to follow these mode annotation conventions throughout this manual.

Finally, we warn the user that {\tt mode} declarations can be error-prone,
and since errors in mode declarations do not show up while running the
predicates interactively, unexpected behaviour may be witnessed in compiled
code, optimised to take modes into account (currently not performed by
\ourprolog).  However, despite this danger, {\tt mode} annotations can be
a good source of documentation, since they express the programmer's
intention of data flow in the program.


\subsubsection{Tabling Directives}\label{tabling_directives}
\index{tabling!directives}\index{directives!tabling}
%-----------------------------------------------------
\index{{\tt auto\_table}}
Memoization is often necessary to ensure that programs terminate, and
can be useful as an optimization strategy as well.  The underlying
engine of XSB is based on SLG, a memoization strategy, which,
in our version, maintains a table of calls and their answers for each
predicate declared as {\em tabled}.  Predicates that are not declared
as tabled execute as in Prolog, eliminating the expense of tabling
when it is unnecessary.

The simplest way to use tabling is to include the directive

\demo{:- auto\_table.}

\noindent
anywhere in the source file.  {\tt auto\_table} declares predicates
tabled so that the program will terminate.

To understand precisely how {\tt auto\_table} does this, it is
necessary to mention a few properties of SLG.  For programs which have
no function symbols, or where function symbols always have a limited
depth, SLG resolution ensures that any query will terminate after it
has found all correct answers.  In the rest of this section, we
restrict consideration to such programs.

Obviously, not all predicates will need to be tabled for a program to
terminate.  The {\tt auto\_table} compiler directive tables only those
predicates of a module which appear to static analysis to contain an
infinite loop, or which are called directly through {\tt tnot/1}.  It
is perhaps more illuminating to demonstrate these conditions through
an example rather than explaining them.  For instance, in the program.

%tls commented out minipage because latex was formatting badly,
\begin{center}
%\begin{minipage}{3in}
\begin{verbatim}
:- auto_table. 

p(a) :- s(f(a)). 

s(X) :- p(f(a)).

r(X) :- q(X,W),r(Y).

m(X) :- tnot(f(X)).

:- mode ap1(-,-,+).
ap1([H|T],L,[H|L1]) :- ap1(T,L,L1).

:- mode ap(+,+,-).
ap([],F,F).
ap([H|T],L,[H|L1]) :- ap(T,L,L1).

mem(H,[H|T]).
mem(H,[_|T]) :- mem(H,T).
\end{verbatim}
%\end{minipage}
\end{center}

\noindent
The compiler prints out the messages
\begin{verbatim}
% Compiling predicate s/1 as a tabled predicate
% Compiling predicate r/1 as a tabled predicate
% Compiling predicate m/1 as a tabled predicate
% Compiling predicate mem/2 as a tabled predicate
\end{verbatim}

Terminating conditions were detected for {\tt ap1/3} and {\tt ap/3}, but
not for any of the other predicates.

{\tt auto\_table} gives an approximation of tabled programs which we
hope will be useful for most programs.  The minimal set of tabled
predicates needed to insure termination for a given program is
undecidible.  
\comment{
Practically, refining the set of tabled predicates
deduced by {\tt auto\_table} is still an open research problem.
}
It should be noted that the presence of meta-predicates such as {\tt
call/1} makes any static analysis useless, so that the {\tt
auto\_table} directive should not be used in such cases.

Predicates can be explicitly declared as tabled as well, through the
{\tt table/1}.  When {\tt table/1} is used, the directive takes the
form

\demo{:- table(F/A).}

\noindent
where {\tt F} is the functor of the predicate to be tabled, and {\tt A} its
arity.  

\index{{\tt suppl\_table}}
Another use of tabling is to filter out redundant solutions for
efficiency rather than termination.  In this case, suppose that the
directive {\tt edb/1} were used to indicate that certain predicates were
likely to have a large number of clauses.  Then the action of the declaration
{\tt :- suppl\_table} in the program:
\begin{verbatim}
:- edb(r1/2).
:- edb(r2/2).
:- edb(r3/2).

:- suppl_table.

join(X,Z):- r1(X,X1),r2(X1,X2),r3(X2,Z).
\end{verbatim}
would be to table {\tt join/2}.  The {\tt suppl\_table} directive is
the XSB analogue to the deductive database optimization, {\em
supplementary magic templates} \cite{BeRa91}.  {\tt suppl\_table/0} is
shorthand for {\tt suppl\_table(2)} which tables all predicates
containing clauses with two or more {\tt edb} facts or tabled
predicates.  By specifying {\tt suppl\_table(3)} for instance, only
predicates containing clauses with three or more {\tt edb} facts or
tabled predicates would be tabled.  This flexibility can prove useful
for certain data-intensive applications.


\subsubsection{Indexing Directives}\label{indexing_directives}
\index{indexing!directives}\index{directives!indexing}
%-------------------------------------------------------------

The XSB compiler usually generates an index on the principal 
functor of the first argument of a predicate.  Indexing on the appropriate 
argument of a predicate may significantly speed up its execution time.  
In many cases the first argument of a predicate may not be the most
appropriate argument for indexing and changing the order of arguments
may seem unnatural.  In these cases, the user may generate an index
on any other argument by means of an indexing directive.  This is a
directive of the form:

\demo{:- index Functor/Arity-IndexArg.}

\noindent
indicating that an index should be created for predicate 
{\tt Functor}/{\tt Arity} on its ${\tt IndexArg}^{\rm th}$ argument.
One may also use the form:

\demo{:- index(Functor/Arity, IndexArg, HashTableSize).}

\noindent
which allows further specification of the size of the hash table to use for
indexing this predicate if it is a {\em dynamic} (i.e., asserted) predicate.
For predicates that are dynamically loaded, this directive can be used to
specify indexing on more than one argument, or indexing on a combination
of arguments (see its description on page~\pageref{index_dynamic}).
For a compiled predicate the size of the hash table is computed automatically,
so {\tt HashTableSize} is ignored.

All of the values {\tt Functor}, {\tt Arity}, {\tt IndexArg} (and possibly
{\tt HashTableSize}) should be ground in the directive.  More specifically,
{\tt Functor} should be an atom, {\tt Arity} an integer in the range 0..255,
and {\tt IndexArg} an integer between 0 and {\tt Arity}.  If {\tt IndexArg}
is equal to~0, then no index is created for that predicate. An {\tt index}
directive may be placed anywhere in the file containing the predicate it
refers to.

As an example, if we wished to create an index on the third argument 
of predicate {\tt foo/5}, the compiler directive would be:

\demo{:- index foo/5-3.}


\subsubsection{Unification Factoring}\label{transformational_indexing}
\index{indexing!transformational}
When the clause heads of a predicate have portions of arguments common
to several clauses, indexing on the principal functor of one argument
may not be sufficient.  Indexing may be improved in such cases by the
use of unification factoring.  Unification Factoring is a program
transformation that ``factors out'' common parts of clause heads,
allowing differing parts to be used for indexing, as illustrated by
the following example:
\begin{center}
\tt
\begin{tabular}{ccc}
\begin{tabular}{l}
p(f(a),X) :- q(X). \\
p(f(b),X) :- r(X).
\end{tabular}
& $\longrightarrow$ &
\begin{tabular}{l}
p(f(X),Y) :- \_\$p(X,Y). \\
\_\$p(a,X) :- q(X). \\
\_\$p(b,X) :- r(X).
\end{tabular}
\end{tabular}
\end{center}
The transformation thus effectively allows $p/2$ to be indexed
on atoms $a/0$ and $b/0$.  Unification Factoring is transparent
to the user; predicates created by the transformation are internal
to the system and do not appear during tracing.

The following compiler directives control the use of unification
factoring:\footnote{Unification factoring was once called
transformational indexing, hence the abbreviation {\tt ti} in the
compiler directives}.
\begin{description}
\item[{\tt :- ti(F/A).}] Specifies that predicate $F/A$ should be
	compiled with unification factoring enabled.
\item[{\tt :- ti\_off(F/A).}] Specifies that predicate $F/A$ should be
	compiled with unification factoring disabled.
\item[{\tt :- ti\_all.}] Specifies that all predicates defined in the
	file should be compiled with unification factoring enabled.
\item[{\tt :- ti\_off\_all.}] Specifies that all predicates defined in
	the file should be compiled with unification factoring disabled.
\end{description}
By default, higher-order predicates (more precisely, predicates named
{\it apply\/} with arity greater than 1) are compiled with unification
factoring enabled.  It can be disabled using the {\tt ti\_off}
directive.  For all other predicates, unification factoring must be
enabled explicitly via the {\tt ti} or {\tt ti\_all} directive.  If
both {\tt :- ti(F/A).} ({\tt :- ti\_all.}) and {\tt :- ti\_off(F/A).}
({\tt :- ti\_off\_all.}) are specified, {\tt :- ti\_off(F/A).} ({\tt
:- ti\_off\_all.}) takes precedence.  Note that unification factoring
may have no effect when a predicate is well indexed to begin
with.  For example, unification factoring has no effect on the
following program:
\begin{center}
\tt
\begin{tabular}{l}
p(a,c,X) :- q(X). \\
p(b,c,X) :- r(X).
\end{tabular}
\end{center}
even though the two clauses have $c/0$ in common.  The user may
examine the results of the transformation by using the {\tt ti\_dump}
compiler option (see Section~\ref{compiler_options}).

\subsubsection{Other Directives} \label{other-directives}
%==============================================

XSB has other directives not found in other Prolog systems.

\begin{description}
\desc{:- hilog $atom_1, \ldots, atom_n$.}
	Declares symbols $atom_1$ through $atom_n$ as HiLog symbols.
	The {\tt hilog} declaration should appear {\em before} any use of
	the symbols.  See Chapter~\ref{Syntax} for a purpose of this
 	declaration.
\desc{:- ldoption($Options$).}
        This directive is only recognized in the header file ({\tt .H} file) 
	of a foreign module. See the chapter {\it Foreign Language
Interface} in Volume 2 for its explanation.
\desc{:- compiler\_options($OptionsList$).}
	Indicates that the compiler options in the list $OptionsList$
	should be used to compile this file.  This must appear at the
	beginning of the file.  Thes options will override any others,
	including those given in the compilation command.  The options
	may be optionally prefixed with \verb|+| or \verb|-| to
	indicate that they should be set on or off.  (No prefix
	indicates the option should be set on.)

\end{description}

\subsection{Inline Predicates}\label{inline_predicates}
\index{Compiler!inlines}\index{inlines!Compiler}
%======================================================

{\em Inline predicates} represent ``primitive'' operations in the
WAM.  Calls to inline predicates are compiled into a sequence of WAM
instructions in-line, i.e. without actually making a call to the
predicate.  Thus, for example, relational predicates (like {\tt >/2},
{\tt >=/2}, etc.) compile to, essentially, a subtraction followed by
a conditional branch.  Inline predicates are expanded specially by
the compiler and thus {\em cannot be redefined by the user without
changing the compiler}.  The user does not need to import these
predicates from anywhere.  There are available no matter what options
are specified during compiling.

Table~\ref{inlinepredicatetable} lists the inline predicates of
XSB \version.  Those predicates that start with \verb|_$|
are internal predicates that are also expanded in-line during
compilation.

\begin{table}[htbp]\centering{\tt
\begin{tabular}{lllll}
\verb|'='/2|    &\verb|'<'/2|	&\verb|'=<'/2|  &\verb|'>='/2| &\verb|'>'/2| \\
\verb|'=:='/2|  &\verb|'=\='/2|	&is/2           &\verb|'@<'/2| &\verb|'@=<'/2|\\
\verb|'@>'/2|	&\verb|'@>='/2|	&\verb|'=='/2|	&\verb|'\=='/2|&fail/0  \\
true/0		&var/1		&nonvar/1	&halt/0	       &'!'/0   \\
'\_\$cutto'/1	&'\_\$savecp'/1	&'\_\$builtin'/1
\end{tabular}}
\caption{The Inline Predicates of \ourprolog}\label{inlinepredicatetable}
\end{table}

We warn the user to be very cautious when defining predicates whose functor
starts with {\tt \_\$}  since the names of these predicates may interfere with
some of \ourprolog's internal predicates.  The situation may be particularly
severe for predicates like {\tt '\_\$builtin'/1} that are treated specially
by the XSB compiler.





%%% Local Variables: 
%%% mode: latex
%%% TeX-master: "manual1"
%%% End: 
