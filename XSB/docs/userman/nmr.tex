\chapter{Libraries for Non-Monotonic and Quantitative Reasoning} \label{library_utilities}
%====================================================

This chapter describes libraries, preprocessors and meta-interpreters
for non-monotonic reasoning.  We note that the implementation of FLORA
\index{FLORA} provides sophisticated capabilities for non-monotonic
reasoning and is described in a separate manual obtainable from {\tt
xsb.sourceforge.net}.


\section{XNMR}

\subsection{Introduction}

The XNMR package attempts to provide an unified environment for
several forms of non-monotonic reasoning. It consists of a new
top-loop for XSB which benefits from the integration of XSB with the
stable model generator SMODELS \cite{NiSi97} and XSB to compute
stable and answer set semantics for logic programs.

\paragraph*{The XNMR Top-Loop}

The XNMR top-loop is the interface between XNMR and the user. It adds
extra functionalities to the standard top-loop of XSB to provide means
to compute different semantics  for Prolog programs.

There are two distinct modes for the XNMR top-loop. The \texttt{nmr}
mode allows for the computation of (partial) stable model semantics
and answerset semantics of normal logic programs. The
\texttt{answerset} mode is designed to compute the a non-explosive
partial answerset semantics of logic programs with explicit negation.

\subsection{The \texttt{nmr} mode}

The \texttt{nmr} mode is the default mode of XNMR. It provides support
for computing the stable model and answerset semantics of normal logic
programs. This mode can be identified by the \verb#nmr| ?-# prompt. If
you change the mode to a different one, you can come back to
\texttt{nmr} mode using the predicate \texttt{set\_nmr\_mode/0}.

The prompt works much like a standard Prolog top-level. The difference
arises when the user issues a query which has an answer that depends
on undefined goals, i.e. a non-empty delay list. In this case, the
user can either ignore the delay list and assume the query (with the
given instantiations for the variables) is undefined under the
well-founded semantics, or it can ask XNMR to call SMODELS to compute
different semantics for the residual program.

Three options are available in such cases: 

\begin{enumerate}

\item[`s'] computes and prints all (partial) stable models of the residual
  program, one at a time;

\item[`t'] computes and prints the (partial) stable models of the residual
  program where the query is true, one at a time;

\item[`a'] computes the answerset semantics of the residual program,
  ie. it says \verb#yes# if the query is true in all stable models of the
  residual program; in this mode, the - symbol, prepended to
  predicates, denote their negative counterpart.

%{\sc TLS: what symbol are you using for explicit negation?}
% -p is the explicit negative counterpart of p  --lfcastro

\end{enumerate}

Figure~\ref{fig:stable} shows the `s' operator in action. The
`exwfs.P' program is shown in figure~\ref{fig:exwfs}. 

At the ``more answers'' prompt, when there is a delay list, the user
can type \verb#s<cr># and obtain the stable models of the residual program.
Note that one model is computed and printed at each time. After a
model is printed, the user can ask for the next one by typing \verb#;<cr>#,
or she can give up by pressing only \verb#<cr>#.

\begin{figure}
\label{fig:exwfs}
\centering
\scriptsize
%\begin{boxedverbatim} TLS: manual did not have include file.
\begin{verbatim}
:- table win/1, p/0, q/0, r/0.

win(X) :- move(X,Y), tnot(win(Y)).

move(a,b).
move(b,a).
move(b,c).
move(c,d).

p :- tnot(q).
q :- tnot(p).
r :- p.
r :- q.
\end{verbatim}
%\end{boxedverbatim}
\caption{The exwfs.P program.}
\end{figure}


\begin{figure}
\label{fig:stable}
\centering
\scriptsize
%\begin{boxedverbatim} TLS: manual did not have include file.
\begin{verbatim}
> xsb xnmr
[xsb_configuration loaded]
[sysinitrc loaded]
[packaging loaded]
[sModels loaded]

XSB Version 2.01 (Gouden Carolus) of August 20, 1999
[i686-pc-linux-gnu; mode: optimal; engine: chat; scheduling: batched]
nmr| ?- [exwfs].
[exwfs loaded]

yes
nmr| ?- win(X).

X = c ? ;

X = b
DELAY LIST = [tnot(win(a))] ? s

Stable Models: 
  {win(a)} ? ;

  {win(b)} ? ;
  no

X = a
DELAY LIST = [tnot(win(b))] ? s

Stable Models: 
  {win(b)} ? ;

  {win(a)} ? ;
  no

no
nmr| ?- 
\end{verbatim}
%\end{boxedverbatim}
\caption{Using the `s' operator}
\end{figure}


Figure~\ref{fig:tstable} shows the use of the `t' option. Note that
the models computed with this option form a subset of the models
computed with the `s' option, namely those models where the query is
true.

\begin{figure}
\label{fig:tstable}
\centering
\scriptsize
%\begin{boxedverbatim} TLS: manual did not have include file.
\begin{verbatim}
lfmobile:~/Programs/Prolog/WFS> xsb xnmr
[xsb_configuration loaded]
[sysinitrc loaded]
[packaging loaded]
[sModels loaded]

XSB Version 2.01 (Gouden Carolus) of August 20, 1999
[i686-pc-linux-gnu; mode: optimal; engine: chat; scheduling: batched]
nmr| ?- [exwfs].
[exwfs loaded]

yes
nmr| ?- win(X).

X = c ? ;

X = b
DELAY LIST = [tnot(win(a))] ? t

Stable Models: 
  {win(b)} ? ;
  no

X = a
DELAY LIST = [tnot(win(b))] ? t

Stable Models: 
  {win(a)} ? ;
  no

no
nmr| ?- 
\end{verbatim}
%\end{boxedverbatim}
\caption{Using the `t' operator}
\end{figure}

Finally, figure~\ref{fig:answerset} shows one application of the `a'
operation, to compute the answerset semantics of the residual
program. You can see that, when we use `s' to compute the stable
models of the residual program, all computed models have the query as
true. Therefore, it's clear that the query is true under the answerset
semantics of the residual program. 

\begin{figure}
\label{fig:answerset}
\centering
\scriptsize
%\begin{boxedverbatim} TLS: manual did not have include file.
\begin{verbatim}
[xsb_configuration loaded]
[sysinitrc loaded]
[packaging loaded]
[sModels loaded]

XSB Version 2.01 (Gouden Carolus) of August 20, 1999
[i686-pc-linux-gnu; mode: optimal; engine: chat; scheduling: batched]
nmr| ?- [exwfs].
[Compiling ./exwfs]
[exwfs compiled, cpu time used: 0.2490 seconds]
[exwfs loaded]

yes
nmr| ?- r.

DELAY LIST = [q]
DELAY LIST = [p] ? s

Stable Models: 
  {r; q} ? ;

  {r; p} ? ;
  no

no
nmr| ?- r.

DELAY LIST = [q]
DELAY LIST = [p] ? a
  yes

no
nmr| ?- 
\end{verbatim}
%\end{boxedverbatim}
\caption{Using the `a' option}
\end{figure}

\subsection{The \texttt{answerset} mode}

The \texttt{answerset} mode is entered using the predicate
\texttt{set\_answerset\_mode/0}. This mode has been designed to deal
with logic programs with explicit negation.

Explicit negated predicates are represented by prepending their names
with \verb#-#. For example, if \verb#p# is a predicate, its
explicitly-negated counterpart is \verb#-p#.

In this mode, when a query is issued, XNMR computes the answerset
semantics for all possible instantiations of the given query, and says
\verb#yes# to all instantiations which are true under this semantics. 

After that, XNMR performs the same computation for the
explicitly-negated counterpart of the query. It, then, says \verb#no# to
all computed instantiations. 

It is assumed that the knowledge about all other instantiations not
shown is unknown.



\section{Extended Logic Programs}  \label{library_utilities:wfsx}
\index{WFSX}
\index{negation!explicit negation}
As explained in the section {\it Using Tabling in XSB}, XSB can
compute normal logic programs according to the well-founded semantics.
In fact, XSB can also compute {\em Extended Logic Programs}, which
contain an operator for explicit negation (written using the symbol
{\tt -} in addition to the negation-by-failure of the well-founded
semantics (\verb|\+| or {\tt not}).  Extended logic programs can be
extremely useful when reasoning about actions, for model-based
diagnosis, and for many other uses \cite{AlPe95}.  The library, {\sf
slx} provides a means to compile programs so that they can be executed
by XSB according to the {\em well-founded semantics with explicit
negation} \cite{ADP95}.  Briefly, WFSX is an extension of the
well-founded semantics to include explicit negation and which is based
on the {\em coherence principle} in which an atom is taken to be
default false if it is proven to be explicitly false, intuitively:
\[
-p \Rightarrow not\ p.
\]

This section is not intended to be a primer on extended logic
programming or on WFSX semantics, but we do provide a few sample
programs to indicate the action of WFSX.  Consider the program
{\small 
\begin{verbatim}
s:- not t.

t:- r.
t.

r:- not r.
\end{verbatim}
}
If the clause {\tt -t} were not present, the atoms {\tt r, t, s} would
all be undefined in WFSX just as they would be in the well-founded
semantics.  However, when the clause {\tt t} is included, {\tt t}
becomes true in the well-founded model, while {\tt s} becomes false.
Next, consider the program
{\small 
\begin{verbatim}
s:- not t.

t:- r.
-t.

r:- not r.
\end{verbatim}
}
In this program, the explicitly false truth value for {\tt t} obtained
by the rule {\tt -t} overrides the undefined truth value for {\tt t}
obtained by the rule {\tt t:- r}.  The WFSX model for this program
will assign the truth value of {\tt t} as false, and that of {\tt s}
as true.  If the above program were contained in the file {\tt
test.P}, an XSB session using {\tt test.P} might look like the
following:
{\small
\begin{verbatim}
              > xsb
              
              | ?- [slx].
              [slx loaded]
            
              yes
              | ?- slx_compile('test.P').
              [Compiling ./tmptest]
              [tmptest compiled, cpu time used: 0.1280 seconds]
              [tmptest loaded]
            
              | ?- s.
              
              yes
              | ?- t.

              no
              | ?- naf t.
              
              yes
              | ?- r.

              no
              | ?- naf r.
              
              no
              | ?- und r.
              
              yes
\end{verbatim}
}
In the above program, the query {\tt ?- t.} did not succeed,  because
{\tt t} is false in WFSX: accordingly the query {\tt naf t} did
succeed, because it is true that t is false via negation-as-failure,
in addition to {\tt t} being false via explicit negation.  Note that
after being processed by the SLX preprocessor, {\tt r} is undefined
but does not succeed, although {\tt und r} will succeed.

We note in passing that programs under WFSX can be paraconsistent.
For instance in the program.
{\small
\begin{verbatim}
              p:- q.

              q:- not q.
              -q.
\end{verbatim}
}
both {\tt p} and {\tt q} will be true {\em and} false in the WFSX
model.  Accordingly, under SLX preprocessing, both {\tt p} and {\tt
naf p} will succeed.

\begin{description}
\ournewitem{slx\_compile(+File)}{slx}
Preprocesses and loads the extended logic program named {\tt File}.
Default negation in {\tt File} must be represented using the operator
{\tt not} rather than using {\tt tnot} or \verb|\+|.  If {\tt L} is an
objective literal (e.g. of the form $A$ or $-A$ where $A$ is an atom),
a query {\tt ?- L} will succeed if {\tt L} is true in the WFSX model,
{\tt naf L} will succeed if {\tt L} is false in the WFSX model, and
{\tt und L} will succeed if {\tt L} is undefined in the WFSX model.
\end{description}


\section{Generalized Annotated Programs}  \label{library_utilities:gap}
\index{Generalized Annotated Programs}

Generalized Annotated Programs (GAPs) \cite{KiSu92} offer a powerful
computational framework for handling paraconsistentcy and quantitative
information within logic programs.  The tabling of XSB is well-suited
to implementing GAPs, and the gap library provides a meta-interpreter
that has proven robust and efficient enough for a commercial
application in data mining.  The current meta-interpreter is limited
to range-restricted programs.

A description of GAPs along with full documentation for this
meta-interpreter is provided in \cite{Swif99a} (currently also
available at {\tt http://www.cs.sunysb.edu/$\sim$tswift}).  Currently, the
interface to the GAP library is through the following call.

\begin{description}
\ournewitem{meta(?Annotated\_atom)}{gap} 
%
If {\tt Annotated\_atom} is of the form {\tt
Atom:[Lattice\_type,Annotation]} the meta-interpreter computes bindings
for {\tt Atom} and {\tt Annotation} by evaluating the program
according to the definitions provided for {\tt Lattice\_type}.
\end{description}


