\chapter{Constraint Packages}
%===========================

Constraint packages are an important part of modern logic programming,
but approaches to constraints differ both in their semantics and in
their implementation.  At a semantic level, {\em Constraint Logic
  Programming} associates constraints with logical variables, and
attempts to determine solutions that are inconsistent with or entailed
by those constaints.  At an implementational level, the constraints
can either be manipulated by accessing attributed variables or by
adding {\em constraint handling rules} to a program.  The former
approach of attributed variables can be much more efficient than
constraint handling rules (which are themselves implemented through
attributed variables) but are much more difficult to use than
constraint handling rules.  These variable-based approaches differ
from that of {\em Answer Set Programming} in which a constraint
problem is formulated as a set of rules, which are consistent if a
stable model can be constructed for them.

XSB supports all of these approaches.  Two packages based on
attributed variables are presented in this chapter: CLP(R) and the
{\tt bounds} package, which provides a simple library for handling
finite domains.  XSB's CHR package is described in Chapter~\ref{chr},
and XSB's Answer Set Programming Package, {\tt XASP} is described in
Chapter~\ref{xasp}.

Before describing the individual packages, we note that these packages
can be freely used with variant tabling, the mechanisms for which
handle attributed variables.  However in \version{}, calling a
predicate $P$ that is tabled using call subsumption will raise an
error if the call to $P$ contains any constrained variables
(attributed variables).

\section{{\tt clpr}: The CPL(R) package} \label{sec:clpr}
\index{Prologs!SWI}
%
The CLP(R) library supports solutions of linear equations and
inequalities over the real numbers and the lazy treatment of nonlinear
equations~\footnote{The CLP(R) package is based on the clpqr package
  included in SWI Prolog version 5.6.49.  This package was originally
  written by Christian Holzbaur and ported to SWI by Leslie De Konick.
  Terrance Swift ported the package to XSB and and wrote this XSB
  manual section.}.  In displaying sets of equations and disequations,
the library removes redundancies, performs projections, and provides
for linear optimization.  The goal of the XSB port is to provide the
same CLP(R) functionality as in other platforms, but also to allow
constraints to be used by tabled predicates.  This section provides a
general introduction to the CLP(R) functionality available in XSB, for
further information on the API described in Section~\ref{sec:clpr-api}
see {\tt http://www.ai.univie.ac.at/clpqr}, or the Sicstus Prolog
manual (the CLP(R) library should behave similarly on XSB and Sicstus
at the level of this API).

The {\tt clpr} package may be loaded by the command {\tt [clpr]}.
Loading the package imports exported predicates from the various files
in the {\tt clpr} package into {\tt usermod} (see Volume 1, Section
3.3) so that they may be used in the interpreter.  Modules that use
the exported predicates need to explicitly import them from the files
in which they are defined (e.g. {\tt bv}, as shown below).

XSB's tabling engine supports the use of attributed variables
(Section~\ref{sec:attributed-variables}), which in turn have been used
to port real constraints to XSB under the CLP(R) library of Christian
Holzbauer \cite{Holz95}.  Constraint equations are represented using
the Prolog syntax for evaluable functions (Volume 1, Section 6.2.1).
Formally:

{\it
\begin{tabbing}
12 \= 12345678901234567890 \= 12345678901234567890 \=	\kill
\> ConstraintSet --$>$   \>  C $|$ C {\tt ,} C   \\
\\
\> C --$>$ \> Expr {\tt =:=} Expr \> {\rm equation} \\
\>    \> $|$ Expr {\tt =} Expr \> {\rm equation} \\
\>    \> $|$ Expr {\tt $<$} Expr \> {\rm strict inequation} \\
\>    \> $|$ Expr {\tt $>$} Expr \> {\rm strict inequation} \\
\>    \> $|$ Expr {\tt =$<$} Expr \> {\rm nonstrict inequation} \\
\>    \> $|$ Expr {\tt $>$=} Expr \> {\rm nonstrict inequation} \\
\>    \> $|$ Expr {\tt =/=} Expr \> {\rm disequation} \\
\\
\> Expr --$>$  \> variable \> {\rm Prolog variable} \\
\> \> $|$ number \> {\rm floating point number} \\
\> \> $|$ {\tt +} Expr \\
\> \> $|$ {\tt -} Expr \\
\> \> $|$ Expr {\tt +}  Expr \\
\> \> $|$ Expr {\tt -} Expr \\
\> \> $|$ Expr {\tt *} Expr \\
\> \> $|$ Expr {\tt /} Expr \\
\> \> $|$ {\tt abs($Expr$)} \\
`\> \> $|$ {\tt sin($Expr$)} \\
\> \> $|$ {\tt cos($Expr$)} \\
\> \> $|$ {\tt tan($Expr$)} \\
\> \> $|$ {\tt pow($Expr$,$Expr$)} \> {\rm raise to the power} \\
\> \> $|$ {\tt exp($Expr$,$Expr$)} \> {\rm raise to the power} \\
\> \> $|$ {\tt min($Expr$,$Expr$)} \> {\rm minimum of two expressions} \\
\> \> $|$ {\tt max($Expr$,$Expr$)} \> {\rm maximum of two expressions} \\
\> \> $|$ \verb|#|(Expr) \> {\rm symbolic numerical constants} 
\end{tabbing}
}

\subsection{The CLP(R) API} \label{sec:clpr-api}
%
From the command line, it is usually easiest to load the {\tt clpr}
package and call the predicates below directly from {\tt usermod} (the
module implicitly used by the command line).  However, when calling
any of these predicates from compiled code, they must be explicitly
imported from their modules (e.g. {\tt \{\}} must be explicitly
imported from {\tt clpr}).  Figure~\ref{fig:clpr} shows an example of
how this is done.
%
\begin{figure} \label{fig:clpr}
{\small 
\begin{verbatim}
   :- import {}/1 from clpr.

   root(N, R) :-
   root(N, 1, R).
   root(0, S, R) :- !, S=R.
   root(N, S, R) :-
	N1 is N-1,
	{ S1 = S/2 + 1/S },
	root(N1, S1, R).
\end{verbatim}
}
\caption{Example of a file with a CLP(R) predicate}
\end{figure}
%
`
\begin{description}
\ourmoditem{\{+Constraints\}} {clpr} \index{\texttt{\{\}/1}}
%
When the CLP(R) package is loaded, inclusion of equations in braces
({\tt \{\}}) adds {\tt Constraints} to the constraint store where they
are checked for satisfiability. 

{\bf Example:}
{\small
\begin{verbatim}
     | ?- [clpr].
     [clpr loaded]
     [itf loaded]
     [dump loaded]
     [bv_r loaded]
     [nf_r loaded]

     yes

     | ?- {X = Y+1, Y = 3*X}.

     X = -0.5000
     Y = -1.5000;

     yes
\end{verbatim}
}
{\bf Error Cases}
\bi
\item 	{\tt Constraints} is not instantiated
\bi
\item 	{\tt instantiation\_error}
\ei
%
\item 	{\tt Constraints} is not an equation, an inequation or a disequation
\bi
\item 	{\tt domain\_error('constraint relation',Rel)}
\ei
\item {\tt Constraints} contains an expression {\tt Expr} that is not
  a numeric expression 
\bi
\item 	{\tt domain\_error('numeric expression',Expr)}
\ei
\ei

\ourmoditem{entailed(+Constraint)} {clpr} \index{\texttt{entailed/1}}
%
Succeeds if {\tt Constraint} is logically implied by the current
constraint store.  {\tt entailed/1} does not change the constraint
store.
%

{\bf Example:}
{\small
\begin{verbatim}
| ?- {A =< 4},entailed(A =\= 5).
 { A =< 4.0000 }

yes   
\end{verbatim}
}
{\bf Error Cases}
\bi
\item 	{\tt Constraints} is not instantiated
\bi
\item 	{\tt instantiation\_error}
\ei
%
\item 	{\tt Constraints} is not an equation, an inequation or a disequation
\bi
\item 	{\tt domain\_error('constraint relation',Rel)}
\ei
\ei

\ourrepeatnewitem{inf(+Expr,-Val)} {clpr} \index{\texttt{inf/2}}

\ourrepeatnewitem{sup(+Expr,-Val)} {clpr} \index{\texttt{sup/2}}

\ourrepeatnewitem{minimize(Expr)} {clpr} \index{\texttt{minimize/1}}

\ourmoditem{maximize(Expr)} {clpr} \index{\texttt{maximize/1}}
%
These four related predicates provide various mechanisms to compute
the maximum and minimum of expressions over variables in a constraint
store.  In the case where the expression is not bounded from above
over the reals {\tt sup/2} and {\tt maximize/1} will fail; similarly
if the expression is not bounded from below {\tt inf/2} and {\tt
minimize/1} will fail.

{\bf Examples:}
{\small
\begin{verbatim}
| ?- {X = 2*Y,Y >= 7},inf(X,F).
 { X >= 14.0000 }
 { Y = 0.5000 * X }

X = _h8841
Y = _h9506
F = 14.0000

| ?- {X = 2*Y,Y >= 7},minimize(X).
X = 14.0000
Y = 7.0000

| ?- {X = 2*Y,Y =< 7},maximize(X-2).

X = 14.0000
Y = 7.0000

| ?- {X = 2*Y,Y =< 7},sup(X-2,Z).
 { X =< 14.0000 }
 { Y = 0.5000 * X }

X = _h8975
Y = _h9640
Z = 12.0000

yes
| ?- {X = 2*Y,Y =< 7},maximize(X-2).

X = 14.0000
Y = 7.0000

yes
\end{verbatim}
}

\ourrepeatnewitem{inf(+Expr,-Val, +Vector, -Vertex)} {clpr} \index{\texttt{inf/4}}

\ourmoditem{sup(+Expr,-Val, +Vector, -Vertex)} {clpr} \index{\texttt{sup/4}}
%
These predicates work like {\tt inf/2} and {\tt sup/2} with the
following addition.  {\tt Vector} is a list of Variables, and for each
variable $V$ in {\tt Vector}, the value of $V$ at the extremal point
{\tt Val} is returned in corresponding position in the list {\tt
  Vertex}.

{\bf Example:}
{\small 
\begin{verbatim}
| ?= { 2*X+Y =< 16, X+2*Y =< 11,X+3*Y =< 15, Z = 30*X+50*Y},
     sup(Z, Sup, [X,Y], Vertex).
 { X + 3.0000 * Y =< 15.0000 }
 { X + 0.5000 * Y =< 8.0000 }
 { X + 2.0000 * Y =< 11.0000 }
 { Z = 30.0000 * X + 50.0000 * Y }

X = _h816
Y = _h869
Z = _h2588
Sup = 310.0000
Vertex = [7.0000,2.0000]
\end{verbatim}
}

\ourmoditem{bb\_inf(+IntegerList,+Expr,-Inf,-Vertex, +Eps)}  {clpr}
\index{\texttt{bb\_inf/4}}
%
Works like {\tt inf/2} in {\tt Expr} but assumes that all the
variables in {\tt IntegerList} have integral values.  {\tt Eps} is a
positive number between $0$ and $0.5$ that specifies how close an
element of {\tt IntegerList} must be to an integer to be considered
integral -- i.e. for such an {\tt X}, {\tt abs(round(X) - X) < Eps}.
Upon success, {\tt Vertex} is instantiated to the integral values of
all variables in {\tt IntegerList}.  {\tt bb\_inf/5} works properly for
non-strict inequalities only.

{\bf Example:}
{\small
\begin{verbatim}
| ?- {X > Y + Z,Y > 1, Z > 1},bb_inf([Y,Z],X,Inf,Vertex,0).
 { Z > 1.0000 }
 { Y > 1.0000 }
 { X - Y - Z > 0.0000 }

X = _h14286
Y = _h10914
Z = _h13553
Inf = 4.0000
Vertex = [2.0000,2.0000]

yes
\end{verbatim}
}
%
{\bf Error Cases}
\bi
\item 	{\tt IntegerList} is not instantiated
\bi
\item 	{\tt instantiation\_error}
\ei
\ei

\ourmoditem{bb\_inf(+IntegerList,+Expr,-Inf)}  {clpr}
\index{\texttt{bb\_inf/3}}
%
Works like {\tt bb\_inf/5}, but with the neighborhood, {\tt Eps}, set
to {\tt 0.001}.

{\bf Example}
{\small
\begin{verbatim}
|?- {X >= Y+Z, Y > 1, Z > 1}, bb_inf([Y,Z],X,Inf)
 { Z > 1.0000 }
 { Y > 1.0000 }
 { X - Y - Z >= 0.0000 }

X = _h14289
Y = _h10913
Z = _h13556
Inf = 4.

yes
\end{verbatim}
}

\ourmoditem{dump(+Variables,+NewVars,-CodedVars} {dump} 
\index{\texttt{dump/3}}
%
For a list of variables {\tt Variables} and a list of variable names
{\tt NewVars}, returns in {\tt CodedVars} the constraints on the
variables, without affecting the constraint store.

{\bf Example:}
{\small
\begin{verbatim}
| ?- {X > Y+1, Y > 2},
     dump([X,Y], [x,y], CS).
 { Y > 2.0000 }
 { X - Y > 1.0000 }

X = _h17748
Y = _h17139
CS = [y > 2.0000,x - y > 1.0000];
\end{verbatim}
}
%
{\bf Error Cases}
\bi
\item 	{\tt Variables} is not instantiated to a list of variables
\bi
\item 	{\tt instantiation\_error}
\ei
\ei

\ourmoditem{projecting\_assert(+Clause)}{dump}
\index{projecting\_assert/1}
\index{constraints!asserting dynamic code with}
%
In XSB, when a subgoal is tabled, the tabling system automatically
determines the relevant projected constraints for an answer and copies
them into and out of a table.  However, when a clause with constrained
variables is asserted, this predicate must be used rather than {\tt
  assert/1} in order to project the relevant constraints.  This
predicate works with either standard or trie-indexed dynamic code.

{\bf Example:}
{\small
\begin{verbatim}
| ?- {X > 3},projecting_assert(q(X)).
 { X > 3.0000 }

X = _h396

yes
| ?- listing(q/1).
q(A) :-
    clpr : {A > 3.0000}.

yes
| ?- q(X),entailed(X > 2).
 { X > 3.0000 }

X = _h358

yes
| ?- q(X),entailed(X > 4).

no
\end{verbatim}
}
\end{description}

\section{The {\tt bounds} Package} \label{sec:bounds}
%
\index{Prologs!Sicstus}
\index{Prologs!SWI}

\version{} of XSB does not support a full-fledged CLP(FD) package.
However it does support a simplified package that maintains an upper
and lower bound for logical variables.  {\tt bounds} can thus be used
for simple constraint problems in the style of finite domains, as long
as these problems that do not rely on too heavily on propagation of
information about constraint domains~\footnote{The {\tt bounds}
  package was written by Tom Schrijvers, and ported to XSB from SWI
  Prolog version 5.6.49 by Terrance Swift, who also wrote this manual
  section.}

Perhaps the simplest way to explain the functionality of {\tt bounds}
is by example.  The query
%
{\small
\begin{verbatim}
|?- X in 1..2,X #> 1.
\end{verbatim}
}
%
\noindent
first indicates via {\tt X in 1..2} that the lower bound of {\tt X} is
{\tt 1} and the higher bound {\tt 2}, and then constraints {\tt X},
which is not yet bound, to be greater than {\tt 1}.  Applying this
latter constraint to {\tt X} forces the lower bound to equal the upper
bound, instantiating {\tt X}, so that the answer to this query is {\tt
  X = 2}.

Next, consider the slightly more complex query
%
{\small
\begin{verbatim}
|?- X in 1..3,Y in 1..3,Z in 1..3,all_different([X,Y,Z]),X = 1, Y = 2.
\end{verbatim}
}
%
\noindent
{\tt all\_different/3} constraints {\tt X}, {\tt Y} and {\tt Z} each
to be different, whatever their values may be.  Accordingly, this
constraint together with the bound restrictions, implies that
instantiating {\tt X} and {\tt Y} also causes the instantiation of
{\tt Z}.  In a similar manner, the query
%
{\small
\begin{verbatim}
|?- X in 1..3,Y in 1..3,Z in 1..3,sum([X,Y,Z],#=,9),
\end{verbatim}
}
%
\noindent
onstrains the sum of the three variables to equal {\tt 9} -- and in
this case assigns them a concrete value due to their domain
restrictions.

In many constraint problems, it does not suffice to know whether a set
of constraints is satisfiable; rather, concrete values may be needed
that satisfy all constraints.  One way to produce such values is
through the predicate {\tt labelling/2}
%
{\small
\begin{verbatim}
|?- X in 1..5,Y in 1..5,X #< Y,labeling([max(X)],[X,Y]))
\end{verbatim}
}
%
\noindent
In this query, it is specified that {\tt X} and {\tt Y} are both to be
instantiated not just by any element of their domains, but by a value
that assigns {\tt X} to be the maximal element consistent with the
constraints.  Accordingly {\tt X} is instantiated to {\tt 4} and {\tt
  Y} to {\tt 5}.

Because constraints in {\tt bounds} are based on attributed variables
which are handled by XSB's variant tabling mechanisms, constrained
variables can be freely used with variant tabling as the folowing
fragment shows:
%
{\small
\begin{verbatim}
table_test(X):- X in 2..3,p(X).

:- table p/1.
p(X):- X in 1..2.

?- table_test(Y).

Y = 2
\end{verbatim}
}
%

For a more elaborate example, we turn to the {\em SEND MORE MONEY}
example, , in which the problem is to assign numbers to each of the
letters {\em S,E,N,D,M,O,R,Y} so that the number {\em SEND} plus the
number {\em MORE} equals the number {\em MONEY}.  Borrowing a solution
from the SWI manual~\cite{SWI-manual}, the {\tt bounds} package solves
this problem as:
%
{\small
\begin{verbatim}
send([[S,E,N,D], [M,O,R,E], [M,O,N,E,Y]]) :-
    Digits = [S,E,N,D,M,O,R,Y],
    Carries = [C1,C2,C3,C4],
    Digits in 0..9,
    Carries in 0..1,
    M #= C4,
    O + 10 * C4 #= M + S + C3,
    N + 10 * C3 #= O + E + C2,
    E + 10 * C2 #= R + N + C1,
    Y + 10 * C1 #= E + D,
    M #>= 1,
    S #>= 1,
    all_different(Digits),
    label(Digits).
\end{verbatim}
}

In many cases, it may be useful to test whether a given constraint is
true or false.  This can be done by unifying a variable with the truth
value of a given constraint -- i.e.  by {\em reifying} the constraint.
As an example, the query
%
{\small
\begin{verbatim}
|?-  X in 1..10, Y in 1..10,Z in 0..1,X #< Y, X #= Y #<=> Z,label([Z]).
\end{verbatim}
}
%
\noindent
sets the bounded variable {\tt Z} to the truth value of {\tt X \#= Y},
or {\tt 0}~\footnote{The current version of the {\tt bounds} package
  does not always seem to propagate entailment into the values of
  reified variables.}.

A reader familiar with the finite domain library of
Sicstus~\cite{sicstus-manual} will have noticed that the syntax of
{\tt bounds} is consistent with that library.  It is important to note
however, {\tt bounds} maintains only the upper and lower bounds of a
variables as its attributes, (along, of course with constraints on
those variables) rather than an explicit vector of permissable values.
As a result, {\tt bounds} may not be suitable for large or complex
constraint problems.

\subsection{The {\tt bounds} API}

Note that {\tt bounds} does not perform error checking, but instead
relies on the error checking of lower-level comparison and arithmetic
operators.

\begin{description}

\ournewitem{in(-Variable,+Bound)}{bounds}\index{\texttt{in/2}}
%
Adds the constraint {\tt Bound} to {\tt Variable}, where {\tt Bound}
should be of the form {\tt Low..High}, with {\tt Low} and {\tt High}
instantiated to integers.  This constraint ensures that any value of
{\tt Variable} must be greater than or equal to {\tt Low} and less
than or equal to {\tt High}.  Unlike some finite-domain constraint
systems, it does {\em not} materialize a vector of currently allowable
values for {\tt Variable}.

Variables that have not had their domains explicitly constrained are
considered to be in the range {\tt min\_integer}..{\tt max\_integer}.

%\ourrepeatitem{\#/\(Expr1,Expr2)}{bound}\index{\texttt{\#>/2}}
%\ourrepeatitem{\#\\/(Expr1,Expr2)}{bound}\index{\texttt{\#>/2}}
%\ourrepeatitem{\#\(Expr1,Expr2)}{bound}\index{\texttt{\#>/2}}
%\ournewitem{\#\(Expr1,Expr2)}{bound}\index{\texttt{\#>/2}}

\ourrepeatnewitem{\#>(Expr1,Expr2)}{bounds}\index{\texttt{\#>/2}}
\ourrepeatnewitem{\#<(Expr1,Expr2)}{bounds}\index{\texttt{\#</2}}
\ourrepeatnewitem{\#>=(Expr1,Expr2)}{bounds}\index{\texttt{\#>=/2}}
\ourrepeatnewitem{\#=<(Expr1,Expr2)}{bounds}\index{\texttt{\#=</2}}
\ourrepeatnewitem{\#=(Expr1,Expr2)}{bounds}\index{\texttt{\#=/2}}
\ournewitem{\#\\=(Expr1,Expr2)}{bounds} %\index{\texttt{\#\=/2}}
%
Ensures that a given relation holds between {\tt Expr1} and {\tt
  Expr2}.  Within these constraints, expressions may contain the
functions {\tt +/2}, {\tt -/2}, {\tt */2}, {\tt +/2}, {\tt +/2}, {\tt
  +/2}, {\tt mod/2}, and {\tt abs/1} in addition to integers and
variables.

\ourrepeatnewitem{\#<=>(Const1,Const2)}{bounds}\index{\texttt{\#>/2}}
\ourrepeatnewitem{\#=>(Const1,Const2)}{bounds}\index{\texttt{\#>/2}}
\ournewitem{\#<=(Const1,Const2)}{bounds}\index{\texttt{\#>/2}}
%
Constrains the truth-value of {\tt Const1} to have the speficied
logical relation (``iff'', ``only-if'' or ``if'') to {\tt Const2},
where {\tt Const1} and {\tt Const2} have one of the six relational
operators above.  

\ournewitem{all\_different(+VarList)}{bounds}\index{\texttt{all\_different/1}}
%
{\tt VarList} must be a list of variables: constrains all variables in
{\tt VarList} to have different values.

\ournewitem{sum(VarList,Op,?Value)}{bounds}\index{\texttt{sum/3}}
%
{\tt VarList} must be a list of variables and {\tt Value} an integer
or variable: constrains the sum of all variables in {\tt VarList} to
have the relation {\tt Op} to {\tt Value} (see preceding example).

\ournewitem{labeling(+Opts,+VarList}{bounds}\index{\texttt{labelling/2}}
%
This predicate succeeds if it can assign a value to each variable in
{\tt VarList} such that no constraint is violated.  Note that
assigning a value to each constrained variable is equivalent to
deriving a solution that satisfies all constraints on the variables,
which may be intractible depending on the constraints.  {\tt Opts}
allows some control over how value assignment is performed in deriving the solution.
%
\begin{itemize}
\item {\tt leftmost} Assigns values to variables in the order in which
  they occur.  For example the query:
%
{\small
\begin{verbatim}
|?- X in 1..4,Y in 1..3,X #< Y,labeling([leftmost],[X,Y]),writeln([X,Y]),fail.
[1,2]
[1,3]
[2,3]

no
\end{verbatim}
}
%
instantiates {\tt X} and {\tt Y} to all values that satisfy their
constraints, and does so by considering each value in the domain of
{\tt X}, checking whether it violates any constraints, then
considering each value of {\tt Y} and checking whether it violates any
constraints.
%
\item {\tt ff} This ``first-fail'' strategy assignes values to
  variables based on the size of their domains, from smallest to
  largest.  By adopting this strategy, it is possible to perform a
  smaller search for a satisfiable solution because the most
  constrained variables may be considered first (though the bounds of
  the variable are checked rather than a vector of allowable values).  
%
\item {\tt min} and {\tt max} This strategy labels variables in the
  order of their minimal lower bound or maximal upper bound.
%
\item {\tt min(Expr)} and {\tt max(Expr)} This strategy labels the
  variables so that their assignment causes {\tt Expr} to have a
  minimal or maximal value.  Consider for example how these strategies
  would affect the labelling of the preceding query:
{\small
\begin{verbatim}
|?- X in 1..4,Y in 1..3,X #< Y,labeling([min(Y)],[X,Y]),writeln([X,Y]),fail.
[1,2]

no
|?- X in 1..4,Y in 1..3,X #< Y,labeling([max(X)],[X,Y]),writeln([X,Y]),fail.
[2,3]

no
\end{verbatim}
}
\end{itemize}

\ournewitem{label(+VarList)}{bounds}\index{\texttt{label/1}}
%
Shorthand for {\tt labeling([leftmost],+VarList)}.

\ournewitem{indomain(?Var)}{bounds}\index{\texttt{}}
%
Unifies {\tt Var} with an element of its domain, and upon sucessive
backttrakcing, with all other elements of its domain.

\ournewitem{serialized(+BeginList,+Durations}{bounds}\index{\texttt{serialized/2}}
% 
{\tt serialized/2} can be useful for scheduling problems.  As input it
takes a list of variables or integers representing the beginnings of
temporal events, along with a list of non-negative intergers
indicating the duration of each event in {\tt BeginList}.  The effect
of this predicate is to constrain each of the events in {\tt
  BeginList} to have a start time such that their durations do not
overlap.  As an example, consier the query
%
{\small
\begin{verbatim}
|?- X in 1..10, Y in 1..10, serialized([X,Y],[8,1]),label([X,Y]),writeln((X,Y)),fail.
\end{verbatim}
}
%
In this query event {\tt X} is taken to have duration of {\tt 8}
units, while event {\tt Y} is taken to have duration of {\tt 1} unit.
Executing this query will instantiate {\tt X} and {\tt Y} to many
different values, such as {\tt (1,9)}, {\tt (1,10)}, and {\tt (2,10)}
where {\tt X} is less than {\tt Y}, but also {\tt (10,1)}, {\tt
  (10,2)} and many others where {\tt Y} is less than {\tt X}.  Refining the query as
%
{\small
\begin{verbatim}
X in 1..10, Y in 1..10, serialized([X,Y],[8,1]),X #< Y,label([X,Y]),writeln((X,Y)),fail.
\end{verbatim}
}
%
removes all solutions where {\tt Y} is less than {\tt X}.

\ournewitem{lex\_chain(+List)}{bounds}\index{\texttt{}}
%
{\tt lex\_chain/1} takes as input a list of lists of variables and
integers, and enforces the constraint that each element in a given
list is less than or equal to the elements in all succeeding lists.
As an example, consider the query
%
{\small
\begin{verbatim}
|?- X in 1..3,Y in 1..3,lex_chain([[X],[2],[Y]]),label([X,Y]),writeln([X,Y]),fail.
[1,2]
[1,3]
[2,2]
[2,3]
\end{verbatim}
}
%
{\tt lex\_chain/1} ensures that {\tt X} is less than or equal to {\tt
  2} which is less than or equal to {\tt Y}.

% Not 100% sure what they mean here.
%\ournewitem{tuples\_in/2}{bounds}\index{\texttt{}}

\end{description}

