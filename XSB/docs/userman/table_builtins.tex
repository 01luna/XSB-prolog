%=======================================================

\newcommand{\retn}{\code{ret/n}}

%--------------------------------------------------------------------------

\section{Tabling Declarations and Builtins} \label{sec:TablingPredicates}

In XSB, tables are designed so that they can be used transparently by
computations.  However, it is necessary to first inform the system of
which predicates should be evaluated using tabled resolution
(\refsec{sec:CompilerOptions}) along with the properties to be used,
such as call variance or call subsumptiion
(\refchap{chap:TablingOverview}).  Further, it is often useful to be
able to explicitly inspect a table, or to alter its state.  The
predicates described in this section are provided for these purposes.
In order to ground the discussion of these predicates, we continue our
overview of tables and table creation from
\refchap{chap:TablingOverview}.  For a detailed description of the
implementation of table access routines in XSB, the reader is referred
to~\cite{RRSSW98,TST99,CuSW99b,Swif14} and other papers listed in the
bibliography.

% - - - - - - - - - - - - - - - - - - - - - - - - - - - - - - - - - - - - -

\subsubsection*{Tables and Table Entries}

\index{TableEntryHandle}
\index{ReturnHandle}
%Abstractly, at a subgoal-level a table $\cT$ can be seen as a set of
Abstractly, a table $\cT$ can be seen as a triple $\langle
S,\cA,Status \rangle$ where $S$ is a subgoal, $\cA$ is its associated
answer set, and $Status$ its status~---~whether the table is
\texttt{complete} or \texttt{incomplete}, along with tabling
properties it uses (e.g., incremental or non-incremental, cf. Chapter
\ref{chap:TablingOverview} for a discussion of tabling properties).
XSB's table inspection built-ins sometimes use a {\em
  TableEntryHandle} to efficiently access $\cT$ and a {\em
  ReturnHandle} to access $\cA$.  Often is is useful to access or
manipulate the set of all (subgoal-level) tables for some tabled
predicate \code{p/n}.  We thus sometimes abuse terminology slightly by
referring to this set as a {\em predicate-level} table.

At execution time, invocation of a tabled subgoal $S$ leads to the
classification of $S$ according to the properties associated with its
predicate, as well as its possible creation of a table for $S$.  Each
occurrence of a subgoal that is not yet completely evaluated can be
classified as either (a) a \emph{generator}, of answers or (b) a
\emph{consumer} of those answers. 

% If call subsumption is used, and
%there are incomplete subgoals $S$ and $S\theta$, where $S$ subsumes
%$S\theta$, and $S\theta$ consumes answers from $S$, $S$ is also called
%a {\em producer}.

%Creation of a table entry thus
%relies not only on the call and on the subgoals already present in the
%table, but also upon whether call-variance or call-subsumption is used
%(cf. \cite{TST99}).

% - - - - - - - - - - - - - - - - - - - - - - - - - - - - - - - - - - - - -

% - - - - - - - - - - - - - - - - - - - - - - - - - - - - - - - - - - - - -

\subsubsection*{Skeletons and Predicate Specifications}

\index{skeleton}

A \emph{skeleton} for a functor \code{f/n} is a structure of the form
\code{f($Arg_1$,\ldots,$Arg_n$)} where each $Arg_i$ is a distinct
variable.  Similarly the skeleton of a term is the skeleton formed
from the principal functor of the term, so that skeletons from the
terms \code{f(1,2)} and \code{f(A,B)} are the same.  A \emph{return
  skeleton} is a specific application of this notion to answers.  From
it, one may discern the size of the template for a given subgoal.
Below, we assume that a predicate specification for a predicate
\code{p} and arity \code{n}, represented as \code{PredSpec} below, can
be given either using the notation \code{p/n} or as a skeleton,
\code{p($t_1$,\ldots,$t_n$)}.

% - - - - - - - - - - - - - - - - - - - - - - - - - - - - - - - - - - - - -
\comment{
| TLS: Unnecessary, as exceptions are described elsewhere
| \subsubsection*{Exceptions}
| 
| Exceptions caught by the following predicates include:
| \begin{description}
| \item[Instantiation Error] Argument is a variable.
| \item[Type Error] Argument is not a predicate specification or callable term.
| \item[Table Error] Argument does not contain a tabled predicate, or a valid
|         reference to a table component.
| \end{description}
}

%--------------------------------------------------------------------------

\subsection{Declaring and Modifying Tabled Predicates}
\label{sec:TablePred:Decl&Mod}
\index{tabling!directives}\index{tabling!strategy selection}

\begin{description}
% - - - - - - - - - - - - - - - - - - - - - - - - - - - - - - - - - - - - -

\index{tabling!call subsumption}
\index{tabling!call variance}
\index{tabling!answer subsumption}
\index{tabling!incremental}
\index{tabling!opaque}
\index{tabling!dynamic predicates}
%\index{tabling!private}
%\index{tabling!shared}
\index{tabling!sound completion}
\index{tabling!subgoal abstraction}
\index{tabling!answer abstraction}
\index{tabling!interned ground terms}

\label{table-declaration}
\ourstandarditem{table(+Operations)}{table/1}{Tabling}
%
{\tt table/1} can be used either as a compiler declaration or as an
executable directive.  Used as a compiler declaration, it indicates
that each predicate denoted by the command is to be compiled using (a
particular form of) tabling, and may indicate that the predicate
itself is dynamic.
%or thread-shared or thread-private.
Executed as a directive in a state of execution where no clauses exist
for each denoted predicate {\tt table/1} ensures that any clauses
asserted for each predicate use tabling and may indicate the mode of
tabling to be used.  The parameter {\tt Operations} can take one of
three forms:
%
\begin{enumerate}
\item {\tt Operations} is a predicate indicator, a skeleton, or
  a comma-list or list of predicate indicators or skeletons.
%
\item {\tt Operations} is a term indicating that a predicate is to be
  tabled with a particular form of answer subsumption
  (cf. Section~\ref{sec:table-aggregation}).
%
\item {\tt Operations} has the form {\tt Predspec as Options} where
\bi
\item {\tt PredSpec} is a predicate indicator, a skeleton, or a
  comma-list or list of predicate indicators or skeletons.
%
\item {\tt Options} is either a table option or a list of table
  options.  In \version, the following table options are supported
\bi
\item {\tt dyn} which causes the predicate(s) to be treated as dynamic
  in addition to being tabled, and is equivalent to {\tt ?- dynamic
    PredSpec}~\footnote{The option {\tt dynamic} can be used, but
    because the atom dynamic is an operator in XSB, the declaration
    requires parentheses, e.g.: {\tt table p/n as (dynamic).}}
%
\item {\tt subsumptive} which causes the table evaluation method of
  the predicate(s) to use call subsumption.
%
\item {\tt variant} which causes the table evaluation method of the
  predicate(s) to use call variance.
%
\item {\tt intern} which causes all ground subterms of subgoals and 
answers entered into the table for the predicate(s) to be interned.
%
\item {\tt incremental} which causes the table evaluation method of
  the predicate(s) to be incremental.
%
\item {\tt opaque} which indicates that each preidcate $p/n$ in {\tt
  PredSpec} is such that incremental tables can depend on tables for
  $p/n$, but that tables for $p/n$ are not themselves incrementally
  maintained.
%
%\item {\tt private} which causes the predicate(s) to be treated as
%  thread private in addition to being tabled.
%
%\item {\tt shared} which causes the predicate(s) to be treated as
%  thread shared in addition to being tabled.
%
\item {\tt index(}{\em index-spec}{\tt )} ensures that the predicates
  in {\em PredSpec} are subsumptively tabled and are indexed using a
  technique called (perhaps unimaginatively) {\em table indexing}.  See
  Section~\ref{sec:table-index} for details.
  
\item {\tt subgoal\_abstract(n)}  which enables size-based subgoal abstraction for the predicate(s).
%
\item {\tt answer\_abstract(n)} which enables depth-n answer
  abstraction for the predicate(s).
%
\item {\tt max\_answers(n)} which for any $P$ in {\tt PredSpec}, the
  maximum number of answers for any tabled subgoal of $P$ is limited
  to {\tt n}, using sound completion.
%
\ei
\ei
\end{enumerate}

If the directive 

{\tt  table $PredList$ as $Options$.}

is executed, various checks are performed on $Options$.  These checks
are (mostly) performed before any predicates are declared as dynamic
or options changed, and reduce the possibility of leaving some $p/n$
in $PredList$ with inconsistent attributes, which could cause an error
to be thrown during program execution.
%
\begin{itemize}
\item If a predicate in {\tt Predlist} has been declared as {\tt
  incremental} it may be changed to {\tt opaque} at any time;
  similarly, a predicate that is {\tt opaque} may be changed to {\tt
    incremental}
%
\item If a predicate in {\tt Predlist} has been declared to use call
  variance it may be changed to use call subsumption at any time;
  similarly, a predicate that uses call subsumption may be changed to
  use call variance.
%
\item Otherwise, an attempt to change an attribute of $p/n$ in
  $PredList$ -- i.e. whether $p/n$ is tabled or not, dynamic or not
%  and thread-private or thread-shared
  -- will throw a permission error.
\end{itemize}

In addition, regardless of the state of predicates in $PredList$, if
options contains an unsupported set of declarations, a permission
error will be thrown (see Section~\ref{sec:tabling-compatibility} for
a list of supported and non-supported combinations of tabling modes
and predicate properties).  {\tt Options} throws a table error in the
following cases:
%
\bi
\item {\tt Options} contains {\tt dynamic} and either {\tt opaque} or
  {\tt incremental}.  Tabled dynamic incremental code is not yet
  supported in XSB.

\item {\tt Options} contains ({\tt incremental} or {\tt opaque}) and
  ({\tt subsumptive} or {\tt shared})

\item {\tt Options} contains {\tt subsumptive} and ({\tt variant} or
  {\tt shared} or {\tt subgoal\_abstract/1} or {\tt
    answer\_abstract/1})

\item {\tt Options} contains {\tt intern} and ({\tt dynamic} or {\tt
subsumptive} or {\tt approximate} or {\tt incremental} or {\tt opaque}
or {\tt answer\_abstract} or {\tt subgoal\_abstract})
%
%\item {\tt Options} contains both {\tt private} and {\tt shared}
%
\item {\tt Options} contains both {\tt incremental} and {\tt opaque}
\ei

{\bf Error Cases}

Error cases are summarized as follows.  Let {\tt Operations} be of the
form {\tt PredSpec} or {\tt PredSpec} as {\tt Options}.  Then if 
%
\bi
\item {\tt PredSpec} or is a variable or a comma list containing a variable
\bi
\item {\tt instantiation\_error}
\ei
\item An element of {\tt PredSpec} is neither a variable nor a
  predicate indicator, nor a skeleton.
\bi
\item 	{\tt type\_error(callable,PredSpec)}
\ei
\item A predicate in {\tt PredSpec} has been previously defined to be
  static or foreign and {\tt Options} contains {\tt dynamic} or {\tt
    dyn}
\bi
\item 	{\tt permission\_error(modify,static\_predicate)}
\ei
\item {\tt Options} is a variable or a list containing a variable
\bi
\item {\tt instantiation\_error}
\ei
\item {\tt Options} contains an element {\tt Option} that isn't a table option (as described above)
\bi
\item {\tt domain\_error(table\_option,Option)}
\ei
\item {\tt Options} contains a non-supported combination of  elements (as described above)
\bi
\item {\tt permission\_error}
\ei
\item An option in {\tt Options} would modify a predicate in {\tt
  PredSpec} in a manner that is not allowed (as described above) 
\bi
\item {\tt permission\_error}
\ei
\ei

% - - - - - - - - - - - - - - - - - - - - - - - - - - - - - - - - - - - - -

% \ourstandarditem{use\_subsumptive\_tabling\ \ +PredSpec$_1$, \ldots, +PredSpec$_k$.}{use\_subsumptive\_tabling/1}{Tabling}
%
%
% Declares each \emph{tabled predicate} denoted by \code{PredSpec$_i$}
% to use call-subsumption, thus overriding the system default of call
% variance~\footnote{If XSB is called with the {\tt -S} option the
%   default will be to use call subsumption}.  The tabling strategy can
% be changed at will through the invocation of this and the following
% predicate.  Note, however, that the table for the predicate must be
% empty at the time of the change.

% - - - - - - - - - - - - - - - - - - - - - - - - - - - - - - - - - - - - -

% \ourstandarditem{use\_variant\_tabling\ \ +PredSpec$_1$, \ldots, +PredSpec$_k$.}{use\_variant\_tabling/1}{Tabling}
%
% Declares each \emph{tabled predicate} denoted by \code{PredSpec$_i$}
% to use call-variance, thus overriding the system default if XSB is
% called with the {\tt -S} option.  The same comments apply as for {\tt
%   use\_subsumptive\_tabling/1}.

% - - - - - - - - - - - - - - - - - - - - - - - - - - - - - - - - - - - - -
\end{description}

%--------------------------------------------------------------------------

\subsection{Predicates for Table Inspection} \label{sec:table-inspection}
\label{sec:TablePred:Inspection}
\index{tabling!table inspection}
\index{tabling!call subsumption}
\index{tabling!call variance}

Often, the higher level inspection predicates described in
Section~\ref{sec:suspend-analyze} are the best bet for analyzing
tables and other aspects of the state of computation.  However, for
some purposes, a finer level of control is needed, which these
predicates provide.  In this section we describe inspection predciates
that can be used to quickly examine a collection of tables.  In the
next section, we describe lower-level inspection predicates that are
special-purpose, and may not be needed by most users.

For explanitory purposes, we maintain two running examples in this
section and the next.  The first uses tabling based on call variance:
%
\begin{center}
\begin{tabular}{cc}\hline \hline
\multicolumn{2}{c}{\rule{0ex}{2.5ex}\textbf{Call Variance Example}} \\ \hline \hline
\rule[-2ex]{0ex}{5ex} \textbf{Program} & \textbf{Table} \\
\begin{minipage}{14.5em}
\begin{verbatim}
:- table p/2 as variant.

p(1,2).
p(1,3).
p(1,_).
p(2,3).
\end{verbatim}
\end{minipage}
&
\begin{tabular}{|c|c|c|} \hline
  Subgoal & Answer Set & Status\\ \hline \hline
  p(1,Y) & p(1,2) & complete \\ 
         & p(1,3) & \\
         & p(1,Y) & \\ \hline
  p(X,3) & p(1,3) & complete \\ 
         & p(2,3) & \\ \hline
\end{tabular} \\
\vspace*{-2ex} \\ \hline \hline
\end{tabular}
\end{center}

\noindent
and the second uses tabling based on call subsumption::

\begin{center}
\begin{tabular}{cc}\hline \hline
\multicolumn{2}{c}{\rule{0ex}{2.5ex}\textbf{Call Subsumption Example}} \\ \hline \hline
\rule[-2ex]{0ex}{5ex} \textbf{Program} & \textbf{Table} \\
\begin{minipage}{17em}
\begin{verbatim}
:- table q/2 as subsumptive.
q(a,b).
q(b,c).
q(a,c).
\end{verbatim}
\end{minipage}
&
\begin{tabular}{|c|c|c|} \hline
  Subgoal & Answer Set & Status \\ \hline \hline
  q(X,Y) & q(a,b) & complete \\
         & q(b,c) & \\
         & q(a,c) & \\ \hline
  q(a,Y) & \textit{q(a,b)} & complete \\
         & \textit{q(a,c)} & \\ \hline
  q(X,c) & \textit{q(b,c)} & complete \\
         & \textit{q(a,c)} & \\ \hline
\end{tabular} \\
\vspace*{-2ex} \\ \hline \hline
\end{tabular}
\end{center}

\noindent
Note that in the call subsumption example, the subgoals \code{q(a,Y)} and
\code{q(X,c)} are subsumed by, and hence obtain their answers from,
the subgoal \code{q(X,Y)}\@.

\vspace{2ex}

\begin{description}

% - - - - - - - - - - - - - - - - - - - - - - - - - - - - - - - - - - - - -
% - - - - - - - - - - - - - - - - - - - - - - - - - - - - - - - - - - - - -

\ourstandarditem{get\_calls\_for\_table(+PredSpec,?Call)}{get\_calls\_for\_table/2}{Tabling}
%
Identifies through backtracking all tabled subgoals whose predicate is
that of \code{PredSpec} and that unify with \code{Call}.
\code{PredSpec} is left unchanged while \code{Call} contains the
unified result.  Its behavior is shown in Example~\ref{ex:gcft}.

\comment{Error if either \code{PredSpec} is a nontabled pred (STRUCTs
and STRINGs are caught here) or a variable.  Fails if \code{PredSpec}
is any other type or if \code{Call} does not unify with any call in
the table.}

\begin{minipage}{6.1in}
\begin{example}{{\bf (get\_calls\_for\_table/2)}}\label{ex:gcft}\ \\
\\
\hspace*{-0.5em}\begin{tabular}{l@{\hspace{1.5em}}r}
\begin{tabular}{c} \hline \hline
\rule{0ex}{2.5ex}\textbf{Variant Predicate} \\ \hline \hline
\vspace*{-1ex} \\
\begin{minipage}{17.35em}
\begin{small}
\begin{verbatim}
|?- get_calls_for_table(p(1,3),Call).  

Call = p(_h142,3);

Call = p(1,_h143);

no
| ?- get_calls_for_table(p/2,Call).

Call = p(_h137,3);

Call = p(1,_h138);

no
\end{verbatim}
\end{small}
\end{minipage} \\
\vspace*{-1ex} \\ \hline \hline
\end{tabular}
 &
\begin{tabular}{c} \hline \hline
\rule{0ex}{2.5ex}\textbf{Subsumptive Predicate} \\ \hline \hline
\vspace*{-1ex} \\
\begin{minipage}{17.85em}
\begin{small}
\begin{verbatim}
| ?- get_calls_for_table(q(X,Y),Call). 

X = _h80
Y = _h94
Call = q(a,_h167);

X = _h80
Y = _h94
Call = q(_h166,c);

X = _h80
Y = _h94
Call = q(_h166,_h167);

no
\end{verbatim}
\end{small}
\end{minipage} \\
\vspace*{-1ex} \\ \hline \hline
\end{tabular}
\end{tabular}
\end{example}
\end{minipage} \\

\comment{
The second example backtracks through all entries in the table, since
only skeletal information is used from the first argument.
}

% - - - - - - - - - - - - - - - - - - - - - - - - - - - - - - - - - - - - -

\ourstandarditem{get\_returns\_for\_call(+Subgoal,?AnswerTerm)}{get\_returns\_for\_call/2}{Tabling}
%
Succeeds through backtracking for each answer of the subgoal
\code{Subgoal} which unifies with \code{AnswerTerm}.  Fails if
\code{Subgoal} is not a tabled subgoal or \code{AnswerTerm} does not
unify with any of its answers or if {\tt Subgoal} has no answers.

The answer is created in its entirety, including fresh variables so
that {\tt Subgoal} is \emph{not} further instantiated.  Of course the
user may unify {\tt Subgoal} with its answer if desired.  Example
~\ref{ex:grfc} illustrates its behavior.

\comment{Fails or gives error whenever \code{get_call/3} does.}

\begin{minipage}{6.1in}
\begin{example}{{\bf (get\_returns\_for\_call/2})}  \label{ex:grfc}\ \\
\\
\hspace*{-0.5em}\begin{tabular}{l@{\hspace{2.5em}}r}
\begin{tabular}{c} \hline \hline
\rule{0ex}{2.5ex}\textbf{Variant Predicate} \\ \hline \hline
\vspace*{-1ex} \\
\begin{minipage}{16.5em}
\begin{small}
\begin{verbatim}
| ?- get_returns_for_call(p(1,Y),
                          AnsTerm).

Y = _h88
AnsTerm = p(1,_h161);

Y = _h88
AnsTerm = p(1,3);

Y = _h88
AnsTerm = p(1,2);

no
| ?- get_returns_for_call(p(X,Y),
                          AnsTerm).

no
| ?- get_returns_for_call(p(1,2),
                          AnsTerm).

no
\end{verbatim}
\end{small}
\end{minipage} \\
\vspace*{-1ex} \\ \hline \hline
\end{tabular}
 &
\begin{tabular}{c} \hline \hline
\rule{0ex}{2.5ex}\textbf{Subsumptive Predicate} \\ \hline \hline
\vspace*{-1ex} \\
\begin{minipage}{16.5em}
\begin{small}
\begin{verbatim}
| ?- get_returns_for_call(q(a,Y),
                          AnsTerm).

Y = _h88
AnsTerm = q(a,c);

Y = _h88
AnsTerm = q(a,b);

no
| ?- get_returns_for_call(q(X,c),
                          AnsTerm).

X = _h80
AnsTerm = q(b,c);

X = _h80
AnsTerm = q(a,c);

no
\end{verbatim}
\end{small}
\end{minipage} \\
\vspace*{-1ex} \\ \hline \hline
\end{tabular}
\end{tabular}
\end{example}
\end{minipage} \\

% - - - - - - - - - - - - - - - - - - - - - - - - - - - - - - - - - - - - -

\ourrepeatstandarditem{get\_residual(\#CallTerm,?DelayList)}{get\_residual/2}{Tabling}
%
\ourstandarditem{variant\_get\_residual(\#CallTerm,?DelayList)}{variant\_get\_residual/2}{Tabling}
%
{\tt get\_residual/2} backtracks through the answers to each
\emph{completed} subgoal in the table that unifies with
\code{CallTerm}.  For each such answer $A$, \code{CallTerm} is unified
with $A$, and \code{DelayList} with a delay list of $A$ if $A$ is
conditional, and otherwise with the empty list.

\index{residual program} Since the delay list of an answer consists of
those literals whose truth value is unknown in the well-founded model
of the program (see Chapter~\ref{chap:TablingOverview})
\code{get\_residual/2} is useful to examine portions of the residual
program.  Example \ref{ex:residual-program} illustrates such a use.

\begin{example}{{\bf (get\_residual/2)}} \label{ex:residual-program}
For the following program and table
     \begin{center}
     \begin{tabular}{cc}
     \begin{minipage}{2.1in}
     {\tt
          :- table p/2. \\
          p(1,2). \\
          p(1,3):- tnot(p(2,3)). \\
          p(2,3):- tnot(p(1,3)). 
     }
     \end{minipage}
     &
     \begin{tabular}{||l|l||}   \hline
     {\em Subgoal}                 & {\em Answers} \\ \hline \hline
     p(1,X)                     & p(1,2) \\ 
                                & p(1,3):- tnot(p(2,3)) \\ \hline
     p(1,3)                     & p(1,3):- tnot(p(2,3)) \\ \hline
     p(2,3)                     & p(2,3):- tnot(p(1,3)) \\ \hline
     \end{tabular}
     \end{tabular}
     \end{center}
the completed subgoals are {\tt p(1,X)}, {\tt p(1,3)}, and {\tt
p(2,3)}.  Calls to {\tt get\_residual/2} will act as follows
%
\begin{center}
\begin{small}
%\begin{minipage}{2.5in}
\begin{verbatim} 
| ?- get_residual(p(X,Y),List).

X = 1       % from subgoal p(1,X)
Y = 2
List = [];

X = 1       % from subgoal p(1,X)
Y = 3
List = [tnot(p(2,3))];

X = 1       % from subgoal p(1,3)
Y = 3
List = [tnot(p(2,3))];

X = 2       % from subgoal p(2,3)
Y = 3
List = [tnot(p(1,3))];

no
\end{verbatim}
%\end{minipage}
\end{small}
\end{center}
\end{example}

For other purposes, it may be desired to examine the answers for a
particular subgoal, rather than for all subgoals that unifiy with {\tt
  CallTerm}.  In this case, {\tt variant\_get\_residual/2} can be
used, which backtracks through all answers for {\tt CallTerm} if {\tt
  CallTerm} is a tabled subgoal with answers, and fails otherwise.
For the above example, {\tt variant\_get\_residual/2} behaves as
follows:

\begin{center}
\begin{small}
%\begin{minipage}{2.5in}
\begin{verbatim} 
| ?- variant_get_residual(p(X,Y),List).

no
| ?- variant_get_residual(p(1,Y),List).

X = 1       % from subgoal p(1,X)
Y = 2
List = [];

X = 1       % from subgoal p(1,X)
Y = 3
List = [tnot(p(2,3))];

no
\end{verbatim}
%\end{minipage}
\end{small}
\end{center}

{\bf Error Cases}
\bi
\item {\tt CallTerm} is not a callable term
\bi
\item {\tt type\_error(callable\_term,CallTerm)}
\ei
\item {\tt CallTerm} does not correspond to a tabled predicate
\bi
\item 	{\tt permission\_error(table access,non-tabled predicate,{\tt CallTerm})}
\ei
\ei

% - - - - - - - - - - - - - - - - - - - - - - - - - - - - - - - - - - - - -

\ourrepeatstandarditem{table\_state(+Subgoal,?Strategy,?CallType,?AnsSetStatus)}{table\_state/1}{Tabling}

\ourstandarditem{table\_state(+TableEntryHandle,?Strategy,?CallType,?AnsSetStatus)}{table\_state/4}{Tabling}
%
May succeed whenever \code{Subgoal} is a subgoal in the table, or
\code{TableEntryHandle} is a valid reference to a table entry.  In
either case, certain arguments 2 through 4 unify with constants
representing properties of the table.  Taken together, these
properties provide a detailed description of current state of the
given subgoal within an evaluation.  The combinations valid in the
current version of XSB and their specific meaning is given in the
following table.  Notice that not only can these combinations describe
the characteristics of a subgoal in the table, but they are also
equipped to predict how {\tt CallTerm} would have been treated had it
been called at that moment.

\begin{center}
\begin{small}
\begin{tabular}{|c|c|l|l|} \hline
\code{Strategy} & \code{CallType}       & \code{AnsSetStatus}
                & \multicolumn{1}{c|}{Description} \\ \hline \hline
        &       & \code{complete}       & Self explanatory. \\ \cline{3-4}
        &       &                       & An incremental table that has been \\
        & \code{producer}  & {\code{incremental\_needs\_reeval}} 
                                        & invalidated, and is therefore inconsistent \\
        &       &                       & with a KB and needs recomputation \\ 
        &       &                       & (which will be lazily done). \\ \cline{3-4}
\code{variant}  &   & \code{incomplete}
                & Self explanatory. \\ \cline{2-4}
        & \code{no\_entry}      & \code{undefined}
                & The call does not appear in the table. \\ \hline
        &       & \code{complete} & Self explanatory. \\ \cline{3-4}
        & \raisebox{1.5ex}[0ex]{\code{producer}}        & \code{incomplete}
                & Self explanatory. \\ \cline{2-4}
        &       &       & The call is in the table and is properly \\
        &       & \raisebox{1.5ex}[0ex]{\code{complete}}
                & subsumed by a completed producer. \\ \cline{3-4}
        &  \raisebox{1.5ex}[0ex]{\code{subsumed}}
                &       & The call is in the table and is properly \\
        &       & \raisebox{1.5ex}[0ex]{\code{incomplete}}
                & subsumed by an incomplete producer. \\ \cline{2-4}
        &       &       & The call is not in the table, but if it were \\
\code{subsumptive}      &       & \code{complete}
                & to be called, it would consume from a \\
        &       &       & completed producer. \\ \cline{3-4}
        &       &       & The call is not in the table, but if it had \\
        & \code{no\_entry}      & \code{incomplete}
                & been called at this moment, it would \\
        &       &       & consume from an incomplete producer. \\ \cline{3-4}
        &       &       & The call is not in the table, but if it had \\
        &   & \code{undefined}   & been called at this moment, it would be \\
        &       &       & a producer. \\ \hline
\code{undefined}        & \code{undefined}      & \code{undefined}
                & The given predicate is not tabled. \\ \hline
\end{tabular}
\end{small}
\end{center}

\comment{
%\ourmoditem{print\_incomplete\_tables}{print\_incomplete\_tables/0}{tables}
%
%{\bf {\em This predicate is obsolescent, and will eventually be
%    replaced by {\tt get\_incomplete\_subgoals/1} and its extensions.}}

%These predicates, which can be useful for debugging, print out each
%incomplete subgoal in the current state of the evaluation, followed
%by the ordinal number of the SCC to which that subgoal belongs.  This
%information describes the dependencies among tabled predicates.  In
%local evaluation (the default evaluation method for XSB) all subgoals
%in SCC $m$ depend on all subgoals in SCC $n$ if $m < n$.  Furthermore,
%all subgoals in a given SCC depend on one another~\footnote{This
%  assumes that there is no early completion, which can remove
%  dependencies. In batched evaluation, the dependencies are less exact
%  -- see \cite{SaSw98} for details, as SCCs represent a dag of
%  dependencies rather than a chain as in local evaluation.}.  As its
%name implies, {\tt print\_incomplete\_tables/[0,1]} print out SCC
%information only for incomplete tables; for full information about the
%SCCs of a computation, forest logging must be used
%(cf. Section~\ref{sec:forest-trace}).
%
%In {\tt print\_incomplete\_tables/0}, the information is output to
%{\tt stdout}.

%{\bf Example:} For the program
%
%\begin{verbatim}
%:- table q/2.
%q(0,_):- !,print_incomplete_tables.
%q(3,A):- q(5,A).
%q(N,A):- N1 is N - 1,q(N1,A).
%\end{verbatim}
%the goal {\tt ?- q(5,foo)} will produce the output 
%
%{\small
%%\begin{verbatim}
%q(5, foo)- scc(1).
%q(4, foo)- scc(1).
%q(3, foo)- scc(1).
%q(2, foo)- scc(2).
%q(1, foo)- scc(3).
%q(0, foo)- scc(4).
%\end{verbatim}
%}
%%
%%{\bf Error Cases ({\tt print\_incomplete\_tables/1})}
%%\bi
%%\item 	{\tt Stream\_or\_alias} is a variable
%%\bi
%\item {\tt instantiation\_error}
%\ei
%\item {\tt Stream\_or\_alias} is neither a variable, nor a stream term
%  nor an alias.  
%\bi
%\item 	{\tt domain\_error(stream\_or\_alias,Stream\_or\_alias)}
%\ei
%\item 	{\tt Stream\_or\_alias} is not associated with an open stream
%\bi
%\item 	{\tt existence\_error(stream,Stream\_or\_alias)}
%\ei
%\ei
}
\index{Prolog flags!{\tt exception\_pre\_action}}
\index{strongly connected components (SCCs)}
\index{tabling!and exceptions}
\ourmoditem{get\_scc\_dumpfile(-Filename)}{get\_scc\_dumpfile/1}{tables}
%
If the Prolog flag {\tt exception\_pre\_action} is set to {\tt
  print\_incomplete\_tables} (its default setting is {\tt none}), then
when an exception is thrown, incomplete tables and their SCC
information are printed to an ``SCC dumpfile''.  Note that the
information is output for the state of execution where the error was
thrown, and so is more informative than an action taken when the error
is caught.  (No file is generated unless the exception is thrown over
at least one incomplete table.)  Creation of an SCC dumpfile can
triggered by any error condition, rather than by tha more restricted
set of tripwire conditions (cf. Section~\ref{sec:tripwire}) and so
provide a complementary functionality.

This predicate returns the name of the last such file generated and
fails if there is no such file.  Files are written to the {\tt
  \$XSBDIR/etc} directory with the prefix {\tt scc\_dump\_}.  Users
are responsible for removing these files.

Note that XSB backtraces (Section~\ref{sec:backtrace}) provide
information about the context in which an exception is thrown, but the
SCC dumpfile provides explicit SCC information along with argument.
values for tabled predicates.

{\bf Error Cases}
\bi
\item {\tt Filename} is a not a variable
\bi
\item {\tt instantiation\_error}
\ei
\ei

%\index{residual program}
%\index{\texttt{get\_residual/2}}
%\index{\texttt{variant\_get\_residual/2}}
%\index{Incremental Dependency Graph (IDG)}
%\index{residual dependency graph}
%\ourrepeatmoditem{get\_residual\_sccs(+Subgoal,+Answer,-SCCList)}{get\_residual\_sccs/3}{tables}
%\ourmoditem{get\_residual\_sccs(+Subgoal,+Answer,-SCCList,-DepList,-SignList)}{get\_residual\_sccs/5}{tables}
%%
%As discussed in Section~\ref{sec:non-strat}, answers that are
%undefined in the well-founded semantics are stored in XSB along with
%their delay lists, forming a residual program.  This residual program
%can be materialized through the various predicates discussed above, in
%particular {\tt get\_residual/2} and {\tt variant\_get\_residual/2}.
%
%At times it can be useful to view the residual program as a directed
%graph, for instance in order to understand why a given answer might be
%undefined.  In a manner somewhat analogous to the incremental
%dependency graph (Section ~\ref{sec:incremental_tabling}) the {\em
%  residual dependency graph} is a directed graph whose nodes are
%subgoal/atom pairs and whose edges are labelled with: 1) a sign
%indicating whether the edge is positive or negative; and 2) the label
%{\em depends on} or {\em affects}.

%\begin{example} \rm
%Consider the program 
%% 
%{\tt 
%\begin{tabbing}
%fooo\=fooooooooooooooooooooooooooooooo\=ooooooooooooo\=\kill
% \>  :- table p/2. \\
%\>           p(1,2). \\
%\%>           p(1,3):- tnot(p(2,3)).  \\
%\>           p(2,3):- tnot(p(1,3)). \> p(2,3):- r(a).\\
%\>           r(a):- tnot(r(b)) \\
%\>           r(b):- tnot(r(a)).   
%\end{tabbing}
%}
%%
%to which the query {?- p(1,X)} was made, generating the tables:
%\begin{center}
%\begin{tabular}{||l|l||}   \hline
%     {\em Subgoal}                 & {\em Answers} \\ \hline \hline
%     p(1,X)                     & p(1,2) \\ 
%                                & p(1,3):- tnot(p(2,3))| \\ \hline
%     p(1,3)                     & p(1,3):- tnot(p(2,3))| \\ \hline
%     p(2,3)                     & p(2,3):- tnot(p(1,3))| \\ \hline
%                                & p(2,3):- tnot(r(a))| \\ \hline
%     r(a)                       & r(a):- tnot(r(b))| \\ \hline
%     r(b)                       & r(b):- tnot(r(a))| \\ \hline
%\%end{tabular}
%\end{center}

%The residual dependency graph for this program and query would have a
%node for each subgoal/answer combination with an undefined truth
%value, and a dependency edge for nodes $S_1/A_1$ and $S_2/A_2$ if
%$A_2$ occurs in a literal in the delay list for $S_1/A_1$, and the
%original subgoal for $A_2$ was $S_2$ in the subcomputation for $S_1$.
%The edge also has a sign indicating whether $A_2$ occurs positively or
%negatively in the delay list for $A_1$.  In this example, the residual
%dependency graph could be represented as 
%%
%\begin{verbatim}
%     depends_on(p(1,X),p(1,3),p(2,3),p(2,3),neg).
%     depends_on(p(1,3),p(1,3),p(2,3),p(2,3),neg).
%     depends_on(p(2,3),p(2,3),p(1,3),p(1,3),neg).
%     depends_on(p(2,3),p(2,3),r(a),r(a),pos).
%     depends_on(r(a),r(a),r(b),r(b),neg).
%     depends_on(r(b),r(b),r(a),r(a),neg).
%\end{verbatim}
%\end{example}
%
%\index{termination!radial restraint} 
%\index{\texttt{u\_not/1}} 
%%
%Using the residual dependency graph, a user may be able to determine
%why an answer $A$ to a subgoal $S$ was unexpectedly undefined either
%because $S/A$ was involved in or depended on a loop through negation;
%or because $S/A$ depended on some other answer that was undefined
%because of the use of bounded rationality
%(Section~\ref{sec:tabling-termination}) or because of floundering and
%the use of {\tt u\_not/1}.  
%
%The residual dependency graph can be constructed in a straightforward
%way from {\tt variant\_get\_residual/2}.  {\tt
%  get\_residual\_sccs/[3,5]} provides an alternate view that is
%slightly higher-level and much faster.  Given a subgoal/answer pair as
%input, each of these predicates constructs SCC-based information about
%the residual dependency graph via structures of the form:
%%
%\begin{center}
%{\tt ret(Subgoal,Answer,SCCIndex)}.
%\end{center}
%%
%where {\tt SCC} is a numerical index for the SCCs of Subgoal. Two
%subgoals are in the same SCC iff they have the same index, however no
%other dependency information can be otherwise directly inferred from
%the index~\footnote{The actual number for
%  each SCC index depends on how the residual dependency graph happens
%  to be traversed; as a result it is best to rely on the index only as
%  a ``generated'' name for each SCC.}.
%
%To obtain dependency information, {\tt get\_residual\_sccs/5} also returns a
%list indicating the direct dependencies among the SCCs, along with a
%list indicating whether given SCCs contain a negative edge.  For the
%example above, the SCC information would have a form such as:
%\begin{verbatim}
%[ ret(p(1,X),p(1,3),1), ret(p(1,3),p(1,3),2), ret(p(2,3),p(2,3),2),
%  ret(r(a),r(a),3), ret(r(b),r(b),3) ]
%\end{verbatim}
%%
%The dependency list would have a form such as:
%\begin{verbatim}
%[ depends(1,2), depends(2,3) ]
%\end{verbatim}
%while the sign list would have a form such as:
%\begin{verbatim}
%[ sign(1,no_neg), sign(2,neg), sign(3,neg) ]
%\end{verbatim}
%If it is necessary to know which subgoal(s) in {\tt SCC1} directly
%depends on which subgoal(s) in {\tt SCC2}, the information can be
%easily reconstructed from the output of {\tt
%  get\_residual\_sccs/[4,5]} using {\tt variant\_get\_residual/2}.  A
%similar approach can be used to determine the actual edges within a
%given SCC.
%
%SCC detection is implemented using Tarjan's algorithm~\cite{Tarj72} in
%C working directly on XSB's data structures.  The algorithm is
%%$\cO(|V| + |E|)$ where $|V|$ is the number of vertices and $|E|$ the
%number of edges in the dependency graph.  As a result, {\tt
%  get\_residual\_sccs/3} provides an efficient means to materialize
%the high-level topography of the dependency graph~\footnote{Currently,
%  the materialization of dependency information between SCCs is
%  implemented in a naive manner, so that {\tt get\_residual\_sccs/6}
%  is $\cO(|V|^2)$.}.

%These predicates implement Tarjan's algorithm~\cite{Tarj72} in C
%working directly on XSB's data structures.  The algorithm is $\cO(|V|
%+ |E|)$ where $|V|$ is the number of vertices and $|E|$ the number of
%edges in the dependency graph.  As a result, these predicates provide
%an efficient means to materialize the dependency graph, even if SCC
%information per se is not required

%%%%%%%%%%%% moved to inspection.tex 
  
%\index{radial restraint}
%\index{termination!radial restraint}
%\index{\texttt{u\_not/1}}
%\index{\texttt{get\_residual\_sccs/5}}
%\ourrepeatmoditem{explain\_u\_val(+Subgoal,+Answer,-Reason)}{explain\_u\_val/3}{tables}	
%\ourmoditem{explain\_u\_val(+Subgoal,+Answer,-Sccs,-Deps,-Signs,-Reason)}{explain\_u\_val/6}{tables}	
%%
%The XSB predicate
%%
%{\tt explain\_u\_val(+Subgoal,+Answer,?Reason)}
%\noindent
%can be used to query why {\tt Answer} is undefined when derived in an
%evaluation of {\tt Subgoal}.  {\tt Reason} may be
%\begin{itemize}
%\item {\tt negative\_loops(cycle)} if the derivation of {\tt Answer} involves a
%  loop through though negation that includes {\tt Answer} itself.
%%
%\item {\tt negative\_loops(dependent)} if the derivation of {\tt
%  Answer} depends on an atom that is involved in a loop through though
%  negation.
%%
%\item {\tt unsafe\_negation} if the derivation of {\tt Answer} depends
%  on a negative subgoal that is non-ground (XSB does not automatically
%  perform subgoal reordering).  The action of making a non-ground
%  subgoal undefined is performed by {\tt u\_not/1}.
%%
%\item {\tt bounded\_rationality} if the derivation of answer depends
%  on bounded rationality based on radial restraint~\cite{GroS13}.
%\end{itemize}
%%
%These reasons are not exclusive, and complex derivations may well
%involve several of the above reasons.
%
%{\tt explain\_u\_val/[3,6]} is based on the structures returned by
%{\tt get\_residual\_sccs/[3,5]}.  While {\tt
%  get\_residual\_sccs/[3,5]} is reasonably fast, it can take a
%peceptable time to analyze large residual programs containing many
%thousands of SCCs.  Accordingly, {\tt explain\_u\_val/6} can reuse
%dependency structures returned by {\tt get\_residual\_sccs/[3,5]},
%which can be useful for justification systems and other applciations.
%
%\begin{example} \rm
%After executing the query {\tt p} to the program
%%
%\begin{verbatim}
%:- table p/0, q/0, r/0, s/1.
%p:- q,tnot p.                 p:- s(f(f(f(f(0))))).
%
%q:- tnot r.                   r:- tnot q.
%
%s(f(X)):- s(X).               s(0).
%\end{verbatim}
%%
%where the bounded rationality depth has been set to 3.  The query {\tt
%  explain\_u\_val(p,P,Reason)} will bind {\tt Reason} to {\tt
%  negative\_loops(cycle)}, to {\tt negative\_loops(dependent)}, and
%to {\tt bounded\_rationality} (this ordering is not guaranteed).
%\end{example}
%

% - - - - - - - - - - - - - - - - - - - - - - - - - - - - - - - - - - - - -

\comment{
%\ournewitem{table\_state(+CallTerm,?State)}{Tabling}
%\index{\texttt{table\_state/2}}
%
%Functions similarly to, and is superseded by, \code{table\_state/4}.
%The value of \code{State} may be one of the following atoms with the
%given meaning.
%
%\begin{description}
%\item[{\tt not\_yet\_called}] iff the predicate corresponding to
%Call has been declared tabled, but there is no table entry for call.
%\item[{\tt complete}] iff the table entry for {\tt Call} contains
%all solutions.
%\item[{\tt incomplete}] iff the table entry for {\tt Call} may not contain
%all solutions.
%\item[{\tt undef}] iff the predicate corresponding to {\tt Call} 
%has not been declared tabled.
%\end{description}

| Exceptions:
|     \begin{description}
|     \item[{\tt type\_error}]
|         Argument 1 is not callable.
|     \end{description}
}

\subsection{Predicates for Table Inspection: Lower-level} \label{sec:table-inspection-low}
%\label{sec:TablePred:Inspection}
\index{tabling!table inspection}
\index{tabling!call subsumption}
\index{tabling!call variance}

The predicates described in this section are primarily for the use of
those who want to write libraries and packages for XSB.  For the
purposes of most users similar and more complete functionality is
provided by other inspection predicates.

In this section, the user should be aware that skeletons that are
dynamically created (e.g., by \code{functor/3}) are located in {\tt
  usermod} (refer to \refsec{Modules}).  In such a case, the tabling
predicates below may not behave in the desired manner if the tabled
predicates themselves have not been imported into {\tt usermod}.

\subsubsection*{Answers, Returns, and Return Templates}
%
\index{answer substitution}
\index{substitution factor}
%
Given a table entry $(S,\cA,Status)$, the vector of variables in $S$ is
sometimes called the {\em substitution factor} of $S$.  The order of
arguments in the substitution factor corresponds to the order of
distinct variables in a left-to-right traversal of $S$.  Each answer
in $\cA$ substitutes values for the variables in the substitution
factor of $S$; this substitution is sometimes called an {\em answer
  substitution}.  The table inspection predicates allow access to
substitution factors and answer substitutions through a family of
terms called {\em return templates} and whose principle functors have
the form \retn, where \code{n} is the size of the substitution factor.

\begin{example}
Let $S =$ \code{p(X,f(Y))} be a tabled subgoal.  Using a return
template, the substitution factor can be depicted as \code{ret(X,Y)},
while the answer substitution \code{\{X=a,Y=b\}} is depicted as
\code{ret(a,b)}.  Note that the application of the answer substitution
to the generator subgoal yields the answer \code{p(a,f(b))}.

To take a slightly more complex example, consider the subgoal
\code{q(X)} where {\tt X} is an attributed variable whose attribute is
\code{f(Z,Y,Y)}.  In this case the substitution factor is
\code{ret(X,Z,Y)}.\fillBox
\end{example}

XSB overloads return templates to maintain substitutions between
generator subgoals and consuming subgoals when call subsumption is
used.  The return template for a consuming subgoal is a substitution
that maps variables of its generator to subterms of the consuming
subgoal.  This template can then be used to select answers from the
generator that unify with the consuming call.
%Note, then, that a return template of
%a \emph{subsumed} subgoal may show partial instantiations.  Return
%templates are also represented as \retn{} terms in the manner
%described above.

\begin{example}
Let \code{p/2} of the previous example be evaluated using call
subsumption and let the subgoal $S =$ \code{p(A,f(B))} be present in
its table.  Further, let $S_1$: \code{p(A,f(B))} and $S_2$:
\code{p(g(Z),f(b))} be two consuming subgoals of $S$\@.  Then the
\emph{return template} of $S_1$ is \code{ret(A,B)} and that of $S_2$
is \code{ret(g(Z),b)}.  $S_1$, being a variant of $S$, selects answers
for $S$ such that \code{\{X=A,Y=B\}}\@, i.e., all answers of $S$.
$S_2$, on the other hand, selects only \emph{relevant} answers of $S$,
those that satisfy \code{\{X=g(Z),Y=b\}}.\fillBox
\end{example}


\subsubsection*{Description of Low-level Inspection Predicates}
\begin{description}

% - - - - - - - - - - - - - - - - - - - - - - - - - - - - - - - - - - - - -
\ourstandarditem{get\_call(+CallTerm,-TableEntryHandle,-ReturnTemplate)}{get\_call/3}{Tabling}

%
\index{tabling!call subsumption} \index{tabling!call variance} If call
variance is used for the predicate corresponding to {\tt CallTerm},
then this predicate searches the table for an entry whose subgoal is a
\emph{variant} of \code{CallTerm}.  If subsumption is used, then this
predicate searches for some entry that subsumes (properly or not) {\tt
  CallTerm}.  In either case, should the entry exist, then the handle
to this entry is assigned to the second argument, while its return
template is constructed in the third argument.  These latter two
arguments must be uninstantiated at call time.
Example~\ref{ex:get-call} illustrates its behavior.

{\bf Error Cases}
\bi
\item {\tt CallTerm} is not a callable term
\bi
\item {\tt type\_error(callable\_term,CallTerm)}
\ei
\item {\tt CallTerm} does not correspond to a tabled predicate
\bi
\item 	{\tt permission\_error(table access,non-tabled predicate,{\tt CallTerm})}

\ei
\ei
\ \\
\begin{minipage}{6in}
\begin{example}{{\bf (get\_call/2)}} \label{ex:get-call}\ \\
 \\
\begin{tabular}{l@{\hspace{4em}}r}
\begin{tabular}{c} \hline \hline
\rule{0ex}{2.5ex}\textbf{Variant Predicate} \\ \hline \hline
\vspace*{-1ex} \\
\begin{minipage}{14.5em}
\begin{small}
\begin{verbatim}
| ?- get_call(p(X,Y),Ent,Ret).

no
| ?- get_call(p(1,Y),Ent,Ret).

Y = _h92
Ent = 136039108
Ret = ret(_h92);

no
| ?- get_call(p(X,3),Ent,Ret).

X = _h84
Ent = 136039156
Ret = ret(_h84);

no
| ?- get_call(p(1,3),Ent,Ret).

no
\end{verbatim}
\end{small}
\end{minipage} \\
\vspace*{-1ex} \\ \hline \hline
\end{tabular}
 &
\begin{tabular}{c} \hline \hline
\rule{0ex}{2.5ex}\textbf{Subsumptive Predicate} \\ \hline \hline
\vspace*{-1ex} \\
\begin{minipage}{14.5em}
\begin{small}
\begin{verbatim}
| ?- get_call(q(X,Y),Ent,Ret).

X = _h80
Y = _h94
Ent = 136043988
Ret = ret(_h80,_h94);

no
| ?- get_call(q(a,Y),Ent,Ret).

Y = _h88
Ent = 136069412
Ret = ret(a,_h88);

no
| ?- get_call(q(X,c),Ent,Ret).

X = _h80
Ent = 136069444
Ret = ret(_h80,c);

no
\end{verbatim}
\end{small}
\end{minipage} \\
\vspace*{-1ex} \\ \hline \hline
\end{tabular}
\end{tabular}
\end{example}
\end{minipage} \\


\ourrepeatstandarditem{get\_calls(\#Subgoal,-TableEntryHandle,-ReturnTemplate)}{get\_calls/3}{Tabling}

\ourmoditem{get\_calls(\#Subgoal,-TableEntryHandle,-ReturnTemplate,-CallNodeHandle)}{get\_calls/4}{Tables}
%
Identifies through backtracking each tabled subgoal $S$ that unifies
with \code{Subgoal}.  For each such $S$, {\tt get\_calls/3} assigns
the handle of the table entry is assigned to the second argument, and
its return template is constructed in the third.  These latter two
arguments must be uninstantiated at call time.  {\tt get\_calls/4}
also returns the handle to the call node structure if {\tt Subgoal} is
incrementally tabled.  {\tt get\_calls/3} is a standard predicate in
XSB, although {\tt get\_calls/4} is not.

{\bf Error Cases}
\bi
\item {\tt Subgoal} is non-variable and does not represent a term that
  is callible in the present state of the system: 
\bi
 \item 	{\tt type\_error}
\ei
\item {\tt Subgoal} represents a callable term that does not
  correspond to a predicate that is tabled, or defined as dynamic and
  incremental:
\bi
\item 	{\tt permission\_error}
\ei
\ei

Example~\ref{ex:get-calls} illustrates its behavior.


\begin{minipage}{6in}
\begin{example}{{\bf (get\_calls/3)}} \label{ex:get-calls}\ \\
\\
\begin{tabular}{l@{\hspace{4em}}r}

\begin{tabular}{c} \hline \hline
\rule{0ex}{2.5ex}\textbf{Variant Predicate} \\ \hline \hline
\vspace*{-1ex} \\
\begin{minipage}{15em}
\begin{small}
\begin{verbatim}
| ?- get_calls(p(X,Y),Ent,Ret).

X = _h80
Y = 3
Ent = 136039156
Ret = ret(_h80);

X = 1
Y = _h94
Ent = 136039108
Ret = ret(_h94);

no
| ?- get_calls(p(X,3),Ent,Ret).

X = _h80
Ent = 136039156
Ret = ret(_h80);

X = 1
Ent = 136039108
Ret = ret(3);

no
| ?- get_calls(p(1,3),Ent,Ret).

Ent = 136039156
Ret = ret(1);

Ent = 136039108
Ret = ret(3);

no
\end{verbatim}
\end{small}
\end{minipage} \\
\vspace*{-1ex} \\ \hline \hline
\end{tabular}
 &
\begin{tabular}{c} \hline \hline
\rule{0ex}{2.5ex}\textbf{Subsumptive Predicate} \\ \hline \hline
\vspace*{-1ex} \\
\begin{minipage}{15em}
\begin{small}
\begin{verbatim}
| ?- get_calls(q(X,Y),Ent,Ret).

X = a
Y = _h94
Ent = 136069412
Ret = ret(a,_h94);

X = _h80
Y = c
Ent = 136069444
Ret = ret(_h80,c);

X = _h80
Y = _h94
Ent = 136043988
Ret = ret(_h80,_h94);

no
| ?- get_calls(q(a,Y),Ent,Ret).

Y = _h88
Ent = 136069412
Ret = ret(a,_h88);

Y = c
Ent = 136069444
Ret = ret(a,c);

Y = _h88
Ent = 136043988
Ret = ret(a,_h88);

no
\end{verbatim}
\end{small}
\end{minipage} \\
\vspace*{-1ex} \\ \hline \hline
\end{tabular}
\end{tabular}
\end{example}
\end{minipage} \\
 
% - - - - - - - - - - - - - - - - - - - - - - - - - - - - - - - - - - - - -

\ourstandarditem{get\_returns(+TableEntryHandlex,\#ReturnTemplate)}{get\_returns/2}{Tabling}
%
Backtracks through the answers for the subgoal whose table entry is
referenced through the first argument, \code{TableEntryHandle}, and
instantiates \code{ReturnTemplate} with the variable bindings
corresponding to the answer.

The supplied values for the entry handle and return skeleton should be
obtained from some previous invocation of a table-inspection predicate
such as {\tt get\_call/3} or {\tt get\_calls/3}.  Its behavior is
illustrated in Example~\ref{ex:get-returns}.

\begin{minipage}{6in}
\begin{example}{{\bf get\_returns/2}} \label{ex:get-returns}\ \\
\\
\begin{tabular}{l@{\hspace{4em}}r}
\begin{tabular}{c} \hline \hline
\rule{0ex}{2.5ex}\textbf{Variant Predicate} \\ \hline \hline
\vspace*{-1ex} \\
\begin{minipage}{15em}
\begin{small}
\begin{verbatim}
| ?- get_calls(p(X,3),Ent,Ret),
     get_returns(Ent,Ret).

X = 2
Ent = 136039156    % p(X,3)
Ret = ret(2);

X = 1
Ent = 136039156    
Ret = ret(1);

X = 1
Ent = 136039108    % p(1,Y)
Ret = ret(3);

X = 1
Ent = 136039108
Ret = ret(3);

no
\end{verbatim}
\end{small}
\end{minipage} \\
\vspace*{-1ex} \\ \hline \hline
\end{tabular}
 &
\begin{tabular}{c} \hline \hline
\rule{0ex}{2.5ex}\textbf{Subsumptive Predicate} \\ \hline \hline
\vspace*{-1ex} \\
\begin{minipage}{15em}
\begin{small}
\begin{verbatim}
| ?- get_calls(q(a,c),Ent,Ret),
     get_returns(Ent,Ret).

Ent = 136069412    % q(a,Y)
Ret = ret(a,c);

Ent = 136069444    % q(X,c)
Ret = ret(a,c);

Ent = 136043988    % q(X,Y)
Ret = ret(a,c);

no
| ?- get_calls(q(c,a),Ent,Ret),
     get_returns(Ent,Ret).

no
\end{verbatim}
\end{small}
\end{minipage} \\
\vspace*{-1ex} \\ \hline \hline
\end{tabular}
\end{tabular}
\end{example}
\end{minipage} \\

% - - - - - - - - - - - - - - - - - - - - - - - - - - - - - - - - - - - - -

\ourstandarditem{get\_returns\_and\_tvs(+TableEntryHandle,\#ReturnTemplate,-TruthValue)}{get\_returns\_and\_tvs/3}{Tabling}
%
Identical to \code{get\_returns/2}, but also obtains the truth value
of a given answer, setting {\tt TruthValue} to {\tt t} if the answer
is unconditional and to {\tt u} if it is conditional.  If a
conditional answer has multiple delay lists, this predicate will
succeed only once, so that using this predicate may be more efficient
than {\tt get\_residual/2} (although less informative).

% - - - - - - - - - - - - - - - - - - - - - - - - - - - - - - - - - - - - -

\ourstandarditem{get\_returns(+TableEntryHandle,\#ReturnSkeleton,-ReturnHandle)}{get\_returns/3}{Tabling}
%
Functions identically to \code{get\_returns/2}, but also obtains a
handle to the answer given in the second argument.

% - - - - - - - - - - - - - - - - - - - - - - - - - - - - - - - - - - - - -
\end{description}


%--------------------------------------------------------------------------

\subsection{Abolishing Tables and Table Components}
\label{sec:TablePred:Deleting}
\index{tabling!table abolishing}

The following predicates are used to {\em abolish} tables: to ensure
that they are not used by new computations and to reclaim their space
when it is safe to do so.  The use of the word ``tables'' in this
section is rather unspecific.  For the purpose of deletion a table can
either refer to a single subgoal and its answers, or to all subgoals
and answers for a tabled predicate.  Predicates are provided to
abolish tables not only for particular predicates and subgoals, but
for all tabled predicates, all tabled predicates in a module.
%, and in
%the multi-threaded engine all thread-private tabled predicates or all
%thread-shared tabled predicates.
Overall, these predicates share similar characteristics.

\paragraph{Abolishing a Table that is being Computed}
An incomplete tabled subgoal $S$ may not be directly abolished by the
user.  This restriction is made since if $S$ is incomplete there may
be pointers to $S$ from various elements of the current execution
environment, and removing all of these pointers may be difficult to
do. (Not to mention that abolishing an incomplete table has a murky
semantics.)  Accordingly, calling an {\tt abolish\_xxx} predicate on
an incomplete table raises an error.  

However, note that incomplete tables may be abolished
\emph{automatically} by XSB on exceptions, and when the interpreter
level is resumed.  Because tabled computation is more complex than
Prolog computation, error handling must be correspondingly more
complex.  Suppose an exception is thrown over some incomplete table,
so that the system looks for some {\tt catch/3} or similar call that
will catch the error.  In order to ensure safe space reclamation, XSB
looks for the catcher $C$ that is nearest to the throw, but is also
{\em between} SCCs.  Both XSB's command line interpeter and the
interpreter XSB uses when embedded in a process use a top-level {\tt
  catch/3} goal, which is considered to be ``between'' SCCs, so that a
thrown error will eventually be caught.

Because of the complexity of error handling in tabled computations, it
is usually best to ensure that user-level catches are close to where
an exception may be thrown so that there is no goal to an incomplete
table between the thrower and catcher. In such a case XSB's error
handling mechanism conforms to the ISO standard for Prolog.

\paragraph{View Consistency and Table Garbage Collection}
If one of the table abolish predicates is called when the current
execution environment contains a failure continuation (i.e., a choice
point) to an answer $A$ in a {\em completed} table $T$, space for $T$
is not immediately reclaimed.  Rather the space for $T$ will be
reclaimed by the {\em table garbage collector} at a later point.  More
precisely, if the current global tabling envonment (including
suspended states) has either
\begin{itemize}
\item a choice point that points to an answer $A$ in $T$; 
\item or a (heap) delay list that points to a subgoal $S$ in $T$
\end{itemize}
we say that $T$ is {\em active}.  Also, since tables can be abolished
and rederived during the course of an evaluation, the table deletion
system marks the tables with versions.  Accordingly, if a tabled
predicate $P_{version}$ or subgoal $S_{version}$ to be abolished is
active in the current environment, reclamation of space for that
version of $P$ or $S$ will be delayed until no answers for
$P_{version}$ or $S_{version}$ are active.  Meanwhile the older
version of table will be available for backtracking, ensuring view
consistency for the choice points.  New calls to $P$ or $S$, however,
will force rederivation of a new table version, rather than using the
abolished information.
\index{view consistency}

\index{tabling!conditional answers} \predref{get\_residual/2}
\paragraph{Maintenance of the Residual Program}
When conditional answers are present, abolishing a specific table or
call may lead to semantic or implementational complications.  Consider
the conditional answer {\tt r(a,b):- undef|} from
Figure~\ref{ex:deletion}.  If the predicate {\tt r/2} (or subgoal {\tt
  r(a,X)}) is abolished and later rederived, the rederivation of {\tt
  r(a,X)} might have different semantics than the original derivation
(e.g. if {\tt undef} depended on a database predicate whose definition
has changed).  From an implementation perspective, if space for {\tt
  r(a,X))} is reclaimed, then the call {\tt get\_residual(p(a,X),Y)}
may core dump, even if there are no choice points for completed tables
anywhere in the choice point stack.  To address this problem, by
default abolishing a subgoal $S$ (predicate $P$) will abolish all
subgoals (predicates) that (transtively) depend on $S$
($P$)~\footnote{Dao Tran Minh contributed to implementing this
  functionality.}.  In this case the goal {\tt
  abolish\_table\_subgoal(r(a,X))} would cause the deletion of {\tt
  p(a,X)} while the goal {\tt abolish\_table\_pred(r/2)} would cause
the deletion of {\tt p/2}, since there are tabled subgoals of {\tt
  p/2} that depend on {\tt r/2}.  Only dependencies from subgoals or
answers to the answers that are conditional on them are taken into
account for table deletion: thus the deletion {\tt r(a,X)} deletes
{\tt p(a,X)}, but not {\tt undef}.

%--------------------------------------------------------------------------------------------
\begin{figure}[htb]
\begin{center}
\begin{tabular}{cc}\hline \hline
%\multicolumn{2}{c}{\rule{0ex}{2.5ex}\textbf{Table Deletion Example}} \\ \hline \hline
\rule[-2ex]{0ex}{5ex} \textbf{Program} & \textbf{Table} \\
\begin{minipage}{14.5em}
\begin{verbatim}
:- table p/2, r/2.
p(X,Y):- r(X,Y).

r(a,b):- undef.
r(a,c):- undef.
r(a,d):- undef.
r(a,e):- undef.

:- table  s/0, t/0.
s:- tnot(t).

t:- tnot(undef).

:- table undef/0.
undef :- tnot(undef).
\end{verbatim}
\end{minipage}
&
\begin{tabular}{|c|c|c|} \hline
  Subgoal & Answer Set & Status\\ \hline \hline
  p(a,X)  & p(a,b):- r(a,b)$|$ & complete \\ 
          & p(a,c):- r(a,c)$|$ & \\ \hline
  p(b,X)  & p(b,d):- r(b,d)$|$ & complete \\ 
          & p(b,d):- r(b,e)$|$ & \\ \hline \hline
  r(a,X)  & r(a,b):- undef$|$  & complete \\ 
          & r(a,c):- undef$|$  & \\ \hline
  r(b,X)  & r(b,d):- undef$|$  & complete \\ 
          & r(b,d):- undef$|$  & \\ \hline \hline
  s       & s:- tnot(t)$|$     & complete \\ \hline \hline
  t       & t:- tnot(undef)$|$ & complete \\ \hline \hline
  undef   & undef:- tnot(undef)$|$ & complete \\ \hline \hline
\end{tabular} \\
\vspace*{-2ex} \\ \hline \hline
\end{tabular}
\end{center}
\caption{Example for Deleting Tables (Call-Variance)} \label{ex:deletion}
\end{figure}
%--------------------------------------------------------------------------------------------

\index{Prolog flags!{\tt table\_gc\_action}}
Users with programs that give rise to conditional answers in completed
tables are encouraged to maintain this default behavior.  However the
default behavior may be changed either by setting a Prolog flag:
%
\begin{verbatim}
?- set_prolog_flag(table_gc_action,abolish_tables_singly).
\end{verbatim}
%
or by calling a 2-ary abolish command with {\tt abolish\_tables\_singly}
in the options list.

\paragraph{Abolishing Incremental Tab;es}
In XSB, incremental tables react to changes in underlying dynamic
predicates and/or external events (cf. Section
\ref{sec:incremental_tabling}).  To support this, XSB maintains an
Incremental Dependency Graph (IDG) among incrementally tabled subgoals
and incremental dynamic predicates.  When an incremental table $T$ is
abolished, the IDG needs to be restructured.  Fortunately, with lazy
incremental tabling as used by XSB, the only maintenance needed for
the IDG outside of $T$ is to delete direct links between other IDG
tables and $T$.  In addition, all tables that depend on $T$ are
incrementally invalidated.  As a result, if some $T'$ which had
previously depended on $T$ is called after $T$ was abolished, $T'$
will be seend to be incrementally invalid and will be recomputed.
This recomputation will re-insert $T$ into the IDG in a manner that
reflects the new state of the program.

%\index{tabling!multi-threaded}
%\paragraph{Multiple Threads}
%In the multi-threaded engine abolishing tables private to a thread
%behaves exactly as in the sequential engine, regardless of whether the
%tables are complete or incomplete, or contain conditional answers.  In
%addition, when a thread $T$ exits (by normal termination or via an
%exception), tables private to $T$ are abolished automatically and
%their space reclaimed, as are any incomplete shared tables owned by
%$T$ in local evaluation.  Shared tables can be abolished by the user
%at any time, but their space will not be reclaimed until there is a
%single active thread.

%- - - - - - - - - - - - - - - - - - - - - - - - - - - - - - - - - - - - -
\subsubsection{Table Deletion Predicates}

\begin{description}
\index{tabling!abolishes}
\ourstandarditem{abolish\_table\_pred(+Pred)}{abolish\_table\_pred/1}{Tabling}
%
Invalidates all tabled subgoals for the predicate denoted by the
predicate or term indicator \code{Pred}.  If any subgoal for {\tt
  Pred} contains an answer $A$ that is active in the current
enviornment, {\tt Pred} space reclamation for the {\tt Pred} tables
will be delayed until $A$ is no longer active; otherwise the space for
the {\tt Pred} tables will be reclaimed immediately.  

\index{tabling!abolishes!transitive vs. single}
\index{tabling!conditional answers}
\index{Prolog flags!{\tt table\_gc\_action}}
If {\tt Pred} has a subgoal that contains a conditional answer, the
default behavior will be to transitively abolish any tabled predicates
with subgoals having answers that depend on any conditional answers of
$S$.  This default may be changed either by setting a Prolog flag:
%
\begin{verbatim}
?- set_xsb_flag(table_gc_action,abolish_tables_singly).
\end{verbatim}
% 
or by calling {\tt abolish\_table\_pred/2} with the appropriate
option.  If the transitive abolishes are turned off, and {\tt Pred}
contains a conditional answer, the warning

{\tt abolish\_table\_pred/[1,2] is deleting a table with
  conditional\ answers: \\ delay dependencies may be corrupted.}

will be issued.  

%\index{tabling!abolishes!multi-threading}
%\index{tabling!multi-threaded}
%In the multi-threaded engine, if {\tt Pred} is shared, reclamation for
%{\tt Pred} will be delayed until there is a single active thread and
%no answer in {\tt Pred} is active in the current execution
%environment.  Otherwise, the behavior of {\tt abolish\_table\_pred/1}
%is the same as in the sequential engine.

\index{tabling!incremental}
Finally, {\tt abolish\_table\_pred/1} will throw an error if the
predicate to be abolished is incremental.  
%
%This is because abolishing some incremental tables but not others
%will leave dangling poinrs in the data structures used for uncremental updates.  
%
Until {\tt abolish\_table\_pred/[1,2]} is extended to support
incremental tables, use {\tt abolish\_table\_subgoal/[1,2]} or {\tt
  abolish\_all\_tables/0}.

{\bf Error Cases}
\bi
%
\item {\tt Pred} is not instantiated
 \bi 
 \item 	{\tt instantiation\_error}
 \ei
%
\item {\tt PredSpec} is not a predicate\_indicator or a term\_indicator
 \bi
 \item 	{\tt domain\_error(predicate\_or\_term\_indicator,Pred)}
 \ei
%
\item {\tt PredSpec} does not indicate a tabled predicate
  \bi
 \item 	{\tt table\_error}
 \ei
%
\item {\tt PredSpec} indicates an incrementally tabled predicate.  
\bi
 \item 	{\tt permission\_error}
 \ei
\item There is currently an incomplete table for an atomic subgoal of
  {\tt Pred}.
\bi
 \item 	{\tt permission\_error}
 \ei
%
\ei

\index{Prolog flags!{\tt table\_gc\_action}}
\ourstandarditem{abolish\_table\_pred(+CallTerm,+Options)}{abolish\_table\_pred/2}{Tabling}
%
Behaves as {\tt abolish\_table\_pred/1}, but allows the default {\tt
  table\_gc\_action} to be over-ridden with a flag, which can be either 
{\tt abolish\_tables\_transitively} or {\tt abolish\_tables\_singly}.

{\bf Error Cases} Error cases are the same as {\tt
  abolish\_table\_pred/1} but with the additions: 
\bi
\item {\tt Options} is a variable, or contains a variable as an element
\bi
\item {\tt instantiation\_error}
\ei
\item {\tt Options} is not a list
\bi
\item {\tt type\_error(list,Options)}
\ei
\item {\tt Options} contains an option {\tt O} that is not a
  table abolish option.
\bi
\item {\tt domain\_error([abolish\_tables\_transitively, abolish\_tables\_singly,O)}
\ei
\ei
% - - - - - - - - - - - - - - - - - - - - - - - - - - - - - - - - - - - - -

\index{tabling!abolishes!transitive vs. single}
\ourstandarditem{abolish\_table\_subgoals(+Subgoal)}{abolish\_table\_subgoals/1}{Tabling}
%
Invalidates the table for any subgoal that unifies with {\tt
 Subgoal}.  If a subgoal $S$ unifying with {\tt Subgoal} contains
an answer $A$ that is active in the current enviornment, the table
entry for $S$ will not be reclaimed until $A$ is no longer active;
otherwise the space for $S$ will be reclaimed immediately.

\index{Prolog flags!{\tt table\_gc\_action}}
If $S$ contains a conditional answer, the default behavior will be to
transitively abolish any subgoals that depend on any conditional
answers of $S$.  This default may be changed either by setting an XSB
flag:
%
\begin{verbatim}
?- set_xsb_flag(table_gc_action,abolish_tables_singly).
\end{verbatim}
% 
or by calling {\tt abolish\_table\_subgoal/2} with the appropriate
option.  If the transitive abolishes are turned off, and $S$ contains
a conditional answer, the warning

{\tt abolish\_table\_subgoal/1 is deleting a table with
  conditional\ answers: \\ delay dependencies may be corrupted.}

will be issued.  

%\index{tabling!abolishes!multi-threading}
%In the multi-threaded engine, if $S$ is a subgoal for a predicate that
%is shared, reclamation for $S$ will be delayed until there is a single
%active thread and no answer in $S$ is active in the current execution
%environment.  Otherwise, the behavior of {\tt abolish\_table\_subgoal/1}
%is the same as in the sequential engine on tabled predicates that are
%thread-private.

\index{Incremental Dependency Graph (IDG)} 
\index{tabling!abolishes!incremental}
%
For incremental tables, {\tt abolish\_table\_subgoal/[1,2]} not only
deletes the table structures for {\tt Subgoal}, but pointers to {\tt
  Subgoal} in the Incremental Dependency Graph (IDG), after
invalidating all subgoals that depend on {\tt Subgoal}. The node and
edges for {\tt Subgoal} will be reinserted into the IDG when {\tt
  Subgoal} is re-evaluated, either lazily or by an explicit update
command.

%  incremental tables.
%If a call $G$ is abolished, all calls that $G$ depends on will also be
%abolished, so that the dependency structures that support incremental
%tabling will remain in a consistent state.

{\bf Error Cases}
\bi
\item The term spec {\tt Subgoal} does not correspond to a tabled predicate:
\bi
\item 	{\tt table\_error}
\ei
\item The term spec {\tt Subgoal} unifies with a tabled subgoal that is incomplete:
\bi
\item 	{\tt permission\_error}
\ei
\item The term spec {\tt Subgoal} is a cyclic term::
\bi
\item 	{\tt table\_error}
\ei
%\item {\tt Subgoal} unifies with a table that is formed from a
%  predicate that does not use variant tabling.  
%\bi
%\item 	{\tt table\_error}
%t\ei
\ei
%

\index{Prolog flags!{\tt table\_gc\_action}}
\ourstandarditem{abolish\_table\_subgoals(+Subgoal,+Options)}{abolish\_table\_subgoals/2}{Tabling}
%
Behaves as {abolish\_table\_subgoals/1}, but allows the default {\tt
  table\_gc\_action} to be over-ridden with a flag, which can be either 
{\tt abolish\_tables\_transitively} or {\tt abolish\_tables\_singly}.

{\bf Error Cases} Error cases are the same as {\tt
  abolish\_table\_subgoal/1} but with the additions: 
\bi
\item {\tt Options} is a variable, or contains a variable as an element
\bi
\item {\tt instantiation\_error}
\ei
\item {\tt Options} is not a list
\bi
\item {\tt type\_error(list,Options)}
\ei
\item {\tt Options} contains an option {\tt O} that is not a
  table abolish option.
\bi
\item {\tt domain\_error([abolish\_tables\_transitively, abolish\_tables\_singly,O)}
\ei
\item {\tt Subgoal} is a subgoal for a predicate that is tabled with
  subsimptive indexing.
  \bi
\item {\tt permission\_error}. \footnote{Tables that are subsumptively
  indexed do not always directly associate subgoals with tables, but
  instead may generalize a subgoal.  For this reason it is not always
  straightforward how to abolish a subgoal in such a case. Such tables
  may be abolished via {\tt abolish\_table\_pred}/[1,2].}
  \ei \ei

% - - - - - - - - - - - - - - - - - - - - - - - - - - - - - - - - - - - - -

\ourrepeatstandarditem{abolish\_table\_subgoal(+Subgoal)}{abolish\_table\_subgoal/1}{Tabling}
\ourstandarditem{abolish\_table\_subgoal(+Subgoal,+Options)}{abolish\_table\_subgoal/2}{Tabling}
%
These predicates behave as {\tt abolish\_table\_subgoals/[1,2]}.
However rather than abolishing all tables whose subgoal {\em unifies}
with {\tt Subgoal} they only abolish the table whose subgoal is a {\em
  variant} of {\tt Subgoal}, if such a table exists.

% - - - - - - - - - - - - - - - - - - - - - - - - - - - - - - - - - - - - -

%\index{tabling!abolishes!multi-threading}
\index{tabling!abolishes}
\ourstandarditem{abolish\_all\_tables}{abolish\_all\_tables/0}{Tabling}
%
In the single-threaded engine, removes all tables presently in the
system and frees all the memory held by XSB for these structures.
Predicates that have been declared tabled remain so, but information
in their table is deleted.  {\tt abolish\_all\_tables/0} works
directly on the memory structures allocated for table space.  This
makes it very fast for abolishing a large amount of tables, and to
maintain its speed it throws an error if the current execution
enviornment contains any incomplete tables, or any active completed
tables.  {\tt abolish\_all\_tables/0} can be used regardless of
whether there are incremental tables, or tables that use call or
answer subsumption.

%In the multi-threaded engine {\tt abolish\_all\_tables/0} additionally
%raises an error unless it is called when there is a single active
%thread.  In that case, all shared tables are abolished as well as all
%private tables for the main thread.
%An error will be thrown if any completed
%answer $A$ is active in the current environment, regardless of whether
%$A$ is thread-private or thread-shared.  



{\bf Error Cases}
\bi
\item There are incomplete tables at the time of the predicate's call;
\bi
\item 	{\tt permission\_error}
\ei
\item The current execution environment has an active completed table $T$
\bi
\item 	{\tt permission\_error}
\ei
%\item (Multi-threaded engine only) More than one thread is active:
%\bi
%\item 	{\tt table\_error}
%\ei
%
%
\ei

% - - - - - - - - - - - - - - - - - - - - - - - - - - - - - - - - - - - - -

\index{tabling!abolishes!incremental}
\index{tabling!abolishes!}
%\ourstandarditem{abolish\_nonincremental\_tables(+Option)}{abolish\_nonincremental\_tables/1}{Tabling}
\ourstandarditem{abolish\_nonincremental\_tables}{abolish\_nonincremental\_tables/0}{Tabling}
%
Abolishes all tabled calls for predicates that are {\em not} declared
to be incremental.~\footnote{Calls for predicates that are declared as
  opaque are considered to be non-incremental.} This predicate
allows XSB to function in a manner similar to that of a deductive
database: incremental tables will be automatically updated when the
data they depends on changes; while non-incremental tables, which may
have become invalid, can be abolished.  As currently implemented, {\tt
  abolish\_nonincremental\_tables/1} traverses through each
nonincremental tabled predicate, $Pred$, and if $Pred$ has any
incomplete subgoals, a permission error will be thrown.  However,
unlike with {\tt abolish\_all\_tables/0} no errors will be thrown if
there are active completed tables: rather these tables will be marked
for deletion and their space later garbage collected.  In addition, no
error will be thrown if there are incomplete incremental subgoals.

\comment{
The {\em Option} argument currently allows control of a single parameter:

\bi
\item {\tt on\_incomplete(+Action)} where {\tt Action} may equal
  either {\tt error} or {\tt skip}.  If {\tt Action} is {\tt error}, a
  permission error is thrown if there is currently an incomplete
  non-incremental table.  If {\tt Action} is {\tt skip}, no action
  will be taken for any incomplete non-incremental table.  
\ei 
%
The argument may be specified either as a list, i.e., {\tt
  [on\_incomplete(...)]} or as {\tt on\_incomplete(...)}.
}

{\bf Error Cases}
\bi
\item There are incomplete nonincremental tables at the time of the predicate's call;
\bi
\item 	{\tt permission\_error}
\ei
\ei

% - - - - - - - - - - - - - - - - - - - - - - - - - - - - - - - - - - - - -

%\index{tabling!abolishes!multi-threading}
%\index{tabling!abolishes!}
%\ourstandarditem{abolish\_all\_private\_tables}{abolish\_all\_private\_tables/0}{Tabling}
%%
%In the multi-threaded engine, removes all tables private to the thread
%and frees all the memory held by XSB for these structures, including
%space for conditional answers.  Predicates that have been declared
%tabled remain so, but information in their table is deleted.  Like
%{\tt abolish\_all\_tables/0}, {\tt abolish\_all\_private\_tables/0}
%works directly on the memory structures allocated for table space.
%This makes it very fast for abolishing a large amount of tables, and
%to maintain its speed it throws an error if the current execution
%enviornment contains any incomplete tables, or any active completed
%tables.  {\tt abolish\_all\_private\_tables/0} can be used regardless of
%whether there are incremental tables, or tables that use call or
%answer subsumption.

%In the single-threaded engine, {\tt abolish\_all\_private\_tables/0}
%is defined as {\tt abolish\_all\_tables/0}.

%{\bf Error Cases}
%\bi
%\item There are incomplete tables at the time of the predicate's call;
%\bi
%\item 	{\tt permission\_error}
%\ei
%\item The current execution environment for the thread has an active
%  private table $T$ for the current thread
%\bi
%\item 	{\tt table\_error}
%\ei
%%
%\ei

% - - - - - - - - - - - - - - - - - - - - - - - - - - - - - - - - - - - - -

%\index{tabling!abolishes!multi-threading}
%\index{tabling!abolishes!}
%\ourstandarditem{abolish\_all\_shared\_tables}{abolish\_all\_shared\_tables/0}{Tabling}
%%
%In the multi-threaded engine, removes all tables private to the thread
%and frees all the memory held by XSB for these structures, including
%space for conditional answers.  Predicates that have been declared
%tabled remain so, but information in their table is deleted.  {\tt
%  abolish\_all\_private\_tables/0} works directly on the memory
%structures allocated for table space.  This makes it very fast for
%abolishing a large amount of tables, and to maintain its speed it
%throws an error if the current execution enviornment contains any
%incomplete tables, or any active completed tables.  {\tt
%  abolish\_all\_private\_tables/0} can be used regardless of whether
%there are incremental tables, or tables that use call or answer
%subsumption.  In addition, {\tt abolish\_all\_shared\_tables/0} raises
%an error unless it is called when there is a single active thread.  If
%called with a single active thread, all shared tables are abolished,
%but private tables for the main thread are unaffected.
%
%
%{\bf Error Cases}
%\bi
%\item There are incomplete tables at the time of the predicate's call;
%\bi
%\item 	{\tt permission\_error}
%\ei
%\item The current execution environment has an active table $T$
%\bi
%\item 	{\tt permission\_error}
%\ei
%\item More than one thread is active:
%\bi
%\item 	{\tt table\_error}
%\ei
%%
%%
%\ei

% - - - - - - - - - - - - - - - - - - - - - - - - - - - - - - - - - - - - -
\comment{
% \ournewitem{abolish\_tables(+OptionsList)}{Tabling}
% \index{\texttt{abolish\_tables/1}}

% Removes tables as specified by options in {\tt OptionsList}, which is
% a list of terms of the form:

% \begin{itemize}

% \item {\tt private} In the multi-threaded engine, this option
%   abolishes all tables private to the calling thread.  In the
%   sequential engine, this option abolishes all tables.

% \item {\tt shared} In the multi-threaded engine, this option abolishes
%   all thread-shared tables.  Space reclamation for thread-shared
%   tables may not be performed until there is a single active thread.
%   In the sequential engine, this option has no effect.

% \item {\tt module(+Module)} Abolishes all tables in a given module,
%   whether they are shared or private.  If {\tt Module} is set to {\tt
%     usermod}, all tables not explicitly placed into a module are
%   abolished.

% \item {\tt pred(+PredSpecifier)} Abolishes the tabled predicate
%   referred to by {\tt PredSpecifier}, regardless of whether it is
%   shared or private.  

% \end{itemize}

% When {\tt OptionsList} contains several options, the options are taken
% to be disjunctive in that any table that is covered by at least one of
% the options will be abolished.  For instance
% \begin{verbatim}
%     abolish_tables([private,module(cdf),predicate(f/2)])
% \end{verbatim}
% abolishes all tables that are private {\em or} in the module {\tt cdf}
% {\em or} have been created by calls to the predicate {\tt f/2}.
}
% - - - - - - - - - - - - - - - - - - - - - - - - - - - - - - - - - - - - -
\index{tabling!abolishes!}
\ourstandarditem{abolish\_module\_tables(+Module)}{abolish\_module\_tables/1}{Tabling}
%
Given a module name (or the default module, {\tt usermod}), this
predicate abolishes all tables for each tabled predicate in {\tt
  Module}.  It is implemented using a series of calls to {\tt
  abolish\_table\_pred/1} and so inherits the behavior of that
predicate.

% - - - - - - - - - - - - - - - - - - - - - - - - - - - - - - - - - - - - -

\ourstandarditem{gc\_tables(-Number)}{gc\_tables/1}{Tabling}
\index{garbage collection!tables}
%
When a tabled subgoal or predicate is abolished, reclamation of its
space may be postponed if the subgoal or predicate has an answer that
is active in the current environment.  A garbage collection routine is
called at various points in execution to check which answers are
active in the current environment, and to reclaim the space for
subgoals and predicates with no active answers.  In particular, space
for all abolished tables is reclaimed whenever the engine re-executes
the main command-line or C thread interpreter code.  However in rare
situations this strategy may not be adequate.  For this reason, the
user can explicitly call the table garbage collector to reclaim space
for any deleted tabled predicates or subgoals that no longer have
active answers.

{\tt gc\_tables/1} always succeeds,  unifying {\tt Number}
%to $-1$ if
%garbage collection was not attempted (due to multiple active threads)
%and otherwise
to the number of tables still unreclaimed at the end of garbage
collection.

{\bf Error Cases}
\bi
\item {\tt Number} is not a variable
\bi
\item 	{\tt type\_error(variable)}
\ei
\ei

% - - - - - - - - - - - - - - - - - - - - - - - - - - - - - - - - - - - - -

\ourstandarditem{delete\_return(+TableEntryHandle,+ReturnHandle)}{delete\_returns/2}{Tabling}
%
Removes the answer indicated by \code{ReturnHandle} from the table
entry referenced by \code{TableEntryHandle}.  The value of each
argument should be obtained from some previous invocation of a
table-inspection predicate.

This predicate is low-level so no error checking is done.  In \version
, this predicate does not reclaim space for deleted returns, but
simply marks the returns as invalid.

{\em Warning: } While useful for purposes such as tabled aggregation,
{\tt delete\_return/2} can be difficult to use, both from an
implementation and semantic perspective.  

\ourstandarditem{invalidate\_tables\_for(+DynamicPredGoal,+Mode)}{invalidate\_tables\_for/2}{Tabling}
%
{\em Note that using incremental tabling provides a simpler and much
  more powerful approach to maintaining dependencies of tables on
  dynamic code.  {\tt invalidate\_tables\_for/2} should only be used
  in cases where incremental tabling is not available (e.g.,
  subsumptive tabling).}

This predicate supports invalidation of tables.  Tables may become
invalid if dynamic predicates on which they depend change, due to
asserts or retracts.  By default XSB does not change or delete tables
when they become invalid; it is the user's responsibility to know when
a table is no longer valid and to use the {\tt abolish\_table\_*}
primitives to delete any table when its contents become invalid.

This predicate gives the XSB programmer some support in managing
tables and deleting them when they become invalid.  To use this
predicate, the user must have previously added clauses to the dynamic
predicate, {\tt invalidate\_table\_for/2}.  That predicate should be
defined to take a goal for a dynamic predicate and a mode indicator
and abolish (some) tables (or table calls) that might depend on (any
instance of) that fact.  \\ {\tt
invalidate\_tables\_for(+DynamicPredGoal),+Mode} simply backtracks
through calls to all unifying clauses of \\ {\tt
invalidate\_table\_for(+DynamicPredGoal,+Mode)}.  The {\tt Mode}
indicator can be any term as long as the two predicates agree on how
they should be used.  The intention is that {\tt Mode} will be either
'assert' or 'retract' indicating the kind of database change being
made.

Consider a simple example of the use of these predicates: Assume the
definition of tabled predicate {\tt ptab/3} depends on dynamic predicate
{\tt qdyn/2}.  In this case, the user could initially call:
\begin{verbatim}
      :- assert((invalidate_table_for(qdyn(_,_),_) :-
                    abolish_table_pred(ptab(_,_,_)))).
\end{verbatim}
to declare that when {\tt qdyn/2} changes (in any way), the table for {\tt ptab/3}
should be abolished.  Then each time a fact such as {\tt qdyn(A,B)} is
asserted to, or retracted from, {\tt qdyn/2}, the user could call
\begin{verbatim}
      :- invalidate_table_for(qdyn(A,B),_).
\end{verbatim}

The user could use the hook mechanisms in XSB (Chapter~\ref{hooks}) to
automatically invoke \\ {\tt invalidate\_tables\_for} whenever {\tt
  assert} and/or {\tt retract} is called.


% - - - - - - - - - - - - - - - - - - - - - - - - - - - - - - - - - - - - -

\subsection{Indexing Subsumptive Tables} \label{sec:table-index}

%\index{table\_index}

%Tables are implemented in XSB using tries, which provide powerful
%indexing capabilities.  By default every table has a trie-index based
%on the left-to-right ordering of its arguments.  This means that any
%lookup to a table which is bound on an initial sequence of the tabled
%predicate's arguments is fully indexed.  But if an initial sequence of
%a table lookup is unbound and only a later argument is bound, indexing
%can be very poor.  To address this, XSB allows the user to provide
%{\tt table\_index} declarations to improve indexed access to tables.
%In fact a {\tt table\_index} can be used to provide very powerful and
%general indexed access to any predicate.

%\ourstandarditem{table\_index(+PredSpec,+IndexSpec)}{table\_index/2}{Tabling}
%
%{\tt table\_index} is a directive that causes the compiler to generate
%code to create tables that provide the requested indexing, and is
%normally used with subsumptive tables.  {\tt PredSpec}, of the form
%{\tt PredName/Arity}, specifies the predicate to be indexed (and
%subsumptively tabled.)  {\tt IndexSpec} is a list of index
%specifications.  Each index specification is a {\tt +}-term that
%indicates a set of argument positions for which indexes are required.
%For example, {\tt [1+2+3, 1, 2+3, 2]} indicates that four indexes are
%desired; a multiple argument index on arguments 1, 2, and 3; a single
%argument index on argument 1; a multi-argument index on 2 and 3; and a
%single argument index on argument 2.  In this case, for a call to the
%indicated predicate, if arguments 1, 2, and 3 are bound, that index
%will be used; if not, but argument 1 is bound, then that index will be
%used; if not but arguments 2 and 3 are bound, then that index will be
%used, and finally if argument 2 is bound, then that index will be
%used.  If none of these situations obtain, then a {\tt table\_error}
%will be thrown.

%The declared indexes should describe modes under which the predicate
%will be called.  Since the transformation assumes that the predicate
%will be called in all indicated modes, it will normally abstract all
%calls to a more general call.  So any call will be abstracted to the
%call to the base predicate that is bound only on argument positions
%that appear in {\em all} indexes.  For the example above the first
%call will result in a completely open call to the base predicate,
%since the intersection of the four indexes is empty.  An index of {\tt
%  0} can be used (as the last index in the index specification) to
%indicate that any call is permitted, and this will always ensure the
%fully open call is made initially.  For example an index specification
%of {\tt [1,0]} causes the first call to be abstracted to the fully
%open call and the subsumptive table to be completely filled.  Then
%every subsequent call will use that constructed table.  A call to a
%{\tt table\_index}-ed predicate that is not of a mode explicitly
%declared in the index specification may not be optimally indexed.

%The declared indexes should describe modes under which the
%predicate will be called.  Since the transformation assumes that the
%predicate will be called in all indicated modes, it will normally
%abstract all calls to a more general call.  So any call will be
%abstracted to the call to the base predicate that is bound only on
%argument positions that appear in {\em all} indexes.  For the example
%above the first call will result in a completely open call to the base
%predicate, since the intersection of the four indexes is empty.  An
%index of {\tt 0} can be used (as the last index in the index
%specification) to indicate that any call is permitted, and this will
%always ensure the fully open call is made initially.  For example an
%index specification of {\tt [1,0]} causes the first call to be
%abstracted to the fully open call and the subsumptive table to be
%completely filled.  Then every subsequent call will use that
%constructed table.  A call to a {\tt table\_index}-ed predicate that
%is not of a mode explicitly declared in the index specification may
%not be optimally indexed.

%As mentioned above, a predicate indicated as {\tt table\_index}-ed
%should {\bf not} be declared as (normally) tabled.  The indexes are
%created using subsumptive tables.  An attempt is made to use the
%smallest number of tables as possible, but each index does take
%(perhaps significant) memory.  For example the above four indexes can
%be accommodated using only two subsumptive tables: with argument
%orders of {\tt [1,2,3,4]} and {\tt [2,3,1,4]}.

%The tables of table-indexed predicates can be abolished by \\ {\tt
%  abolish\_table\_pred}, which will abolish all the generated
%subsumptive tables.  But the tables cannot be removed by {\tt
%  abolish\_table\_subgoals/1} and its finer variants.  Also these
%predicates cannot be declared as {\tt incremental}.

%A deeper discussion of how {\tt table\_index/2} can be used and its
%Implications can be found in \cite{warren-tdbu}.

Within XSB, tabled calls and their answers are both implemented using
tries, which provide powerful indexing capabilities: by default every
table has a trie-index based on the left-to-right ordering of its
arguments.  This mechanism works very well for nearly all tabling
operations and is a major factor in XSB's tabling speed.

However, the use of tries may be slow in certain cases when call
subsumption is used.  Recall that in call subsumption, a goal
$G_{subsumed}$ may re-use the table of a subsuming call
$G_{subsumer}$.  If the initial prefix of $G_{subsumer}$ is used by
$G_{subsumed}$, the answer trie will provide excellent indexing: for
instance the table for the goal {\tt p(a,X,Y)} will properly index the
subsumed call {\tt p(a,b,Y)}.  But if this is not the case, and the
table for the $G_{subsuming}$ is large, indexing can be very poor: for
instance the table for goal {\tt p(a,X,Y)} may provide poor indexing
for {\tt p(a,X,c)}.  (This would occur if there are a number of
  different instantiations of {\tt X} in the answer trie for {\tt
    p(a,X,Y)}.)

To address this, XSB allows the user to specify indexing of call
subsumption via the {\tt index/1} option of a tabling declaration.

{\em :- table PredSpec as index(IndexSpec).}

This option causes the compiler to generate code to create tables that
provide the requested indexing.  {\tt IndexSpec} is a list of index
specifications.  Each index specification is a {\tt +/2}-term that
indicates a set of argument positions for which indexes are required.
For instance the index specification {\tt [1+2+3, 1, 2+3, 2]}
indicates that four indexes are desired; a multiple argument index on
arguments 1, 2, and 3; a single argument index on argument 1; a
multi-argument index on 2 and 3; and a single argument index on
argument 2.

Although the syntax used to specify table indexing is similar to that
used to specify (non-trie) indexing on dynamic predicates, its
behavior is somewhat different.  First, trie indexing provides full
indexing of each term in each argument of an answer, and the {\tt
  index/1} option of the {\tt table} directive makes use of tries to
create multiple answer tries for different orders of arguments.  And
second, {\bf {\em  the index declaration is expected to be a true mode
declaration so that any calls outside of that declaration will throw
errors.}}

This operational aspect of table indexing is perhaps best seen on a
simpler example:

{\tt :- table triple/3 as index([1,2,3]).}

Suppose a goal {\tt triple(e1,Rel,Target)} is called.  This call will
create a table $T_{1,2,3}$ whose trie has argument order
1,2,3 \footnote{Due to the indexing specification this first order is
  the default order used by tables, but the form of the first ordering
  may be different for different specifications.}.  In creating
$T_{1,2,3}$ the original call is abstracted to {\tt
  triple(Source,Rel,Target)}.  Once $T_{1,2,3}$ is completed, other
calls with a nonground first argument can efficiently use this trie.
Next, if a call {\tt triple(Source,r1,Target)} is made, a new table
with argument order 2,1,3 is made for an abstracted call {\tt
  triple(Rel,Source,Target)}.  Finally, if the call {\tt
  triple(Source,Rel,e2)} is made, a third abstracted table will be
created with argument order 3,1,2.

Table indexing is performed by hidden rules generated by a program
transformation executed by XSB's compiler.  These hidden rules ensure
that the proper tries are created on demand and the tries thus created
are properly reused.  As part of the transformation, the compiler will
try to support the desired indexing with as few tries as possible, in
order to save space.  All tables created by the transformation can be
abolished by {\tt abolish\_table\_pred/[1,2]}, which will abolish all
the generated subsumptive tables.  But the tables cannot be removed by
{\tt abolish\_table\_subgoals/1} and its finer variants.  See
Section~\ref{} for the compatability of table indexing with other
tabling options.

A fuller discussion of how {\tt table\_index/2} can be used and its
Implications can be found in \cite{warren-tdbu}.

\paragraph*{Example}
Consider the declaration

{\tt table p/4 as index([3+2+1, 1, 2+3, 2]})

that was discussed above.  The compiler can satisfy this specification
using only two subsumptive tables havng with argument orders {\tt
  [1,2,3,4]} and {\tt [2,3,1,4]}.  Note that this is because the {\tt
  3+2+1} is equivalent to {\tt 1+2+3}, so an argument ordering of {\tt
  [1,2,3,4]} satisfies both {\tt 1+2+3} and {\tt 1}.  Similarly the
argumet ordering {\tt [2,3,1,4]} -- or any argument ordering beginning
with {\tt [2,3...]}. satisfies both {\tt 2+3} and {2}.

{\em Exceptions}
\bi
\item A call is made to a predicate in {\tt PredSpec} that is not of declared mode:
\bi
\item {\tt table\_error}
\ei
\item The tabling declaration contains an option that is not supported
  together with table indexing.
  \bi
  \item {\tt permission\_error}
  \ei
\ei

%In addition, an index of {\tt 0} can be used (as the last index in the
%index specification) to indicate that any call is permitted, and this
%will always ensure the fully open call is made initially.  For example
%an index specification of {\tt [1,0]} causes the first call to be
%abstracted to the fully open call and the subsumptive table to be
%completely filled.  Then every subsequent call will use that
%constructed table.  A call to a {\tt table\_index}-ed predicate that
%is not of a mode explicitly declared in the index specification may
%not be optimally indexed.

%As mentioned above, a predicate indicated as {\tt table\_index}-ed
%should {\bf not} be declared as (normally) tabled.  The indexes are
%created using subsumptive tables.  An attempt is made to use the
%smallest number of tables as possible, but each index does take
%(perhaps significant) memory.  For example the above four indexes can
%be accommodated using only two subsumptive tables: with argument
%orders of {\tt [1,2,3,4]} and {\tt [2,3,1,4]}.

\end{description}


%==========================================================================

%%%
%%% OLD JUNK
%%%

\comment{

%\ournewitem{table\_prop(?Pred\_indicator,+Type,?Old\_val,+New\_val)}{Tabling}
%\index{\texttt{table\_prop/4}}
%This predicate is described together with the following one.

%\ournewitem{table\_prop(?Pred\_indicator,+Type,-Value)}{Tabling}\index{\texttt{table\_prop/3}}
%This predicate is used either to provide information about a table
%property, or to reset one of properties to a particular value.  These
%values can also be set by the table declarations in the source code.
%{\tt table\_prop} provides a method for doing so without having to
%recompile code.
%
%In previous versions {\tt table\_prop} was used to reset hash sizes for
%calls and returns.  It is not used in \version, but is maintained since
%in future versions {\tt table\_prop} may be used for other table
%properties such as subgoal subsumption.

}

%In \version\ the properties visible to the user are the call and return
%hash sizes.  Each tabled predicate contains two types of hash tables.
%A {\em call hash table} is used to find whether the variant of a call
%to a tabled predicate has occurred before in the course of a
%computation.  Each tabled call has a {\em return hash table} which is
%used during a computation to determine whether a given return needs to
%be added to the table.  Appropriate indexing for call and return
%lookup, including proper hash table sizes, is crucial to good
%performance.  

%Hash table sizes are non-negative integers denoting the number of
%buckets in the hash table.  When {\tt Pred\_indicator} denotes a
%tabled predicate, the predicate can take one of two modes depending on
%its arity.  In {\tt table\_prop/4}, the {\tt New\_value} is provided by
%the user and {\tt Old\_value} is unified with the precious value of
%the table property.  In {\tt table\_prop/3} the present value of the 
%property is returned.

%Presently, the hash table sizes can only be reset when there are no tabled
%calls to {\tt Pred\_indicator}.  This can be ensured by a call to 
%{\tt abolish\_table\_pred/1}
%
%Future versions of \ourprolog\ may include additional indexing
%strategies as configurable table properties.

\comment{

    Exceptions:
    \begin{description}
%    \item[{\tt table\_error}]
%       Attempt to reset hash table size for predicate with tabled calls.
    \item[{\tt table\_error}]
        Argument 1 is not a tabled predicate.
    \end{description}

}
%\ounewitem{table\_copy(+From,-To)} \index{\texttt{table\_copy/2}}
%Table copy provides an explicit mechanism for the user to reduce the
%amount of copying into and out of a table.  {\tt table\_copy/2}
%succeeds when {\tt From} is ground, and returns a copy of that
%structure (or list) in the table space.  No copying of that structure
%will be required when calls or returns containing it are tabled, and
%no copying will be necessary when returns using it are resolved with
%goals. 
%
%For an example of the use of table copy, see Section~\ref{dcg_tabling}

\comment{

%\ournewitem{breg\_retskel(+Breg,+Arity,-ReturnSkeleton,-SubgoalStructurePtr)}
%\index{\texttt{breg\_retskel/4}}
%If Breg is bound to a pointer to the tabling choice point and 
%Arity is bound to the arity of the predicate corresponding to 
%the choice point, {\tt breg\_retskel} succeeds after binding 
%{\tt ReturnSkeleton} to ret/n as in {\tt get\_calls/3} and 
%{\tt SubgoalStructurePtr} to the pointer to the subgoal structure.

%Warning:This built-in works for tabling predicates only.

}

%========================================================

\end{description}



%%% Local Variables: 
%%% mode: latex
%%% TeX-master: "manual1"
%%% End: 
